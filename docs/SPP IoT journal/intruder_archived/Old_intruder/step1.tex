% !TEX root = ../SPP_journal.tex
\subsection{Step 1: Induced Obstacle Computation \label{sec:intruder_iocomp}}
The goal of this section is to compute, for each lower priority vehicle $\veh{i}$, the time-varying obstacle induced by each higher priority vehicle $\veh{j}, j < i$, denoted by $\ioset_i^j(t)$. As before, once $\ioset_i^j(t)$ is computed, one can then solve \eqref{eq:HJIVI_i} with the union of all obstacles induced by higher priority vehicles as the total obstacle set $\obsset_i(t)$. 

Note that since there are no moving vehicle obstacles for the highest priority vehicle, $\obsset_1(t) = \soset$. To compute the obstacle set $\ioset_i^j(t)$ where $i> 1$, we first compute the ``base" obstacles using any of the three methods outlined in Section \ref{sec:incomp}. Base obstacles correspond to the states which a vehicle can reach when it is not avoiding an intruder. Computation of these base obstacles would requires information of a corresponding ``base" BRS of $\veh{j}$; the process for computing this set is outlined in Step 2. In this section, we assume that the sequence of base obstacles, $\boset_i^j(t)$, is known. Given $\boset_i^j(t)$, we now show how to compute the obstacle set $\obsset_i(t)$. 

The induced obstacles are given by the states a vehicle can reach while avoiding the intruder, on top of the base obstacles. Since a vehicle avoids the intruder for a maximum of $\iat$, these states can be given by the $\iat$-horizon FRS of the base obstacle. Regardless of what control is used by $\veh{j}$ for avoidance, it still remains within the FRS $\ioset_i^j(t) := \frs_{\mathcal{O}}(\iat, \boset_i^j(t-\iat), \emptyset, \ham_{\mathcal{O}})$, which is the set of all possible states that $\veh{j}$ can reach after a duration of $\iat$ starting from inside $\boset_i^j(t-\iat)$. Here, the Hamiltonian is given by

\begin{equation}
\ham_{\mathcal{O}}(\state_j, p) = \min_{\ctrl_j} \min_{\dstb_j} p \cdot f_j (\state_j, \ctrl_j, \dstb_j)
\end{equation}
