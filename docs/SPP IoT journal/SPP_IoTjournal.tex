%%%%%%%%%%%%%%%%%%%%%%%%%%%%%%%%%%%%%%%%%%%%%%%%%%%%%%%%%%%%%%%%%%%%%%%%%%%%%%%%
%2345678901234567890123456789012345678901234567890123456789012345678901234567890
%        1         2         3         4         5         6         7         8

\documentclass[journal]{IEEEtran}  
%\documentclass[12pt, draftcls, onecolumn]{IEEEtran}      

\IEEEoverridecommandlockouts                              % This command is only
                                                          % needed if you want to
                                                          % use the \thanks command
%\overrideIEEEmargins
% See the \addtolength command later in the file to balance the column lengths
% on the last page of the document

\usepackage{mathtools}    % need for sub equations
\usepackage{amsfonts}
\usepackage{graphicx}   % need for figures
\usepackage{subcaption}
\usepackage{epsfig} 
\usepackage{cancel}
\usepackage{amssymb}
\usepackage{color}
\usepackage{bm}
\usepackage[ruled,vlined,titlenotnumbered]{algorithm2e} 
\usepackage{todonotes} \setlength{\marginparwidth}{2.5cm} 
\usepackage{float}
\usepackage{cite}

\newcommand{\MCnote}{\textcolor{red}}
\newcommand{\SBnote}{\textcolor{blue}}

\newcommand{\R}{\mathbb{R}} % Real number
\newcommand{\dist}{\text{dist}} % Distance
\newcommand{\rc}{R_c} % Capture radius
\newcommand{\cradius}{\rc}
\newcommand{\N}{N} % number of agents

\newcommand{\veh}{Q} % vehicle
\newcommand{\intr}{I} % Intruder index
\newcommand{\state}{x} % state
\newcommand{\ctrl}{u} % control
\newcommand{\dstb}{d} % disturbance
\newcommand{\pos}{p} % position
\newcommand{\npos}{h} % non-position states

\newcommand{\traj}{\zeta}
\newcommand{\errstate}{e}

\newcommand{\fdyn}{f} % full dynamics
\newcommand{\cset}{\mathcal{U}} % Control set
\newcommand{\cfset}{\mathbb{U}} % control function set
\newcommand{\dset}{\mathcal{D}} % disturbance
\newcommand{\dfset}{\mathbb{D}} % disturbance function set
\newcommand{\obsset}{\mathcal{G}} % Obstacle (the one used to solve PDE)
\newcommand{\dz}{\mathcal{Z}} % danger zone
\newcommand{\sep}{\mathcal{S}} % Separation region
\newcommand{\buff}{\mathcal{B}} % Buffer region

\newcommand{\valfunc}{V} % value function
\newcommand{\valfuncfwd}{W} % value function for forwards reachable set
\newcommand{\brs}{\mathcal{V}} % backwards reachable set
\newcommand{\frs}{\mathcal{W}} % forwards reachable set
\newcommand{\pfrs}{\mathcal{P}} % projected forwards reachable set
\newcommand{\targetset}{\mathcal{L}} % target set
\newcommand{\ham}{H} % Hamiltonian
\newcommand{\fc}{l} % Final condition
\newcommand{\ic}{l} % Initial condition
\newcommand{\obsfunc}{g} % Obstacle function
\newcommand{\costate}{\lambda}

\newcommand{\disckernel}{\Omega} % Discriminating kernel

\newcommand{\edt}{t^\text{EDT}} % earliest departure time
\newcommand{\ldt}{t^\text{LDT}} % latest departure time
\newcommand{\sta}{t^\text{STA}} % scheduled time of arrival
\newcommand{\ioset}{\mathcal{O}} % Induced obstacle
\newcommand{\boset}{\mathcal{M}} % Base obstacle
\newcommand{\sosetp}{\mathcal{S}} % static obstacle in position space
\newcommand{\soset}{\ioset^\text{static}} % static obstacle in state space
\newcommand{\iat}{t^\text{IAT}} % intruder avoidance time
\newcommand{\wcttr}{t^\text{WC}} % worst case TTR

\newcommand{\basicham}{\ham^\text{basic}}

\newcommand{\tsa}{\underline{t}} % time of start of avoidance
\newcommand{\tea}{\bar{t}} % time of end of avoidance
\newcommand{\nva}{\bar{k}} % Number of Vehicles to Avoid (NVA)
\newcommand{\brd}{t^\text{BRD}} % Buffer Region Duration (BRD)
\newcommand{\trd}{t^\text{RD}} % Remaining Duration (RD)
\newcommand{\rvs}{\mathcal{N}^\text{RP}} % Re-Planning Vehicle Set

\newcommand{\errorbound}{\mathcal{E}} % Error ``bubble" between vehicle and tracking reference
\newcommand{\tracklaw}{\kappa} % Robust tracking law

\newtheorem{assumption}{Assumption}
\newtheorem{alg}{Algorithm}
\newtheorem{remark}{Remark}
\newtheorem{observation}{Observation}

\title{\LARGE \bf Large-Scale Robust Sequential Path Planning}

\author{Somil Bansal*, Mo Chen*, Ken Tanabe, and Claire J. Tomlin
\thanks{This work has been supported in part by NSF under CPS:ActionWebs (CNS-931843), by ONR under the HUNT (N0014-08-0696) and SMARTS (N00014-09-1-1051) MURIs and by grant N00014-12-1-0609, by AFOSR under the CHASE MURI (FA9550-10-1-0567). The research of M. Chen has received funding from the ``NSERC'' program and ``la Caixa" Foundation, respectively.}
\thanks{* Both authors contributed equally to this work. All authors are with the Department of Electrical Engineering and Computer Sciences, University of California, Berkeley. \{somil, mochen72, k-tanabe, tomlin\}@eecs.berkeley.edu}
}

\begin{document}
\maketitle
\thispagestyle{empty}
\pagestyle{empty}

%%%
\begin{abstract}
To be written.
\end{abstract}

% Introduction
% !TEX root = SPP2.tex
\section{Introduction}
Recently, there has been an immense surge of interest in using unmanned aerial vehicles (UAVs) for civil purposes. The applications of UAVs extend well beyond package delivery, and include aerial surveillance, disaster response, and other important tasks \cite{Tice91, Debusk10, Amazon16, AUVSI16, BBC16}. Many of these applications will involve UAVs flying in an urban environment, potentially in close proximity of humans. As a result, government agencies such as the Federal Aviation Administration (FAA) and National Aeronautics and Space Administration (NASA) of the United States are urgently trying to develop new scalable ways to organize an air space in which potentially thousands of UAVs can fly \cite{FAA13, NASA16}.

One essential problem that needs to be addressed is how a group of vehicles in the same vicinity can reach their destinations while avoiding collision with each other. Several previous studies have attempted to address this problem. In some of these studies, specific control strategies for the vehicles or moving entities are assumed, and approaches such as induced velocity obstacles have been used \cite{Fiorini98, Chasparis05, Vandenberg08}. Other researchers have used ideas involving virtual potential fields to maintain collision avoidance while maintaining a specific formation \cite{Saber02, Chuang07}. Although interesting results emerge from these previous studies, simultaneous trajectory planning and collision avoidance are not considered. 

In the past, trajectory planning and collision avoidance problems in safety-critical systems have been studied using reachability analysis, which provides guarantees on the success and safety of optimal system trajectories \cite{Barron90, Mitchell05, Bokanowski10, Margellos11, Fisac15}. In reachability analysis, one computes the reachable set, defined as the set of states from which the system can be driven to a target set. Reachability analysis has been successfully used in applications involving systems with no more than two vehicles, such as pairwise collision avoidance \cite{Mitchell05}, automated in-flight refueling \cite{Ding08}, two-player reach-avoid games \cite{Huang11}, and many others \cite{Bayen07}.

%In addition to the guarantees reachability theory provides and the evident flexibility of reachability theory for analyzing vastly different systems with nonlinear dynamics, many numerical tools for solving reachability problems are also available, making the approach practically appealing \cite{Mitchell05, Sethian96, Osher02, LSToolbox}.

Despite the advantages of reachability analysis, it cannot be directly applied to scenarios involving complex high dimensional systems such as multi-vehicle systems. The computation of reachable sets involves solving a Hamilton-Jacobi (HJ) partial differential equation (PDE) on a grid representing a discretization of the state space, causing an exponential scaling of computation complexity with respect to the dimension of the system, or roughly speaking, with the number of vehicles present.

In this paper, we build on the work in \cite{Chen15}, and assume a reasonable structure in the multi-vehicle path planning problem. In the sequential path planning (SPP) scheme, vehicles are assigned some priority. Higher-priority vehicles may ignore the lower-priority vehicles, who must take into account the presence of higher-priority vehicles by treating them as induced time-varying obstacles. Unlike the work in \cite{Chen15}, we incorporate disturbances for all vehicles and consider three different assumptions on the information each of the vehicles may have access to, making the sequential path planning substantially more practical. For each of the assumed information patterns, we propose a reachability-based method to compute the induced obstacles that would guarantee collision avoidance as well as successful transit to the destination. We demonstrate and compare our proposed methods through numerical simulations.

% Problem Formulation
% !TEX root = nextUAVsched.tex
\section{Problem Formulation \label{sec:formulation}}
Consider $N$ vehicles $P_i,i=1\ldots,N$, each trying to reach one of $N$ target sets $\target_i,i=1\ldots,N$, while avoiding obstacles and collision with each other. Each vehicle $i$ has states $\x_i\in \R^{n_i}$ and travels on a domain $\amb=\obs \cup \free\in\R^p$, where $\obs$ represents the obstacles that each vehicle must avoid, and $\free$ represents all other states in the domain on which vehicles can move. Each vehicle $i = 1,2,\ldots,N$ moves with the following dynamics for $t\in[\tnow_i, \tf_i]$:

\begin{equation} \label{eq:dyn}
\dotx_i = f_i (t, \x_i, \ctrl_i), \quad\x_i(\ti_i) = \x_i^0 
\end{equation}

\noindent where $\x_i^0$ represents the initial condition of vehicle $i$, and $\ctrl_i(\cdot)$ represents the control function of vehicle $i$. In general, $f_i(\cdot,\cdot,\cdot)$ depends on the specific dynamic model of vehicle $i$, and need not be of the same form across different vehicles. Denote $\pos_i\in\R^p$ the subset of the states that represent the position of the vehicle. Given $\pos_i^0\in\free$, we define the admissible control function set for $P_i$ to be the set of all control functions such that $\pos_i(t) \in \free \forall t\ge \ti_i$. Denote the joint state space of all vehicles $\x \in \R^n$ where $n = \sum_i n_i$, and their joint control $\ctrl$.

We assume that the control functions $\ctrl_i(\cdot)$ are drawn from the set $\ctrlf_i := \{\ctrl_i: [\tnow_i, \tf_i] \rightarrow \ctrlin_i, \text{measurable}$\footnote{
A function $f:X\to Y$ between two measurable spaces $(X,\Sigma_X)$ and $(Y,\Sigma_Y)$ is said to be measurable if the preimage of a measurable set in $Y$ is a measurable set in $X$, that is: $\forall V\in\Sigma_Y, f^{-1}(V)\in\Sigma_X$, with $\Sigma_X,\Sigma_Y$ $\sigma$-algebras on $X$,$Y$.}\} where $\ctrlin_i \in \R^{n^\ctrl_i}$ is the set of allowed control inputs. Furthermore, we assume $f_i(t,\x_i, \ctrl_i)$ is bounded, Lipschitz continuous in $\x_i$ for any fixed $t,\ctrl_i$, and measurable in $t, \ctrl_i$ for each $\x_i$. Therefore given any initial state $\x_i^0$ and any control function $\ctrl_i(\cdot)$, there exists a unique, continuous trajectory $\x_i(\cdot)$ solving (\ref{eq:dyn}) \cite{coddington55}.

The goal of each vehicle $i$ is to arrive at $\target_i \subset \R^{n_i}$ at or before some scheduled time of arrival (STA) $\tf_i$ in minimum time, while avoiding obstacles and danger with all other vehicles. The target sets $\target_i$ can be used to represent desired kinematic quantities such as position and velocity and, in the case of non-holonomic systems, quantities such as heading angle.  $\tnow_i$ can be interpreted as the earliest start time (EST) of vehicle $i$, before which the vehicle may not depart from its initial state. Further, we define $\ti_i$, the latest (acceptable) start time (LST) for vehicle $i$. Our problem can now be thought of as determining the LST $\ti_i$ for each vehicle to get to $\target_i$ at or before the STA $\tf_i$, and finding a control to do this safely. If the LST is before the EST $\ti_i < \tnow_i$, then it is infeasible for vehicle $i$ to arrive at $\target_i$ at or before the STA $\tf_i$. Comparing $\ti_i$ and $\tnow_i$ is feasibility problem that may arise in practice; however, for simplicity of presentation, we will assume that $\tnow_i\le \ti_i \forall i$.

Danger is described by sets $\danger_{ij}(\x_j) \subset \amb$. In general, the definition of $\danger_{ij}$ depends on the conditions under which vehicles $i$ and $j$ are considered to be in an unsafe configuration, given the state of vehicle $j$. Here, we define danger to be the situation in which the two vehicles come within a certain radius $\Rc$ of each other: $\danger_{ij}(\x_j) = \{\x_i: \| \pos_i - \pos_j\|_2 \le \Rc \}$. Such a danger zone is also used by the FAA \cite{paglione99}. An illustration of the problem setup is shown in Figure \ref{fig:formulation}.

\begin{figure}
	\centering
	\includegraphics[width=0.35\textwidth]{"fig/formulation"}
	\caption{An illustration of the problem formulation with three vehicles. Each vehicle $P_i$ seeks to reach its target set $\target_i$ by time $t=\tf_i$, while avoiding physical obstacles $\obs$ and the danger zones of other vehicles.}
	\label{fig:formulation}
\end{figure}

In general, the above problem must be analyzed in the joint state space of all vehicles, making the solution intractable. In this paper, we will instead consider the problem of performing path planning of the vehicles in a sequential manner. Without loss of generality, we consider the problem of first fixing $i=1$ and determining the optimal control for vehicle $1$, the vehicle with the highest priority. The resulting optimal control $\ctrl_1$ sends vehicle $1$ to $\target_1$ in minimum time. 

Then, we plan the minimum time trajectory for each of the vehicles $2,\ldots,N$, in decreasing order of priority, given the previously-determined trajectories for higher-priority vehicles $1,\ldots,i-1$. We assume that all vehicles have complete information about the states and trajectories of higher-priority vehicles, and that all vehicles adhere to their planned trajectories. Thus, in planning its trajectory, vehicle $i$ treats higher-priority vehicles as known time-varying obstacles. 

With the above sequential path planning (SPP) protocol and assumptions, our problem now reduces to the following for vehicle $i$. Given $\x_j(\cdot), j=1,\ldots,i-1$, determine $\ctrl_i(\cdot)$ that maximizes $\ti_i$ and such that $x_i(\tau) \in \target_i, \tau\le \tf_i$.

% Background material
% !TEX root = ./STP_IoTjournal.tex
\section{Background \label{sec:background}}
In this section, we present the basic STP algorithm \cite{Chen15c} in which disturbances are ignored and perfect information of vehicles’ positions is assumed. This simplification allows us to clearly present the basic STP algorithm. However, in presence of disturbances, it is no longer possible to commit to exact trajectories (and hence positions), since the disturbance $\dstb_i(\cdot)$ is \textit{a priori} unknown. Thus, disturbances and incomplete information significantly complicate the STP scheme. We next present the robust trajectory tracking algorithm \cite{Bansal2017} that can be used to make basic STP approach robust to disturbances as well as to an imperfect knowledge of other vehicles' positions. All of these algorithms use time-varying reachability analysis to provide goal satisfaction and safety guarantees; therefore, we start with an overview of time-varying reachability.

% !TEX root = ../STP_IoTjournal.tex
\subsection{Time-Varying Reachability Background \label{sec:HJIVI}}
We will be using reachability analysis to compute a backward reachable set (BRS) $\brs$ given some target set $\targetset$, time-varying obstacle $\obsset(t)$, and the Hamiltonian function $\ham$ which captures the system dynamics as well as the roles of the control and disturbance. The BRS $\brs$ in a time interval $[t, t_f]$ will be denoted by

\begin{equation}
\brs(t, t_f) \quad \text{ (backward reachable set)}
\end{equation}

Several formulations of reachability are able to account for time-varying obstacles \cite{Bokanowski11, Fisac15} (or state constraints in general). For our application in STP, we utilize the time-varying formulation in \cite{Fisac15}, which accounts for the time-varying nature of systems without requiring augmentation of the state space with the time variable. In the formulation in \cite{Fisac15}, a BRS is computed by solving the following \textit{final value} double-obstacle HJ VI:

\begin{equation}
\label{eq:HJIVI_BRS}
\begin{aligned}
\max \Big\{ \min \{&D_t \valfunc(t, \state) + \ham(t, \state, \nabla \valfunc(t, \state)), \fc(\state) - \valfunc(t, \state) \}, \\
&-\obsfunc(t, \state) - \valfunc(t, \state) \Big\} = 0, \quad t \le t_f \\
&\valfunc(t_f, \state) = \max\{\fc(\state), -\obsfunc(t_f, \state)\}
\end{aligned}
\end{equation}

%In a similar fashion, the FRS is computed by solving the following \textit{initial value} HJ PDE:
%
%\begin{equation}
%\label{eq:HJIVI_FRS}
%\begin{aligned}
%D_t \valfuncfwd(t, \state) + &\ham(t, \state, \nabla \valfuncfwd(t, \state)) = 0 , \quad t \ge t_0  \\
%&\valfuncfwd(t_0, \state) = \max\{\fc(\state), -\obsfunc(t_0, \state)\}
%\end{aligned}
%\end{equation}
%
In \eqref{eq:HJIVI_BRS}, the function $\ic(\state)$ is the implicit surface function representing the target set $\targetset = \{\state: \ic(\state) \le 0\}$. Similarly, the function $\obsfunc(t, \state)$ is the implicit surface function representing the time-varying obstacles $\obsset(t) = \{\state: \obsfunc(t,\state)\le 0\}$. The BRS $\brs(t, t_f)$ is given by

%\begin{equation}
%\label{eq:implicitValfuncs}
%\begin{aligned}
%\brs(t, t_f) &= \{\state: \valfunc(t, \state) \le 0\} \\
%\frs(t_0, t) &= \{\state: \valfuncfwd(t, \state) \le 0 \}
%\end{aligned}
%\end{equation}
\begin{equation}
\label{eq:implicitValfuncs}
\brs(t, t_f) = \{\state: \valfunc(t, \state) \le 0\}
\end{equation}

Some of the reachability computations will not involve an obstacle set $\obsset(t)$, in which case we can simply set $\obsfunc(t, \state) \equiv \infty$ which effectively means that the outside maximum is ignored in \eqref{eq:HJIVI_BRS}.

The Hamiltonian, $\ham(t, \state, \nabla \valfunc(t,\state))$, depends on the system dynamics, and the role of control and disturbance. Whenever $\ham$ does not depend explicit on $t$, we will drop the argument. In addition, the Hamiltonian is an optimization that produces the optimal control $\ctrl^*(t, \state)$ and optimal disturbance $\dstb^*(t, \state)$, once $\valfunc$ is determined. For BRSs, whenever the existence of a control (``$\exists \ctrl$'') or disturbance is sought, the optimization is a minimum over the set of controls or disturbance. Whenever a BRS characterizes the behavior of the system for all controls (``$\forall \ctrl$'') or disturbances, the optimization is a maximum. We will introduce precise definitions of reachable sets, expressions for the Hamiltonian, expressions for the optimal controls as needed for the many different reachability calculations we use.
%% !TEX root = STP_journal.tex
\section{STP Without Disturbances and With Perfect Information\label{sec:basic}}
In this section, we introduce the basic STP algorithm assuming that there is no disturbance affecting the vehicles, and that each vehicle knows the exact position of higher-priority vehicles. \SBnote{Although in practice, such assumptions do not hold, the description of the basic STP algorithm will introduce the notation needed for describing the subsequent, more realistic versions of STP.} We also show simulation results for the basic STP algorithm. The majority of the content in this section is taken from \cite{Chen15c}.

\subsection{Theory}
Recall that the STP vehicles $\veh_i, i=1,\ldots,N$, are each assigned a strict priority, with $\veh_j$ having a higher priority than $\veh_i$ if $j<i$. In the absence of disturbances, we can write the dynamics of the STP vehicles as

\begin{equation}
\label{eq:dyn_no_dstb}
\begin{aligned}
\dot\state_i &= \fdyn_i(\state_i, \ctrl_i), t \le \sta_i \\
\ctrl_i &\in \cset_i, \qquad i = 1 \ldots, \N
\end{aligned}
\end{equation}

%\noindent with trajectories denoted by $\traj_i(s; \state^0_i, \ldt, \ctrl_(\cdot))$.

In STP, each vehicle $\veh_i$ plans the trajectory to its target set $\targetset_i$ while avoiding static obstacles $\soset_i$ and the obstacles $\ioset_i^j(t)$ induced by higher-priority vehicles $\veh_j, j<i$. Path planning is done sequentially starting from the first vehicle and proceeding in descending priority, $\veh_1, \veh_2, \ldots, \veh_{\N}$ so that each of the trajectory planning problems can be done in the state space of only one vehicle. During its trajectory planning process, $\veh_i$ ignores the presence of lower-priority vehicles $\veh_k, k>i$, and induces the obstacles $\ioset_k^i(t)$ for $\veh_k, k>i$.

From the perspective of $\veh_i$, each of the higher-priority vehicles $\veh_j, j<i$ induces a time-varying obstacle denoted $\ioset_i^j(t)$ that $\veh_i$ needs to avoid\footnote{Note that the index $k$ in $\ioset_k^i$ denotes vehicles with lower priority than $\veh_i$, and the index $j$ in $\ioset_i^j(t)$ denotes vehicles with higher priority than $\veh_i$.}. Therefore, each vehicle $\veh_i$ must plan its trajectory to $\targetset_i$ while avoiding the union of all the induced obstacles as well as the static obstacles. Let $\obsset_i(t)$ be the union of all the obstacles that $\veh_i$ must avoid on its way to $\targetset_i$:

\begin{equation}
\label{eq:obsseti}
\obsset_i(t)  = \soset_i \cup \bigcup_{j=1}^{i-1} \ioset_i^j(t)
\end{equation}

With full position information of higher priority vehicles, the obstacle induced for $\veh_i$ by $\veh_j$ is simply

\begin{equation}
\label{eq:ioset}
\ioset_i^j(t) = \{\state_i: \|\pos_i - \pos_j(t)\|_2 \le \rc \}
\end{equation}

Each higher priority vehicle $\veh_j$ plans its trajectory while ignoring $\veh_i$. Since trajectory planning is done sequentially in descending order or priority, the vehicles $\veh_j, j<i$ would have planned their trajectories before $\veh_i$ does. Thus, in the absence of disturbances, $\pos_j(t)$ is \textit{a priori} known, and therefore $\ioset_i^j(t), j<i$ are known, deterministic moving obstacles, which means that $\obsset_i(t)$ is also known and deterministic. Therefore, the trajectory planning problem for $\veh_i$ can be solved by first computing the BRS $\brs_i^\text{basic}(t, \sta_i)$, defined as follows:
%
\begin{equation}
\label{eq:BRS_basic}
\begin{aligned}
\brs_i^\text{basic}(t, \sta_i) = & \{y: \exists \ctrl_i(\cdot) \in \cfset_i, \state_i(\cdot) \text{ satisfies \eqref{eq:dyn_no_dstb}}, \\
& \forall s \in [t, \sta_i],\state_i(s) \notin \obsset_i(s), \\
& \exists s \in [t, \sta_i], \state_i(s) \in \targetset_i, \state_i(t) = y\}
\end{aligned}
\end{equation}
%
The BRS $\brs(t, \sta_i)$ can be obtained by solving \eqref{eq:HJIVI_BRS} with $\targetset = \targetset_i$, $\obsset(t) = \obsset_i(t)$, and the Hamiltonian 
%
\begin{equation}
\label{eq:basicham}
\ham_i^\text{basic}(\state_i, \costate) = \min_{\ctrl_i\in\cset_i} \costate \cdot \fdyn_i(\state_i, \ctrl_i)
\end{equation}
%
\SBnote{Note that $\brs(t, \sta_i)$, by definition, does not contain any states from which it is inevitable to avoid the danger zone $\dz_{ij}$ (and $\obsset_i$ in general).} Given $\brs(t, \sta_i)$, the optimal control for reaching $\targetset_i$ while avoiding $\obsset_i(t)$ is then given by
%
\begin{equation}
\label{eq:basicOptCtrl}
\ctrl_i^\text{basic}(t, \state_i) = \arg \min_{\ctrl_i\in\cset_i} \costate \cdot \fdyn_i(\state_i, \ctrl_i)
\end{equation}
%
\noindent from which the trajectory $\state_i(\cdot)$ can be computed by integrating the system dynamics, which in this case are given by \eqref{eq:dyn_no_dstb}. In addition, the latest departure time $\ldt_i$ can be obtained from the BRS $\brs(t, \sta_i)$ as $\ldt_i = \arg \sup_t \{\state_i^0 \in \brs(t, \sta_i)\}$. In summary, the basic STP algorithm is given as follows:

\begin{alg}
\label{alg:basic}
\textbf{Basic STP algorithm}: Suppose we are given initial conditions $\state_i^0$, vehicle dynamics \eqref{eq:dyn_no_dstb}, target sets $\targetset_i$, and static obstacles $\soset_i, i = 1\ldots, \N$. For each $i$ in ascending order starting from $i=1$ (which corresponds to descending order of priority),
\begin{enumerate}
\item determine the total obstacle set $\obsset_i(t)$, given in \eqref{eq:obsseti}. In the case $i=1$, $\obsset_i(t) = \soset_i ~ \forall t$;
\item compute the BRS $\brs_i^\text{basic}(t, \sta_i)$ defined in \eqref{eq:BRS_basic}. The latest departure time $\ldt_i$ is then given by $\arg \sup_t \{\state^0_i \in \brs_i^\text{basic}(t, \sta_i)\}$;
\item determine the trajectory $\state_i(\cdot)$ using vehicle dynamics \eqref{eq:dyn_no_dstb}, with the optimal control  $\ctrl_i^\text{basic}(\cdot)$ given by \eqref{eq:basicOptCtrl};
\item given $\state_i(\cdot)$, compute the induced obstacles $\ioset_k^i(t)$ for each $k>i$. In the absence of disturbances, $\ioset_k^i(t)$ is given by \eqref{eq:ioset}.
\end{enumerate}
\end{alg}

\MCnote{Note that Step 1, which determines the total obstacle set, can be updated in a recursive manner by adding a new set of induced obstacles for each next vehicle: $\obsset_{i+1}(t) = \obsset_i(t) \cup \ioset_{i+1}^i(t)$. In addition, in implementation, Step 4 can be simplified by storing $\obsset_i(t)$ as a look-up table with the maximum dimensionality across all vehicle state spaces. When a vehicle plans its trajectory, irrelevant dimensions of $\obsset_i(t)$ can be ignored. This observation keeps the computational complexity of our algorithm linear with respect to the number of vehicles.}

\MCnote{As previously mentioned, the basic STP algorithm, as well as all subsequent variants of STP algorithms, will \textit{always} return a feasible trajectory that arrives at the target on time, as long as a feasible trajectory exists in the \textit{absence} of other vehicles. This is because a vehicle can simply depart early enough to avoid being blocked by higher-priority vehicles. In fact, the latest departure time $\ldt_i$ quantifies exactly when each vehicle needs to depart to arrive on time.}
% !TEX root = SPP2.tex
\subsection{Method 3: Robust Trajectory Tracking\label{sec:rtt}}
Although it is impossible to commit to and track an exact trajectory in the presence of disturbances, it may still be possible to \textit{robustly} track a \textit{nominal} trajectory with a bounded error at all times. If this can be done, then the tracking error bound can be used to determine the induced obstacles. Here, computation is done in two phases: the \textit{planning phase} and the \textit{disturbance rejection phase}. In the planning phase, we compute a nominal trajectory $\state_{r,j}(\cdot)$ that is feasible in the absence of disturbances. In the disturbance rejection phase, we compute a bound on the tracking error.%\MCnote{don't need to explain where error comes bound, imo}%, caused by a vehicle's inability to exactly track the nominal trajectory in the presence of disturbances. 

In the planning phase, planning is done for a reduced control set $\cset^p\subset\cset$, as some margin is needed to reject unexpected disturbances while tracking the nominal trajectory. In the disturbance rejection phase, we determine the error bound independently of the nominal trajectory. Let $\state_j$ and $\state_{r,j}$ denote the states of the actual vehicle $\veh_j$ and the virtual evader, respectively, and define the tracking error $e_j=\state_j-\state_{r,j}$. When the error dynamics are independent of the absolute state as in \eqref{eq:edyn} (and also (7) in \cite{Mitchell05}), we can obtain error dynamics of the form
\begin{equation}
\label{eq:edyn} % Error dynamics
\begin{aligned}
\dot{e_j} &= \fdyn_{e_j}(e_j, \ctrl_j, \ctrl_{r,j},\dstb_j), \\
\ctrl_j &\in \cset_j, \ctrl_{r,j} \in \cset^p_j, \dstb_j \in \dset_j, \quad t \leq 0
\end{aligned}
\end{equation}

To obtain bounds on the tracking error, we first conservatively estimate the error bound around any reference state $\state_{r,j}$, denoted $\errorbound_j = \{e_j: \|\pos_{e_j}\|_2 \le R_{\text{EB}}\}$,
%\begin{equation} \label{eqn:err}
%\errorbound_j = \{e_j: \|\pos_{e_j}\|_2 \le R_{\text{EB}} \}, 
%\end{equation}
\noindent where $\pos_{e_j}$ denotes the position coordinates of $e_j$ and $R_{\text{EB}}$ is a design parameter. We next solve a reachability problem with its complement $\errorbound_j^c$, the set of tracking errors violating the error bound, as the target in the space of the error dynamics. From $\errorbound_j^c$, we compute the following BRS:
\begin{equation} \label{eqn:errBound}
\begin{aligned}
&\brs^{\text{EB}}_{j}(t, 0) = \{y: \forall \ctrl_j(\cdot) \in \cfset_j, \exists \ctrl_{r, j}(\cdot) \in \cfset^\pos_j, \exists \dstb_j(\cdot) \in \dfset_i, \\
& e_j(\cdot) \text{ satisfies \eqref{eq:edyn}}, e_j(t) = y, \exists s \in [t, 0], e_j(s) \in \errorbound_j^c\}, 
\end{aligned}
\end{equation}
where the Hamiltonian to compute the BRS is given by:
\begin{equation}
\begin{aligned}
H^{\text{EB}}_{j}(e_j, \costate) &= \max_{\ctrl_j \in \cset_j} \min_{\ctrl_r \in \cset^\pos_j, \dstb_j \in \dset_j} \costate \cdot \fdyn_{e_j}(e_j, \ctrl_j, \ctrl_{r,j}, \dstb_j).
\end{aligned}
\end{equation}

Letting $t \to -\infty$, we obtain the infinite-horizon control-invariant set $\disckernel_j := \lim_{t \to -\infty} \left( \brs^{\text{EB}}_{j}(t, 0) \right)^c$. If $\disckernel_j$ is nonempty, then the tracking error $e_j$ at flight time is guaranteed to remain within $\disckernel_j \subseteq \errorbound_j$ provided that the vehicle starts inside $\disckernel_j$ and subsequently applies the feedback control law
\begin{equation}
\label{eq:robust_tracking_law}
\tracklaw_j(e_j) = \arg\max_{\ctrl_j \in \cset_j} \min_{\ctrl_r \in\cset^\pos_j, \dstb_j \in \dset_j} \costate \cdot \fdyn_{e_j}(e_j,\ctrl_j,\ctrl_{r, j},\dstb_j).
\end{equation}

The induced obstacles by each higher-priority vehicle $\veh_j$ can thus be obtained by: 
\begin{equation} 
\label{eqn:rttObs}
\begin{aligned}
\ioset_i^j(t) &=  \{\state_i: \exists y \in \pfrs_j(t), \|\pos_i - y\|_2 \le \rc \} \\
\pfrs_j(t) &= \{\pos_j: \exists \npos_j, (\pos_j, \npos_j) \in \disckernel_j  + \state_{r,j}(t)\},
\end{aligned}
\end{equation}
\noindent where the ``$+$'' in \eqref{eqn:rttObs} denotes the Minkowski sum\footnote{The Minkowski sum of sets $A$ and $B$ is the set of all points that are the sum of any point in $A$ and $B$.}. Finally, we can obtain the total obstacle set $\obsset_i(t)$ using \eqref{eq:ioset}. %Intuitively, if $\veh_j$ is tracking $\state_{r,j}(t)$, then it will remain within the error bound $\disckernel_j$ around $\state_{r,j}(t) ~\forall t$. This is precisely the set $\pfrs_j(t)$. The induced obstacles can then be obtained by augmenting a danger zone around this set. Finally, we can obtain the total obstacle set $\obsset_i(t)$ using \eqref{eq:obsseti}.

Since each vehicle $\veh_j$, $j<i$, can only be guaranteed to stay within $\disckernel_j$, we must make sure during the path planning of $\veh_i$ that at any given time, the error bounds of $\veh_i$ and $\veh_j$, $\disckernel_i$ and $\disckernel_j$, do not intersect. This can be done by augmenting the total obstacle set by $\disckernel_i$:%This can be done by choosing the induced obstacle to be the Minkowski sum\footnote{The Minkowski sum of sets $A$ and $B$ is the set of all points that are the sum of any point in $A$ and $B$.} of the error bounds. Thus,

\begin{equation} 
\label{eqn:rttAugObs}
\tilde{\obsset}_i(t) = \obsset_i(t) + \disckernel_i.
\end{equation}

Finally, given $\disckernel_i$, we can guarantee that $\veh_i$ will reach its target $\targetset_i$ if $\disckernel_i \subseteq \targetset_i$; thus, in the path planning phase, we modify $\targetset_i$ to be $\tilde{\targetset}_i := \{\state_i: \disckernel_i + \state_i \subseteq \targetset_i\}$, and compute a BRS, with the control authority $\cset^\pos_i$, that contains the initial state of the vehicle. Mathematically,

\begin{equation}
\label{eq:rttBRS}
\begin{aligned}
\brs_i^\text{rtt}(t, \sta_i) = & \{y: \exists \ctrl_i(\cdot) \in \cfset^p_i, \state_i(\cdot) \text{ satisfies \eqref{eq:dyn_no_dstb}},\\
&\forall s \in [t, \sta_i], \state_i(s) \notin \tilde{\obsset}_i(t), \\
& \exists s \in [t, \sta_i], \state_i(s) \in \tilde{\targetset}_i, \state_i(t) = y\}
\end{aligned}
\end{equation}

The Hamiltonian to compute $\brs_i^\text{rtt}(t, \sta_i)$ and the optimal control for reaching $\tilde{\targetset}_i$ are given by \eqref{eq:basicSPPHam} and \eqref{eq:optCtrl} respectively.
%$\brs_i^\text{rtt}(t, \sta_i)$ can be obtained by solving \eqref{eq:HJIVI_BRS} using the Hamiltonian: 
%\begin{equation}
%\label{eq:RTTham}
%\ham_i^\text{rtt}(\state_i, \costate) = \min_{\ctrl_i \in \cset^\pos_i } \costate \cdot \fdyn_i(\state_i, \ctrl_i)
%\end{equation}
%
%The corresponding optimal control for reaching $\tilde{\targetset}_i$ is given by:
%\begin{equation}
%\label{eq:RTTOptCtrl}
%\ctrl_i^\text{rtt}(t) = \arg \min_{\ctrl_i \in \cset^\pos_i } \costate \cdot \fdyn_i(\state_i, \ctrl_i).
%\end{equation}
The nominal trajectory $\state_{r,i}(\cdot)$ can thus be obtained by using vehicle dynamics \eqref{eq:dyn_no_dstb}, with the optimal control  $\ctrl_i^\text{rtt}(\cdot)$. From the resulting nominal trajectory $\state_{r,i}(\cdot)$, the overall control policy to reach $\targetset_i$ can be obtained via \eqref{eq:robust_tracking_law}.

% Intruder files
% !TEX root = ../SPP_IoTjournal.tex
\section{Response to Intruders \label{sec:intruder}}
In Section \ref{sec:rtt}, we presented RTT algorithm that can take into account disturbances in vehicles' dynamics. However, if a vehicle not in the set of SPP vehicles enters the system, or even worse, if this vehicle is an adversarial intruder, the original plan can lead to vehicles entering into each other's danger zones. If vehicles do not plan with an additional safety margin that takes a potential intruder into account, a vehicle trying to avoid the intruder may effectively become an intruder itself, leading to a domino effect. %In this section, we propose a method to allow vehicles to avoid an intruder while maintaining the SPP structure.

In general, the effect of an intruder on the vehicles in structured flight can be entirely unpredictable, since the intruder in principle could be adversarial in nature, and the number of intruders could be arbitrary. Therefore, we make the following two assumptions: %for our analysis to produce reasonable results, two assumptions about the intruders must be made.

\begin{assumption}
\label{as:avoidOnce}
At most one intruder (denoted as $\veh_I$ here on) affects the SPP vehicles at any given time. The intruder exits the altitude level affecting the SPP vehicles after a duration of $\iat$. 
\end{assumption}

Let the time at which intruder appears in the system be $\tsa$ and the time at which it disappears be $\tea$. Assumption \ref{as:avoidOnce} implies that $\tea \leq \tsa + \iat$. Thus, any vehicle $\veh_i$ would need to avoid the intruder $\veh_{\intr}$ for a maximum duration of $\iat$. %This assumption can be valid in situations where intruders are rare, and that some fail-safe or enforcement mechanism exists to force the intruder out of the altitude level affecting the SPP vehicles. 
Note that we do not pose any restriction on $\tsa$; however, we assume that once the intruder appears, it stays for a maximum duration of $\iat$.
%in addition, after avoiding the intruder, Qi can safely assume that it would not need to avoid another intruder

\begin{assumption}
\label{as:dynKnown}
The dynamics of the intruder are known and given by $\dot\state_\intr = f_\intr(\state_\intr, \ctrl_\intr, \dstb_\intr)$.
\end{assumption}

Assumption \ref{as:dynKnown} is required for HJ reachability analysis. In situations where the dynamics of the intruder are not known exactly, a conservative model of the intruder may be used instead. We also denote the initial state of the intruder as $\state_{\intr}^0.$ Note that $\state_{\intr}^0$ is unknown.

Our goal is to design a control policy that ensures separation with the intruder and other SPP vehicles, and ensures a successful transit to the destination. However, depending on the initial state of the intruder and its control policy, a vehicle may arrive at different states after avoiding the intruder. Therefore, a control policy that ensures a successful transit to the destination needs to account for all such possible states, which is a path planning problem with multiple initial states and a single destination, and is hard to solve in general. Thus, we divide the intruder avoidance problem into two sub-problems: (i) we first design a control policy that ensures a successful transit to the destination if no intruder appears and that successfully avoid the intruder, if it does. (ii) after the intruder disappears at $\tea$, we replan the trajectories of the affected vehicles. Following the same theme and assumptions, authors in \cite{chen2016robust} present an algorithm to avoid an intruder in SPP formulation; however, once the intruder disappears, the algorithm might need to replan the trajectories for all SPP vehicles. Since the replanning is done in real-time, it should be fast and scalable with the number of SPP vehicles, rendering the method in \cite{chen2016robust} unsuitable for practical implementation.  

In this work, we propose a novel intruder avoidance algorithm, which will need to replan trajectories only for a fixed number of vehicles, irrespective of the total number of SPP vehicles. Moreover, this number is a design parameter, which can be chosen based on the resources available during the run time. Intuitively, one can think about dividing the flight space of vehicles such that at any given time, any two vehicles are far enough from each other so that an intruder can only affect atmost $\nva$ vehicles in a duration of $\iat$ despite its best efforts. In this method, we build upon this intuition and show that such a division of space is indeed possible. The advantage of such an approach is that after the intruder disappears, we only have to replan the trajectory of \textit{atmost $\nva$} vehicles, which can be efficiently done in real-time if $\nva$ is low enough, thus making this approach particularly suitable for practical systems. %Thus the proposed method guarantees that \textit{atmost $k$} vehicles are affected by the presence of intruder, regardless of the number of SPP vehicles, and hence the replanning can be efficiently done in real-time. 

In Sections \ref{sec:intruder_avoid}, we discuss the intruder avoidance control and explain the sensing range required to avoid the intruder. In Sections \ref{sec:case1} and \ref{sec:case2}, we compute a space division of state-space such that atmost $\nva$ vehicles need to apply the avoidance maneuver computed in Section \ref{sec:intruder_avoid}, regardless of the initial state of the intruder. %However, we still need to ensure that vehicles do not collide with each other while avoiding the intruder. The induced obstacles that reflect this possibility are computed in Section \ref{sec:intruderObs}. 
Trajectopry planning and replanning are discussed in Sections \ref{sec:path_planning} and \ref{sec:replan} respectively.
% !TEX root = ../../SPP_IoTjournal.tex
\subsubsection{Separation region} \label{sec:sepRegion_case1}
Recall that the separation region denotes the set of states of the intruder for which a vehicle is forced to apply an avoidance maneuver. In this section our goal is to find $\sep_j(\tsa_j)$, the separation region of $\veh_j$ at the avoid start time. By virtue of Observation \ref{obs1_case1}, we will use $\tsa_j$ and $\tsa$ interchangeably here on.
%
%Consider any $\tsa \in \R$. In this section, our goal is to find the set of all states $\state_{\intr}^0 := \state_{\intr}(\tsa)$ for which $\veh_j$ is forced to apply an avoidance maneuver. We refer to this set as \textit{separation region}, and denote it as $\sep_j(\tsa)$. 

As discussed in Section \ref{sec:intruder_avoid}, $\veh_j$ needs to apply avoidance maneuver at time $\tsa$ only if $\state_{\intr j}(\tsa) \in \partial \brs^{\text{A}}_{j}(0, \iat)$. To compute set $\sep_j(\tsa)$, we thus need to translate these relative states to a set in the state space of the intruder. Therefore, if all possible states of $\veh_j$ at time $\tsa$ are known, then $\sep_j(\tsa)$ can be trivially computed.

Recall from Section \ref{sec:distb} that the base obstacle $\boset_j(t)$ at time $t$ represents all possible states of $\veh_j$ at time $t$, if the intruder doesn't appear in the system until that time. This is precisely the set that we are interested in to compute the separation region.
%
%at time $t$ represents all states that $\veh_j$ can be in at time $t$ in the presence of disturbances, but in the absence of an intruder. In particular, if the intruder doesn't appear in the system until time $t$, then  captures all possible states of $\veh_j$ at time $t$, precisely the set that we are interested in to compute the separation region.
Depending on the information known to a lower-priority vehicle $\veh_i$ about $\veh_j$'s control strategy, we can use one of the three methods described in Section 5 in \cite{chen2016robust} (and Section \ref{sec:distb} of this paper) to compute the base obstacles $\boset_j(\tsa)$. In particular, the base obstacles are respectively given by equations (25), (31) and (37) in \cite{chen2016robust} for the centralized control, the least restrictive control and the robust trajectory tracking algorithms (the three proposed algorithms to account for disturbances in STP). We will explain the computation of the base obstacles further in Section \ref{sec:path_planning}.

Given $\boset_j(\tsa)$, $\sep_j(\tsa)$ can be obtained as:
\begin{equation} \label{eqn:sepRegion_case1}
\sep_j(\tsa) = \boset_j(\tsa) + \partial \brs^{\text{A}}_{j}(0, \iat), ~\tsa \in \R,
\end{equation}
where the ``$+$'' in \eqref{eqn:sepRegion_case1} denotes the Minkowski sum. Since $\sep_j(\tsa)$ represents the set of all states of $\veh_\intr$ for which $\veh_j$ must apply an avoidance maneuver, Observation \ref{obs1_case1} implies that it is sufficient to consider the scenarios where $\state_{\intr}^0 := \state_{\intr}(\tsa) \in \sep_j(\tsa)$.
% !TEX root = ../SPP_IoTjournal.tex
\subsection{Buffer Region} \label{sec:buffRegion}
In section \ref{sec:sepRegion}, we computed sets $\sep_j(\cdot)$ such that $\veh_j$ avoids the intruder only if $\state_{\intr}(t) \in \sep_j(t)$. But to ensure that atmost $\nva$ vehicles need to replan their trajectories after the intruder disappears, we need to make sure that the intruder can cause atmost $\nva$ vehicles to deviate from their planned trajectories. Equivalently, we want to ensure that atmost $\nva$ vehicles need to avoid the intruder. 

Intuitively, we want to make sure that at any given time the separation regions of any two vehicles are far enough from each other (that is, there is a ``buffer" region between two separation regions) such that it will take at least $\brd := \iat / \nva$ time for the intruder to go from the separation region of one vehicle to that of the other vehicle. This means that there is a ``buffer" time interval of $\brd$ between any $t_1$, $t_2 \in [\tsa, \tea]$ where $\veh_{\intr}$ is in the separation regions of two different vehicles at $t_1$ and $t_2$, e.g. $\state_{\intr}(t_1) \in \sep_j(t)$ and $\state_{\intr}(t_2) \in \sep_i(t)$, $i \neq j$. Thus, in the worst case, the intruder can force atmost $\nva$ vehicles to apply avoidance maneuver in a duration of $\iat$. 

We next focus on computing the buffer region between any two vehicles $\veh_j$ and $\veh_i$, $j < i$. Without loss of generality, we can assume that the intruder appears at the separation region of a vehicle at $t = \tsa$, because if it doesn't then the vehicles need not account for intruders until it reaches the boundary of the separation region of a vehicle. To compute the buffer region, we consider the following two cases:     

\subsubsection{Case1- $\state_{\intr}^0 \in \sep_j(\tsa)$} \label{sec:buffCase1}
Given the relative dynamics $\state_{i, \intr}$ in \eqref{eq:reldyn}, we compute the set of states from which the joint states of $\veh_{\intr}$ and $\veh_{i}$ can enter danger zone $\dz_{i\intr}$ within a duration of $\brd$ when both $\veh_{i}$ and $\veh_{\intr}$ are using \textit{optimal control to collide} with each other. This set of states is given by the backwards reachable set $\brs^{\text{B}}_i(\tau, \brd)$: \MCnote{Might be clearer to put ``exists'' before every expression? Mostly to make it easier to differentiate from the case where there is a ``for all''... not sure.}

\begin{equation} \label{eqn:buffBRS}
\begin{aligned}
\brs^{\text{B}}_{i}(t, \brd) = & \{y: \exists \ctrl_i(\cdot) \in \cfset_i, \ctrl_\intr(\cdot) \in \cfset_\intr, \dstb_i(\cdot) \in \dfset_i, \\
& \dstb_\intr(\cdot) \in \dfset_\intr, \state_{i, \intr}(\cdot) \text{ satisfies \eqref{eq:reldyn}},\\
& \exists s \in [t, \brd], \state_{i, \intr}(s) \in \targetset^{\text{B}}_{i}, \state_{i, \intr}(t) = y\},
\end{aligned}
\end{equation}
where 
\begin{equation}
\begin{aligned}
\targetset^{\text{B}}_{i} &= \{\state_{i, \intr}: \|\pos_{i, \intr}\|_2 \le \rc\} \\
H^{\text{B}}_{i}(\state_{i, \intr}, \costate) &= \min_{\ctrl_i \in \cset_i, \ctrl_\intr \in \cset_\intr, \dstb_i \in \dset_i, \dstb_\intr \in \dset_\intr} \costate \cdot f_r(\state_{i, \intr}, \ctrl_i, \ctrl_\intr, \dstb_i, \dstb_\intr)
\end{aligned}
\end{equation}

The interpretation of set $\brs^{\text{B}}_{i}(0, \brd)$ is that if the separation region of $\veh_i$ is outside the boundary of this set and $\veh_{\intr}$ is at the boundary of $\targetset^{\text{B}}_{i}$ (in relative coordinates), then $\veh_{\intr}$ and $\veh_{i}$ cannot enter the danger zone $\dz_{i\intr}$ for a duration of $\brd$, irrespective of control applied by them. If we augment this set on the separation region of the $\veh_j$, then we get the same property in the state space of $\veh_i$:
\begin{equation} \label{eqn:buffRegion}
\buff_{ij}(t) = \sep_j(t) + \brs^{\text{B}}_{i}(0, \brd).
\end{equation} 

Finally, during the path planning of $\veh_i$, we need to ensure that $\veh_i$ is far enough from the boundary of $\buff_{ij}(t)$ such that $\veh_{\intr}$ and $\veh_{i}$ cannot enter the danger zone $\dz_{i\intr}$ for the remaining duration of $\trd := \iat - \brd$. Thus, during the path planning of $\veh_i$, we need to ensure that $\veh_i$ is outside the augmented buffer region:
\begin{equation} \label{eqn:augbuffRegion1}
\tilde{\buff}_{ij}(t) = \buff_{ij}(t) + \brs^{\text{S}}_{i}(0, \trd),
\end{equation}
where $\brs^{\text{S}}_{i}(0, \trd)$ can be computed as described in Section \ref{sec:sepRegion}.

\subsubsection{Case2- $\state_{\intr}^0 \in \sep_i(\tsa)$} \label{sec:buffCase2}
This case can be treated in a similar fashion as Section \ref{sec:buffCase1}. We can now look at the same problem from $\veh_i$'s perspective and compute the augmented buffer region $\tilde{\buff}_{ji}(t)$ as:
\begin{equation} \label{eqn:augbuffRegion2}
\tilde{\buff}_{ij}(t) = \boset_j(t) + \brs^{\text{S}}_{j}(0, \trd) + \brs^{\text{B}}_{j}(0, \brd) + \brs^{\text{S}}_{i}(0, \iat).
\end{equation}
During the path planning of $\veh_i$, we need to ensure that $\veh_i$ is outside $\tilde{\buff}_{ji}(t)$. 
% !TEX root = ../../SPP_IoTjournal.tex
\subsubsection{Obstacle Computation} \label{sec:intruderObs_case2}
In this section, we want to find the set of states that $\veh_i$ needs to avoid in order to avoid entering in the danger zone of $\veh_j$. We consider the following two mutually exclusive and exhaustive cases:
\begin{enumerate}
\item Case A: The intruder affects $\veh_i$, but not $\veh_j$, i.e., $\tsa_i < \infty$ and $\tsa_j = \infty$.
\item Case B: The intruder first affects $\veh_i$ and then $\veh_j$, i.e., $\tsa_i < \tsa_j < \infty$.
\end{enumerate}
For each case, we compute the set of states that $\veh_i$ needs to avoid at time $t$ to avoid entering in $\dz_{ij}$. We also let $_2^A\ioset_i^j(\cdot)$ and $_2^B\ioset_i^j(\cdot)$ denote the set of obstacles corresponding to Case A and Case B respectively. 
\begin{itemize}[leftmargin=*] 
\item \label{sec:intruderObs_case2_caseA} Case A: In this case, we need to ensure that $\veh_i$ does not collide with $\veh_j$ while it is avoiding the intruder. Since $\veh_j$ is not avoiding the intruder in this case, the set of possible states of $\veh_j$ at time $t$ is given by $\boset_j(t)$. To compute $_2^A\ioset_i^j(\cdot)$, we begin with the following observation: 
\begin{observation} \label{obs1_case2_caseA}
By Observation \ref{obs1_case1Obs}, it is sufficient to consider the scenarios where $\tsa = \tsa_i \in [t-\iat, t]$. Since $\veh_i$ can be forced to apply an avoidance maneuver for the time interval $[\tsa_i, \tsa_i+\iat]$, to compute obstacles at time $t$ for a given $\tsa_i$, we need to make sure that $\veh_i$ avoid all states at time $t$ that can lead to a collision with $\veh_j$ during the interval $[t, \tsa_i+\iat]$ for some avoidance control. Therefore, it is sufficient to consider the scenario $\tsa_i = t$ as that will maximize the avoidance duration $[\tsa_i, \tsa_i+\iat]$ for the obstacle computation at time $t$.  
\end{observation}

Mathematically, $\veh_i$ needs to avoid all states at time $t$ that can reach $\mathcal{K}^{\text{A2}}(\tau)$ for some control action of $\veh_i$ during time duration $[t, \tau]$. $\mathcal{K}^{\text{A2}}(\tau)$ here is given by:
\begin{equation}
\begin{aligned}
\mathcal{K}^{\text{A2}}(\tau) = & \tilde{\boset}_j(\tau),\\
\tilde{\boset}_j(s) = & \{\state_j: \exists (y, h) \in \boset_j(s), \|\pos_j - y\|_2 \le \rc \}.
\end{aligned}
\end{equation}
$\tilde{\boset}_j(s)$ represent the set of all states that are in potential collision with $\veh_j$ at time $s$. Note that since the intruder is present in the system for a maximum duration of $\iat$ and since $\tsa_i = t$ (by Observation \ref{obs1_case2_caseA}), we have that $\tau \in [t, t+\iat]$. 

Avoiding $\mathcal{K}^{\text{A2}}(\cdot)$ will ensure that $\veh_i$ and $\veh_j$ will not enter into each other's danger zones regardless of the avoidance maneuver applied by $\veh_i$. The set of states that $\veh_i$ needs to avoid at time $t$ is given by the following BRS: 
\begin{equation} \label{eq:ObsBRS_case2_caseA}
\begin{aligned}
\brs^{\text{A2}}_{i}(t, t+\iat) = & \{y: \exists \ctrl_i(\cdot) \in \cfset_i, \exists \dstb_i(\cdot) \in \dfset_i, \\
& \state_i(\cdot) \text{ satisfies \eqref{eq:dyn}}, \state_i(t) = y, \\
& \exists s \in [t, t+\iat], \state_i(s) \in \mathcal{K}^{\text{A2}}(s) \}.
\end{aligned}
\end{equation}
The Hamiltonian $\ham^{\text{A2}}_{i}$ to compute $\brs^{\text{A2}}_{i}(t, t+\iat)$ is given by:
\begin{equation} \label{eqn:BRS_obsham_case2_caseA}
\ham^{\text{A2}}_{i}(\state_i, \costate) = \min_{\ctrl_i \in \cset_i, \dstb_i \in \dset_i} \costate \cdot f_i (\state_i, \ctrl_i, \dstb_i).
\end{equation}
$\brs^{\text{A2}}_{i}(t, t+\iat)$ represents the set of all states of $\veh_i$ at time $t$ from which it is possible for $\veh_i$ to reach $\mathcal{K}^{\text{A2}}(\tau)$ for some $\tau \geq t$. Thus, the induced obstacle in this case is given as
\begin{equation} \label{eq:intruderObs_case2_caseA}
_2^A\ioset_i^j(t) = \brs^{\text{A2}}_{i}(t, t+\iat).
\end{equation}

\item \label{sec:intruderObs_case2_caseB} Case B: In this case, the intruder first affects $\veh_i$ and then $\veh_j$. $\veh_i$ and $\veh_j$ apply their first avoidance maneuver at $\tsa_i$ and $\tsa_j$ respectively. Since the intruder appears for a maximum duration of $\iat$ and $\tsa_i = \tsa$, from the perspective  of $\veh_i$, $\veh_j$ can apply any control during the duration $[\tsa_j, \tsa_i + \iat]$ and hence can be anywhere in the set $\frs_{j}^{\mathcal{O}}(\tsa_j, \tau)$ at $\tau \in [\tsa_j, \tsa_i + \iat]$, where $\frs_{j}^{\mathcal{O}}$ denotes the FRS of base obstacle $\boset_j(\tsa_j)$ computed forward for a duration of $(\tsa_i + \iat - \tsa_j)$. $\veh_i$ thus needs to make sure that it avoids all states at time $t$ that can reach $\frs_{j}^{\mathcal{O}}(\tsa_j, \tau)$, regardless of the control applied by $\veh_i$ during $[t, \tau]$. We now make the following key observation:
\begin{observation} \label{obs1_case2_caseB}
Observation \ref{obs1_case1_caseA} implies that $\frs_{j}^{\mathcal{O}}(\tau_2, \tau) \subseteq \frs_{j}^{\mathcal{O}}(\tau_1, \tau)$ if $\tau > \tau_2 > \tau_1$. Therefore, the biggest obstacle, $\frs_{j}^{\mathcal{O}}(\tsa_j, \tau)$, is induced by $\veh_j$ at $\tau$ if $\tsa_j$ is as early as possible. Hence, it is sufficient for $\veh_i$ to avoid this obstacle to ensure collision avoidance with $\veh_j$ at time $\tau$. Given the separation and buffer regions between $\veh_i$ and $\veh_j$, we must have $\tsa_j - \tsa_i \geq \brd$. Hence, the biggest obstacle is induced by $\veh_j$ when $\tsa_j = \tsa_i + \brd$. 
\end{observation}
Intuitively, Observation \ref{obs1_case2_caseB} implies that the biggest obstacle is induced by $\veh_j$ when intruder forces $\veh_i$ to apply the avoidance maneuver and \textit{immediately} begins traveling through the buffer region between two vehicles to force $\veh_j$ to apply an avoidance maneuver after a duration of $\brd$. Therefore, $\veh_i$ needs to avoid $\mathcal{K}^{\text{B2}}(\tau)$ at time $\tau > t$, where 
\begin{equation} \label{eqn:obs_case2_caseB_help1}
\mathcal{K}^{\text{B2}}(\tau) =  \bigcup_{\tsa_i \in [t-\iat, t], \tau \leq \tsa_i + \iat} \frs_{j}^{\mathcal{O}}(\tsa_i + \brd, \tau), \tau>t,
\end{equation}
where we have substituted $\tsa_j = \tsa_i + \brd$. In \eqref{eqn:obs_case2_caseB_help1}, $\tsa = \tsa_i \in [t-\iat, t]$ due to Observation \ref{obs1_case1Obs} and $\tau \leq \tsa_i + \iat$ because the intruder can appear for a maximum duration of $\iat$. \eqref{eqn:obs_case2_caseB_help1} can be equivalently written as:  
\begin{equation} \label{eqn:obs_case2_caseB_help2}
\begin{aligned}
\mathcal{K}^{\text{B2}}(\tau) = & \bigcup_{\tsa_i \in [\tau-\iat, t]} \frs_{j}^{\mathcal{O}}(\tsa_i + \brd, \tau), t < \tau \leq t + \iat\\
\mathcal{K}^{\text{B2}}(\tau) = & \frs_{j}^{\mathcal{O}}(\tau-\iat+\brd, \tau), t < \tau \leq t + \iat,
\end{aligned}
\end{equation}
where the second equality holds because of Observation \ref{obs1_case1_caseA}. The set of states that $\veh_i$ needs to avoid at time $t$ is thus given by the following BRS:  
\begin{equation} \label{eq:ObsBRS_case2_caseB}
\begin{aligned}
\brs^{\text{B2}}_{i}(t, t+\iat) = & \{y: \exists \ctrl_i(\cdot) \in \cfset_i, \exists \dstb_i(\cdot) \in \dfset_i, \\
& \state_i(\cdot) \text{ satisfies \eqref{eq:dyn}}, \state_i(t) = y, \\
& \exists s \in [t+\brd, t+\iat], \state_i(s) \in \tilde{\mathcal{K}}^{\text{B2}}(s) \},\\
\tilde{\mathcal{K}}^{\text{B2}}(s) = & \{\state_i: \exists (y, h) \in \mathcal{K}^{\text{B2}}(s), \|\pos_i - y\|_2 \le \rc \}.
\end{aligned}
\end{equation}
The Hamiltonian $\ham^{\text{B2}}_{i}$ to compute $\brs^{\text{B2}}_{i}(t, t+\iat)$ is given by:
\begin{equation} \label{eqn:BRS_obsham_case2_caseB}
\ham^{\text{B2}}_{i}(\state_i, \costate) = \min_{\ctrl_i \in \cset_i, \dstb_i \in \dset_i} \costate \cdot f_i (\state_i, \ctrl_i, \dstb_i).
\end{equation}
Finally, the induced obstacle in this case is given as
\begin{equation} \label{eq:intruderObs_case2_caseB}
_2^B\ioset_i^j(t) = \brs^{\text{B2}}_{i}(t, t+\iat).
\end{equation}
\end{itemize}
% !TEX root = ../SPP_IoTjournal.tex
\subsection{Optimal Avoidance Controller} \label{sec:intruder_avoid}
To compute the optimal avoidance control for $\veh_i$ in presence of $\veh_{\intr}$, we compute the set of states from which the joint states of $\veh_{\intr}$ and $\veh_i$ can enter danger zone $\dz_{i\intr}$ despite the best efforts of $\veh_i$ to avoid $\veh_{\intr}$. 

We define relative dynamics of the intruder $\veh_{\intr}$ with state $\state_\intr$ with respect to $\veh_i$ with state $\state_i$:
\begin{equation}
\label{eq:reldyn}
\begin{aligned}
\state_{\intr, i} &= \state_\intr - \state_i \\
\dot \state_{\intr, i} &= f_r(\state_{\intr, i}, \ctrl_i, \ctrl_\intr, \dstb_i, \dstb_\intr)
\end{aligned}
\end{equation}
Given the relative dynamics, we compute the set of states from which the joint states of $\veh_{\intr}$ and $\veh_{i}$ can enter danger zone $\dz_{i\intr}$ in a duration of $\iat$ despite the best efforts of $\veh_i$ to avoid $\veh_{\intr}$. Under the relative dynamics \eqref{eq:reldyn}, this set of states is given by the backwards reachable set $\brs^{\text{S}}_i(\tau, \iat),~ \tau \in [0, \iat]$:

\begin{equation} \label{eqn:avoidBRS}
\begin{aligned}
\brs^{\text{S}}_{i}(\tau, \iat) = & \{y: \forall \ctrl_i(\cdot) \in \cfset_i, \exists \ctrl_\intr(\cdot) \in \cfset_\intr, \exists \dstb_i(\cdot) \in \dfset_i, \\
& \exists \dstb_\intr(\cdot) \in \dfset_\intr, \state_{\intr, i}(\cdot) \text{ satisfies \eqref{eq:reldyn}},\\
& \exists s \in [\tau, \iat], \state_{\intr, i}(s) \in \targetset^{\text{S}}_{i}, \state_{\intr, i}(\tau) = y\},
\end{aligned}
\end{equation}
where 
\begin{equation}
\begin{aligned}
\targetset^{\text{S}}_{i} = & \{\state_{\intr, i}: \|\pos_{\intr, i}\|_2 \le \rc\} \\
H^{\text{S}}_{i}(\state_{\intr, i}, \costate) = & \max_{\ctrl_i \in \cset_i} \left( \right.\\
&\left. \min_{\ctrl_\intr \in \cset_\intr, \dstb_i \in \dset_i, \dstb_\intr \in \dset_\intr} \costate \cdot f_r(\state_{\intr, i}, \ctrl_i, \ctrl_\intr, \dstb_i, \dstb_\intr) \right)
\end{aligned}
\end{equation}

We refer to $\brs^{\text{S}}_i(\tau, \iat)$ as \textit{avoid region} hereon. Once the value function $\valfunc^{\text{S}}_{i}(\cdot)$ is computed, the optimal avoidance control ${\ctrl^{\text{S}}_{i}}$ can be obtained as:
\begin{equation} \label{eqn:optAvoidCtrl}
{\ctrl^{\text{S}}_{i}} = \arg \max_{\ctrl_i \in \cset_i} \left( \min_{\ctrl_\intr \in \cset_\intr, \dstb_i \in \dset_i, \dstb_\intr \in \dset_\intr} \costate \cdot f_r(\state_{\intr, i}, \ctrl_i, \ctrl_\intr, \dstb_i, \dstb_\intr) \right)
\end{equation}

Let $\partial \brs^{\text{S}}_{i}(\cdot, \iat)$ denotes the boundary of the set $\brs^{\text{S}}_{i}(\cdot, \iat)$. The interpretation of $\brs^{\text{S}}_{i}(\tau, \iat)$ is that if $\state_{\intr, i}(t) \in \partial \brs^{\text{S}}_{i}(\tau, \iat)$, then $\veh_i$ can successfully avoid the intruder for a duration of $(\iat - \tau)$ using the optimal avoidance control in \eqref{eqn:optAvoidCtrl}. In the worst case, $\veh_i$ might need to avoid the intruder for a duration of $\iat$; thus, we must have that  
$\state_{\intr, i}(\tsa) \in \left(\brs^{\text{S}}_{i}(0, \iat)\right)^C$. Equivalently, every SPP vehicle should be able to detect the intruder at a distance of $\dsen$, where

\begin{equation} \label{eqn:sen_distance}
\dsen = \max\{ \|a\|: a \in \brs^{\text{S}}_{i}(0, \iat) \}.
\end{equation} 

%Under normal circumstances when the intruder $\veh_{\intr}$ is far away, we have $\valfunc^{\text{S}}_{i}(0, \state_{\intr, i}(t)) > 0$; as $\veh_{\intr}$ gets closer to $\veh_i$, $\valfunc^{\text{S}}_{i}(0, \state_{\intr, i}(t))$ decreases. If $\veh_i$ applies the control ${\ctrl^{\text{S}}_{i}}$ when $\valfunc^{\text{S}}_{i}(0, \state_{\intr, i}(t)) = 0$, then collision avoidance between $\veh_i$ and $\veh_{\intr}$ is guaranteed for a duration of $\iat$ under the worst-case intruder control strategy $\ctrl_\intr$.
% !TEX root = ../SPP_IoTjournal.tex
\subsection{Replanning after intruder avoidance} \label{sec:replan}
As discussed in Section \ref{sec:path_planning}, the intruder can force some SPP vehicles to deviate from their planned nominal trajectory; therefore, goal satisfaction is no longer guaranteed once a vehicle is forced to apply an avoidance maneuver. Therefore, we have to replan the trajectories of these vehicles once $\veh_{\intr}$ disappears. The set of all vehicles $\veh_i$ for whom we need to replan the trajectories, $\rvs$, can be obtained by checking if a vehicle $\veh_i$ applied any avoidance control during $[\tsa, \tea]$, e.g.,
\begin{equation} \label{eq:RVS}
\rvs = \{\veh_i: \tsa_i < \infty, i \in \{1, \ldots, \N \} \}. 
\end{equation}  

Note that due to the presence of separation and buffer regions, at most $\nva$ vehicles can be affected by $\veh_{\intr}$, e.g. $|\rvs| \leq \nva$. Goal satisfaction controllers which ensure that these vehicles reach their destinations can be obtained by solving a new SPP problem, where the starting states of the vehicles are now given by the states they end up in, denoted $\tilde{\state}_i^0$, after avoiding the intruder. Note that we can pick $\nva$ beforehand and design buffer regions accordingly. Thus, by picking compatible $\nva$ based on the available computation resources during run-time, we can ensure that this replanning can be done in real time. Moreover, flexible path-planning algorithms such as FaSTrack \cite{herbert2017fastrack} can be used that can perform planning in real-time.   

Let the optimal control policy corresponding to this liveness controller be denoted ${\ctrl^{\text{L}}_{i}}(t, \state_i)$. The overall control policy that ensures intruder avoidance, collision avoidance with other vehicles, and successful transition to the destination for vehicles in $\rvs$ is given by:

\begin{equation} \label{eqn:full_controller}
\ctrl_i^{\text{RP}}(t) = 
\left \{ 
\begin{array}{ll}
{\ctrl^*_{i}}(t, \state_i) & t \leq \tea\\
{\ctrl^{\text{L}}_{i}}(t, \state_i) & t > \tea
\end{array}
\right.
\end{equation}

Note that in order to re-plan using a SPP method, we need to determine feasible $\sta_i$ for all vehicles. This can be done by computing an FRS:
\begin{equation} \label{eq:re-planFRS}
\begin{aligned} 
\frs_i^{\text{RP}}(\tea, t) = & \{y \in \R^{n_i}: \exists \ctrl_i(\cdot) \in \cfset_i, \forall \dstb_i(\cdot) \in \dfset_i, \\
& \state_i(\cdot) \text{ satisfies \eqref{eq:dyn}}, \state_i(\tea) = \tilde{\state}_i^0, \\
& \state_i(t) = y, \forall s \in [\tea, t], \state_i(s) \notin \obsset_i^{\text{RP}}(s) \},
\end{aligned}
\end{equation}
\noindent where $\tilde{\state}_i^0$ represents the state of $\veh_i$ at $t = \tea$; $\obsset_i^{\text{RP}}(\cdot)$ takes into account the fact that $\veh_i$ now needs to avoid all other vehicles in $(\rvs)^C$ and is defined in a way analogous to \eqref{eq:obsseti}. The FRS in \eqref{eq:re-planFRS} can be obtained by solving %the HJ VI in \eqref{eq:HJIVI_FRS} with the following Hamiltonian:
\begin{equation}
\begin{aligned}
\max \Big\{&D_t \valfuncfwd_i^{\text{RP}}(t, \state_i) + \ham_i^{\text{RP}}(t, \state_i, \nabla \valfuncfwd_i^{\text{RP}}(t, \state_i)), \\
&\qquad - \obsfunc_i^{\text{RP}}(t, \state_i) - \valfuncfwd_i^{\text{RP}}(t, \state_i) \Big\} = 0\\
&\valfuncfwd_i^{\text{RP}}(\tsa, \state_i) = \max\{\fc_i^{\text{RP}}(\state_i), -\obsfunc_i^{\text{RP}}(\tsa, \state_i)\} \\
&\ham_i^{\text{RP}}(\state_i, \costate) = \max_{\ctrl_i \in \cset_i} \min_{\dstb_i \in \dset_i} \costate \cdot f_i (\state_i, \ctrl_i, \dstb_i)
\end{aligned}
\end{equation} 

\noindent where $\valfuncfwd_i^{\text{RP}}, \obsfunc_i^{\text{RP}}, \fc_i^{\text{RP}}$ represent the FRS, obstacles during re-planning, and the initial state of $\veh_i$, respectively. The new $\sta_i$ of $\veh_i$ is now given by the earliest time at which $\frs_i^{\text{RP}}(\tea, t)$ intersects the target set $\targetset_i$, $\sta_i := \arg \inf_t \{ \frs_i^{\text{RP}}(\tea, t) \cap \targetset_i \neq \emptyset \}$. Intuitively, this means that there exists a control policy which will steer the vehicle $\veh_i$ to its destination by that time, despite the worst case disturbance it might experience.

\begin{remark}
Note that even though we have presented the analysis for one intruder, the proposed method can handle multiple intruders as long as only one intruder is present \textit{at any given time} by designing buffer regions during replanning as well.
\end{remark}

We conclude this section by summarizing the overall SPP algorithm: 
%
\begin{algorithm}[tb]
\SetKwInOut{Input}{input}
\SetKwInOut{Output}{output}
	\DontPrintSemicolon
	\caption{The intruder avoidance algorithm: Planning-phase (offline planning)}
	\label{alg:intruder_plan}
	\Input{Set of vehicles $\veh_i, i = 1, \ldots, \N$ in the descending priority order;\newline
	Vehicle dynamics \eqref{eq:dyn} and initial states $\state_i^0$;\newline
	Vehicle destinations $\targetset_i$ and static obstacles $\soset_i$;\newline
	Intruder dynamics $f_{\intr}$ and the maximum avoidance time $\iat$ ;\newline
	Maximum number of vehicles allowed to re-plan their trajectories $\nva$.}
    \Output{The nominal trajectories for all vehicles;\newline 
    The nominal controller $\ctrl^{\text{PP}}$ and the avoidance controller $\ctrl^{\text{A}}$ for all vehicles.}
	\For{\text{$i=1:N$}}{
			%
			\textbf{Avoid region and avoidance control for $\veh_{i}$} \;	
			compute the avoid region $\brs^{\text{A}}_{i}$ using \eqref{eqn:avoidBRS}; \;
			compute the optimal avoidnace controller $\ctrl^{\text{A}}_i$ using \eqref{eqn:optAvoidCtrl}; \;
			output the optimal avoidance controller for $\veh_i$. \;
			%
			\If{\text{$i \neq 1$}}{
			    \textbf{Computation of separation region for $\veh_{i}$} \;	
				\For{\text{$j=1:i-1$}}{
					given the base obstacles $\boset_j(\cdot)$ and the avoid region $\brs^{\text{A}}_{j}$, compute the separation regions in \eqref{eqn:sepRegion_case1} and \eqref{eqn:sepRegion_case2}; 					\;}
				\textbf{Computation of buffer region for $\veh_{i}$} \;	
				\SBnote{Start from here} \;
				\For{\text{$j=1:i-1$}}{
					given the separation regions, compute the relative buffer regions in \eqref{eqn:sepRegion_case1} and \eqref{eqn:sepRegion_case2}; \;}
			}			
			\textbf{Path planning for $\veh_{i}$} \;
			compute the total obstacle set $\obsset_i(t)$ given by \eqref{eq:obsseti}. If $i=1$, $\obsset_i(t) = \soset_i ~ \forall t$;\;
			compute the BRS $\brs_i^\text{basic}(t, \sta_i)$ defined in \eqref{eq:BRS_basic}.;\;
			\textbf{Trajectory and controller of $\veh_{i}$} \;
			compute the optimal controller $\ctrl_i^\text{basic}(\cdot)$ given by \eqref{eq:basicOptCtrl};\;
			determine the trajectory $\state_i(\cdot)$ using vehicle dynamics \eqref{eq:dyn_no_dstb} and the control $\ctrl_i^\text{basic}(\cdot)$; \;
			output the trajectory and optimal controller for $\veh_i$.\;
			\textbf{Induced obstacles by $\veh_{i}$} \;
			given the trajectory $\state_i(\cdot)$, compute the induced obstacles $\ioset_k^i(t)$ given by \eqref{eq:ioset_basic} for all $k>i$.
		}
\end{algorithm}
%
%
%\begin{algorithm}[tb]
%	%\RestyleAlgo{boxed}
%	\DontPrintSemicolon
%	%\AlgoDisplayBlockMarkers\SetAlgoBlockMarkers{}{end}%
%	%\SetAlgoNoEnd
%	\caption{Intruder avoidance algorithm (offline planning)}
%	\label{alg:intruder_plan}
%	Given initial conditions $\state_i^0$, vehicle dynamics \eqref{eq:dyn}, intruder dynamics in Assumption \ref{as:dynKnown}, target sets $\targetset_i$, and static obstacles $\soset_i, i = 1\ldots, 		    \N$\;
%	\For{\text{$i=1:N$}}{
%			compute the avoid region of $\veh_i$ using \eqref{eqn:avoidBRS}. The optimal control to avoid the intruder can be obtained by using \eqref{eqn:optAvoidCtrl}.\;
%			determine the separation region and buffer regions given by \eqref{eqn:sepRegion_case1}, \eqref{eqn:buffRegion_case1} and \eqref{eqn:buffRegion_case2}. \;
%			compute the induced obstacles for $\veh_i$ by $\veh_j$, given by \eqref{eq:intruderObs_case1_caseA}, \eqref{eq:intruderObs_case1_caseB}, \eqref{eq:intruderObs_case2_caseA}, 	\eqref{eq:intruderObs_case2_caseB} and \eqref{eq:ObsBRS_static}. \;
%			compute the total obstacle set $\obsset_i(t)$, given in \eqref{eq:obsseti_intr}. In the case $i=1$, $\obsset_i(t) = \brs^{\text{S}}_{i}(t, t+\iat) ~ \forall t$ \;
%			given $\obsset_i(t)$, compute the BRS $\brs^{\text{PP}}_{i}(t, \sta_i)$ defined in \eqref{eqn:intrBRS1}; the nominal control strategy is given by \eqref{eqn:PPPolicy}\;
%		}
%\end{algorithm}
%
%\begin{alg}
%\label{alg:intruder}
%\textbf{Intruder Avoidance algorithm (offline planning)}: Given initial conditions $\state_i^0$, vehicle dynamics \eqref{eq:dyn}, intruder dynamics in Assumption \ref{as:dynKnown}, target sets $\targetset_i$, and static obstacles $\soset_i, i = 1\ldots, \N$, for each $i$,
%\begin{enumerate}
%\item compute the avoid region of $\veh_i$ using \eqref{eqn:avoidBRS}. The optimal control to avoid the intruder can be obtained by using \eqref{eqn:optAvoidCtrl}.
%\item determine the separation region and buffer regions given by \eqref{eqn:sepRegion_case1}, \eqref{eqn:buffRegion_case1} and \eqref{eqn:buffRegion_case2}.
%\item compute the induced obstacles for $\veh_i$ by $\veh_j$, given by \eqref{eq:intruderObs_case1_caseA}, \eqref{eq:intruderObs_case1_caseB}, \eqref{eq:intruderObs_case2_caseA}, \eqref{eq:intruderObs_case2_caseB} and \eqref{eq:ObsBRS_static}. 
%\item compute the total obstacle set $\obsset_i(t)$, given in \eqref{eq:obsseti_intr}. In the case $i=1$, $\obsset_i(t) = \brs^{\text{S}}_{i}(t, t+\iat) ~ \forall t$;
%\item given $\obsset_i(t)$, compute the BRS $\brs^{\text{PP}}_{i}(t, \sta_i)$ defined in \eqref{eqn:intrBRS1}; the nominal control strategy is given by \eqref{eqn:PPPolicy};
%\end{enumerate}
%
%\textbf{Intruder Avoidance algorithm (online re-planning)}: For each vehicle $\veh_i \in \rvs$ which performed avoidance in response to the intruder,
%\begin{enumerate}
%\item compute $\frs_i^{\text{RP}}(\tea, t)$ using $\eqref{eq:re-planFRS}$. The new $\sta_i$ for $\veh_i$ is given by $\arg \inf_t \{ \frs_j^{\text{RP}}(\tea, t) \cap \targetset_j \neq \emptyset \}$;
%\item given $\sta_i$, $\tilde{\state}_i^0$, vehicle dynamics \eqref{eq:dyn}, target set $\targetset_i$, and obstacles $\obsset_i^{\text{RP}}$, use any of the three SPP methods discussed in \cite{chen2016robust} for re-planning. 
%\end{enumerate}
%\end{alg}

% Simulations
% !TEX root = ../SPP_IoTjournal.tex
\section{City Environment Simulation \label{sec:bayArea_sim}}
We next use SPP algorithm to design trajectories for a 200 vehicle UAV system where UAVs are flying through a multi-city region.

%%%%%%%%%%%%%%%%%%%%%%%%%%%%%%%%%%%%%%%%%%%%%%%%%%%%%%%%%%%%%%%%%%%%%%%%%%%%%%%%
%\addtolength{\textheight}{1cm}   % This command serves to balance the column lengths
                                  % on the last page of the document manually. It shortens
                                  % the textheight of the last page by a suitable amount.
                                  % This command does not take effect until the next page
                                  % so it should come on the page before the last. Make
                                  % sure that you do not shorten the textheight too much.

\bibliographystyle{IEEEtran}
\bibliography{IEEEabrv,references}
\end{document}
