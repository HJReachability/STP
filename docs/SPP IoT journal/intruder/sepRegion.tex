% !TEX root = ../SPP_IoTjournal.tex
\subsection{Separation and Buffer Regions} \label{sec:sepRegion}

\SBnote{
\begin{itemize}
\item Introduction of the section:
\begin{itemize}
\item Main summary: In the next two sections, we are computing separation between any two vehicles such that only one applies the avoidnace maneuver at a time and atmost $\nva$ vehicles need to apply avoidnace maneuver during the entire duration. 
\item To understand what this means we introduce the notation of avoid start time. Introduce the notation and give its mathematical definition in terms of relative state and avoidnace set computed in the previous section. Now explain that only one vehicle need to apply avoidance control at a time is equivalent to $t_i \neq t_j$. Only $\nva$ vehicles need to apply avoidnace control is equivalent to $t_i - t_j \geq \brd$.  
\item If we find the set of all states of $\veh_i$ such that this separation is ensured between $\veh_i$ and $\veh_m$ for all $m<i$, then during the path planning of $\veh_i$, we can then ensure that $\veh_i$ is indeed in one of these states and we will be done (as the path planning is sequential). So we next focus on finding all the states of $\veh_i$ at time $t$ such that this separation is ensured between two vehicles $\veh_j$ and $\veh_i$, $j <i$ at time $t$ for all intruding scenarios.
\end{itemize}
\item Section1: $t_j < t_i$.
\begin{itemize}
\item We next make the following observations: (1) it is sufficient to consider the scenarios where $\tsa$ is within $\iat$ distance. (2) WLOG, initial state of the intruder belongs to a boundary of vehicle $j$.
\item Section11: First find the set of all states of intruder for which $\veh_j$ is forced to apply an avoidance maneuver. Explain what that set is in relative co-ordinates for different $\tsa$. To compute the set in absolute co-ordinates we need to augment the relative states to the states of $\veh_j$. Explain, how we can compute these states of $\veh_j$ (by FRS).
\item Section12: Next we want to make sure that $\veh_i$ is far away from this set such that the intruder will need atleast a duration of $\brd$ to reach any state such that $\state_{\intr,i}$ is on the boundary of the avoid set of $\veh_i$. In relative co-ordinates, if we compute min-min set and augment it on set in Section 11, and ensure that the boundary of the avoid region is outside this augmented set then we are done. Draw a table listing the duration of each set. Mention the duration relationships. Draw a diagram showing different sets. 
\item Section13: Obstacle computation. First compute the directly induced obstacle by $\veh_j$ at time $t$ for all possible $\tsa$. Then compute the obstacle that $\veh_i$ need to avoid at any future time for different $\tsa$. Therefore, compute the BRS that it needs to avoid at $\tsa$.    
\end{itemize}
\item Section2: $t_i < t_j$.
\end{itemize}
}

In the next two sections, our goal is to ensure that any two vehicles are separated enough from each other such that only one of the vehicles is \textit{forced} to apply the intruder avoidance maneuver at any given time. Moroever, atmost $\nva$ vehicles are \textit{forced} to apply avoidnace maneuver during the entire duration of $\iat$. 

For vehicle $\veh_m$, let $\avoidt_{m} := \{t: \state_{\intr,m}(t) \in \brs^{\text{S}}_{m}(t-\tsa, \iat), t \in [\tsa, \tea]\}$. $\avoidt_{m}$ is thus the set of all times at which $\veh_m$ is forced to apply an intruder avoidance maneuver. We define the avoid start time $\tsa_m$ as:
\begin{equation}
\tsa_m  = 
\left \{ 
\begin{array}{ll}
\min\{t: t \in  \avoidt_{m}\} & \mbox{if~} \avoidt_{m} \neq \emptyset\\
\infty & \mbox{otherwise}
\end{array}
\right.
\end{equation}  

Only one vehicle applying the avoidance maneuver at a time in the presence of an intruder is thus equivalent to $\tsa_i \neq \tsa_j \forall i \neq j$. Similarly, if we ensure that $\forall i \neq j, |\tsa_i - \tsa_j| \geq \iat / \nva  := \brd$, then atmost $\nva$ vehicles are forced to apply the avoidance maneuver during the time interval $[\tsa, \tea]$. We refer to these two conditions as \textit{separation requirements} hereon. For any given time $t$, if we could find the set of all states of $\veh_i$ such that the separation requirements hold for $\veh_i$ and $\veh_m \forall ~ m<i$, then during the path planning of $\veh_i$, we can ensure that $\veh_i$ is in one of these states at time $t$. The sequential path planning will therefore guarantee that the separation requirements hold for every SPP vehicle pair. Hereon, we thus focus on finding all the states $\state_i(t)$ such that the separation requirements are ensured between vehicles $\veh_j$ and $\veh_i$, $j <i$ at time $t$. We consider the following two mutually exclusive and exhaustive cases: $\tsa_i < \tsa_j$ and $\tsa_i \geq \tsa_j$. %As we will see shortly, our construction ensures that $\tsa_i \neq \tsa_j$.