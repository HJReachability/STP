% !TEX root = ../SPP_IoTjournal.tex
\subsection{Replanning after intruder avoidance} \label{sec:replan}
After the intruder disappears, liveness controllers which ensure that the vehicles reach their destinations can be obtained by solving a new SPP problem, where the starting states of the vehicles are now given by the states they end up in, denoted $\tilde{\state}_j^0$, after avoiding the intruder. The set of all vehicles $\veh_j$ for whom we need to replan the trajectories, $\rvs$, can be obtained by checking if a vehicle $\veh_j$ applied any avoidnace control during $[\tsa, \tea]$, e.g.,
\begin{equation} \label{eq:RVS}
\rvs = \{j \in \{1, \ldots, \N \}: \valfunc^{\text{S}}_{j}(0, \state_{\intr, j}(t)) \leq 0, t \in [\tsa, \tea] \}. 
\end{equation}  

Let the optimal control policy corresponding to this liveness controller be denoted ${\ctrl^{\text{L}}_{j}}(t)$. The overall control policy that ensures intruder avoidance, collision avoidance with other vehicles, and successful transition to the destination for vehicles $j \in \rvs$ is given by:

\begin{equation*}
\ctrl_j^*(t) = 
\left \{ 
\begin{array}{ll}
{\ctrl^{\text{A}}_{j}}(t) & t \leq \tea\\
{\ctrl^{\text{L}}_{j}}(t) & t > \tea
\end{array}
\right.
\end{equation*}

Note that in order to replan using a SPP method, we need to determine feasible $\sta$s for all vehicles $j \in \rvs$. This can be done by computing an FRS:
\begin{equation} \label{eq:replanFRS}
\begin{aligned} 
\frs_j^{\text{RP}}(\tea, t) = & \{y \in \R^{n_j}: \exists \ctrl_j(\cdot) \in \cfset_j, \forall \dstb_j(\cdot) \in \dfset_j, \\
& \state_j(\cdot) \text{ satisfies \eqref{eq:dyn}}, \state_j(\tea) = \tilde{\state}_j^0, \\
& \state_j(t) = y\},
\end{aligned}
\end{equation}
where $\tilde{\state}_j^0$ represents the state of $\veh_j$ at $t = \tea$. The FRS in \eqref{eq:replanFRS} can be obtained by solving the HJ VI in \eqref{eq:HJIVI_FRS} with the following Hamiltonian:
\begin{equation}
\ham_j^{\text{RP}}(\state_j, \costate) = \max_{\ctrl_j \in \cset_j} \min_{\dstb_j \in \dset_j} \costate \cdot f_j (\state_j, \ctrl_j, \dstb_j). 
\end{equation} 
The new $\sta$ of $\veh_j$ is now given by the earliest time at which $\frs_j^{\text{RP}}(\tea, t)$ intersects the target set $\targetset_j$, $\sta_j := \arg \inf_t \{ \frs_j^{\text{RP}}(\tea, t) \cap \targetset_j \neq \emptyset \}$. Intuitively, this means that there exists a control policy which will steer the vehicle to its destination by that time, despite the worst case disturbance it might experience.

\SBnote{This does not seem correct. We have to compute STA for each vehicle in the order of priority and take into account the obstacles that they might experience while computing FRS.}