% !TEX root = ../SPP_IoTjournal.tex
\subsection{Buffer Region} \label{sec:buffRegion}
In section \ref{sec:sepRegion}, we computed sets $\sep_j(\cdot)$ such that $\veh_j$ avoids the intruder only if $\state_{\intr}(t) \in \sep_j(t)$. But to ensure that atmost $\nva$ vehicles need to replan their trajectories after the intruder disappears, we need to make sure that the intruder can cause atmost $\nva$ vehicles to deviate from their planned trajectories. Equivalently, we want to ensure that atmost $\nva$ vehicles need to avoid the intruder. 

Intuitively, we want to make sure that at any given time the separation regions of any two vehicles are far enough from each other (that is, there is a ``buffer" region between two separation regions) such that it will take at least $\brd := \iat / \nva$ time for the intruder to go from the separation region of one vehicle to that of the other vehicle. This means that there is a ``buffer" time interval of $\brd$ between any $t_1$, $t_2 \in [\tsa, \tea]$ where $\veh_{\intr}$ is in the separation regions of two different vehicles at $t_1$ and $t_2$, e.g. $\state_{\intr}(t_1) \in \sep_j(t)$ and $\state_{\intr}(t_2) \in \sep_i(t)$, $i \neq j$. Thus, in the worst case, the intruder can force atmost $\nva$ vehicles to apply avoidance maneuver in a duration of $\iat$. 

We next focus on computing the buffer region between any two vehicles $\veh_j$ and $\veh_i$, $j < i$. Without loss of generality, we can assume that the intruder appears at the separation region of a vehicle at $t = \tsa$, because if it doesn't then the vehicles need not account for intruders until it reaches the boundary of the separation region of a vehicle. To compute the buffer region, we consider the following two cases:     

\subsubsection{Case1- $\state_{\intr}^0 \in \sep_j(\tsa)$} \label{sec:buffCase1}
Given the relative dynamics $\state_{i, \intr}$ in \eqref{eq:reldyn}, we compute the set of states from which the joint states of $\veh_{\intr}$ and $\veh_{i}$ can enter danger zone $\dz_{i\intr}$ within a duration of $\brd$ when both $\veh_{i}$ and $\veh_{\intr}$ are using \textit{optimal control to collide} with each other. This set of states is given by the backwards reachable set $\brs^{\text{B}}_i(\tau, \brd)$: \MCnote{Might be clearer to put ``exists'' before every expression? Mostly to make it easier to differentiate from the case where there is a ``for all''... not sure.}

\begin{equation} \label{eqn:buffBRS}
\begin{aligned}
\brs^{\text{B}}_{i}(t, \brd) = & \{y: \exists \ctrl_i(\cdot) \in \cfset_i, \ctrl_\intr(\cdot) \in \cfset_\intr, \dstb_i(\cdot) \in \dfset_i, \\
& \dstb_\intr(\cdot) \in \dfset_\intr, \state_{i, \intr}(\cdot) \text{ satisfies \eqref{eq:reldyn}},\\
& \exists s \in [t, \brd], \state_{i, \intr}(s) \in \targetset^{\text{B}}_{i}, \state_{i, \intr}(t) = y\},
\end{aligned}
\end{equation}
where 
\begin{equation}
\begin{aligned}
\targetset^{\text{B}}_{i} &= \{\state_{i, \intr}: \|\pos_{i, \intr}\|_2 \le \rc\} \\
H^{\text{B}}_{i}(\state_{i, \intr}, \costate) &= \min_{\ctrl_i \in \cset_i, \ctrl_\intr \in \cset_\intr, \dstb_i \in \dset_i, \dstb_\intr \in \dset_\intr} \costate \cdot f_r(\state_{i, \intr}, \ctrl_i, \ctrl_\intr, \dstb_i, \dstb_\intr)
\end{aligned}
\end{equation}

The interpretation of set $\brs^{\text{B}}_{i}(0, \brd)$ is that if the separation region of $\veh_i$ is outside the boundary of this set and $\veh_{\intr}$ is at the boundary of $\targetset^{\text{B}}_{i}$ (in relative coordinates), then $\veh_{\intr}$ and $\veh_{i}$ cannot enter the danger zone $\dz_{i\intr}$ for a duration of $\brd$, irrespective of control applied by them. If we augment this set on the separation region of the $\veh_j$, then we get the same property in the state space of $\veh_i$:
\begin{equation} \label{eqn:buffRegion}
\buff_{ij}(t) = \sep_j(t) + \brs^{\text{B}}_{i}(0, \brd).
\end{equation} 

Finally, during the path planning of $\veh_i$, we need to ensure that $\veh_i$ is far enough from the boundary of $\buff_{ij}(t)$ such that $\veh_{\intr}$ and $\veh_{i}$ cannot enter the danger zone $\dz_{i\intr}$ for the remaining duration of $\trd := \iat - \brd$. Thus, during the path planning of $\veh_i$, we need to ensure that $\veh_i$ is outside the augmented buffer region:
\begin{equation} \label{eqn:augbuffRegion1}
\tilde{\buff}_{ij}(t) = \buff_{ij}(t) + \brs^{\text{S}}_{i}(0, \trd),
\end{equation}
where $\brs^{\text{S}}_{i}(0, \trd)$ can be computed as described in Section \ref{sec:sepRegion}.

\subsubsection{Case2- $\state_{\intr}^0 \in \sep_i(\tsa)$} \label{sec:buffCase2}
This case can be treated in a similar fashion as Section \ref{sec:buffCase1}. We can now look at the same problem from $\veh_i$'s perspective and compute the augmented buffer region $\tilde{\buff}_{ji}(t)$ as:
\begin{equation} \label{eqn:augbuffRegion2}
\tilde{\buff}_{ij}(t) = \boset_j(t) + \brs^{\text{S}}_{j}(0, \trd) + \brs^{\text{B}}_{j}(0, \brd) + \brs^{\text{S}}_{i}(0, \iat).
\end{equation}
During the path planning of $\veh_i$, we need to ensure that $\veh_i$ is outside $\tilde{\buff}_{ji}(t)$. 