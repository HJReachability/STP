% !TEX root = ../SPP_IoTjournal.tex
\subsection{Buffer Region} \label{sec:buffRegion}
In section \ref{sec:sepRegion}, we computed sets $\sep_j(\cdot)$ such that $\veh_j$ avoids the intruder only if $\state_{\intr}(t) \in \sep_j(t)$. But to ensure that atmost $\nva$ vehicles need to replan their trajectories after the intruder disappears, we need to make sure that the intruder can cause atmost $\nva$ vehicles to deviate from their planned trajectories. Equivalently, we want to ensure that atmost $\nva$ vehicles need to avoid the intruder. 

Intuitively, we want to make sure that at any given time the separation regions of any two vehicles are far enough from each other such that it will take at least $\brd := \iat / \nva$ time for the intruder to go from the separation region of one vehicle to that of the other vehicle. This means that there is a ``buffer" time interval of $\brd$ between 
any $t_1$, $t_2 \in [\tsa, \tea]$ where $\veh_{\intr}$ is in the separation regions of two different vehicles at $t_1$ and $t_2$, e.g. $\state_{\intr}(t_1) \in \sep_j(t)$ and $\state_{\intr}(t_2) \in \sep_i(t)$, $i \neq j$. Thus, in the worst case, the intruder can force atmost $\nva$ vehicles to apply avoidance maneuver in a duration of $\iat$.          


\SBnote{start from here}

But to ensure that atmost one vehicle avoids the intruder, we also need to make sure that no other vehicle $\veh_i, i>j$, avoids the intruder if it appears inside $sep_j(\tsa)$. This can be achieved by ensuring that vehicle $\veh_i$ is far enough from $\sep_j(\cdot)$ (that is, there is a ``buffer" region between $\veh_i$ and $\sep_j(\cdot)$), such that intruder appearing inside $sep_j(\cdot)$ cannot enter the danger zone $\dz_{i\intr}$. We denote this buffer region as $\buff^i(t)$, and the separation region augmented with the buffer region as $\tilde{\sep}^j_i(t)$.

$\buff^i(t)$ can be computed using the relative dynamics 

However, to ensure that atmost one vehicle needs to apply the avoidance maneuver, we need to make sure that no other vehicle $\veh{i}, i \neq j$, needs to avoid the intruder if intruder starts inside $\brs_\text{C}$. This can  be achieved by ensuring that $\brs_{\text{C},i}$ and $\brs_{\text{C},j}$ do not intersect.





\SBnote{The folliwng need to be moved to the buffer region section}
when both $\veh_{j}$ and $\veh_{\intr}$ are using \textit{optimal control to collide} with each other for a duration of $\iat$. 
\begin{equation} \label{eqn:optAvoid}
\begin{aligned}
\brs^{\text{S}}_{j}(t, \iat) = & \{y: \exists \ctrl_j(\cdot) \in \cfset_j, \ctrl_\intr(\cdot) \in \cfset_\intr, \dstb_j(\cdot) \in \dfset_j, \\
& \dstb_\intr(\cdot) \in \dfset_\intr, \state_{\intr, j}(\cdot) \text{ satisfies \eqref{eq:reldyn}},\\
& \exists s \in [t, \iat], \state_{\intr, j}(s) \in \targetset^{\text{S}}_{j}, \state_{\intr, j}(t) = y\},
\end{aligned}
\end{equation}
where 
\begin{equation}
\begin{aligned}
\targetset^{\text{S}}_{j} &= \{\state_{\intr, j}: \|\pos_{\intr, j}\|_2 \le \rc\} \\
H^{\text{S}}_{j}(\state_{\intr, j}, \costate) &= \min_{\ctrl_j \in \cset_j, \ctrl_\intr \in \cset_\intr, \dstb_i \in \dset_i, \dstb_\intr \in \dset_\intr} \costate \cdot f_r(\state_{\intr, j}, \ctrl_j, \ctrl_\intr, \dstb_j, \dstb_\intr)
\end{aligned}
\end{equation}