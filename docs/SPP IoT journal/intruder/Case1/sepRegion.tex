% !TEX root = ../../SPP_IoTjournal.tex
\subsubsection{Separation region} \label{sec:sepRegion_case1}
Consider any $\tsa \in \R$. In this section, we find the set of all states $\state_{\intr}^0 := \state_{\intr}(\tsa)$ for which $\veh_j$ is forced to apply an avoidnace maneuver. We refer to this set as \textit{separation region}, and denote it as $\sep_j(\tsa)$. As discussed in Section \ref{sec:intruder_avoid}, $\veh_j$ needs to apply avoidnace maneuver at time $\tsa$ only if $\state_{\intr, j}(\tsa) \in \partial \brs^{\text{A}}_{j}(0, \iat)$. To compute set $\sep_j(\tsa)$, we thus need to translate this set in relative coordinates to a set in absolute coordinates, for which we need to know all possible states of $\veh_j$ at time $\tsa$.

Depending on the information known to a lower-priority vehicle $\veh_i$ about $\veh_j$'s control strategy, we can use one of the three methods described in Section 5 in \cite{chen2016robust} to compute the ``base" obstacles $\boset_j(\tsa)$, the obstacles that would have been induced by $\veh_j$ in the presence of disturbances, but in the absence of an intruder. The base obstacles are respectively given by equations (25), (31) and (37) in \cite{chen2016robust} for centralized control, least restrictive control and robust trajectory tracking algorithms.

Given $\boset_j(\tsa)$, $\sep_j(\tsa)$ can be obtained as:
\begin{equation} \label{eqn:sepRegion_case1}
\sep_j(\tsa) = \boset_j(\tsa) + \partial \brs^{\text{A}}_{j}(0, \iat), ~\tsa \in \R,
\end{equation}
where the ``$+$'' in \eqref{eqn:sepRegion_case1} denotes the Minkowski sum. Since $\sep_j(\tsa)$ represents the set of all states of $\veh_\intr$ for which $\veh_j$ is forced to apply an avoidnace maneuver, we have $\state_{\intr}(\tsa) \in \sep_j(\tsa)$ in Case1.




