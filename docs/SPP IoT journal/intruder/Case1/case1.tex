% !TEX root = ../../SPP_IoTjournal.tex
\subsection{Separation and Buffer Regions- Case1} \label{sec:case1}
In the next two sections, our goal is to ensure that any two vehicles are separated enough from each other so that %only one of the vehicles is \textit{forced} to apply the intruder avoidance maneuver at any given time. Moroever, 
atmost $\nva$ vehicles should be \textit{forced} to apply avoidnace maneuver during the duration $[\tsa, \tea]$. To capture this mathematically, we define 
\begin{equation*}
\avoidt_{m} := \{t: \state_{\intr,m}(t) \in \brs^{\text{A}}_{m}(t-\tsa, \iat), t \in [\tsa, \tea]\}
\end{equation*} 
$\avoidt_{m}$ is the set of all times at which $\veh_m$ is forced to apply an intruder avoidance maneuver. We also define \textit{avoid start time}, $\tsa_m$, for $\veh_m$ as:
\begin{equation}
\tsa_m  = 
\left \{ 
\begin{array}{ll}
\min_{t \in  \avoidt_{m}} t & \mbox{if~} \avoidt_{m} \neq \emptyset\\
\infty & \mbox{otherwise}
\end{array}
\right.
\end{equation}  
Therefore, $\tsa_m \in [\tsa, \tea]$ denotes the first time at which $\veh_m$ applies an avoidance maneuver and defined to be $\infty$ if $\veh_m$ never applies an avoidance maneuver. %Only one vehicle applying the avoidance maneuver at a time is thus equivalent to $\forall i \neq j \tsa_i \neq \tsa_j$ whenever . Similarly, 
Therefore, if we ensure that 
\begin{equation} \label{eqn:sep_cond}
\forall i \neq j, min(\tsa_i, \tsa_j)< \infty \implies |\tsa_i - \tsa_j| \geq \frac{\iat}{\nva} := \brd,
\end{equation}
then atmost $\nva$ vehicles are forced to apply the avoidance maneuver during the time interval $[\tsa, \tea]$. We refer to the condition in \eqref{eqn:sep_cond} as \textit{separation requirement} hereon. 

For any given time $t$, if we could find the set of all states of $\veh_i$ such that the separation requirement holds for $\veh_i$ and $\veh_m$ for all $m<i$ and for all intruder strategies, then during the path planning of $\veh_i$, we can ensure that $\veh_i$ is in one of these states at time $t$. The sequential path planning will therefore guarantee that the separation requirement holds for every SPP vehicle pair. Thus, hereon we focus on finding all states $\state_i(t)$ such that the separation requirements are ensured between vehicles $\veh_j$ and $\veh_i$, $j <i$ at time $t$ for all possible intruder scenarios (meaning all possible $\tsa, \tea, \state_{\intr}^0$ and $\ctrl_{\intr}(\cdot))$. For our analysis, we consider the following two mutually exclusive and exhaustive cases: $\tsa_j \leq \tsa_i, \tsa_j < \infty$ and $\tsa_i < \tsa_j, \tsa_i <\infty$. %As we will see shortly, our construction ensures that $\tsa_i \neq \tsa_j$.
In this section, we consider Case1: $\tsa_j \leq \tsa_i, \tsa_j < \infty$. Case2 is discussed in the next section.  

In Case1, the intruder forces $\veh_j$ to apply avoidance control before or at the same time as $\veh_i$. %If $\tsa_i = \tsa_j = \infty$, then none of $\veh_i$ and $\veh_j$ need to apply avoidance control and hence trivially won't be involved in re-planning. 
To ensure the separation requirement in this case, we begin with the following observation which narrows down the intruder scenarios that we need to consider:
\begin{observation} \label{obs1_case1}
Without loss of generality, we can assume that the intruder appears at the boundary of the avoid region of $\veh_j$, e.g. $\state_{\intr, j}(\tsa) \in \partial \brs^{\text{A}}_{j}(0, \iat)$. Otherwise, vehicles $\veh_j$ and $\veh_i$ need not account for the intruder until it reaches the boundary of the avoid region of $\veh_j$. Also note that since $\infty < \tsa_j \leq \tsa_i$, $\veh_{\intr}$ reaches the boundary of the avoid region of $\veh_j$ first. Equivalently, we can assume that $\tsa_j = \tsa$.
\end{observation}