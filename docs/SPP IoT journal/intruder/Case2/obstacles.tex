% !TEX root = ../../SPP_IoTjournal.tex
\subsubsection{Obstacle Computation} \label{sec:intruderObs_case1}
In this section, we want to find the set of states that $\veh_i$ needs to avoid to avoid entering in the danger zone of $\veh_j$. We consider the following two mutually exclusive and exhaustive cases:
\begin{enumerate}
\item CaseA: intruder affects $\veh_i$, but not $\veh_j$.
\item CaseB: intruder first affects $\veh_i$ and then $\veh_j$.
\end{enumerate}
For each case, we compute the set of states that $\veh_i$ needs to avoid at time $t$ to avoid entering in $\dz_{ji}$. We also let $_2^A\ioset_i^j(\cdot)$ and $_2^B\ioset_i^j(\cdot)$ denote the set of obstacles corresponding to CaseA and CaseB respectively. 
\SBnote{Use a text superscript to differentiate between $\mathcal{K}$s.}

\begin{itemize}[leftmargin=*] 
\item \label{sec:intruderObs_case2_caseA} CaseA: In this case, we need to ensure that $\veh_i$ does not collide with the obstacle set $\boset_j(\cdot)$ or any static obstacles when it is avoiding the intruder. We begin with the following observation: 
\begin{observation} \label{obs1_case2_caseA}
By Observation \ref{obs1_case2Obs}, it is sufficient to consider the scenarios where $\tsa = \tsa_i \in [t-\iat, t]$. Since $\veh_i$ can be forced to apply an avoidance maneuver for a duration of $[\tsa_i, \tsa_i+\iat]$, to compute obstacles at time $t$ for a given $\tsa_i$, we need to make sure that $\veh_i$ avoid all states at time $t$ that can lead to a collision with $\veh_j$ or static obstacles during the interval $[t, \tsa_i+\iat]$ for some avoidnace control. Therefore, it is sufficient to consider the scenario $\tsa_i = t$ as that will maximize the avoidnace duration $[t, \tsa_i+\iat]$ for the obstacle computation at time $t$.  
\end{observation}

Mathematically, $\veh_i$ needs to avoid all states at time $t$ that can reach $\mathcal{K}(\tau)$ for some control action of $\veh_i$ during time duration $[t, \tau]$. $\mathcal{K}(\tau)$ here is given by:
\begin{equation}
\begin{aligned}
\mathcal{K}(\tau) = & \tilde{\boset}_j(s) \cup \soset_i\\
\tilde{\boset}_j(s) = & \{\state_j: \exists (y, h) \in \boset_j(s), \|\pos_j - y\|_2 \le \rc \}.
\end{aligned}
\end{equation}
Note that since the intruder is present in the system for a maximum duration of $\iat$ and since $\tsa_i = t$ (by Observation \ref{obs1_case2_caseA}), we have that $\tau \in [t, t+\iat]$. 

Avoiding $\mathcal{K}(\cdot)$ will ensure that $\veh_i$ and $\veh_j$ will not enter into each other's danger zones regardless of the avoidance maneuver applied by them. The set of states that $\veh_i$ needs to avoid at time $t$ is given by the following BRS: 
\begin{equation} \label{eq:ObsBRS_case2_caseA}
\begin{aligned}
\brs^{\text{A2}}_{i}(t, t+\iat) = & \{y: \exists \ctrl_i(\cdot) \in \cfset_i, \exists \dstb_i(\cdot) \in \dfset_i, \\
& \state_i(\cdot) \text{ satisfies \eqref{eq:dyn}}, \state_i(t) = y, \\
& \exists s \in [t, t+\iat], \state_i(s) \in \mathcal{K}(s) \}.
\end{aligned}
\end{equation}
The Hamiltonian $\ham^{\text{A2}}_{i}$ to compute $\brs^{\text{A2}}_{i}(t, t+\iat)$ is given by:
\begin{equation} \label{eqn:BRS_obsham_case2_caseA}
\ham^{\text{A2}}_{i}(\state_i, \costate) = \min_{\ctrl_i \in \cset_i} \min_{\dstb_i \in \dset_i} \costate \cdot f_i (\state_i, \ctrl_i, \dstb_i).
\end{equation}
Finally, the induced obstacle in this case is given as
\begin{equation} \label{eq:intruderObs_case2_caseA}
_2^A\ioset_i^j(t) = \brs^{\text{A2}}_{i}(t, t+\iat).
\end{equation}

\item \label{sec:intruderObs_case2_caseB} CaseB: In this case, the intruder first affects $\veh_i$ and then $\veh_j$. $\veh_i$ and $\veh_j$ apply their first avoidnace maneuver at $\tsa_i$ and $\tsa_j$ respectively. From the perspective  of $\veh_i$, $\veh_j$ can apply any control during the duration $[\tsa_j, \tsa_i + \iat]$ and hence can be anywhere in the set $\frs_{j}^{\mathcal{O}}(\tsa_j, \tau)$ at $\tau \in [\tsa_j, \tsa_i + \iat]$, where $\frs_{j}^{\mathcal{O}}$ denotes the FRS of base obstacle $\boset_j(\tsa_j)$ computed forward for a duration of $(\tsa_i + \iat - \tsa_j)$. $\veh_i$ thus needs to make sure that it avoids all states at time $t$ that can reach $\frs_{j}^{\mathcal{O}}(\tsa_j, \tau)$, regardless of the control applied by $\veh_i$ during $[t, \tau]$. We now make the following key observation:
\begin{observation} \label{obs1_case2_caseB}
Observation \ref{obs1_case1_caseA} implies that $\frs_{j}^{\mathcal{O}}(\tau_2, \tau) \subseteq \frs_{j}^{\mathcal{O}}(\tau_1, \tau)$ if $\tau > \tau_2 > \tau_1$. Therefore, the biggest obstacle is induced by $\veh_j$ at $\tau$ if $\tsa_j$ is as early as possible and it is sufficient to avoid this obstacle to ensure collision avoidnace with $\veh_j$ at time $\tau$. Given the separation and buffer regions between $\veh_i$ and $\veh_j$, we must have $\tsa_j - \tsa_i \geq \brd$. Hence, the biggest obstacle is induced by $\veh_j$ when $\tsa_j = \tsa_i + \brd$. 
\end{observation}
Intuitively, Observation \ref{obs1_case2_caseB} implies that the biggest obstacle is induced by $\veh_j$ when intruder forces $\veh_i$ to apply the avoidance maneuver and \textit{immediately} ``travels" through the buffer region between two vehicles to affect $\veh_j$ after a duration of $\brd$. Therefore, $\veh_i$ needs to avoid $\mathcal{K}(\tau)$ at time $\tau > t$, where 
\begin{equation} \label{eqn:obs_case2_caseB_help1}
\mathcal{K}(\tau) =  \bigcup_{\tsa_i \in [t-\iat, t], \tau - \tsa_i \leq \iat} \frs_{j}^{\mathcal{O}}(\tsa_i + \brd, \tau), \tau>t,
\end{equation}
where we have substituted $\tsa_j = \tsa_i + \brd$. In \eqref{eqn:obs_case2_caseB_help1} $\tsa = \tsa_i \in [t-\iat, t]$ (By Observation \ref{obs1_case2Obs}) and $\tau - \tsa_i \leq \iat$ because the intruder can appear for a maximum duration of $\iat$. \eqref{eqn:obs_case2_caseB_help1} can be equivalently written as:  
\begin{equation} \label{eqn:obs_case2_caseB_help1}
\begin{aligned}
\mathcal{K}(\tau) = & \bigcup_{\tsa_i \in [\tau-\iat, t]} \frs_{j}^{\mathcal{O}}(\tsa_i + \brd, \tau), t < \tau \leq t + \iat\\
\mathcal{K}(\tau) = & \frs_{j}^{\mathcal{O}}(\tau-\iat+\brd, \tau), t < \tau \leq t + \iat,
\end{aligned}
\end{equation}
where the second equality holds because of Observation \ref{obs1_case1_caseA}. The set of states that that $\veh_i$ needs to avoid at time $t$ is thus given by the following BRS:  
\begin{equation} \label{eq:ObsBRS_case2_caseB}
\begin{aligned}
\brs^{\text{B2}}_{i}(t, t+\iat) = & \{y: \exists \ctrl_i(\cdot) \in \cfset_i, \exists \dstb_i(\cdot) \in \dfset_i, \\
& \state_i(\cdot) \text{ satisfies \eqref{eq:dyn}}, \state_i(t) = y, \\
& \exists s \in [t+\brd, t+\iat], \state_i(s) \in \mathcal{K}(s) \}.
\end{aligned}
\end{equation}
The Hamiltonian $\ham^{\text{B2}}_{i}$ to compute $\brs^{\text{B2}}_{i}(t, t+\iat)$ is given by:
\begin{equation} \label{eqn:BRS_obsham_case2_caseB}
\ham^{\text{B2}}_{i}(\state_i, \costate) = \min_{\ctrl_i \in \cset_i} \min_{\dstb_i \in \dset_i} \costate \cdot f_i (\state_i, \ctrl_i, \dstb_i).
\end{equation}
Finally, the induced obstacle in this case is given as
\begin{equation} \label{eq:intruderObs_case2_caseB}
_2^B\ioset_i^j(t) = \brs^{\text{B2}}_{i}(t, t+\iat).
\end{equation}
\end{itemize}