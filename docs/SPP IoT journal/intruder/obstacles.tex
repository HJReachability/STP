% !TEX root = ../SPP_IoTjournal.tex
\subsection{Obstacle Computation} \label{sec:intruderObs}

In sections \ref{sec:sepRegion} and \ref{sec:buffRegion}, we computed a separation between any two vehicles, such that itruder can affect atmost $\nva$ vehicles during a duration of $\iat$. However, we need to make sure that while applying avoidance control a vehicle does not enter the danger zone of other vehicle. In this section, we compute the obstacles that reflect this possibility. In particular, we want to find the set of states that a lower priority vehicle $\veh_i$ needs to avoid to avoid entering in the danger zone of a higher priority vehicle $\veh_j$, $j < i$. To find such states, we consider the following five mutually exclusive and exhaustive cases:
\begin{enumerate}
\item Intruder does not affect $\veh_j$ or $\veh_i$ during their flight.
\item Intruder affects $\veh_j$, but not $\veh_i$.
\item Intruder affects $\veh_i$, but not $\veh_j$.
\item Intruder first affects $\veh_j$ and then $\veh_i$.
\item Intruder first affects $\veh_i$ and then $\veh_j$.
\end{enumerate}
We will compute the set of states that $\veh_i$ needs to avoid to avoid a collision with $\veh_j$ for each of the five cases. Let $^k\ioset_i^j(\cdot)$ denotes the set of obstacles corresponding to case $k$ above. 

\subsubsection{Case1} \label{sec:intruderObs_case1}
When the intruder does not affect any of the two vehicles, $\veh_i$ simply needs to avoid the set of base obstacles $\boset_j(t)$. Therfore, 
\begin{equation} \label{eq:intruderObs_case1}
\begin{aligned}
^1\ioset_i^j(t) & = \{\state_i: \exists y \in \pfrs_j(t), \|\pos_i - y\|_2 \le \rc \}\\
\pfrs_j(t) & = \{p_j: \exists \npos_j, (p_j, \npos_j) \in \boset_j(t) \}
\end{aligned}
\end{equation}


\subsubsection{Case2} \label{sec:intruderObs_case2}
To compute the obstacles that $\veh_i$ needs to avoid at time $t$ for the remaining four cases, it is sufficient to consider the scenarios where $\tsa \in [t-\iat, t]$. This is because if $\tsa < t - \iat$, then $\veh_i$ and/or $\veh_j$ will already be in the replanning phase at time $t$ (see assumption \ref{as:avoidOnce}) and hence the two vehicles cannot be in conflict at time $t$. On the other hand, if $\tsa > t$, then we need not account for the intruder as it has not appeared in the system yet. \SBnote{Actually, we compute obstacles at time $t'$ in a way such that their effect doesn't propagate to $t < t'$, but not sure if we need to menbtion this.}

The induced obstacles for Case2 at time $t$ are given by the states that $\veh_j$ can reach while avoiding the intruder, starting from some state in $\boset_j(\tsa), \tsa \in [t-\iat, t]$. These states can be obtained by computing a FRS from the base obstacles.
\begin{equation} \label{eq:ObsFRS_case2}
\begin{aligned}
\frs_{j}^{\mathcal{O}}(0, \tau) = & \{y: \exists \ctrl_j(\cdot) \in \cfset_j, \exists \dstb_j(\cdot) \in \dfset_j, \\
& \state_j(\cdot) \text{ satisfies \eqref{eq:dyn}}, \state_j(0) \in \boset_j(t-\tau), \\
& \state_j(\tau) = y\}.
\end{aligned}
\end{equation}
$\frs_{j}^{\mathcal{O}}(0, \tau)$ represents the set of all possible states that $\veh_j$ can reach after a duration of $\tau$ starting from inside $\boset_j(t-\tau)$. This FRS can be obtained by solving the HJ VI in \eqref{eq:HJIVI_FRS} with the following Hamiltonian:
\begin{equation}
\ham_{j}^{\mathcal{O}}(\state_j, \costate) = \max_{\ctrl_j \in \cset_j} \max_{\dstb_j \in \dset_j} \costate \cdot f_j (\state_j, \ctrl_j, \dstb_j).
\end{equation} 
Since $\tau \in [0, \iat]$, the induced obstacles in this case can be obtained as:
\begin{equation} \label{eq:intruderObs_case2} 
\begin{aligned}
^2\ioset_i^j(t) & = \{\state_i: \exists y \in \pfrs_j(t), \|\pos_i - y\|_2 \le \rc \}\\
\pfrs_j(t) & = \{p_j: \exists \npos_j, (p_j, \npos_j) \in \bigcup_{\tau \in [0, \iat]} \frs_{j}^{\mathcal{O}}(0, \tau) \}
\end{aligned}
\end{equation}

\begin{observation} \label{obs1_case2}
Note that by the definition of base obstacles, $\boset_j(t+\tau_2) \subset \frs_{j}^{\text{BO}}(0, \tau_2-\tau_1) ~\forall t, \tau_2 > \tau_1$, where $\frs_{j}^{\text{BO}}(0, \tau_2-\tau_1)$ denotes the FRS of $\boset_j(t+\tau_1)$ computed for a duration of $\tau_2-\tau_1$. Therefore, we have that $\frs_{j}^{\mathcal{O}}(0, \tau) \subset \frs_{j}^{\mathcal{O}}(0, \iat) ~\forall \tau \in [0, \iat)$. 
\end{observation}

Using observation \ref{obs1_case2}, $\pfrs_j(t)$ in \eqref{eq:intruderObs_case2} can be equivalently written as:
\begin{equation} \label{eq:intruderObs_case2_inter}
\pfrs_j(t) = \{p_j: \exists \npos_j, (p_j, \npos_j) \in \frs_{j}^{\mathcal{O}}(0, \iat) \}.
\end{equation}


\subsubsection{Case3} \label{sec:intruderObs_case3}
In this case, we need to ensure that $\veh_i$ doesn't collide with the obstacle set $\boset_j(\cdot)$ even when it is avoiding the intruder. In particular, we can compute a region around the obstacles $\boset_j(\cdot)$ such that for all disturbances, $\veh_i$ can avoid colliding with obstacles for $\iat$ seconds regardless of its avoidance control, if $\veh_i$ starts outside this region. 

Suppose that the intruder appears in the system at time $\tsa = t - \iat + \tau, \tau \in [0, \iat]$. In this case, we need to ensure that $\veh_i$ does not collide with the obstacle $\boset_j(t + \tau)$ at time $t + \tau$, regardless of its control $\ctrl_i(s)$ and disturbance $\dstb_i(s)$ for the time interval $s \in [\tsa, t + \tau]$. It is, therefore, sufficient to avoid the $\tau$-horizon BRS of $\boset_j(t + \tau)$ at time $t$. This argument applies for all $\tau \in [0, \iat]$. Mathematically,

\begin{equation} \label{eq:intruderObs_case3}
^3\ioset_i^j(t) = \bigcup_{\tau \in [0, \iat]} \brs^{\mathcal{G}}_{i}(0, \tau)
\end{equation}
where $\brs^{\mathcal{G}}_{i}(0, \tau)$ represents BRS of $\tilde{\boset}_j(t+\tau)$ computed backwards for $\tau$ seconds. Formally, 
\begin{equation}  \label{eq:ObsBRS_case3}
\begin{aligned}
\brs^{\mathcal{G}}_{i}(0, \tau) = & \{y: \exists \ctrl_i(\cdot) \in \cfset_i, \exists \dstb_i(\cdot) \in \dfset_i, \\
& \state_i(\cdot) \text{ satisfies \eqref{eq:dyn}}, \state_i(0) = y, \\
& \state_i(\tau) \in \tilde{\boset}_j(t+\tau)\} \\
\tilde{\boset}_j & = \{\state_j: \exists (y, h) \in \boset_j(t), \|\pos_j - y\|_2 \le \rc \}.
\end{aligned}
\end{equation}

The Hamiltonian $\ham^{\mathcal{G}}_{i}$ to compute $\brs^{\mathcal{G}}_{i}(\cdot)$ is given by:
\begin{equation} \label{eqn:BRS_obsham}
\ham^{\mathcal{G}}_{i}(\state_i, \costate) = \min_{\ctrl_i \in \cset_i} \min_{\dstb_i \in \dset_i} \costate \cdot f_i (\state_i, \ctrl_i, \dstb_i)
\end{equation}

\SBnote{I need to work on the next 2 cases, but will do that later. ALso, Case5 is clearer compared to Case4 so that might be a good starting point.}

\subsubsection{Case4} \label{sec:intruderObs_case4}
In this case, intruder $\veh_{\intr}$ appears at the boundary of set $\sep_j(\tsa)$ and after forcing $\veh_j$ to apply avoidance control for some duration $t_1 \in \mathbb{R}$, it reaches at the boundary of set $\sep_i(\tsa + t_1 + t_2)$ at time $(\tsa + t_1 + t_2) \in \mathbb{R}$. Once $\veh_j$ starts applying avoidance control at time $\tsa$, it might deviate from its planned trajectory. From the perspective of $\veh_i$, $\veh_j$ can apply any control for the next $\iat$ duration. Furthermore, $\veh_i$ itself might need to apply avoidnace maneuver starting at time $(\tsa + t_1 + t_2)$ and hence to compute obstacles in this case, we need to make sure that $\veh_i$ does not collide with $\veh_j$ when $\veh_j$ has deviated from its original trajectory \textit{and} $\veh_i$ is avoiding the intruder.   

To compute the induced obstacles in this case, we first make the following observations:
\begin{observation} \label{obs1_case4}
To compute the induced obstacle at time $t$ for this case, it is sufficient to consider the scenario where $t_1 + t_2 \leq t - \tsa \leq \iat$. If $t - \tsa < t_1 + t_2$, then at time $t$, $\veh_i$ does not need to avoid the intruder as it is not on the boundary of $\sep_i(t)$. Thus, $\veh_i$ only needs to make sure that it does not enter the danger zone of $\veh_j$, which has already been ensured in Case2. 
\end{observation}
\begin{observation} \label{obs2_case4}
By the design of buffer region in Section \ref{sec:buffRegion}, we have that $t_2 \geq \brd$. Combining this fact with observation \ref{obs1_case4} implies that $t- \tsa \geq \brd$.
\end{observation}

While avoiding the intruder, $\veh_j$ can reach any state starting from some state in $\boset_j(\tsa), \tsa \in [t-\iat, t]$. These states can be obtained by computing a FRS from the base obstacles as in \eqref{eq:ObsFRS_case2}. To avoid these unsafe states, $\veh_i$ needs to avoid all the states at time $t$ that can maneuver it to an unsafe state under some control. The set of states that $\veh_i$ needs to avoid can thus be given by the BRS of the unsafe states:
\begin{equation}  \label{eq:ObsBRS_case4}
\begin{aligned}
\brs^{\text{C4}}_{i}(t, \tsa + \iat) = & \{y: \exists \ctrl_i(\cdot) \in \cfset_i, \exists \dstb_i(\cdot) \in \dfset_i, \\
& \state_i(\cdot) \text{ satisfies \eqref{eq:dyn}}, \state_i(t) = y, \\
& \exists s \in [t, \tsa + \iat], \state_i(s) \in \tilde{\frs}_j^{\text{C4}}(\tsa, s)\},
\end{aligned}
\end{equation}
where
\begin{equation} \label{eq:ObsFRS_case4}
\begin{aligned}
\tilde{\frs}_{j}^{\text{C4}}(t', s) & = \{\state_j: \exists (y, h) \in \frs_j^{\text{C4}}(t', s), \|\pos_j - y\|_2 \le \rc \},\\
\frs_j^{\text{C4}}(t', s) & = \{y: \exists \ctrl_j(\cdot) \in \cfset_j, \exists \dstb_j(\cdot) \in \dfset_j, \\
& \state_j(\cdot) \text{ satisfies \eqref{eq:dyn}}, \state_j(t') \in \boset_j(t'), \\
& \state_j(s) = y\}.
\end{aligned}
\end{equation} 
The Hamiltonian $\ham^{\text{C4}}_{i}$ to compute $\brs^{\text{C4}}_{i}(\cdot)$ is given by:
\begin{equation} \label{eqn:BRS_obsham}
\ham^{\text{C4}}_{i}(\state_i, \costate) = \min_{\ctrl_i \in \cset_i} \min_{\dstb_i \in \dset_i} \costate \cdot f_i (\state_i, \ctrl_i, \dstb_i).
\end{equation}
Finally, the induced obstacle in this case can be obtained as:
\begin{equation} \label{eq:intruderObs_case4_alt1} 
^4\ioset_i^j(t) = \bigcup_{\tau \in [0, \iat - \brd]} \brs^{\text{C4}}_{i}(t, t + \tau),
\end{equation}
where range of $\tau$ is obtained via observation \ref{obs2_case4}.

Finally, by the definition of base obstacles, we have that $\frs_j^{\text{C4}}(t', s) \subseteq \frs_j^{\text{C4}}(t'', s), ~\forall t' > t''$. The union in \eqref{eq:intruderObs_case4_alt1} can thus be obtained by computing a single BRS:
\begin{equation} \label{eq:intruderObs_case4} 
\begin{aligned}
^4\ioset_i^j(t) & = \{y: \exists \ctrl_i(\cdot) \in \cfset_i, \exists \dstb_i(\cdot) \in \dfset_i, \\
& \state_i(\cdot) \text{ satisfies \eqref{eq:dyn}}, \state_i(t) = y, \\
& \exists s \in [t, t + \iat - \brd], \state_i(s) \in \tilde{\frs}_j^{\text{C4}}(s-\iat, s)\}.
\end{aligned}
\end{equation}

\subsubsection{Case5} \label{sec:intruderObs_case5}
In this case, intruder $\veh_{\intr}$ appears at the boundary of set $\sep_i(\tsa)$ and after forcing $\veh_i$ to apply avoidance control for some duration $t_1 \in \mathbb{R}$, it reaches at the boundary of set $\sep_j(\tsa + t_1 + t_2)$ at time $(\tsa + t_1 + t_2) \in \mathbb{R}$. Similar to Case4, once $\veh_i$ starts applying avoidance control at time $\tsa$, it might deviate from its planned trajectory. $\veh_j$ itself might need to apply avoidnace maneuver starting at time $(\tsa + t_1 + t_2)$ and hence to compute obstacles in this case, we need to make sure that $\veh_i$ does not collide with $\veh_j$ when $\veh_j$ is avoiding the intruder \textit{and} $\veh_i$ has deviated from its original trajectory. It is, therefore, sufficent that $\veh_i$ avoids the following BRS:
\begin{equation}  \label{eq:ObsBRS_case5}
\begin{aligned}
\brs^{\text{C5}}_{i}(t, \tsa + \iat) = & \{y: \exists \ctrl_i(\cdot) \in \cfset_i, \exists \dstb_i(\cdot) \in \dfset_i, \\
& \state_i(\cdot) \text{ satisfies \eqref{eq:dyn}}, \state_i(t) = y, \\
& \exists s \in [t_M, \tsa + \iat], \state_i(s) \in \tilde{\frs}_j^{\text{C5}}(\tsa+t_1+t_2, s)\},
\end{aligned}
\end{equation}
where
\begin{equation} \label{eq:ObsAltSet_case5}
\begin{aligned}
t_M & = max(\tsa+t_1+t_2, t),\\
\tilde{\frs}_{j}^{\text{C5}}(t', s) & = \{\state_j: \exists (y, h) \in \frs_j^{\text{C5}}(t', s), \|\pos_j - y\|_2 \le \rc \},\\
\frs_j^{\text{C5}}(t', s) & = \{y: \exists \ctrl_j(\cdot) \in \cfset_j, \exists \dstb_j(\cdot) \in \dfset_j, \\
& \state_j(\cdot) \text{ satisfies \eqref{eq:dyn}}, \state_j(t') \in \boset_j(t'), \\
& \state_j(s) = y\}.
\end{aligned}
\end{equation} 
The Hamiltonian $\ham^{\text{C5}}_{i}$ to compute $\brs^{\text{C5}}_{i}(\cdot)$ is given by:
\begin{equation} \label{eqn:BRS_obsham}
\ham^{\text{C5}}_{i}(\state_i, \costate) = \min_{\ctrl_i \in \cset_i} \min_{\dstb_i \in \dset_i} \costate \cdot f_i (\state_i, \ctrl_i, \dstb_i).
\end{equation}   

The above set can be interpreted as follows: $\frs_j^{\text{C5}}$ represents the states that $\veh_j$ might be in while avoiding the intruder. Thus to avoid these states, it is sufficient to avoid the BRS of $\frs_j^{\text{C5}}$ (after taking into account the danger zone $\dz_{ij}$) at time $t$, which is given by $\brs^{\text{C5}}_{i}$. Therefore, the induced obstacle in this case can be obtained by taking into account all possible $\tsa, t_1$ and $t_2$:
\begin{equation} \label{eq:intruderObs_case5_alt1} 
^5\ioset_i^j(t) = \bigcup_{\tsa \in [t-\iat, t], t_2 \geq \brd, t_1 + t_2 \in (0, \iat)} \brs^{\text{C5}}_{i}(t, \tsa + \iat),
\end{equation}
where range of $t_2$ is obtained via observation \ref{obs2_case4} and $t_1 + t_2 \leq \iat$ is implied due to assumption \ref{as:avoidOnce}.

Next we make the following observations:
\begin{observation} \label{obs1_case5}
By definition of the base obstacles, $\boset_j(t') \subseteq \frs_j^{\text{BO}}(t'', t')$, where $t'' < t'$ and $\frs_j^{\text{BO}}$ represents the FRS of the base obstacle $\boset_j(t'')$ computed for a duration of $t' - t''$. Therefore, for a given $\tsa$ and $t_2$, we have $\tilde{\frs}_j^{\text{C5}}(\tsa+t_1+t_2, s) \subseteq \tilde{\frs}_j^{\text{C5}}(\tsa+t_2, s) ~\forall t_1 >0$. Therefore, the biggest obstacles are induced for $\veh_i$ when $t_1 = 0$ and hence for the obstacle computation purposes we can assume that $t_1 = 0$. Similar argument can be applied for $t_2$. Therefore, we can assume that $t_2 = \brd$. 
\end{observation}

\begin{observation} \label{obs2_case5}
If $\tsa < t- 2\brd$, then $\tsa + t_1 + t_2 < t - \brd$ (see Observation \ref{obs1_case5}). That is $\veh_j$ applies the avoidance maneuver atleast for a duration of $\brd$ before time $t$. Thus, the intruder have enough time before it must disappear to travel through the buffer region and affect $veh_i$ again. But this scenario is already covered in Case4. So for Case5, we can safely assume that $\tsa \in [t-2*\brd, t]$. 
\end{observation}
Using the above two observations, the induced obstacle and $\brs^{\text{C5}}_{i}$ can be re-written as:
\begin{equation} \label{eq:intruderObs_case5} 
\begin{aligned}
^5\ioset_i^j(t) = & \bigcup_{\tau \in [t+\iat -2\brd, t+\iat]} \brs^{\text{C5}}_{i}(t, \tau),\\
\brs^{\text{C5}}_{i}(t, \tau) = & \{y: \exists \ctrl_i(\cdot) \in \cfset_i, \exists \dstb_i(\cdot) \in \dfset_i, \\
& \state_i(\cdot) \text{ satisfies \eqref{eq:dyn}}, \state_i(t) = y, \\
& \state_i(\tau) \in \tilde{\frs}_j^{\text{C5}}(\tau-\iat+\brd, \tau)\}.
\end{aligned}
\end{equation}

\SBnote{Start from here. Obstacle section is finished; write something about path planning next.}