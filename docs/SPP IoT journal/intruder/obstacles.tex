% !TEX root = ../SPP_IoTjournal.tex
\subsection{Obstacle Computation} \label{sec:intruderObs}

In sections \ref{sec:sepRegion} and \ref{sec:buffRegion}, we computed a separation between any two vehicles, such that itruder can affect atmost $\nva$ vehicles during a duration of $\iat$. However, we need to make sure that while applying avoidance control a vehicle does not enter the danger zone of other vehicle. In this section, we compute the obstacles that reflect this possibility. In particular, we want to find the set of states that a lower priority vehicle $\veh_i$ needs to avoid to avoid entering in the danger zone of a higher priority vehicle $\veh_j$, $j < i$. To find such states, we consider the following five mutually exclusive and exhaustive cases:
\begin{enumerate}
\item Intruder does not affect $\veh_j$ or $\veh_i$ during their flight.
\item Intruder affects $\veh_j$, but not $\veh_i$.
\item Intruder affects $\veh_i$, but not $\veh_j$.
\item Intruder first affects $\veh_j$ and then $\veh_i$.
\item Intruder first affects $\veh_i$ and then $\veh_j$.
\end{enumerate}
We will compute the set of states that $\veh_i$ needs to avoid to avoid a collision with $\veh_j$ for each of the five cases. Let $^k\ioset_i^j(\cdot)$ denotes the set of obstacles corresponding to case $k$ above. 

\subsubsection{Case1} \label{sec:intruderObs_case1}
When the intruder does not affect any of the two vehicles, $\veh_i$ simply needs to avoid the set of base obstacles $\boset_j(t)$. Therfore, $^1\ioset_i^j(t) = \boset_j(t)$.

\subsubsection{Case2} \label{sec:intruderObs_case2}
To compute the obstacles that $\veh_i$ needs to avoid at time $t$ for the remaining four cases, it is sufficient to consider the scenarios where $\tsa \in [t-\iat, t]$. This is because if $\tsa < t - \iat$, then $\veh_i$ and/or $\veh_j$ will already be in the replanning phase at time $t$ (see assumption \ref{as:avoidOnce}) and hence the two vehicles cannot be in conflict at time $t$. On the other hand, if $\tsa > t$, then we need not account for the intruder as it has not appeared in the system yet. \SBnote{Actually, we compute obstacles at time $t'$ in a way such that their effect doesn't propagate to $t < t'$, but not sure if we need to menbtion this.}

The induced obstacles for Case2 at time $t$ are given by the states that $\veh_j$ can reach while avoiding the intruder, starting from some state in $\boset_j(\tsa), \tsa \in [t-\iat, t]$. These states can be obtained by computing a FRS from the base obstacles.
\begin{equation} \label{eq:ObsFRS_case2}
\begin{aligned}
\frs_{j}^{\mathcal{O}}(0, \tau) = & \{y: \exists \ctrl_j(\cdot) \in \cfset_j, \exists \dstb_j(\cdot) \in \dfset_j, \\
& \state_j(\cdot) \text{ satisfies \eqref{eq:dyn}}, \state_j(0) \in \boset_j(t-\tau), \\
& \state_j(\tau) = y\}.
\end{aligned}
\end{equation}
$\frs_{j}^{\mathcal{O}}(0, \tau)$ represents the set of all possible states that $\veh_j$ can reach after a duration of $\tau$ starting from inside $\boset_j(t-\tau)$. This FRS can be obtained by solving the HJ VI in \eqref{eq:HJIVI_FRS} with the following Hamiltonian:
\begin{equation}
\ham_{j}^{\mathcal{O}}(\state_j, \costate) = \max_{\ctrl_j \in \cset_j} \max_{\dstb_j \in \dset_j} \costate \cdot f_j (\state_j, \ctrl_j, \dstb_j).
\end{equation} 
Since $\tau \in [0, \iat]$, the induced obstacles in this case can be obtained as:
\begin{equation} \label{eq:intruderObs_case2_inter}
^2\ioset_i^j(t) = \bigcup_{\tau \in [0, \iat]} \frs_{j}^{\mathcal{O}}(0, \tau).
\end{equation}

Note that by the definition of base obstacles, $\boset_j(t+\tau_2) \subset \frs_{j}^{\text{BO}}(0, \tau_2-\tau_1) ~\forall t, \tau_2 > \tau_1$, where $\frs_{j}^{\text{BO}}(0, \tau_2-\tau_1)$ denotes the FRS of $\boset_j(t+\tau_1)$ computed for a duration of $\tau_2-\tau_1$. Therefore, we have that $\frs_{j}^{\mathcal{O}}(0, \tau) \subset \frs_{j}^{\mathcal{O}}(0, \iat) ~\forall \tau \in [0, \iat)$. Thus, \eqref{eq:intruderObs_case2_inter} can be equivalently written as:
\begin{equation} \label{eq:intruderObs_case2}
^2\ioset_i^j(t) = \frs_{j}^{\mathcal{O}}(0, \iat).
\end{equation}

\subsubsection{Case3} \label{sec:intruderObs_case3}
In this case, we need to ensure that $\veh_i$ doesn't collide with the obstacle set $\boset_j(t)$ even when it is avoiding the intruder. In particular, we can compute a region around the obstacles $\boset_j(\cdot)$ such that for all disturbances, $\veh_i$ can avoid colliding with obstacles for $\iat$ seconds regardless of its avoidance control, if $\veh_i$ starts outside this region. To ensure that a vehicle does not collide with the obstacle $\boset_j(t_1 + t')$ at time $t = t_1 + t'$ starting at $t = t_1$, regardless of its control $\ctrl_i(s)$ and disturbance $\dstb_i(s)$ for the time interval $s \in [t_1, t_1 + t']$, it suffices to avoid the $t'$-horizon BRS of $\boset_i(t_1 + t')$. This argument applies for all $t' \in [0, \iat]$. Mathematically,

\begin{equation} \label{eq:intruderObs_case3}
^3\ioset_i^j(t) = \bigcup_{\tau \in [0, \iat]} \brs^{\mathcal{G}}_{i}(0, \tau)
\end{equation}
where $\brs^{\mathcal{G}}_{i}(0, \tau)$ represents BRS of $\boset_j(t+\tau)$ computed backwards for $\tau$ seconds. Formally, 
\begin{equation}  \label{eq:ObsBRS_case3}
\begin{aligned}
\brs^{\mathcal{G}}_{i}(0, \tau) = & \{y: \exists \ctrl_i(\cdot) \in \cfset_i, \exists \dstb_i(\cdot) \in \dfset_i, \\
& \state_i(\cdot) \text{ satisfies \eqref{eq:dyn}}, \state_i(0) = y, \\
& \exists s \in [0, \tau], \state_i(s) \in \boset_i(t)\}.
\end{aligned}
\end{equation}

The Hamiltonian $\ham^{\mathcal{G}}_{i}$ to compute $\brs^{\mathcal{G}}_{i}(\cdot)$ is given by:
\begin{equation} \label{eqn:BRS_obsham}
\ham^{\mathcal{G}}_{i}(\state_i, \costate) = \min_{\ctrl_i \in \cset_i} \min_{\dstb_i \in \dset_i} \costate \cdot f_i (\state_i, \ctrl_i, \dstb_i)
\end{equation}

\subsubsection{Case4} \label{sec:intruderObs_case4}

\SBnote{start from here}


