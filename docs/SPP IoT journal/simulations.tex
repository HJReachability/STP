% !TEX root = ../SPP_IoTjournal.tex
\section{Simulations \label{sec:simulations}}

Focus on the following aspects:
\begin{itemize}
\item Demonstration of theory (that it avoids collision w/ other vehicles and intruders, and we reach our destinations).
\item Scaling of SPP.
\item Provide some more intuition about the solution that emerge out of theory-- Space-time separation, buffer region between vehicles, trajectories, etc.
\item Reactivity of controller to the actual disturbance (Claire: be very detailed about explaining the setup of simulation)
\item Illustrate the structure that emerge out of SPP algorithm (Almost straight line path w/ different starting times)
\item Illustrate how this structure change with change in disturbance bounds (Straight line trajectories become curvy? )
\item For method-2 results, it may be helpful to include a figure which is showing the division of space among vehicles at some time (probably right before the intruder enters).
\end{itemize}

\SBnote{Also mention the technical details for the simulations, like RTT parameters, relative co-ordinate dynamics, rotation and translation of obstacles, union for obstacles, etc.}  

Let's pick speed $1.5 Km/min$ (or $2.5 Decametre/s$) and turnrate to be $120 rad/min$ ($2 rad/s$). Let's use grid to be $[0, 500] Dm$.

