% !TEX root = ../SPP_IoTjournal.tex
\subsection{SPP Without Disturbances and With Perfect Information\label{sec:basic}}
\SBnote{We need not present the basic SPP in this paper. Let's only present RTT.}
In this section, we give an overview of the basic SPP algorithm assuming that there is no disturbance affecting the vehicles, and that each vehicle knows the exact position of higher-priority vehicles. Although in practice, such assumptions do not hold, the basic SPP algorithm can still serve as a useful approximation in certain situations. In addition, the description of the basic SPP algorithm will introduce the notation needed for describing the subsequent, more realistic versions of SPP. We also show simulation results for the basic SPP algorithm. The majority of the content in this section is taken from \cite{Chen15c}.

Recall that the SPP vehicles $\veh_i, i=1,\ldots,N$, are each assigned a strict priority, with $\veh_j$ having a higher priority than $\veh_i$ if $j<i$. In the absence of disturbances, we can write the dynamics of the SPP vehicles as

\begin{equation}
\label{eq:dyn_no_dstb}
\begin{aligned}
\dot\state_i &= \fdyn_i(\state_i, \ctrl_i), t \le \sta_i \\
\ctrl_i &\in \cset_i, \qquad i = 1 \ldots, \N
\end{aligned}
\end{equation}

%\noindent with trajectories denoted by $\traj_i(s; \state^0_i, \ldt, \ctrl_(\cdot))$.

In SPP, each vehicle $\veh_i$ plans the path to its target set $\targetset_i$ while avoiding static obstacles $\soset_i$ and the obstacles $\ioset_i^j(t)$ induced by higher-priority vehicles $\veh_j, j<i$. Path planning is done sequentially starting from the first vehicle and proceeding in descending priority, $\veh_1, \veh_2, \ldots, \veh_{\N}$ so that each of the path planning problems can be done in the state space of only one vehicle. During its path planning process, $\veh_i$ ignores the presence of lower-priority vehicles $\veh_k, k>i$, and induces the obstacles $\ioset_k^i(t)$ for $\veh_k, k>i$.

From the perspective of $\veh_i$, each of the higher-priority vehicles $\veh_j, j<i$ induces a time-varying obstacle denoted $\ioset_i^j(t)$ that $\veh_i$ needs to avoid\footnote{Note that the index $k$ in $\ioset_k^i$ denotes vehicles with lower priority than $\veh_i$, and the index $j$ in $\ioset_i^j(t)$ denotes vehicles with higher priority than $\veh_i$.}. Therefore, each vehicle $\veh_i$ must plan its path to $\targetset_i$ while avoiding the union of all the induced obstacles as well as the static obstacles. Let $\obsset_i(t)$ be the union of all the obstacles that $\veh_i$ must avoid on its way to $\targetset_i$:

\begin{equation}
\label{eq:obsseti}
\obsset_i(t)  = \soset_i \cup \bigcup_{j=1}^{i-1} \ioset_i^j(t)
\end{equation}

With full position information of higher priority vehicles, the obstacle induced for $\veh_i$ by $\veh_j$ is simply

\begin{equation}
\label{eq:ioset}
\ioset_i^j(t) = \{\state_i: \|\pos_i - \pos_j(t)\|_2 \le \rc \}
\end{equation}

Each higher priority vehicle $\veh_j$ plans its path while ignoring $\veh_i$. Since path planning is done sequentially in descending order or priority, the vehicles $\veh_j, j<i$ would have planned their paths before $\veh_i$ does. Thus, in the absence of disturbances, $\pos_j(t)$ is \textit{a priori} known, and therefore $\ioset_i^j(t), j<i$ are known, deterministic moving obstacles, which means that $\obsset_i(t)$ is also known and deterministic. Therefore, the path planning problem for $\veh_i$ can be solved by first computing the BRS $\brs_i^\text{basic}(t, \sta_i)$, defined as follows:

\begin{equation}
\label{eq:BRS_basic}
\begin{aligned}
\brs_i^\text{basic}(t, \sta_i) = & \{y: \exists \ctrl_i(\cdot) \in \cfset_i, \state_i(\cdot) \text{ satisfies \eqref{eq:dyn_no_dstb}}, \\
& \forall s \in [t, \sta_i],\state_i(s) \notin \obsset_i(s), \\
& \exists s \in [t, \sta_i], \state_i(s) \in \targetset_i, \state_i(t) = y\}
\end{aligned}
\end{equation}

The BRS $\brs(t, \sta_i)$ can be obtained by solving \eqref{eq:HJIVI_BRS} with $\targetset = \targetset_i$, $\obsset(t) = \obsset_i(t)$, and the Hamiltonian 

\begin{equation}
\label{eq:basicham}
\ham_i^\text{basic}(\state_i, \costate) = \min_{\ctrl_i\in\cset_i} \costate \cdot \fdyn_i(\state_i, \ctrl_i)
\end{equation}

The optimal control for reaching $\targetset_i$ while avoiding $\obsset_i(t)$ is then given by

\begin{equation}
\label{eq:basicOptCtrl}
\ctrl_i^\text{basic}(t, \state_i) = \arg \min_{\ctrl_i\in\cset_i} \costate \cdot \fdyn_i(\state_i, \ctrl_i)
\end{equation}

\noindent from which the trajectory $\state_i(\cdot)$ can be computed by integrating the system dynamics, which in this case are given by \eqref{eq:dyn_no_dstb}. In addition, the latest departure time $\ldt_i$ can be obtained from the BRS $\brs(t, \sta_i)$ as $\ldt_i = \arg \sup_t \{\state_i^0 \in \brs(t, \sta_i)\}$. In summary, the basic SPP algorithm is given as follows:

\SBnote{We probably don't need to present this algorithm for this paper.}
\begin{alg}
\label{alg:basic}
\textbf{Basic SPP algorithm}: Suppose we are given initial conditions $\state_i^0$, vehicle dynamics \eqref{eq:dyn_no_dstb}, target sets $\targetset_i$, and static obstacles $\soset_i, i = 1\ldots, \N$. For each $i$ in ascending order starting from $i=1$ (which corresponds to descending order of priority),
\begin{enumerate}
\item determine the total obstacle set $\obsset_i(t)$, given in \eqref{eq:obsseti}. In the case $i=1$, $\obsset_i(t) = \soset_i ~ \forall t$;
\item compute the BRS $\brs_i^\text{basic}(t, \sta_i)$ defined in \eqref{eq:BRS_basic}. The latest departure time $\ldt_i$ is then given by $\arg \sup_t \{\state^0_i \in \brs_i^\text{basic}(t, \sta_i)\}$;
\item determine the trajectory $\state_i(\cdot)$ using vehicle dynamics \eqref{eq:dyn_no_dstb}, with the optimal control  $\ctrl_i^\text{basic}(\cdot)$ given by \eqref{eq:basicOptCtrl};
\item given $\state_i(\cdot)$, compute the induced obstacles $\ioset_k^i(t)$ for each $k>i$. In the absence of disturbances, $\ioset_k^i(t)$ is given by \eqref{eq:ioset}.
\end{enumerate}
\end{alg}