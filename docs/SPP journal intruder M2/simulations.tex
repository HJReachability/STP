% !TEX root = ./SPP_IoTjournal.tex
\section{Simulations \label{sec:simulations}}
We now illustrate the proposed algorithm using a fifty-vehicle example. 

\subsection{Setup \label{sec:simSetup}}
Our goal is to simulate a scenario where UAVs are flying through an urban environment. This setup can be representative of many UAV applications, such as package delivery, aerial surveillance, etc. For this purpose, we grid San Francisco (SF) city in California, US and use it as our state space, as shown in Figure \ref{fig:sf_setup}. 
\begin{figure}[H]
  \centering
  \includegraphics[width=\columnwidth]{"figs/sf_setup"}
  \caption{Simulation setup. A $25 km^2$ area of San Francisco city is used as the state-space for vehicles. SPP vehicles originate from the Blue star and go to one of the four destinations, denoted by circles. Tall buildings in the downtown area are used as static obstacles, represented by the black contours.}
  \label{fig:sf_setup}
\end{figure}
Each box in Figure \ref{fig:sf_setup} represents a $500m \times 500m$ area of SF. The origin point for the vehicles is denoted by the Blue star. Four different areas in the city are chosen as the destinations for the vehicles. Mathematically, the target sets $\targetset_i$ of the vehicles are circles of radius $r$ in the position space, i.e. each vehicle is trying to reach some desired set of positions. In terms of the state space $\state_i$, the target sets are defined as
\begin{equation}
\label{eq:target_sim}
\targetset_i = \{\state_i: \|\pos_i - c_i\|_2 \le r\}
\end{equation}
\noindent where $c_i$ are centers of the target circles. In this simulation, we use $r = 100m$. The four targets are represented by four circles in Figure \ref{fig:sf_setup}. The destination of each vehicle is chosen randomly from these four destinations. Finally, tall buildings in downtown San Francisco are used as static obstacles for the SPP vehicles, denoted by black contours in Figure \ref{fig:sf_setup}.

For this simulation, we use the following dynamics for each vehicle:
\begin{equation}
\label{eq:dyn_i}
\begin{aligned}
\dot{\pos}_{x,i} &= v_i \cos \theta_i + d_{x,i} \\
\dot{\pos}_{y,i} &= v_i \sin \theta_i + d_{y,i}\\
\dot{\theta}_i &= \omega_i, \\
\underline{v} \le v_i \le \bar{v}, & ~|\omega_i| \le \bar{\omega}, ~\|(d_{x,i}, d_{y,i}) \|_2 \le d_{r},
\end{aligned}
\end{equation}
\noindent where $\state_i = (\pos_{x,i}, \pos_{y,i}, \theta_i)$ is the state of vehicle $\veh_i$, $\pos_i = (\pos_{x,i}, \pos_{y,i})$ is the position, $\theta_i$ is the heading, and $d = (d_{x,i}, d_{y,i})$ represents $\veh_i$'s disturbances, for example wind, that affect its position evolution. The control of $\veh_i$ is $u_i = (v_i, \omega_i)$, where $v_i$ is the speed of $\veh_i$ and $\omega_i$ is the turn rate; both controls have a lower and upper bound. To make our simulations as close as possible to real scenarios, we choose velocity and turn-rate bounds as $\underline{v} = 0m/s, \bar{v} = 25m/s, \bar\omega = 2 rad/s$, aligned with the modern UAV specs \cite{UAVspecs1, UAVspecs2}. The disturbance bound is chosen as $d_{r} = 11 m/s$, which corresponds to \textit{strong winds} on Beaufort wind force scale \cite{Windscale}. Note that we have used same dynamics and input bounds across all vehicles for clarity of illustration; however, our method can easily handle more general systems of the form in which the vehicles have different control bounds and dynamics.

The goal of the vehicles is to reach their destinations while avoiding a collision with the other vehicles or the static obstacles. The vehicles also need to account for the possibility of the presence of an intruder for a maximum duration of $\iat = 10s$, whose dynamics are given by \eqref{eq:dyn_i}. The joint state space of this fifty-vehicle system is 150-dimensional (150D), making the joint path planning and collision avoidance problem intractable for direct analysis using HJ reachability. Therefore, we assign a priority order to vehicles and solve the path planning problem sequentially. For this simulation, we assign a random priority order to fifty vehicles and use the algorithm proposed in Section \ref{sec:intruder} to compute a separation between SPP vehicles so that they do not collide with each other or the intruder. 

\subsection{Results \label{sec:simResults}}
In this section, we present the simulation results for $\nva = 3$; occassionally, we also compare the results for different $\nva$s to highlight some key points about the proposed algorithm. As per Algorithm \ref{alg:intruder}, we begin with computing the avoid region $\brs^{\text{A}}_{i}(0, \iat)$. To compute the avoid region, relative dynamics between $\veh_i$ and $\veh_{\intr}$ are required. Given dynamics in \eqref{eq:dyn_i}, the relative dynamics are given by \cite{Mitchell05}:
\begin{equation}
\label{eq:reldyn_i}
\begin{aligned}
\dot{\pos}_{x, \intr, i} &= v_{\intr} \cos \theta_{\intr, i} - v_i + \omega_i {\pos}_{y, \intr, i} + d_{x,i} \\
\dot{\pos}_{y, \intr, i} &= v_i \sin \theta_{\intr, i} - \omega_i {\pos}_{x, \intr, i} + d_{y,i}\\
\dot{\theta}_{\intr, i} &= \omega_{\intr} - \omega_i,
\end{aligned}
\end{equation}
\SBnote{Confirm disturbance in relative dynamics with Mo.}     
where $\state_{\intr, i} = (\pos_{x, \intr, i}, \pos_{y, \intr, i}, \theta_{\intr, i})$ is the relative state between $\veh_{\intr}$ and $\veh_i$. Given relative dynamics, the avoid region can be computed using \eqref{eqn:avoidBRS}. For all the BRS and FRS computations in this simulation, we use Level Set Toolbox \cite{Mitchell07b}. Also, since the vehicle dynamics' are same across all vehicles, we will omit the vehicle index from sets wherever applicable. The avoid region $\brs^{\text{A}}(0, \iat)$ for SPP vehicles is shown in Figure \ref{fig:MaxMin}.
\begin{figure}[H]
  \centering
  \includegraphics[width=\columnwidth]{"figs/bufferRegion_steps"}
  \caption{Base obstacle $\boset(t)$ , Avoid region $\brs^{\text{A}}(0, \iat)$, Separation region $\sep(t)$ and Relative buffer region $\brs^{\text{B}}(0, \brd)$ for vehicles. The three axes represent three states of the vehicles.}
  \label{fig:MaxMin}
\end{figure}
As long as $\veh_{\intr}$ starts outside the avoid region, $\veh_i$ is guarnateed to avoid the intruder for a duration of $\iat$. Given $\brs^{\text{A}}(0, \iat)$, we can compute the minimum required detection range $\dsen$ given by \eqref{eqn:sen_distance}, which turns out to be XYm \SBnote{Confirm detection range with Mo} in this case. So as long as the vehicles can detect the intruder within XYm \SBnote{Confirm detection range with Mo}, the proposed algorithm guarantees collision avoidance with the intruder as well as a safe transit to their respective destinations.   

Next, we compute the separation and buffer regions between vehicles. For the computation of base obstacles, we use RTT method \cite{Bansal2017}. In RTT method, a nominal trajectory is declared by the higher priority vehicles, which is then guaranteed to be tracked with some known error bound in the presence of disturbances. The base obstacles are thus given by a ``bubble" around the nominal trajectory. For further details of RTT method, we refer the interested readers to Section 4C in \cite{Bansal2017}. The position tracking error bound obtained for the strong wind conditions is $35m$ \SBnote{Confirm the tracking radius with Mo and see if it is consistent with the figure}. The overall base obstacle $\boset$ around the nominal tarjectory point $(0, 0, 0)$ is shown in Figure \ref{fig:MaxMin}. Thus, the base obstacles induced by a higher priority vehicle are given by this set augmented on the nominal trajectory, the trajectory that a vehicle will follow if the intruder never appears in the system, and is obtained by executing the control policy ${\ctrl^{\text{PP}}_{i}}(\cdot)$ in \eqref{eqn:PPPolicy}.

Given $\boset$ and $\brs^{\text{A}}(0, \iat)$, we compute the separation region $\sep$ as defined in \eqref{eqn:sepRegion_case1}. Relative buffer region $\brs^{\text{B}}(0, \brd)$, defined in \eqref{eqn:buffBRS_case1}, is computed next. The results are shown in Figure \ref{fig:MaxMin}. Note that since $\nva = 3$, $\brd = 10/3$. Finally, we compute the buffer region as defined in \eqref{eqn:buffRegion_case1}. The resultant buffer region is shown in Blue in Figure \ref{fig:buffRegions}. Thus, if $\veh_j$ is inside the base obstacle set shown in Figure \ref{fig:MaxMin} and $\veh_i$ is outside the Blue region in Figure \ref{fig:buffRegions}, we can ensure that the intruder will have to spend a duration of atleast $\brd$ to go from the boundary of the avoid region of $\veh_j$ to the boundary of the avoid region of $\veh_i$. 
\begin{figure}[H]
  \centering
  \includegraphics[width=\columnwidth]{"figs/bufferRegions_3D"}
  \caption{Buffer regions for different $\nva$ (best visualized with colors). As $\nva$ decreases, a larger buffer is required between vehicles to ensure that the intruder spends more time while traveling through this buffer region so that it forces fewer vehicles to apply an avoidnace maneuver.}
  \label{fig:buffRegions}
\end{figure}
We also computed the buffer regions for $\nva = 2$ and $\nva = 4$. The results are shown in Figure \ref{fig:buffRegions}. Top-down views of these 3D sets are shown in Figure \ref{fig:buffRegions_td}. As evident from the figures, a bigger buffer region is required between vehicles when $\nva$ is smaller. Intuitively, when $\nva$ is smaller, a larger buffer is required to ensure that the intruder spends more time ``traveling" through this buffer region so that it can affect fewer vehicles.             
\begin{figure}[H]
  \centering
  \includegraphics[width=\columnwidth]{"figs/bufferRegions_topdown"}
  \caption{Top-down view of the buffer regions for different $\nva$ shown in Figure {fig:buffRegions} (best visualized with colors). As $\nva$ decreases, $\brd$ increase and a larger buffer is required between vehicles.}
  \label{fig:buffRegions_td}
\end{figure}

Similarly, we sequentially computed the obstacles induced by the higher priority vehicles for each vehicle, i.e. $\obsset(\cdot)$, and the corresponding nominal trajectories, obtained by executing the control policy $\ctrl^{\text{PP}}(\cdot)$ defined in \eqref{eqn:PPPolicy}. The nominal trajectory thus corresponds to the trajectory that a vehicle will follow if the intruder does not appear in the system. The nominal trajectories and the overall obstacles for different vehicles at time $t = -2000$ \SBnote{Confirm time step with Mo and use SI units here.} are shown in Figure \ref{fig:trajObsSim} for the case $\nva = 3$. The nominal trajectories (solid lines) are well separated from each other to ensure safe transition even during a worst-case intruder ``attack". The dashed circle around a vehicle denote the overall obstacle induced by that vehicle for the lower priority vehicles. As expected, all vehicles are outside each other's induced obstacles, as well as static obstacles. %Accounting for this worst-case behavior makes our reachability analysis conservative and this limitation is discussed more Section \ref{sec:discuss}.
Note that in the absence of an intruder the vehicles transit successfully to their destinations with control policy $\ctrl^{\text{PP}}(\cdot)$, but they can deviate from the shown nominal trajectories if an intruder appears in the system.
\begin{figure}[H]
  \centering
  \includegraphics[width=\columnwidth]{"figs/nomTraj"}
  \caption{Nominal trajectories and induced obstacles by different vehicles. The nominal trajectories (solid lines) are well separated from each other to ensure safe transition even in the presnece of an intruder.}
  \label{fig:trajObsSim}
\end{figure}

Finally, in Figure \ref{fig:trajComparison}, we show that how the avoidance control can cause a deviation from the nominal trajectory. As evident from the figure, if vehicle doesn't apply the avoidnace maneuver and continue to track the nominal trajectory in the presence of an intruder, this might lead to a collision with the intruder. On the other hand, if the vehicle applies the avoidance policy $\ctrl^{\text{A}}(\cdot)$, it can successfully avoid a collision with the intruder. \SBnote{Add the figure here.} 

Under the proposed algorithm, the intruder will affect maximum number of vehicles ($\nva$ vehicles), when it appears at the boundary of the avoid region of a vehicle, immediately ``travels" through the buffer region between vehicles and reach the boundary of the avoid region of another vehicle at $\tsa + \brd$ and then the boundary of the avoid region of another vehicle at $\tsa + 2\brd$ and so on. This strategy will make sure that the intruder affects $\nva$ vehicles during a duration of $\iat$. However, the relative buffer region between vehicles is computed under the assumption that both the SPP vehicle and the intruder are trying to collide with each other, which is not necessarily true as a vehicle will be applying the control policy $\ctrl^{\text{PP}}(\cdot)$ unless the intruder forces it to apply an avoidance maneuver, so it is very likely that the intruder will affect less than $\nva$ vehicles even with this best strategy to affect maximum vehicles. This is also evident from Figures \ref{fig:bestStrategy1} and \ref{fig:bestStrategy1}. 
\SBnote{To-dos:
\begin{itemize}
\item Add the explanation of what is happening in figures once we have the figures. Make time points and vehicle numbers precise
\item Explain which vehicles need to replan their trajectories once the intruder disappears in each case. Explicitly define the set. 
\end{itemize}
}  

In Case-1, the intruder forces all 3 vehicles to apply an avoidance maneuver so we need to replan the trajectories of $3 (= \nva)$ vehicles once the intruder disappears. However, a vehicle applies control policy $\ctrl^{\text{PP}}(\cdot)$ while the intruder is traveling through the buffer region, which may not necessarily correspond to the policy that the vehicle will use to \textit{deliberately} collide with the intruder, unlike assumed during the computation of the set $\brs^{\text{B}}(0, \brd)$. Thus, at $\tsa + \brd$, the intruder may not reach the boundary of the avoid region of $\veh_{XY}$.         

\subsection{Discussion \label{sec:discuss}}
The shown simulations illustrate the effectiveness of reachability in ensuring that the SPP vehicles safely reach their respective destinations even in the presence of an intruder. However, they also highlight some of the conservatism that is in-built in the reachability analysis due to the worst case analysis. For example, in the proposed algorithm, we assume the worst-case disturbances and intruder behavior while computing the buffer region and induced obstacles, which results in a large separation between vehicles and hence a lower vehicle density overall, as evident from Figure {fig:trajObsSim}. Similarly, while computing the relative buffer region, we assumed that a vehicle is \textit{deliberatley} trying to collide with the intruder so we once again consider the worst case scenario, as the vehicle will only be applying the nominal control strategy $\ctrl^{\text{PP}}(\cdot)$, which may not be same as the worst-case control strategy. Therefore, even though this worst-case analysis is essential to guarantee safety regardless the actions of SPP vehicles, the intruder and disturbances, a probabilistic safety analysis, that can overocme some of this conservatism, might be more suitable in practical applications.