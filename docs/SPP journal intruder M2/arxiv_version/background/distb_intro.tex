% !TEX root = SPP2.tex
\subsection{STP With Disturbances and Without Intruder\label{sec:distb}}
Disturbances and incomplete information significantly complicate the STP scheme. The main difference is that the vehicle dynamics satisfy \eqref{eq:dyn} as opposed to \eqref{eq:dyn_no_dstb}. Committing to exact trajectories is therefore no longer possible, since the disturbance $d_i(\cdot)$ is \textit{a priori} unknown. Thus, the induced obstacles $\ioset_i^j(t)$ are no longer just the danger zones centered around positions, unlike in \eqref{eq:ioset_basic}. In particular, a lower-priority vehicle needs to account for all possible states that the higher-priority vehicles could be in. To do this, the lower-priority vehicle needs to have some knowledge about the control policy used by each higher-priority vehicle. Three different methods are presented in \cite{Bansal2017} to address the above issues. The methods differ in terms of control policy information that is known to a lower-priority vehicle.
\begin{itemize}
\item \textbf{Centralized control}: A specific control strategy is enforced upon a vehicle; this can be achieved, for example, by some central agent such as an air traffic controller. 
\item \textbf{Least restrictive control}: A vehicle is required to arrive at its targets on time, but has no other restrictions on its control policy. When the control policy of a vehicle is unknown, the least restrictive control can be safely assumed by lower-priority vehicles.
\item \textbf{Robust trajectory tracking}: A vehicle declares a nominal trajectory which can be robustly tracked under disturbances.
\end{itemize}
In each case, a vehicle $\veh_i$ can compute all possible states $\ioset_i^j(t)$ that a higher-priority vehicle $\veh_j$ can be in based on the control strategy information known to the lower priority vehicle. A collision avoidance between $\veh_i$ and $\veh_j$ is thus ensured. We refer to the obstacle $\ioset_i^j(t)$, induced in the presence of disturbances but in the absence of intruders, as \textit{base obstacle} and denote it as $\boset_j(t)$ from here on. Further details of each algorithm are presented in \cite{Bansal2017}.
