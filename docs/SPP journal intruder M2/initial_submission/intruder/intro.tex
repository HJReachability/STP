% !TEX root = ../SPP_IoTjournal.tex
\section{Response to Intruders \label{sec:intruder}}
In this section, we propose a method to allow vehicles to avoid an intruder while maintaining the STP structure.
%
%In general, the effect of an intruder on the vehicles in structured flight can be entirely unpredictable, since the intruder in principle could be adversarial in nature, and the number of intruders could be arbitrary. Therefore, we make the following two assumptions: %for our analysis to produce reasonable results, two assumptions about the intruders must be made.
%
%\begin{assumption}
%\label{as:avoidOnce}
%At most one intruder (denoted as $\veh_I$ here on) affects the STP vehicles at any given time. The intruder exits the altitude level affecting the STP vehicles after a duration of $\iat$. 
%\end{assumption}
%
%Let the time at which intruder appears in the system be $\tsa$ and the time at which it disappears be $\tea$. Assumption \ref{as:avoidOnce} implies that $\tea \leq \tsa + \iat$. Thus, any vehicle $\veh_i$ would need to avoid the intruder $\veh_{\intr}$ for a maximum duration of $\iat$. %This assumption can be valid in situations where intruders are rare, and that some fail-safe or enforcement mechanism exists to force the intruder out of the altitude level affecting the STP vehicles. 
%Note that we do not pose any restriction on $\tsa$; however, we assume that once the intruder appears, it stays for a maximum duration of $\iat$.
%%in addition, after avoiding the intruder, Qi can safely assume that it would not need to avoid another intruder
%
%\begin{assumption}
%\label{as:dynKnown}
%The dynamics of the intruder are known and given by $\dot\state_\intr = f_\intr(\state_\intr, \ctrl_\intr, \dstb_\intr)$.
%\end{assumption}
%
%Assumption \ref{as:dynKnown} is required for HJ reachability analysis. In situations where the dynamics of the intruder are not known exactly, a conservative model of the intruder may be used instead. We also denote the initial state of the intruder as $\state_{\intr}^0.$ Note that $\state_{\intr}^0$ is unknown.
%
Our goal is to design a control policy for each vehicle that ensures separation with the intruder and other STP vehicles, and ensures a successful transit to the destination. %However, depending on the initial state of the intruder and its control policy, a vehicle may arrive at different states after avoiding the intruder. Therefore, a control policy that ensures a successful transit to the destination needs to account for all such possible states, which is a path planning problem with multiple initial states and a single destination, and is hard to solve in general. Thus, we divide the intruder avoidance problem into two sub-problems: (i) we first design a control policy that ensures a successful transit to the destination if no intruder appears and that successfully avoid the intruder, if it does. (ii) after the intruder disappears at $\tea$, we replan the trajectories of the affected vehicles. Following the same theme and assumptions, authors in \cite{chen2016robust} present an algorithm to avoid an intruder in STP formulation; however, once the intruder disappears, the algorithm might need to replan the trajectories for all STP vehicles. Since the replanning is done in real-time, it should be fast and scalable with the number of STP vehicles, rendering the method in \cite{chen2016robust} unsuitable for practical implementation.  

As discussed in Section \ref{sec:formulation}, depending on the initial state of the intruder and its control policy, a vehicle may arrive at different states after avoiding the intruder. To make sure that the vehicle still reaches its destination, a replanning of vehicle's trajectory is required. Since the replanning must be done in real-time, we also need to ensure that only a small number of vehicles require replanning. In this work, a novel intruder avoidance algorithm is proposed, which will need to replan trajectories only for a \textit{small fixed} number of vehicles, irrespective of the total number of STP vehicles. Moreover, this number is a design parameter, which can be chosen based on the resources available during run time. 

Let $\nva$ denote the maximum number of vehicles that we can replan the trajectories for in real-time. Also, let $\brd = \frac{\iat}{\nva}$. We divide our algorithm in two parts: the planning phase and the replanning phase. In the planning phase, our goal is to divide the flight space of vehicles such that at any given time, any two vehicles are far enough from each other so that an intruder needs at least a duration of $\brd$ to travel from the vicinity of one vehicle to that of another. Since the intruder is present for a total duration of $\iat$, this division ensures that it can only affect at most $\nva$ vehicles despite its best efforts. In particular, we compute a separation region for each vehicle such that the vehicle needs to account for the intruder if and only if the intruder is inside this separation region. We then compute a buffer region between the separation regions of any two vehicles such that the intruder requires atleast a duration of $\brd$ to travel through this region. A high-level overview of the planning phase is presented in Algorithm \ref{alg:basic_idea}. The planning phase ensures that after the intruder disappears, \textit{at most $\nva$} vehicles have to replan their trajectories. In the replanning phase, we re-plan the trajectories of affected vehicles so that they reach their destinations safely. %which can be efficiently done in real-time if $\nva$ is low enough. %Thus the proposed method guarantees that \textit{at most $k$} vehicles are affected by the presence of intruder, regardless of the number of STP vehicles, and hence the replanning can be efficiently done in real-time. 

Note that our theory assumes worst-case scenarios in terms of the behavior of the intruder, the effect of disturbances, and the planned trajectories of each STP vehicle. This way, we are able to guarantee safety and goal satisfaction of all vehicles in all possible scenarios given the bounds on intruder dynamics and disturbances. To achieve denser operation of STP vehicles, known information about the intruder, disturbances, and specifies of STP vehicle trajectories may be incorporated; however, these considerations are out of the scope of this paper.

The rest of the section is organized as follows. In Sections \ref{sec:intruder_avoid}, we discuss the intruder avoidance control that a vehicle needs to apply within the separation region. In Sections \ref{sec:case1} and \ref{sec:case2_maintext}, we compute the separation and buffer regions for vehicles. 
%a division of state space is computed such that at most $\nva$ vehicles need to apply the avoidance maneuver computed in Section \ref{sec:intruder_avoid}, regardless of the initial state of the intruder. However, we still need to ensure that vehicles do not collide with each other while avoiding the intruder. The induced obstacles that reflect this possibility are computed in Section \ref{sec:intruderObs}. 
Trajectory planning that maintains the buffer region between every pair of vehicles is discussed in Section \ref{sec:path_planning}. Finally, the replanning of the trajectories of the affected vehicles is discussed in Section \ref{sec:replan}. 
%The planning and the replanning phases are summarized in Algorithms \ref{alg:intruder_plan} and \ref{alg:intruder_replan} respectively. The notations introduced in this section are summarized in Table \ref{table:notation}.