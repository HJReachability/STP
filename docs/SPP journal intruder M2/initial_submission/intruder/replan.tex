% !TEX root = ../SPP_IoTjournal.tex
\subsection{Replanning after intruder avoidance} \label{sec:replan}
As discussed in Section \ref{sec:path_planning}, the intruder can force some STP vehicles to deviate from their planned nominal trajectory; therefore, goal satisfaction is no longer guaranteed once a vehicle is forced to apply an avoidance maneuver. Therefore, we have to replan the trajectories of these vehicles once $\veh_{\intr}$ disappears. The set of all vehicles $\veh_i$ for whom replanning is required, $\rvs$, can be obtained by checking if a vehicle $\veh_i$ applied any avoidance control during $[\tsa, \tsa+\iat]$, i.e.,
\begin{equation} \label{eq:RVS}
\rvs = \{\veh_i: \tsa_i < \infty, i \in \{1, \ldots, \N \} \}. 
\end{equation}  

Note that due to the presence of separation and buffer regions, at most $\nva$ vehicles can be affected by $\veh_{\intr}$, i.e., $|\rvs| \leq \nva$. Goal satisfaction controllers which ensure that these vehicles reach their destinations can be obtained by solving a new STP problem, where the starting states of the vehicles are now given by the states they end up in, denoted $\tilde{\state}_i^0$, after avoiding the intruder. Note that we can pick $\nva$ beforehand and design buffer regions accordingly. Thus, by picking compatible $\nva$ based on the available computation resources during run-time, we can ensure that this replanning can be done in real time. Moreover, flexible trajectory-planning algorithms such as FaSTrack \cite{Herbert2017} can be used that can perform replanning efficiently in real-time.   

Let the optimal control policy corresponding to this liveness controller be denoted ${\ctrl^{\text{L}}_{i}}(t, \state_i)$. The overall control policy that ensures intruder avoidance, collision avoidance with other vehicles, and successful transition to the destination for vehicles in $\rvs$ is given by:

\begin{equation} \label{eqn:full_controller}
\ctrl_i^{\text{RP}}(t) = 
\left \{ 
\begin{array}{ll}
{\ctrl^*_{i}}(t, \state_i) & t \leq \tsa + \iat\\
{\ctrl^{\text{L}}_{i}}(t, \state_i) & t > \tsa + \iat
\end{array}
\right.
\end{equation}

Note that in order to re-plan using a STP method, we need to determine feasible $\sta_i$ for all vehicles. This can be done by computing an FRS:
\begin{equation} \label{eq:re-planFRS}
\begin{aligned} 
\frs_i^{\text{RP}}(\tea, t) = & \{y \in \R^{n_i}: \exists \ctrl_i(\cdot) \in \cfset_i, \forall \dstb_i(\cdot) \in \dfset_i, \\
& \state_i(\cdot) \text{ satisfies \eqref{eq:dyn}}, \state_i(\tea) = \tilde{\state}_i^0, \\
& \state_i(t) = y, \forall s \in [\tea, t], \state_i(s) \notin \obsset_i^{\text{RP}}(s) \},
\end{aligned}
\end{equation}
\noindent where $\tilde{\state}_i^0$ represents the state of $\veh_i$ at $t = \tsa+\iat$; $\obsset_i^{\text{RP}}(\cdot)$ takes into account the fact that $\veh_i$ now needs to avoid all other vehicles in $(\rvs)^C$ and is defined in a way analogous to \eqref{eq:obsseti}. The FRS in \eqref{eq:re-planFRS} can be obtained by solving %the HJ VI in \eqref{eq:HJIVI_FRS} with the following Hamiltonian:
\begin{equation}
\begin{aligned}
\max \Big\{&D_t \valfuncfwd_i^{\text{RP}}(t, \state_i) + \ham_i^{\text{RP}}(t, \state_i, \nabla \valfuncfwd_i^{\text{RP}}(t, \state_i)), \\
&\qquad - \obsfunc_i^{\text{RP}}(t, \state_i) - \valfuncfwd_i^{\text{RP}}(t, \state_i) \Big\} = 0\\
&\valfuncfwd_i^{\text{RP}}(\tsa, \state_i) = \max\{\fc_i^{\text{RP}}(\state_i), -\obsfunc_i^{\text{RP}}(\tsa, \state_i)\} \\
&\ham_i^{\text{RP}}(\state_i, \costate) = \max_{\ctrl_i \in \cset_i} \min_{\dstb_i \in \dset_i} \costate \cdot f_i (\state_i, \ctrl_i, \dstb_i)
\end{aligned}
\end{equation} 

\noindent where $\valfuncfwd_i^{\text{RP}}, \obsfunc_i^{\text{RP}}, \fc_i^{\text{RP}}$ represent the FRS, obstacles during re-planning, and the initial state of $\veh_i$, respectively. The new $\sta_i$ of $\veh_i$ is now given by the earliest time at which $\frs_i^{\text{RP}}(\tea, t)$ intersects the target set $\targetset_i$, $\sta_i := \arg \inf_t \{ \frs_i^{\text{RP}}(\tea, t) \cap \targetset_i \neq \emptyset \}$. Intuitively, this means that there exists a control policy which will steer the vehicle $\veh_i$ to its destination by that time, despite the worst case disturbance it might experience. The replanning phase is summarized in Algorithm \ref{alg:intruder_replan}.
\begin{remark}
Note that even though we have presented the analysis for one intruder, the proposed method can handle multiple intruders as long as only one intruder is present \textit{at any given time}.
\end{remark}
%
\begin{algorithm}
\SetKwInOut{Input}{input}
\SetKwInOut{Output}{output}
	\DontPrintSemicolon
	\caption{The intruder avoidance algorithm: Replanning-phase (real-time planning)}
	\label{alg:intruder_replan}
	\Input{Set of vehicles $\veh_i \in \rvs$ that require replanning;\newline
	Vehicle dynamics \eqref{eq:dyn} and new initial states $\tilde{\state}_i^0$;\newline
	Vehicle destinations $\targetset_i$ and static obstacles $\soset_i$;\newline
	Total base obstacle set $\obsset_i^{\text{RP}}(\cdot)$ induced by all other vehicles in $(\rvs)^C$.}
    \Output{The updated nominal controller ${\ctrl^{\text{L}}_{i}}$ for all vehicles in $\rvs$.}
	\For{\text{$\veh_i \in \rvs$}}{
			%
			\textbf{Computation of the updated $\sta_i$ for $\veh_{i}$} \;	
			given the obstacle set $\obsset_i^{\text{RP}}(\cdot)$, compute the FRS $\frs_i^{\text{RP}}(\tea, t)$ in \eqref{eq:re-planFRS}; \;
			the updated $\sta_i$ for $\veh_i$ is given by $\arg \inf_t \{ \frs_i^{\text{RP}}(\tea, t) \cap \targetset_i \neq \emptyset \}$. \;
			\textbf{Trajectory and controller of $\veh_{i}$} \;
			given the updated STA $\sta_i$, the initial state $\tilde{\state}_i^0$, the total obstacle set $\obsset_i^{\text{RP}}(\cdot)$, the vehicle dynamics \eqref{eq:dyn} and the target set $\targetset_i$, use Algorithm \ref{alg:intruder_plan} to replan the nominal trajectory and controller.
%any of the three sequential trajectory planning methods discussed in \cite{chen2016robust} for re-planning. 
		}
\end{algorithm}
%
%
%\begin{algorithm}[tb]
%	%\RestyleAlgo{boxed}
%	\DontPrintSemicolon
%	%\AlgoDisplayBlockMarkers\SetAlgoBlockMarkers{}{end}%
%	%\SetAlgoNoEnd
%	\caption{Intruder avoidance algorithm (offline planning)}
%	\label{alg:intruder_plan}
%	Given initial conditions $\state_i^0$, vehicle dynamics \eqref{eq:dyn}, intruder dynamics in Assumption \ref{as:dynKnown}, target sets $\targetset_i$, and static obstacles $\soset_i, i = 1\ldots, 		    \N$\;
%	\For{\text{$i=1:N$}}{
%			compute the avoid region of $\veh_i$ using \eqref{eqn:avoidBRS}. The optimal control to avoid the intruder can be obtained by using \eqref{eqn:optAvoidCtrl}.\;
%			determine the separation region and buffer regions given by \eqref{eqn:sepRegion_case1}, \eqref{eqn:buffRegion_case1} and \eqref{eqn:buffRegion_case2}. \;
%			compute the induced obstacles for $\veh_i$ by $\veh_j$, given by \eqref{eq:intruderObs_case1_caseA}, \eqref{eq:intruderObs_case1_caseB}, \eqref{eq:intruderObs_case2_caseA}, 	\eqref{eq:intruderObs_case2_caseB} and \eqref{eq:ObsBRS_static}. \;
%			compute the total obstacle set $\obsset_i(t)$, given in \eqref{eq:obsseti_intr}. In the case $i=1$, $\obsset_i(t) = \brs^{\text{S}}_{i}(t, t+\iat) ~ \forall t$ \;
%			given $\obsset_i(t)$, compute the BRS $\brs^{\text{PP}}_{i}(t, \sta_i)$ defined in \eqref{eqn:intrBRS1}; the nominal control strategy is given by \eqref{eqn:PPPolicy}\;
%		}
%\end{algorithm}
%
%\begin{alg}
%\label{alg:intruder}
%\textbf{Intruder Avoidance algorithm (offline planning)}: Given initial conditions $\state_i^0$, vehicle dynamics \eqref{eq:dyn}, intruder dynamics in Assumption \ref{as:dynKnown}, target sets $\targetset_i$, and static obstacles $\soset_i, i = 1\ldots, \N$, for each $i$,
%\begin{enumerate}
%\item compute the avoid region of $\veh_i$ using \eqref{eqn:avoidBRS}. The optimal control to avoid the intruder can be obtained by using \eqref{eqn:optAvoidCtrl}.
%\item determine the separation region and buffer regions given by \eqref{eqn:sepRegion_case1}, \eqref{eqn:buffRegion_case1} and \eqref{eqn:buffRegion_case2}.
%\item compute the induced obstacles for $\veh_i$ by $\veh_j$, given by \eqref{eq:intruderObs_case1_caseA}, \eqref{eq:intruderObs_case1_caseB}, \eqref{eq:intruderObs_case2_caseA}, \eqref{eq:intruderObs_case2_caseB} and \eqref{eq:ObsBRS_static}. 
%\item compute the total obstacle set $\obsset_i(t)$, given in \eqref{eq:obsseti_intr}. In the case $i=1$, $\obsset_i(t) = \brs^{\text{S}}_{i}(t, t+\iat) ~ \forall t$;
%\item given $\obsset_i(t)$, compute the BRS $\brs^{\text{PP}}_{i}(t, \sta_i)$ defined in \eqref{eqn:intrBRS1}; the nominal control strategy is given by \eqref{eqn:PPPolicy};
%\end{enumerate}
%
%\textbf{Intruder Avoidance algorithm (online re-planning)}: For each vehicle $\veh_i \in \rvs$ which performed avoidance in response to the intruder,
%\begin{enumerate}
%\item compute $\frs_i^{\text{RP}}(\tea, t)$ using $\eqref{eq:re-planFRS}$. The new $\sta_i$ for $\veh_i$ is given by $\arg \inf_t \{ \frs_j^{\text{RP}}(\tea, t) \cap \targetset_j \neq \emptyset \}$;
%\item given $\sta_i$, $\tilde{\state}_i^0$, vehicle dynamics \eqref{eq:dyn}, target set $\targetset_i$, and obstacles $\obsset_i^{\text{RP}}$, use any of the three STP methods discussed in \cite{chen2016robust} for re-planning. 
%\end{enumerate}
%\end{alg}