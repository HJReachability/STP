% !TEX root = ../SPP_IoTjournal.tex
\subsection{STP Without Disturbances and Intruder\label{sec:basic}}
In this section, we give an overview of the basic STP algorithm assuming that there is no disturbance and no intruder affecting the vehicles. 
%Although in practice, such assumptions do not hold, the description of the basic STP algorithm will introduce the notation needed for describing the subsequent, more realistic versions of STP. 
The majority of the content in this section is taken from \cite{Chen15c}. 

Recall that among the STP vehicles $\veh_i, i=1,\ldots,N$, $\veh_j$ has a higher priority than $\veh_i$ if $j<i$. In the absence of disturbances, we can write the dynamics of the STP vehicles as
\begin{equation}
\label{eq:dyn_no_dstb}
\begin{aligned}
\dot\state_i = \fdyn_i(\state_i, \ctrl_i), t \le \sta_i,\quad \ctrl_i &\in \cset_i, 
%\qquad i = 1 \ldots, \N
\end{aligned}
\end{equation}

In STP, each vehicle $\veh_i$ plans its trajectory while avoiding static obstacles $\soset_i$ and the obstacles $\ioset_i^j(t)$ induced by higher-priority vehicles $\veh_j, j<i$. Trajectory planning is done sequentially in descending priority, $\veh_1, \veh_2, \ldots, \veh_{\N}$. During its trajectory planning process, $\veh_i$ ignores the presence of lower-priority vehicles $\veh_k, k>i$.
%, and induces the obstacles $\ioset_k^i(t)$ for $\veh_k, k>i$.
From the perspective of $\veh_i$, each of the higher-priority vehicles $\veh_j, j<i$ induces a time-varying obstacle denoted $\ioset_i^j(t)$ that $\veh_i$ needs to avoid. Therefore, each vehicle $\veh_i$ must plan its trajectory while avoiding the union of all the induced obstacles as well as the static obstacles. Let $\obsset_i(t)$ be the union of all the obstacles that $\veh_i$ must avoid on its way to $\targetset_i$:
\begin{equation}
\label{eq:obsseti}
\obsset_i(t)  = \soset_i \cup \bigcup_{j=1}^{i-1} \ioset_i^j(t)
\end{equation}

With full position information of higher-priority vehicles, the obstacle induced for $\veh_i$ by $\veh_j$ is simply
\begin{equation}
\label{eq:ioset_basic}
\ioset_i^j(t) = \{\state_i: \|\pos_i - \pos_j(t)\|_2 \le \rc \}
\end{equation}

Each higher-priority vehicle $\veh_j$ ignores $\veh_i$. Since trajectory planning is done sequentially in descending order of priority, the vehicles $\veh_j, j<i$ would have planned their trajectories before $\veh_i$ does. Thus, in the absence of disturbances, $\pos_j(t)$ is \textit{a priori} known, and therefore $\ioset_i^j(t), j<i$ are known, deterministic moving obstacles, which means that $\obsset_i(t)$ is also known and deterministic. Therefore, the trajectory planning problem for $\veh_i$ can be solved by first computing the BRS $\brs_i^\text{basic}(t, \sta_i)$, defined as follows:
\begin{equation}
\label{eq:BRS_basic}
\begin{aligned}
\brs_i^\text{basic}(t, \sta_i) = & \{y: \exists \ctrl_i(\cdot) \in \cfset_i, \state_i(\cdot) \text{ satisfies \eqref{eq:dyn_no_dstb}}, \\
& \forall s \in [t, \sta_i],\state_i(s) \notin \obsset_i(s), \\
& \exists s \in [t, \sta_i], \state_i(s) \in \targetset_i, \state_i(t) = y\}
\end{aligned}
\end{equation}

The BRS $\brs(t, \sta_i)$ can be obtained by solving \eqref{eq:HJIVI_BRS} with $\targetset = \targetset_i$, $\obsset(t) = \obsset_i(t)$, and the Hamiltonian 
\begin{equation}
\label{eq:basicham}
\ham_i^\text{basic}(\state_i, \costate) = \min_{\ctrl_i\in\cset_i} \costate \cdot \fdyn_i(\state_i, \ctrl_i)
\end{equation}

The optimal control for reaching $\targetset_i$ while avoiding $\obsset_i(t)$ is then given by
\begin{equation}
\label{eq:basicOptCtrl}
\ctrl_i^\text{basic}(t, \state_i) = \arg \min_{\ctrl_i\in\cset_i} \costate \cdot \fdyn_i(\state_i, \ctrl_i)
\end{equation}
\noindent from which the trajectory $\state_i(\cdot)$ can be computed by integrating the system dynamics, which in this case are given by \eqref{eq:dyn_no_dstb}. In addition, the latest departure time $\ldt_i$ can be obtained from the BRS $\brs(t, \sta_i)$ as $\ldt_i = \arg \sup_t \{\state_i^0 \in \brs(t, \sta_i)\}$. The basic STP algorithm is summarized in Algorithm \ref{alg:basic}.
%
\begin{algorithm}[tb]
\SetKwInOut{Input}{input}
\SetKwInOut{Output}{output}
	\DontPrintSemicolon
	\caption{STP algorithm in the absence of disturbances and intruders}
	\label{alg:basic}
	\Input{STP vehicles $\veh_i$, their dynamics \eqref{eq:dyn_no_dstb}, initial states $\state_i^0$,	destinations $\targetset_i$, static obstacles $\soset_i$}
    \Output{Provably safe trajectories to destinations and goal-satisfaction controllers $\ctrl^\text{basic}(\cdot)$}
	\For{\text{$i=1:N$}}{
			\textbf{Trajectory planning for $\veh_{i}$} \;
			compute the total obstacle set $\obsset_i(t)$ given by \eqref{eq:obsseti}. If $i=1$, $\obsset_i(t) = \soset_i ~ \forall t$;\;
			compute the BRS $\brs_i^\text{basic}(t, \sta_i)$ defined in \eqref{eq:BRS_basic};\;
			\textbf{Trajectory and controller of $\veh_{i}$} \;
			compute the optimal controller $\ctrl_i^\text{basic}(\cdot)$ given by \eqref{eq:basicOptCtrl};\;
			determine the trajectory $\state_i(\cdot)$ using vehicle dynamics \eqref{eq:dyn_no_dstb} and the control $\ctrl_i^\text{basic}(\cdot)$; \;
			output the trajectory and optimal controller for $\veh_i$.\;
			\textbf{Obstacles induced by $\veh_{i}$} \;
			given the trajectory $\state_i(\cdot)$, compute the induced obstacles $\ioset_k^i(t)$ given by \eqref{eq:ioset_basic} for all $k>i$.
		}
\end{algorithm}
%
%\begin{alg}
%\label{alg:basic}
%\textbf{Basic STP algorithm \cite{Chen15c}}: Suppose we are given initial conditions $\state_i^0$, vehicle dynamics \eqref{eq:dyn_no_dstb}, target sets $\targetset_i$, and static obstacles $\soset_i, i = 1\ldots, \N$. For each $i$ in ascending order starting from $i=1$ (which corresponds to descending order of priority),
%\begin{enumerate}
%\item determine the total obstacle set $\obsset_i(t)$, given in \eqref{eq:obsseti}. In the case $i=1$, $\obsset_i(t) = \soset_i ~ \forall t$;
%\item compute the BRS $\brs_i^\text{basic}(t, \sta_i)$ defined in \eqref{eq:BRS_basic}. The latest departure time $\ldt_i$ is then given by $\arg \sup_t \{\state^0_i \in \brs_i^\text{basic}(t, \sta_i)\}$;
%\item determine the trajectory $\state_i(\cdot)$ using vehicle dynamics \eqref{eq:dyn_no_dstb}, with the optimal control  $\ctrl_i^\text{basic}(\cdot)$ given by \eqref{eq:basicOptCtrl};
%\item given $\state_i(\cdot)$, compute the induced obstacles $\ioset_k^i(t)$ for each $k>i$. In the absence of disturbances, $\ioset_k^i(t)$ is given by \eqref{eq:ioset_basic}.
%\end{enumerate}
%\end{alg}