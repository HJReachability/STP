% !TEX root = ../SPP_IoTjournal.tex
\subsection{Optimal Avoidance Controller} \label{sec:intruder_avoid}
To compute the optimal avoidance control for $\veh_i$ in presence of $\veh_{\intr}$, we compute the set of states from which the joint states of $\veh_{\intr}$ and $\veh_i$ can enter danger zone $\dz_{i\intr}$ despite the best efforts of $\veh_i$ to avoid $\veh_{\intr}$. 

We define relative dynamics of the intruder $\veh_{\intr}$ with state $\state_\intr$ with respect to $\veh_i$ with state $\state_i$:
\begin{equation}
\label{eq:reldyn}
\begin{aligned}
\state_{\intr, i} &= \state_\intr - \state_i \\
\dot \state_{\intr, i} &= f_r(\state_{\intr, i}, \ctrl_i, \ctrl_\intr, \dstb_i, \dstb_\intr)
\end{aligned}
\end{equation}
Given the relative dynamics, we compute the set of states from which the joint states of $\veh_{\intr}$ and $\veh_{i}$ can enter danger zone $\dz_{i\intr}$ in a duration of $\iat$ despite the best efforts of $\veh_i$ to avoid $\veh_{\intr}$. Under the relative dynamics \eqref{eq:reldyn}, this set of states is given by the backwards reachable set $\brs^{\text{A}}_i(\tau, \iat),~ \tau \in [0, \iat]$:

\begin{equation} \label{eqn:avoidBRS}
\begin{aligned}
\brs^{\text{A}}_{i}(\tau, \iat) = & \{y: \forall \ctrl_i(\cdot) \in \cfset_i, \exists \ctrl_\intr(\cdot) \in \cfset_\intr, \exists \dstb_i(\cdot) \in \dfset_i, \\
& \exists \dstb_\intr(\cdot) \in \dfset_\intr, \state_{\intr, i}(\cdot) \text{ satisfies \eqref{eq:reldyn}},\\
& \exists s \in [\tau, \iat], \state_{\intr, i}(s) \in \targetset^{\text{A}}_{i}, \state_{\intr, i}(\tau) = y\},
\end{aligned}
\end{equation}
where 
\begin{equation}
\begin{aligned}
\targetset^{\text{A}}_{i} = & \{\state_{\intr, i}: \|\pos_{\intr, i}\|_2 \le \rc\} \\
H^{\text{A}}_{i}(\state_{\intr, i}, \costate) = & \max_{\ctrl_i \in \cset_i} \left( \right.\\
&\left. \min_{\ctrl_\intr \in \cset_\intr, \dstb_i \in \dset_i, \dstb_\intr \in \dset_\intr} \costate \cdot f_r(\state_{\intr, i}, \ctrl_i, \ctrl_\intr, \dstb_i, \dstb_\intr) \right)
\end{aligned}
\end{equation}

We refer to $\brs^{\text{A}}_i(\tau, \iat)$ as \textit{avoid region} hereon. Once the value function $\valfunc^{\text{A}}_{i}(\cdot)$ is computed, the optimal avoidance control ${\ctrl^{\text{A}}_{i}}$ can be obtained as:
\begin{equation} \label{eqn:optAvoidCtrl}
{\ctrl^{\text{A}}_{i}} = \arg \max_{\ctrl_i \in \cset_i} \left( \min_{\ctrl_\intr \in \cset_\intr, \dstb_i \in \dset_i, \dstb_\intr \in \dset_\intr} \costate \cdot f_r(\state_{\intr, i}, \ctrl_i, \ctrl_\intr, \dstb_i, \dstb_\intr) \right)
\end{equation}

Let $\partial \brs^{\text{A}}_{i}(\cdot, \iat)$ denotes the boundary of the set $\brs^{\text{A}}_{i}(\cdot, \iat)$. The interpretation of $\brs^{\text{A}}_{i}(\tau, \iat)$ is that if $\state_{\intr, i}(t) \in \partial \brs^{\text{A}}_{i}(\tau, \iat)$, then $\veh_i$ can successfully avoid the intruder for a duration of $(\iat - \tau)$ using the optimal avoidance control in \eqref{eqn:optAvoidCtrl}. In the worst case, $\veh_i$ might need to avoid the intruder for a duration of $\iat$; thus, we must have that  
$\state_{\intr, i}(\tsa) \in \left(\brs^{\text{A}}_{i}(0, \iat)\right)^C$. Equivalently, every SPP vehicle should be able to detect the intruder at a distance of $\dsen$, where
\begin{equation} \label{eqn:sen_distance}
\dsen = \max\{ \|a\|: a \in \brs^{\text{A}}_{i}(0, \iat) \}.
\end{equation} 
Hereon, we assume that every SPP vehicle can detect the intruder at a distance of $\dsen$.

%Under normal circumstances when the intruder $\veh_{\intr}$ is far away, we have $\valfunc^{\text{S}}_{i}(0, \state_{\intr, i}(t)) > 0$; as $\veh_{\intr}$ gets closer to $\veh_i$, $\valfunc^{\text{S}}_{i}(0, \state_{\intr, i}(t))$ decreases. If $\veh_i$ applies the control ${\ctrl^{\text{S}}_{i}}$ when $\valfunc^{\text{S}}_{i}(0, \state_{\intr, i}(t)) = 0$, then collision avoidance between $\veh_i$ and $\veh_{\intr}$ is guaranteed for a duration of $\iat$ under the worst-case intruder control strategy $\ctrl_\intr$.