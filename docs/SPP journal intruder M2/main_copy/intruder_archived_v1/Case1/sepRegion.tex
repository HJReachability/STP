% !TEX root = ../SPP_IoTjournal.tex
\subsubsection{Separation region} \label{sec:sepRegion_case1}
In this section, we find the set of all states $\state_{\intr}(t)$ for which $\veh_j$ is forced to apply an avoidnace maneuver for a given $\tsa$. We refer to this set as \textit{separation region}, and denote it as $\sep_j(\tsa, t)$. As discussed in Section \ref{sec:intruder_avoid}, $\veh_j$ needs to apply avoidnace maneuver at time $t$ only if $\state_{\intr, j}(t) \in \partial \brs^{\text{A}}_{j}(t-\tsa, \iat)$. To compute set $\sep_j(\tsa, t)$, we thus need to translate this set in relative coordinates to absolute coordinates, for which we need to know all possible states of $\veh_j$ at time $t$.

Depending on the information known to a lower-priority vehicle $\veh_i$ about $\veh_j$'s control strategy, we can use one of the three methods described in Section 5 in \cite{chen2016robust} to compute the ``base" obstacles $\boset_j(t)$, the obstacles that would have been induced by $\veh_j$ in the presence of disturbances, but in the absence of an intruder. The base obstacles are respectively given by equations (25), (31) and (37) in \cite{chen2016robust} for centralized control, least restrictive control and robust trajectory tracking algorithms.

Given $\boset_j(t)$, we compute the set of all possible states of $\veh_j$ at time $t$ given $\tsa$. Regardless of the avoidnace control applied by $\veh_j$ during $[\tsa, t]$, $\state_j(t)$ will still be contained withing the FRS $\frs^{\text{S}}_{j}(\tsa, \iat)$:  
\begin{equation} \label{eq:sepRegionFRS_case1}
\begin{aligned}
\frs_{j}^{\text{S}}(\tsa, t) = & \{y: \exists \ctrl_j(\cdot) \in \cfset_j, \exists \dstb_j(\cdot) \in \dfset_j, \\
& \state_j(\cdot) \text{ satisfies \eqref{eq:dyn}}, \state_j(\tsa) \in \boset_j(\tsa), \\
& \state_j(t) = y\}.
\end{aligned}
\end{equation}
$\frs_{j}^{\text{S}}(\tsa, t)$ represents the set of all possible states that $\veh_j$ can reach after a duration of $(t-\tsa)$ starting from inside $\boset_j(\tsa)$. This FRS can be obtained by solving the HJ VI in \eqref{eq:HJIVI_FRS} with the following Hamiltonian:
\begin{equation}
\ham_{j}^{\text{S}}(\state_j, \costate) = \max_{\ctrl_j \in \cset_j} \max_{\dstb_j \in \dset_j} \costate \cdot f_j (\state_j, \ctrl_j, \dstb_j).
\end{equation} 
Thus, $\sep_j(\tsa, t)$ can be obtained as:
\begin{equation} \label{eqn:sepRegion_case1}
\sep_j(\tsa, t) = \frs_{j}^{\text{S}}(\tsa, t) + \brs^{\text{A}}_{j}(t-\tsa, \iat), ~\tsa \in [t-\iat, t],
\end{equation}
where the ``$+$'' in \eqref{eqn:sepRegion_case1} denotes the Minkowski sum.