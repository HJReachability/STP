% !TEX root = ../SPP_IoTjournal.tex
\section{Sequential Path Planning Problem \label{sec:formulation}}
Consider $\N$ vehicles (also denoted as \textit{SPP vehicles}) which participate in the SPP process $\veh_i, i = 1, \ldots, \N$. We assume their dynamics are given by

\begin{equation}
\label{eq:dyn}
\begin{aligned}
\dot\state_i &= \fdyn_i(\state_i, \ctrl_i, \dstb_i), t \le \sta_i \\
\ctrl_i &\in \cset_i, \dstb_i \in \dset_i, i = 1 \ldots, \N
\end{aligned}
\end{equation}

\noindent where $\state_i \in \R^{n_i}$, $\ctrl_i \in \cset_i$ and $\dstb_i \in \dset_i$, respectively, represent the state, control and disturbance experienced by vehicle $\veh_i$. We partition the state $\state_i$ into the position component $\pos_i \in \R^{n_\pos}$ and the non-position component $\npos_i \in \R^{n_i - n_\pos}$: $\state_i = (\pos_i, \npos_i)$. %We assume that the control functions $\ctrl_i(\cdot), \dstb_i(\cdot)$ are drawn from the set of measurable functions\footnote{A function $f:X\to Y$ between two measurable spaces $(X,\Sigma_X)$ and $(Y,\Sigma_Y)$ is said to be measurable if the preimage of a measurable set in $Y$ is a measurable set in $X$, that is: $\forall V\in\Sigma_Y, f^{-1}(V)\in\Sigma_X$, with $\Sigma_X,\Sigma_Y$ $\sigma$-algebras on $X$,$Y$.}. For convenience, 
We will use the sets $\cfset_i, \dfset_i$ to respectively denote the set of functions from which the control and disturbance functions $\ctrl_i(\cdot), \dstb_i(\cdot)$ are drawn.

% We further assume that the flow field $\fdyn_i: \R^{n_i}\times\cset_i\times\dset_i \rightarrow \R^{n_i}$ is uniformly continuous, bounded, and Lipschitz continuous in $\state_i$ for fixed $\ctrl_i$ and $\dstb_i$. With this assumption, given $\ctrl_i(\cdot) \in \cfset_i, \dstb_i(\cdot) \in \dfset_i$, there exists a unique trajectory solving \eqref{eq:dyn} \cite{EarlA.Coddington1955}. %We will denote trajectories of \eqref{eq:dyn} starting from state $\state^0_i$ at time $t_0$ under control $\ctrl_i(\cdot)$ and disturbance $\dstb_i(\cdot)$ as $\traj_i(t; \state^0_i, t_0, \ctrl_i(\cdot))$. Trajectories satisfy an initial condition and the differential equation \eqref{eq:dyn} almost everywhere:

%\begin{equation}
%\begin{aligned}
%\frac{d}{dt}\traj_i(t; \state^0_i, t_0, \ctrl_i(\cdot)) &= \fdyn(\state^0_i, \ctrl_i, \dstb_i) \\
%\traj_i(t_0; \state^0_i, t_0, \ctrl_i(\cdot)) &= \state^0_i
%\end{aligned}
%\end{equation}

%In addition, we assume that the disturbances $\dstb_i(\cdot)$ are drawn the set of non-anticipative strategies \cite{Mitchell05} $\Gamma$, defined as follows:
%\begin{equation}
%\begin{aligned}
%& \Gamma := \{\mathcal{N}: \cfset_i \rightarrow \dfset_i:  \ctrl_i(r) = \hat{\ctrl}_i(r) \text{ a. e. } r\in[t,s] \\
%& \Rightarrow \mathcal{N}[\ctrl_i](r) = \mathcal{N}[\hat{\ctrl}_i](r) \text{ a. e. } r\in[t,s]\}
%\end{aligned}
%\end{equation}

Each vehicle $\veh_i$ has initial state $\state^0_i$, and aims to reach its target $\targetset_i$ by some scheduled time of arrival $\sta_i$. The target in general represents some set of desirable states, for example the destination of $\veh_i$. %For most of the paper we will make the assumption that $\edt_i$ is early enough for $\veh_i$ to feasibly get to $\targetset_i$ on time; this can be done by letting $\edt_i \rightarrow -\infty$. The assumption on $\edt_i$ is merely for convenience: in situations where $\edt_i$ is $-\infty$. In some situations, we may find that it is infeasible for $\veh_i$ to get to $\targetset_i$ at or before $\sta_i$. Whenever unsure, we may first determine the earliest feasible $\sta_i$ as described in Section \ref{sec:intruder}.
On its way to $\targetset_i$, $\veh_i$ must avoid a set of static obstacles $\soset_i \subset \R^{n_i}$. The interpretation of $\soset_i$ could be a tall building or any set of states that are forbidden for each SPP vehicle. Each vehicle $\veh_i$ must also avoid the danger zones with respect to every other vehicle $\veh_j, j\neq i$. The danger zones in general can represent any joint configurations between $\veh_i$ and $\veh_j$ that are considered to be unsafe. We define the danger zone of $\veh_i$ with respect to $\veh_j$ to be
\begin{equation}
\dz_{ij} = \{(\state_i, \state_j): \|\pos_i - \pos_j\|_2 \le \rc\}
\end{equation}
\noindent whose interpretation is that $\veh_i$ and $\veh_j$ are considered to be in an unsafe configuration when they are within a distance of $\rc$ of each other. In particular, $\veh_i$ and $\veh_j$ are said to have collided, if $(\state_i, \state_j) \in \dz_{ij}$.

In addition to the obstacles and danger zones, an intruder vehicle can also appear in the system, which can potentially have malicious intents. In general, the effect of an intruder on the vehicles in structured flight can be entirely unpredictable, since the intruder in principle could be adversarial in nature, and the number of intruders could be arbitrary. Therefore, we make the following two assumptions: 
\begin{assumption}
\label{as:avoidOnce}
At most one intruder (denoted as $\veh_I$ here on) affects the SPP vehicles at any given time. The intruder exits the altitude level affecting the SPP vehicles after a duration of $\iat$. 
\end{assumption}

Let the time at which intruder appears in the system be $\tsa$ and the time at which it disappears be $\tea$. Assumption \ref{as:avoidOnce} implies that $\tea \leq \tsa + \iat$. Thus, any vehicle $\veh_i$ would need to avoid the intruder $\veh_{\intr}$ for a maximum duration of $\iat$. This assumption can be valid in situations where intruders are rare, and that some fail-safe or enforcement mechanism exists to force the intruder out of the altitude level affecting the SPP vehicles. 
Note that we do not pose any restriction on $\tsa$; however, we assume that once the intruder appears, it stays for a maximum duration of $\iat$.
\begin{assumption}
\label{as:dynKnown}
The dynamics of the intruder are known and given by $\dot\state_\intr = f_\intr(\state_\intr, \ctrl_\intr, \dstb_\intr)$.
\end{assumption}

Assumption \ref{as:dynKnown} is required for HJ reachability analysis. In situations where the dynamics of the intruder are not known exactly, a conservative model of the intruder may be used instead. We also denote the initial state of the intruder as $\state_{\intr}^0.$ Note that $\state_{\intr}^0$ is unknown.

Given the set of SPP vehicles, their targets $\targetset_i$, the static obstacles $\soset_i$, the vehicles' danger zones with respect to each other $\dz_{ij}$, and the intruder dynamics $f_\intr(\cdot)$, our goal is, for each vehicle $\veh_i$, to synthesize a controller which guarantees that $\veh_i$ reaches its target $\targetset_i$ at or before the scheduled time of arrival $\sta_i$, while avoiding the static obstacles $\soset_i$, the danger zones with respect to all other vehicles $\dz_{ij}, j\neq i$, and the intruder vehicle $\veh_{\intr}$, irrespective of the control strategy of the intruder. In addition, we would like to obtain the latest departure time $\ldt_i$ such that $\veh_i$ can still arrive at $\targetset_i$ on time.

In general, the above optimal path planning problem must be solved in the joint space of all $\N$ SPP vehicles and the intruder vehicle. However, due to the high joint dimensionality, a direct dynamic programming-based solution is intractable. Therefore, authors in \cite{Chen15c} proposed to assign a priority to each vehicle, and perform SPP given the assigned priorities. Without loss of generality, let $\veh_j$ have a higher priority than $\veh_i$ if $j<i$. Under the SPP scheme, higher-priority vehicles can ignore the presence of lower-priority vehicles, and perform path planning without taking into account the lower-priority vehicles' danger zones. A lower-priority vehicle $\veh_i$, on the other hand, must ensure that it does not enter the danger zones of the higher-priority vehicles $\veh_j, j<i$ or the intruder vehicle $\veh_{\intr}$; each higher-priority vehicle $\veh_j$ induces a set of time-varying obstacles $\ioset_i^j(t)$, which represents the possible states of $\veh_i$ such that a collision between $\veh_i$ and $\veh_j$ or $\veh_i$ and $\veh_{\intr}$ could occur.

It is straight-forward to see that if each vehicle $\veh_i$ is able to plan a trajectory that takes it to $\targetset_i$ while avoiding the static obstacles $\soset_i$, the danger zones of \textit{higher-priority vehicles} $\veh_j, j<i$, and the danger zone of the \textit{intruder} $\veh_{\intr}$ irrespective of the intruder's control policy, then the set of SPP vehicles $\veh_i, i=1,\ldots,\N$ would all be able to reach their targets safely. Under the SPP scheme, path planning can be done sequentially in descending order of vehicle priority in the state space of only a single vehicle. Thus, SPP provides a solution whose complexity scales linearly with the number of vehicles, as opposed to exponentially with a direct application of dynamic programming approaches. 

However, when an intruder appears in the system, depending on the initial state of the intruder and its control policy, a vehicle may arrive at different states after avoiding the intruder. Therefore, a control policy that ensures a successful transit to the destination needs to account for all such possible states, which is a path planning problem with multiple initial states and a single destination, and is hard to solve in general. Thus, we divide the intruder avoidance problem into two sub-problems: (i) we first design a control policy that ensures a successful transit to the destination if no intruder appears and that successfully avoid the intruder, if it does. (ii) after the intruder disappears at $\tea$, we replan the trajectories of the affected vehicles. Following the same theme and assumptions, authors in \cite{chen2016robust} present an algorithm to avoid an intruder in SPP formulation; however, once the intruder disappears, the algorithm might need to replan the trajectories for all SPP vehicles. Since the replanning is done in real-time, it should be fast and scalable with the number of SPP vehicles, rendering the method in \cite{chen2016robust} unsuitable for practical implementation. Our goal in this work is to present an algorithm that ensures that only a fixed number of vehicles need to replan their trajectories, regardless of the total number of vehicles. Thus, the replanning time is constant and can be done in real time if this constant is small. In particular, we answer the following inter-dependent questions:
\begin{enumerate}
\item How can each vehicle guarantee that it will reach its target set without getting into any danger zones, despite no knowledge of the intruder initial state, the time at which it appears and its control strategy?
\item How can it be ensured that only a small number of vehicles need to replan their trajectories after the intruder disappears from the system? 
\item Can we control the number the vehicles that need to replan their trajectories?
\end{enumerate}