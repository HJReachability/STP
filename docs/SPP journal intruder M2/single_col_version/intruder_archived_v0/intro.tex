% !TEX root = ../SPP_IoTjournal.tex
\section{Response to Intruders \label{sec:intruder}}
\MCnote{Need diagrams illustrating all the different times (BRD, RD, etc.)}

\MCnote{Need to mention augmentation of static obstacles.}

In Section \ref{sec:rtt}, we presented RTT algorithm that can take into account disturbances in vehicles' dynamics. However, if a vehicle not in the set of SPP vehicles enters the system, or even worse, if this vehicle is an adversarial intruder, the original plan can lead to vehicles entering into each other's danger zones. If vehicles do not plan with an additional safety margin that takes a potential intruder into account, a vehicle trying to avoid the intruder may effectively become an intruder itself, leading to a domino effect. %In this section, we propose a method to allow vehicles to avoid an intruder while maintaining the SPP structure.

In general, the effect of an intruder on the vehicles in structured flight can be entirely unpredictable, since the intruder in principle could be adversarial in nature, and the number of intruders could be arbitrary. Therefore, we make the following two assumptions: %for our analysis to produce reasonable results, two assumptions about the intruders must be made.

\begin{assumption}
\label{as:avoidOnce}
At most one intruder (denoted as $\veh_I$ here on) affects the SPP vehicles at any given time. The intruder exits the altitude level affecting the SPP vehicles after a duration of $\iat$. 
\end{assumption}

Let the time at which intruder appears in the system be $\tsa$ and the time at which it disappears be $\tea$. Assumption \ref{as:avoidOnce} implies that $\tea \leq \tsa + \iat$. Thus, any vehicle $\veh_i$ would need to avoid the intruder $\veh_{\intr}$ for a maximum duration of $\iat$. %This assumption can be valid in situations where intruders are rare, and that some fail-safe or enforcement mechanism exists to force the intruder out of the altitude level affecting the SPP vehicles. 
Note that we do not pose any restriction on $\tsa$; however, we assume that once the intruder appears, it stays for a maximum duration of $\iat$.
%in addition, after avoiding the intruder, Qi can safely assume that it would not need to avoid another intruder

\begin{assumption}
\label{as:dynKnown}
The dynamics of the intruder are known and given by $\dot\state_\intr = f_\intr(\state_\intr, \ctrl_\intr, \dstb_\intr)$.
\end{assumption}

Assumption \ref{as:dynKnown} is required for HJ reachability analysis. In situations where the dynamics of the intruder are not known exactly, a conservative model of the intruder may be used instead. We also denote the initial state of the intruder as $\state_{\intr}^0.$ Note that $\state_{\intr}^0$ is unknown.

Our goal is to design a control policy that ensures separation with the intruder and other SPP vehicles, and ensures a successful transit to the destination. However, depending on the initial state of the intruder and its control policy, a vehicle may arrive at different states after avoiding the intruder. Therefore, a control policy that ensures a successful transit to the destination needs to account for all such possible states, which is a path planning problem with multiple initial states and a single destination, and is hard to solve in general. Thus, we divide the intruder avoidance problem into two sub-problems: (i) we first design a control policy that ensures a successful transit to the destination if no intruder appears and that successfully avoid the intruder, if it does. (ii) after the intruder disappears at $\tea$, we replan the trajectories of the affected vehicles. Following the same theme and assumptions, authors in \cite{chen2016robust} present an algorithm to avoid an intruder in SPP formulation; however, once the intruder disappears, the algorithm might need to replan the trajectories for all SPP vehicles. Since the replanning is done in real-time, it should be fast and scalable with the number of SPP vehicles, rendering the method in \cite{chen2016robust} unsuitable for practical implementation.  

In this work, we propose a novel intruder avoidance algorithm, which will need to replan trajectories only for a fixed number of vehicles, irrespective of the total number of SPP vehicles. Moreover, this number is a design parameter, which can be chosen based on the resources available during the run time. Intuitively, one can think about dividing the flight space of vehicles such that at any given time, any two vehicles are far enough from each other so that an intruder can only affect atmost $\nva$ vehicles in a duration of $\iat$ despite its best efforts. In this method, we build upon this intuition and show that such a division of space is indeed possible. The advantage of such an approach is that after the intruder disappears, we only have to replan the trajectory of \textit{atmost $\nva$} vehicles, which can be efficiently done in real-time if $\nva$ is low enough, thus making this approach particularly suitable for practical systems. %Thus the proposed method guarantees that \textit{atmost $k$} vehicles are affected by the presence of intruder, regardless of the number of SPP vehicles, and hence the replanning can be efficiently done in real-time. 

\SBnote{Fix the organization of sections once you finish writing the sections.}
In Sections \ref{sec:intruder_avoid}, we discuss the intruder avoidance control and explain the sensing range required to avoid the intruder. In Sections \ref{sec:sepRegion} and \ref{sec:buffRegion}, we compute a space division of state-space such that atmost $\nva$ vehicles need to apply the avoidance maneuver computed in Section \ref{sec:intruder_avoid}, regardless of the initial state of the intruder. However, we still need to ensure that vehicles do not collide with each other while avoiding the intruder. The induced obstacles that reflect this possibility are computed in Section \ref{sec:intruderObs}. Trajectopry planning and replanning are discussed in Sections \ref{sec:plan} and \ref{sec:replan} respectively.