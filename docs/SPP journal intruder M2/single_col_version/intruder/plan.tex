% !TEX root = ../SPP_IoTjournal.tex
\subsection{Trajectory Planning} \label{sec:path_planning}
In this section, our goal is to plan the trajectory of each vehicle such that it is guaranteed to safely reach its target in the absence of an intruder, and to ensure collision avoidance with vehicles or obstacles if forced to apply an avoidance maneuver. We also need to make sure that the trajectories of the vehicles are such that the separation requirement is satisfied at all times. To obtain such a trajectory, we take into account all the ``obstacles" computed in previous sections which ensure that the vehicle $\veh_i$ will not collide with any other vehicle, as long as it is outside these obstacles.   

Before, we plan such a trajectory, we need to compute one final set of obstacles. In particular, we need to compute the set of states states that $\veh_i$ needs to avoid in order to avoid a collision with static obstacles while it is applying an avoidance maneuver. Since $\veh_i$ applies avoidance maneuver for a maximum duration of $\iat$, this set is given by the following BRS:
\begin{equation} \label{eq:ObsBRS_static}
\begin{aligned}
\brs^{\text{S}}_{i}(t, t+\iat) = & \{y: \exists \ctrl_i(\cdot) \in \cfset_i, \exists \dstb_i(\cdot) \in \dfset_i, \\
& \state_i(\cdot) \text{ satisfies \eqref{eq:dyn}}, \state_i(t) = y, \\
& \exists s \in [t, t+\iat], \state_i(s) \in \mathcal{K}^{\text{S}}(s) \},\\
\mathcal{K}^{\text{S}}(s) = & \{\state_i: \exists (y, h) \in \soset_i, \|\pos_i - y\|_2 \le \rc \}.
\end{aligned}
\end{equation}
The Hamiltonian $\ham^{\text{S}}_{i}$ to compute $\brs^{\text{S}}_{i}(t, t+\iat)$ is given by:
\begin{equation} \label{eqn:BRS_obsham_static}
\ham^{\text{S}}_{i}(\state_i, \costate) = \min_{\ctrl_i \in \cset_i, \dstb_i \in \dset_i} \costate \cdot f_i (\state_i, \ctrl_i, \dstb_i).
\end{equation}
$\brs^{\text{S}}_{i}(t, t+\iat)$ represents the set of all states of $\veh_i$ at time $t$ that can lead to a collision with a static obstacle for some time $\tau > t$ for some control strategy of $\veh_i$.

During the trajectory planning of $\veh_i$, if we use $\buff_{ij}(t)$ and $\buff_{ji}(t)$ as obstacles at time $t$, then the separation requirement is ensured between $\veh_i$ and $\veh_j$ for all intruder strategies and $\tsa = t$. Similarly, if obstacles computed in sections \ref{sec:intruderObs_case1} and \ref{sec:case2_maintext} are used as obstacles in trajectory planning, then we can guarantee collision avoidance between $\veh_i$ and $\veh_j$ while they are avoiding the intruder. Thus, the overall obstacle for $\veh_i$ is given by:
\begin{equation} \label{eq:obsseti_intr}
\begin{aligned}
\obsset_i(t)  =  & \brs^{\text{S}}_{i}(t, t+\iat) \bigcup \\
& \bigcup_{j=1}^{i-1} \left( \buff_{ij}(t) \cup \buff_{ji}(t) \bigcup_{k \in\{1, 2\}} {}_k^A\ioset_i^j(t) \bigcup_{k \in\{1, 2\}} {}_k^B\ioset_i^j(t) \right).
\end{aligned}
\end{equation}

Given $\obsset_i(t)$, we compute a BRS $\brs^{\text{AO}}_{i}(t, \sta_i)$ for trajectory planning that contains the initial state of $\veh_i$ and avoids these obstacles:
\begin{equation} \label{eqn:intrBRS1}
\begin{aligned}
\brs^{\text{PP}}_{i}(t, \sta_i) = & \{y: \exists \ctrl_i(\cdot) \in \cfset_i, \forall \dstb_i(\cdot) \in \dfset_i, \\
& \state_i(\cdot) \text{ satisfies \eqref{eq:dyn}}, \forall s \in [t, \sta_i], \state_i(s) \notin \obsset_i(s), \\
& \exists s \in [t, \sta_i], \state_i(s) \in \targetset_i, \state_i(t) = y \}.
\end{aligned}
\end{equation}
The Hamiltonian $\ham^{\text{PP}}_{i}$ to compute the BRS in \eqref{eqn:intrBRS1} is given by:
\begin{equation} \label{eqn:BRSham}
\ham^{\text{PP}}_{i}(\state_i, \costate) = \min_{\ctrl_i \in \cset_i} \max_{\dstb_i \in \dset_i} \costate \cdot f_i (\state_i, \ctrl_i, \dstb_i)
\end{equation}

Note that $\brs^{\text{PP}}_{i}(\cdot)$ ensures goal satisfaction for $\veh_i$ in the absence of intruder. The goal satisfaction controller is given by:
\begin{equation} \label{eqn:PPPolicy}
{\ctrl^{\text{PP}}_{i}}(t, \state_i) = \arg \min_{\ctrl_i \in \cset_i} \max_{\dstb_i \in \dset_i} \costate \cdot f_i (\state_i, \ctrl_i, \dstb_i)
\end{equation}
When intruder is not present in the system, $\veh_i$ applies the control ${\ctrl^{\text{PP}}_{i}}$ and we get the ``nominal trajectory" of $\veh_i$. Once intruder appears in the system, $\veh_i$ applies the avoidance control ${\ctrl^{\text{A}}_i}$ and hence might deviate from its nominal trajectory. The overall control policy for avoiding the intruder and collision with other vehicles is thus given by:
\begin{equation} \label{eqn:final_nominalcontroller}
{\ctrl^*_{i}}(t) = 
\left \{ 
\begin{array}{ll}
{\ctrl^{\text{PP}}_{i}}(t) & t \leq \tsa_i\\
{\ctrl^{\text{A}}_{i}}(t) & \tsa_i \leq t \leq \tsa + \iat
\end{array}
\right.
\end{equation}

If $\veh_i$ starts within $\brs^{\text{PP}}_{i}$ and uses the control ${\ctrl^*_{i}}$, it is guaranteed to avoid collision with the intruder and other STP vehicles, regardless of the control strategy of $\veh_{\intr}$. Finally, since we use separation and buffer regions as obstacles during the trajectory planning of $\veh_i$, it is guaranteed that $|\tsa_i - \tsa_j| \geq \brd$ for all $j < i$. Therefore, atmost $\nva$ vehicles are forced to apply an avoidance maneuver. The planning phase is summarized in Algorithm \ref{alg:intruder_plan}.
\begin{remark}
Note that given $\brs^{\text{PP}}_{i}$ and ${\ctrl^{\text{PP}}_{i}}$, the base obstacles required for the computation of the separation region in Section \ref{sec:sepRegion_case1} can be computed using equations (25), (31) and (37) in \cite{Chen2016d}. Also, if we use the robust trajectory tracking method to compute the base obstacles, we would need to augment the obstacles in \eqref{eq:obsseti_intr} by the error bound of $\veh_i$, $\disckernel_i$ (for details, see Section 5A-3 in \cite{Chen2016d}). The BRS in \eqref{eqn:intrBRS1} in this case is computed assuming no disturbance in $\veh_i$'s dynamics.
\end{remark}
%
\begin{algorithm}[tb!]
\SetKwInOut{Input}{input}
\SetKwInOut{Output}{output}
	\DontPrintSemicolon
	\caption{The intruder avoidance algorithm: Planning-phase (offline planning)}
	\label{alg:intruder_plan}
	\begin{small}
	\Input{Set of vehicles $\veh_i$ in the descending priority order, their dynamics \eqref{eq:dyn} and initial states $\state_i^0$;\newline
	Vehicle destinations $\targetset_i$ and static obstacles $\soset_i$;\newline
	Intruder dynamics $f_{\intr}$ and the maximum avoidance time $\iat$ ;\newline
	Maximum number of vehicles allowed to re-plan their trajectories $\nva$.}
    \Output{The nominal controller $\ctrl^{\text{PP}}$ and the avoidance controller $\ctrl^{\text{A}}$ for all vehicles.}
	\For{\text{$i=1:N$}}{
			%
			\textbf{Avoid region and avoidance control for $\veh_{i}$} \;	
			compute the avoid region $\brs^{\text{A}}_{i}$ using \eqref{eqn:avoidBRS}; \;
			compute the avoidance controller $\ctrl^{\text{A}}_i$ using \eqref{eqn:optAvoidCtrl}; \;
			output the optimal avoidance controller $\ctrl_i^{\text{A}}$ for $\veh_i$. \;
			%
			\If{\text{$i \neq 1$}}{
			    \textbf{Computation of separation region for $\veh_{i}$} \;	
				\For{\text{$j=1:i-1$}}{
					given the base obstacles $\boset_j(\cdot)$ and the avoid region $\brs^{\text{A}}_{j}$, compute the separation regions in \eqref{eqn:sepRegion_case1} and (45) (in \cite{current_axriv}); 					\;}
				\textbf{Computation of buffer region for $\veh_{i}$} \;	
				\For{\text{$j=1:i-1$}}{
					given the separation regions, compute the relative buffer regions $\brs^{\text{B}}$ in \eqref{eqn:buffBRS_case1} and (46) (in \cite{current_axriv}); \;
					given the relative buffer regions, compute the buffer regions in \eqref{eqn:buffRegion_case1} and (48) (in \cite{current_axriv}); \;}
			}
			\textbf{Computation of obstacles for $\veh_{i}$} \;
			\If{\text{$i \neq 1$}}{
				\For{\text{$j=1:i-1$}}{
					given the base obstacles $\boset_j(\cdot)$, compute the obstacles $_1^A\ioset_i^j(t)$ in \eqref{eq:intruderObs_case1_caseA}, $_1^B\ioset_i^j(t)$ in \eqref{eq:intruderObs_case1_caseB}, $_2^A\ioset_i^j(t)$ in (52) (in \cite{current_axriv}), and $_2^B\ioset_i^j(t)$ in (57) (in \cite{current_axriv}); \;}
			}
			compute the effective static obstacle to avoid ($\brs^{\text{S}}_{i}$) using \eqref{eq:ObsBRS_static}; \;
			\textbf{Trajectory planning for $\veh_{i}$} \;
			compute the total obstacle set $\obsset_i(t)$ given by \eqref{eq:obsseti_intr};\;
			compute the BRS $\brs^{\text{PP}}_{i}(t, \sta_i)$ defined in \eqref{eqn:intrBRS1};\;
			\textbf{The nominal controller of $\veh_{i}$} \;
			compute the nominal controller ${\ctrl^{\text{PP}}_{i}}(\cdot)$ given by \eqref{eqn:PPPolicy};\;
			%determine the trajectory $\state_i(\cdot)$ using vehicle dynamics \eqref{eq:dyn} and the control ${\ctrl^{\text{PP}}_{i}}(\cdot)$; \;
			output the nominal controller for $\veh_i$.\;
			\textbf{Base obstacle induced by $\veh_{i}$} \;
			given the nominal controller ${\ctrl^{\text{PP}}_{i}}(\cdot)$ and the BRS $\brs^{\text{PP}}_{i}(t, \sta_i)$, compute the base obstacles $\boset_i(\cdot)$ using equations (25), (31) or (37) in \cite{Chen2016d}, depending on the information assumed to be known about the higher-priority vehicles.
		}
		\end{small}
\end{algorithm}

%