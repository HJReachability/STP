% !TEX root = ../SPP_IoTjournal.tex
\subsection{Separation and Buffer Regions- Case1} \label{sec:case1}

\SBnote{
\begin{itemize}
\item Introduction of the section:
\begin{itemize}
\item Main summary: In the next two sections, we are computing separation between any two vehicles such that only one applies the avoidnace maneuver at a time and atmost $\nva$ vehicles need to apply avoidnace maneuver during the entire duration. 
\item To understand what this means we introduce the notation of avoid start time. Introduce the notation and give its mathematical definition in terms of relative state and avoidnace set computed in the previous section. Now explain that only one vehicle need to apply avoidance control at a time is equivalent to $t_i \neq t_j$. Only $\nva$ vehicles need to apply avoidnace control is equivalent to $t_i - t_j \geq \brd$.  
\item If we find the set of all states of $\veh_i$ such that this separation is ensured between $\veh_i$ and $\veh_m$ for all $m<i$, then during the path planning of $\veh_i$, we can then ensure that $\veh_i$ is indeed in one of these states and we will be done (as the path planning is sequential). So we next focus on finding all the states of $\veh_i$ at time $t$ such that this separation is ensured between two vehicles $\veh_j$ and $\veh_i$, $j <i$ at time $t$ for all intruding scenarios.
\end{itemize}
\item Section1: $t_j < t_i$.
\begin{itemize}
\item We next make the following observations: (1) it is sufficient to consider the scenarios where $\tsa$ is within $\iat$ distance. (2) WLOG, initial state of the intruder belongs to a boundary of vehicle $j$.
\item Section11: First find the set of all states of intruder for which $\veh_j$ is forced to apply an avoidance maneuver. Explain what that set is in relative co-ordinates for different $\tsa$. To compute the set in absolute co-ordinates we need to augment the relative states to the states of $\veh_j$. Explain, how we can compute these states of $\veh_j$ (by FRS).
\item Section12: Next we want to make sure that $\veh_i$ is far away from this set such that the intruder will need atleast a duration of $\brd$ to reach any state such that $\state_{\intr,i}$ is on the boundary of the avoid set of $\veh_i$. In relative co-ordinates, if we compute min-min set and augment it on set in Section 11, and ensure that the boundary of the avoid region is outside this augmented set then we are done. Draw a table listing the duration of each set. Mention the duration relationships. Draw a diagram showing different sets. 
\item Section13: Obstacle computation. First compute the directly induced obstacle by $\veh_j$ at time $t$ for all possible $\tsa$. Then compute the obstacle that $\veh_i$ need to avoid at any future time for different $\tsa$. Therefore, compute the BRS that it needs to avoid at $\tsa$.    
\end{itemize}
\item Section2: $t_i < t_j$.
\end{itemize}
}

In the next two sections, our goal is to ensure that any two vehicles are separated enough from each other so that %only one of the vehicles is \textit{forced} to apply the intruder avoidance maneuver at any given time. Moroever, 
atmost $\nva$ vehicles should be \textit{forced} to apply avoidnace maneuver during the duration $[\tsa, \tea]$. To capture this mathematically, for every vehicle $\veh_m$, we define 
\begin{equation*}
\avoidt_{m} := \{t: \state_{\intr,m}(t) \in \brs^{\text{A}}_{m}(t-\tsa, \iat), t \in [\tsa, \tea]\}
\end{equation*} 
$\avoidt_{m}$ is the set of all times at which $\veh_m$ is forced to apply an intruder avoidance maneuver. We also define \textit{avoid start time}, $\tsa_m$, for $\veh_m$ as:
\begin{equation}
\tsa_m  = 
\left \{ 
\begin{array}{ll}
\min_{t \in  \avoidt_{m}} t & \mbox{if~} \avoidt_{m} \neq \emptyset\\
\infty & \mbox{otherwise}
\end{array}
\right.
\end{equation}  
Therefore, $\tsa_m \in [\tsa, \tea]$ denotes the first time at which $\veh_m$ applies an avoidance maneuver and defined to be $\infty$ if $\veh_m$ never applies an avoidance maneuver. %Only one vehicle applying the avoidance maneuver at a time is thus equivalent to $\forall i \neq j \tsa_i \neq \tsa_j$ whenever . Similarly, 
Therefore, if we ensure that 
\begin{equation} \label{eqn:sep_cond}
\forall i \neq j, min(\tsa_i, \tsa_j)< \infty \implies |\tsa_i - \tsa_j| \geq \frac{\iat}{\nva} := \brd,
\end{equation}
then atmost $\nva$ vehicles are forced to apply the avoidance maneuver during the time interval $[\tsa, \tea]$. We refer to the condition in \eqref{eqn:sep_cond} as \textit{separation requirement} hereon. 

For any given time $t$, if we could find the set of all states of $\veh_i$ such that the separation requirement holds for $\veh_i$ and $\veh_m$ for all $m<i$ and for all intruder strategies, then during the path planning of $\veh_i$, we can ensure that $\veh_i$ is in one of these states at time $t$. The sequential path planning will therefore guarantee that the separation requirement holds for every SPP vehicle pair. Thus, hereon we focus on finding all states $\state_i(t)$ such that the separation requirements are ensured between vehicles $\veh_j$ and $\veh_i$, $j <i$ at time $t$ for all possible intruder scenarios (meaning all possible $\tsa, \tea, \state_{\intr}^0$ and $\ctrl_{\intr}(\cdot))$. For our analysis, we consider the following two mutually exclusive and exhaustive cases: $\infty < \tsa_j \leq \tsa_i$ and $\infty < \tsa_i < \tsa_j$. %As we will see shortly, our construction ensures that $\tsa_i \neq \tsa_j$.
In this section, we consider Case1: $\infty < \tsa_j \leq \tsa_i$. Case2 is discussed in the next section.  

In Case1, the intruder forces $\veh_j$ to apply avoidance control before or at the same time as $\veh_i$. %If $\tsa_i = \tsa_j = \infty$, then none of $\veh_i$ and $\veh_j$ need to apply avoidance control and hence trivially won't be involved in re-planning. 
To ensure the separation requirement in this case, we begin with the following two observations which narrow down the intruder scenarios that we need to consider:

\begin{observation} \label{obs1_case1}
At any time $t$, it is sufficient to consider the intruder scenarios where $\tsa \in [t-\iat, t]$. This is because if $\tsa < t - \iat$, then $\veh_i$ and/or $\veh_j$ will already be in the replanning phase at time $t$ (see assumption \ref{as:avoidOnce}) and hence we need not ensure separation requirements for them during the planning phase. On the other hand, if $\tsa > t$, then we need not ensure separation requirements at time $t$ as the intruder has not appeared in the system yet. %\SBnote{Actually, we compute obstacles at time $t'$ in a way such that their effect doesn't propagate to $t < t'$, but not sure if we need to mention this.}
\end{observation}

\begin{observation} \label{obs2_case1}
Without loss of generality, we can assume that the intruder appears at the boundary of the avoid region of $\veh_j$, e.g. $\state_{\intr, j}(\tsa) \in \partial \brs^{\text{A}}_{j}(0, \iat)$. Otherwise, vehicles need not account for the intruder until it reaches the boundary of the avoid region of $\veh_j$. Also note that since $\infty < \tsa_j \leq \tsa_i$, $\veh_{\intr}$ reaches the boundary of the avoid region of $\veh_j$ first. 
\end{observation}