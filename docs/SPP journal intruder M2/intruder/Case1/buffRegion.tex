% !TEX root = ../../SPP_IoTjournal.tex
\subsubsection{Buffer Region} \label{sec:buffRegion_case1}
\SBnote{Start from here.}
From Section \ref{sec:sepRegion_case1}, we know that $\state_{\intr}^0 := \state_{\intr}(\tsa) \in \sep_j(\tsa) := \sep_j(\tsa_j)$, where last equivalence holds because $\tsa_j = \tsa$ (by Observation \ref{obs1_case1}). Moroever, since avoid start time of $\veh_i$ is $\tsa_i$, we must have $\state_{\intr, i}(\tsa_i) \in \partial \brs^{\text{A}}_{i}(0, \iat)$.

In this section, we first compute the set of all relative states $\brs^{\text{B}}_{i}$ from which it is possible to reach $\brs^{\text{A}}_{i}(0, \iat)$ for some control applied by $\veh_{\intr}$ and $\veh_i$. If we augment this relative coordinates set $\brs^{\text{B}}_{i}$ on $\sep_j(\tsa)$ to get $\buff_{ij}(\tsa)$, then we can guarantee that $\veh_{\intr}$ cannot reach the boundary (or inside) of $\brs^{\text{A}}_{i}(0, \iat)$ before $t = \tsa + \brd$ irrespective of the control applied by $\veh_{\intr}$ and $\veh_i$ during $[\tsa, t]$, if $\state_{\intr}(\tsa) \in \sep_j(\tsa)$ and $\state_i(\tsa) \in \left(\buff_{ij}(\tsa)\right)^C$. Equivalently, $(\tsa_i - \tsa_j) \geq \brd$, if $\state_i(\tsa) \in \left(\buff_{ij}(\tsa)\right)^C$ ($\state_{\intr}(\tsa) \in \sep_j(\tsa)$ from Section \ref{sec:sepRegion_case1} and $\tsa = \tsa_j$ by Observation 1).

We refer to the set $\buff_{ij}(\tsa)$ as \textit{buffer region} hereon. To compute $\buff_{ij}(\tsa)$, we first compute $\brs^{\text{B}}_{i}$ which is given by the following BRS:
\begin{equation} \label{eqn:buffBRS_case1}
\begin{aligned}
\brs^{\text{B}}_{i}(0, \brd) = & \{y: \exists \ctrl_i(\cdot) \in \cfset_i, \exists \ctrl_\intr(\cdot) \in \cfset_\intr, \exists \dstb_i(\cdot) \in \dfset_i, \\
& \exists \dstb_\intr(\cdot) \in \dfset_\intr, \state_{i, \intr}(\cdot) \text{ satisfies \eqref{eq:reldyn}},\\
& \exists s \in [0, \brd], \state_{i, \intr}(s) \in -\brs^{\text{A}}_{i}(0, \iat),\\
& \state_{i, \intr}(t) = y\},
\end{aligned}
\end{equation}
where 
\begin{equation} 
H^{\text{B}}_{i}(\state_{i, \intr}, \costate) = \min_{\ctrl_i \in \cset_i, \ctrl_\intr \in \cset_\intr, \dstb_i \in \dset_i, \dstb_\intr \in \dset_\intr} \costate \cdot f_r(\state_{i, \intr}, \ctrl_i, \ctrl_\intr, \dstb_i, \dstb_\intr)
\end{equation}

Note that we use $-\brs^{\text{A}}_{i}(0, \iat)$ as the target set for our computation above because we are computing BRS $\brs^{\text{B}}_{i}(0, \brd)$ using the relative state $\state_{i, \intr}$ and not $\state_{\intr, i}$. Also, since $\veh_{\intr}$ appears for a maximum duration of $\iat$, it is sufficient to ensure that $\veh_{\intr}$ and $\veh_i$ do not reach the set $\brs^{\text{A}}_{i}(\brd, \iat)$ instead of $\brs^{\text{A}}_{i}(0, \iat)$, as a duration of $\brd$ is already spent by $\veh_{\intr}$ to reach the avoid region of $\veh_i$.

The buffer region can not be computed by augmenting $\brs^{\text{B}}_{i}(0, \brd)$ on the separation region of the $\veh_j$:
\begin{equation} \label{eqn:buffRegion_case1}
\buff_{ij}(\tsa) = \sep_j(\tsa) + \brs^{\text{B}}_{i}(0, \brd).
\end{equation} 
We can thus ensure that $(\tsa_i - \tsa_j) \geq \brd$ as long as $\state_i(\tsa) \in \left(\buff_{ij}(\tsa)\right)^C$.