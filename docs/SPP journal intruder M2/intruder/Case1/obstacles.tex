% !TEX root = ../../SPP_IoTjournal.tex
\subsubsection{Obstacle Computation} \label{sec:intruderObs_case1}
In sections \ref{sec:sepRegion_case1} and \ref{sec:buffRegion_case1}, we computed a separation between $\veh_i$ and $\veh_j$ such that if $\tsa_i - \tsa_j \geq \brd$. However, we need to make sure that while applying avoidance control these vehicles do not enter in each other's danger zones or collide with the static obstacles. In this section, we compute the obstacles that reflect this possibility. In particular, we want to find the set of states that $\veh_i$ needs to avoid to avoid entering in the danger zone of $\veh_j$. To find such states, we consider the following two cases:
\begin{enumerate}
\item CaseA: intruder affects $\veh_j$, but not $\veh_i$.
\item CaseB: intruder first affects $\veh_j$ and then $\veh_i$.
\end{enumerate}
For each case, we compute the set of states that $\veh_i$ needs to avoid at time $t$ to avoid entering in $\dz_{ji}$. Let $_1^A\ioset_i^j(\cdot)$ and $_1^B\ioset_i^j(\cdot)$ denote the set of obstacles corresponding to CaseA and CaseB respectively. We begin with the following observation: 
\begin{observation} \label{obs1_case1Obs}
To compute obstacles at time $t$, it is sufficient to consider the scenarios where $\tsa \in [t-\iat, t]$. This is because if $\tsa < t - \iat$, then $\veh_j$ and/or $\veh_i$ will already be in the replanning phase at time $t$ (see assumption \ref{as:avoidOnce}) and hence the two vehicles cannot be in conflict at time $t$. On the other hand, if $\tsa > t$, then $\veh_j$ wouldn't apply any avoidance maneuver at time $t$. 
\end{observation}

\begin{itemize}[leftmargin=*] 
\item \label{sec:intruderObs_case1_caseA} CaseA: Note that since $\tsa_j = \tsa$ (by Observation \ref{obs1_case1}), $_1^A\ioset_i^j(\cdot)$ is given by the states that $\veh_j$ can reach while avoiding the intruder, starting from some state in $\boset_j(\tsa), \tsa \in [t-\iat, t]$. These states can be obtained by computing a FRS from the base obstacles.
\begin{equation} \label{eq:ObsFRS_case1_caseA}
\begin{aligned}
\frs_{j}^{\mathcal{O}}(\tsa, t) = & \{y: \exists \ctrl_j(\cdot) \in \cfset_j, \exists \dstb_j(\cdot) \in \dfset_j, \\
& \state_j(\cdot) \text{ satisfies \eqref{eq:dyn}}, \state_j(\tsa) \in \boset_j(\tsa), \\
& \state_j(t) = y\}.
\end{aligned}
\end{equation}
$\frs_{j}^{\mathcal{O}}(\tsa, t)$ represents the set of all possible states that $\veh_j$ can reach after a duration of $(t-\tsa)$ starting from inside $\boset_j(\tsa)$. This FRS can be obtained by solving the HJ VI in \eqref{eq:HJIVI_FRS} with the following Hamiltonian:
\begin{equation}
\ham_{j}^{\mathcal{O}}(\state_j, \costate) = \max_{\ctrl_j \in \cset_j} \max_{\dstb_j \in \dset_j} \costate \cdot f_j (\state_j, \ctrl_j, \dstb_j).
\end{equation} 
Since $\tsa \in [t-\iat, t]$, the induced obstacles in this case can be obtained as:
\begin{equation} \label{eq:intruderObs_case1_caseA} 
\begin{aligned}
_1^A\ioset_i^j(t) & = \{\state_i: \exists y \in \pfrs_j(t), \|\pos_i - y\|_2 \le \rc \}\\
\pfrs_j(t) & = \{p_j: \exists \npos_j, (p_j, \npos_j) \in \bigcup_{\tsa \in [t-\iat, t]} \frs_{j}^{\mathcal{O}}(\tsa, t) \}
\end{aligned}
\end{equation}

\begin{observation} \label{obs1_case1_caseA}
Since the base obstacles represent possible states that a vehicle can be in in the absence of an intruder, the base obstacle at any time $\tau_2$ is contained within the FRS of the base obstacle at any time $\tau_1 (< \tau_2)$, computed forward for a duration of $(\tau_2-\tau_1).$ That is, $\boset_j(\tau_2) \subseteq \frs_{j}^{\mathcal{O}}(\tau_1, \tau_2)$, where $\frs_{j}^{\mathcal{O}}(\tau_1, \tau_2)$, as before, denotes the FRS of $\boset_j(\tau_1)$ computed forward for a duration of $(\tau_2-\tau_1)$. The same argument can be applied for the FRS computed from $\boset_j(\tau_2)$ and $\boset_j(\tau_1)$, i.e. $\frs_{j}^{\mathcal{O}}(\tau_2, \tau_3) \subseteq \frs_{j}^{\mathcal{O}}(\tau_1, \tau_3)$, where $\tau_1 < \tau_2 < \tau_3$.
%Note that by the definition of base obstacles, $\boset_j(t+\tau_2) \subset \frs_{j}^{\text{BO}}(0, \tau_2-\tau_1) ~\forall t, \tau_2 > \tau_1$, where $\frs_{j}^{\text{BO}}(0, \tau_2-\tau_1)$ denotes the FRS of $\boset_j(t+\tau_1)$ computed for a duration of $\tau_2-\tau_1$. Therefore, we have that $\frs_{j}^{\mathcal{O}}(0, \tau) \subset \frs_{j}^{\mathcal{O}}(0, \iat) ~\forall \tau \in [0, \iat)$. 
\end{observation}

Using observation \ref{obs1_case1_caseA}, $\pfrs_j(t)$ in \eqref{eq:intruderObs_case1_caseA} can be equivalently written as:
\begin{equation} \label{eq:intruderObs_case1_caseA_inter}
\pfrs_j(t) = \{p_j: \exists \npos_j, (p_j, \npos_j) \in \frs_{j}^{\mathcal{O}}(t-\iat, t) \}.
\end{equation}

\item \label{sec:intruderObs_case1_caseB} CaseB: In this case, the intruder first affects $\veh_j$ and then $\veh_i$. Therefore, we have $\tsa_j, \tsa_i \in [\tsa, \tea]$. Once $\veh_j$ starts applying avoidance control at time $\tsa = \tsa_j$, it might deviate from its planned trajectory. Thus from the perspective of $\veh_i$, $\veh_j$ can apply any control during $[\tsa, \tea]$. Furthermore, $\veh_i$ itself might need to apply avoidnace maneuver during $[\tsa_i, \tea]$. Thus, the main challenge in this case is to ensure that $\veh_i$ and $\veh_j$ do not enter each other's danger zones even when both have deviated from their planned trajectory and hence can apply \textit{any} control from each other's perspective. Thus at time $t$, $\veh_i$ not only need no avoid the states that $\veh_j$ could be in at time $t$, but also all the states that could lead it to $\dz_{ji}$ \textit{in future} under some control actions of $\veh_i$ and $\veh_j$. To compute this set of states, we make the following key observation:
\begin{observation} \label{obs1_case1_caseB}
For computing $_1^B\ioset_i^j(t)$, it is sufficient to consider $\tsa_i = t$. If $\tsa_i > t$, then $\veh_i$ is not applying any avoidance maneuver at time $t$ and hence should only avoid the states that $\veh_j$ could be in at time $t$. However, this is already ensured during computation of $_1^A\ioset_i^j(t)$. If $\tsa_i < t$, then for a given $\tsa$, $\veh_i$ still needs to avoid the same set of states at time $t$, as it would have if $\tsa_i = t$.  
\end{observation}

Due to the separation and buffer regions, we have $\tsa_i - \tsa_j \geq \brd$. Combining this with Observations \ref{obs1_case1Obs} and \ref{obs1_case1_caseB} implies that $\tsa_j \in [t-\iat, t-\brd]$. From the perspective of $\veh_i$, $\veh_j$ can reach any state in $\frs_{j}^{\mathcal{O}}(\tsa_j, t')$ at time $t' \in [\tsa_j, \tsa_j+\iat]$ starting from some state in $\boset_j(\tsa_j)$ at time $\tsa_j$. Here, $\frs_{j}^{\mathcal{O}}(\tsa_j, t')$ represents the FRS of $\boset_j(\tsa_j)$ computed for a duration of $(t' - \tsa_j)$ and is given by \eqref{eq:ObsFRS_case1_caseA}. Taking into account all possible $\tsa_j \in [t-\iat, t-\brd]$, $\state_j(\tau)$ is contained in the set:
\begin{equation}
\mathcal{K}^{\text{B1}}(\tau) = \bigcup_{\tsa_j \in [\tau-\iat, t-\brd]} \frs_{j}^{\mathcal{O}}(\tsa_j, \tau)
\end{equation} 
at time $\tau \in [t, t-\brd+\iat]$. From Observation \ref{obs1_case1_caseA}, we have $\frs_{j}^{\mathcal{O}}(\tsa_j, \tau) \subseteq \frs_{j}^{\mathcal{O}}(\tau-\iat, \tau)$ for all $\tsa_j \in [\tau-\iat, t-\brd]$. Therefore, $\mathcal{K}^{\text{B1}}(\tau) = \frs_{j}^{\mathcal{O}}(\tau-\iat, \tau)$.

From the perspective of $\veh_i$, it needs to avoid all states at time $t$ that can reach $\mathcal{K}^{\text{B1}}(\tau)$ for some control action of $\veh_i$ during time duration $[t, \tau]$. This will ensure that $\veh_i$ and $\veh_j$ will not enter into each other's danger zones regardless of the avoidance maneuver applied by them. This set of states is given by the following BRS:

\begin{equation}  \label{eq:ObsBRS_case1_caseB}
\begin{aligned}
\brs^{\text{B1}}_{i}(t, t-\brd+\iat) = & \{y: \exists \ctrl_i(\cdot) \in \cfset_i, \exists \dstb_i(\cdot) \in \dfset_i, \\
& \state_i(\cdot) \text{ satisfies \eqref{eq:dyn}}, \state_i(t) = y, \\
& \exists s \in [t, t-\brd+\iat],\\
& \state_i(s) \in \tilde{\mathcal{K}}^{\text{B1}}(s)\},
\end{aligned}
\end{equation}
where
\begin{equation*}
\tilde{\mathcal{K}}^{\text{B1}}(s) = \{\state_j: \exists (y, h) \in \mathcal{K}^{\text{B1}}(s), \|\pos_j - y\|_2 \le \rc \}.
\end{equation*} 
The Hamiltonian $\ham^{\text{B1}}_{i}$ to compute $\brs^{\text{B1}}_{i}(\cdot)$ is given by:
\begin{equation} \label{eqn:BRS_obsham_case1_caseB}
\ham^{\text{B1}}_{i}(\state_i, \costate) = \min_{\ctrl_i \in \cset_i} \min_{\dstb_i \in \dset_i} \costate \cdot f_i (\state_i, \ctrl_i, \dstb_i).
\end{equation}
Finally, the induced obstacle in this case can be obtained as:
\begin{equation} \label{eq:intruderObs_case1_caseB} 
_1^B\ioset_i^j(t) = \brs^{\text{B1}}_{i}(t, t-\brd+\iat).
\end{equation}
\end{itemize}