% !TEX root = ../../SPP_IoTjournal.tex
\subsubsection{Separation region} \label{sec:sepRegion_case1}
Recall that the separation region denotes the set of states of the intruder for which a vehicle is forced to apply an avoidance maneuver. In this section our goal is to find $\sep_j(\tsa_j)$, the separation region of $\veh_j$ at the avoid start time. By virtue of Observation \ref{obs1_case1}, we will use $\tsa_j$ and $\tsa$ interchangeably here on.
%
%Consider any $\tsa \in \R$. In this section, our goal is to find the set of all states $\state_{\intr}^0 := \state_{\intr}(\tsa)$ for which $\veh_j$ is forced to apply an avoidance maneuver. We refer to this set as \textit{separation region}, and denote it as $\sep_j(\tsa)$. 

As discussed in Section \ref{sec:intruder_avoid}, $\veh_j$ needs to apply avoidance maneuver at time $\tsa$ only if $\state_{\intr j}(\tsa) \in \partial \brs^{\text{A}}_{j}(0, \iat)$. To compute set $\sep_j(\tsa)$, we thus need to translate these relative states to a set in the state space of the intruder. Therefore, if all possible states of $\veh_j$ at time $\tsa$ are known, then $\sep_j(\tsa)$ can be trivially computed.

Recall from Section \ref{sec:distb} that the base obstacle $\boset_j(t)$ at time $t$ represents all possible states of $\veh_j$ at time $t$, if the intruder doesn't appear in the system until that time. This is precisely the set that we are interested in to compute the separation region.
%
%at time $t$ represents all states that $\veh_j$ can be in at time $t$ in the presence of disturbances, but in the absence of an intruder. In particular, if the intruder doesn't appear in the system until time $t$, then  captures all possible states of $\veh_j$ at time $t$, precisely the set that we are interested in to compute the separation region.
Depending on the information known to a lower-priority vehicle $\veh_i$ about $\veh_j$'s control strategy, we can use one of the three methods described in Section 5 in \cite{chen2016robust} (and Section \ref{sec:distb} of this paper) to compute the base obstacles $\boset_j(\tsa)$. In particular, the base obstacles are respectively given by equations (25), (31) and (37) in \cite{chen2016robust} for the centralized control, the least restrictive control and the robust trajectory tracking algorithms (the three proposed algorithms to account for disturbances in STP). We will explain the computation of the base obstacles further in Section \ref{sec:path_planning}.

Given $\boset_j(\tsa)$, $\sep_j(\tsa)$ can be obtained as:
\begin{equation} \label{eqn:sepRegion_case1}
\sep_j(\tsa) = \boset_j(\tsa) + \partial \brs^{\text{A}}_{j}(0, \iat), ~\tsa \in \R,
\end{equation}
where the ``$+$'' in \eqref{eqn:sepRegion_case1} denotes the Minkowski sum. Since $\sep_j(\tsa)$ represents the set of all states of $\veh_\intr$ for which $\veh_j$ must apply an avoidance maneuver, Observation \ref{obs1_case1} implies that it is sufficient to consider the scenarios where $\state_{\intr}^0 := \state_{\intr}(\tsa) \in \sep_j(\tsa)$.