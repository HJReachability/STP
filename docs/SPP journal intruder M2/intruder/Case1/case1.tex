% !TEX root = ../../SPP_IoTjournal.tex
\subsection{Separation and Buffer Regions - Case 1} \label{sec:case1}
In the next two sections, our goal is to compute separation and buffer regions for STP vehicles so that %only one of the vehicles is \textit{forced} to apply the intruder avoidance maneuver at any given time. Moreover, 
at most $\nva$ vehicles are \textit{forced} to apply an avoidance maneuver during the duration $[\tsa, \tsa+\iat]$. To capture this mathematically, we define 
\begin{equation} \label{eqn:avoidStartTime}
\avoidt_{i} := \{t: \state_{\intr i}(t) \in \partial\brs^{\text{A}}_{i}(t-\tsa, \iat), t \in [\tsa, \tsa+\iat]\}
\end{equation} 
$\avoidt_{i}$ is the set of all times at which $\veh_i$ must apply an avoidance maneuver. We also define \textit{avoid start time} $\tsa_i$
\begin{equation} \label{eqn:avoidStartTime2}
\tsa_i  = 
\left \{ 
\begin{array}{ll}
\min_{t \in  \avoidt_{i}} t & \mbox{if~} \avoidt_{i} \neq \emptyset\\
\infty & \mbox{otherwise}
\end{array}
\right.
\end{equation}  
Therefore, $\tsa_i$ denotes the first time at which $\veh_i$ applies an avoidance maneuver and defined to be $\infty$ if it never applies one. From \eqref{eqn:avoidStartTime} it follows that at the avoid start time, $\state_{\intr i}(\tsa_i) \in  \partial\brs^{\text{A}}_{i}(\tsa_i-\tsa, \iat)$. %Only one vehicle applying the avoidance maneuver at a time is thus equivalent to $\forall i \neq j \tsa_i \neq \tsa_j$ whenever . Similarly,
 
If we divide the duration of $\iat$ into $\nva$ intervals and ensure that atmost one vehicle can be forced to apply the avoidance maneuver during that entire interval, then, by construction, it is guaranteed that at most $\nva$ vehicles apply the avoidance maneuver during the time interval $[\tsa, \tsa+\iat]$. Mathematically, if we ensure that 
\begin{equation} \label{eqn:sep_cond}
\forall i \neq j, \min(\tsa_i, \tsa_j)< \infty \implies |\tsa_i - \tsa_j| \geq \brd,
\end{equation}
then at most $\nva$ vehicles can be forced to apply the avoidance maneuver by the intruder. We refer to the condition in \eqref{eqn:sep_cond} as \textit{separation requirement} from here on. In sections \ref{sec:case1} and \ref{sec:case2}, our aim is to design buffer regions such that the separation requirement is satisfied.   

For any given time $t$, if we could find the set of all states of $\veh_i$ such that the separation requirement could be violated between $\veh_i$ and $\veh_j$ for some $j<i$ and for some intruder strategy, then during the trajectory planning of $\veh_i$, it can be ensured that $\veh_i$ is not in one of these states at time $t$ by using this set of states as ``obstacles". The sequential trajectory planning will therefore guarantee that the separation requirement holds for every STP vehicle pair. Thus, we can focus on finding all states $\state_i(t)$ such that the separation requirements could be violated between vehicles $\veh_j$ and $\veh_i$, $j <i$ at time $t$ for some possible intruder scenario (meaning some possible $\tsa, \state_{\intr}^0$ and $\ctrl_{\intr}(\cdot))$. For this analysis, it is sufficient to consider the following two mutually exclusive and exhaustive cases: 
\begin{enumerate}
\item Case 1: $\tsa_j \leq \tsa_i, \tsa_j < \infty$
\item Case 2: $\tsa_i < \tsa_j, \tsa_i <\infty$
\end{enumerate}
In this section, we consider Case 1. Case 2 is discussed in the next section.  

In Case 1, the intruder forces $\veh_j$, the higher-priority vehicle, to apply avoidance control before or at the same time as $\veh_i$, the lower-priority vehicle. %If $\tsa_i = \tsa_j = \infty$, then none of $\veh_i$ and $\veh_j$ need to apply avoidance control and hence trivially won't be involved in re-planning. 
To ensure the separation requirement in this case, we begin with the following observation which narrows down the intruder scenarios that we need to consider:
\begin{observation} \label{obs1_case1}
Since $\tsa_j \leq \tsa_i$, $\veh_{\intr}$ reaches the boundary of the avoid region of $\veh_j$ in this case before it reaches the boundary of the avoid region of $\veh_i$. Without loss of generality, we can assume that the intruder \textit{appears} in the system at the boundary of the avoid region of $\veh_j$, i.e., $\state_{\intr j}(\tsa) \in \partial \brs^{\text{A}}_{j}(0, \iat)$. Equivalently, we can assume that $\tsa_j = \tsa$. Otherwise, vehicles $\veh_j$ and $\veh_i$ need not account for the intruder until it reaches the boundary of the avoid region of $\veh_j$. Thus, we will use $\tsa_j$ and $\tsa$ interchangeably in Case 1.
\end{observation}