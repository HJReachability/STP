% !TEX root = ../../SPP_IoTjournal.tex
\subsubsection{Buffer Region} \label{sec:buffRegion_case2}
The idea behind the design of buffer region is same as that in Case 1: we want to make sure that $\veh_{\intr}$ spends at least a duration of $\brd$ to go from the boundary of the avoid region of one SPP vehicle to the boundary of the avoid region of some other SPP vehicle. Mathematically, we want to compute the set of all states $\state_{\intr}$ such that if $\veh_{\intr}$ starts in this set at time $t$, it cannot reach $\sep_j(\cdot)$ before $t_1 = t + \brd$, regardless of the control applied by $\veh_j$ and $\veh_{\intr}$ during interval $[t, t_1]$. Similar to Section \ref{sec:buffRegion_case1}, this set is given by $\brs^{\text{B}}_{j}(0, \brd)$: 
\begin{equation} \label{eqn:buffBRS_case1}
\begin{aligned}
\brs^{\text{B}}_{j}(0, \brd) = & \{y: \exists \ctrl_j(\cdot) \in \cfset_j, \exists \ctrl_\intr(\cdot) \in \cfset_\intr, \exists \dstb_j(\cdot) \in \dfset_j, \\
& \exists \dstb_\intr(\cdot) \in \dfset_\intr, \state_{\intr, j}(\cdot) \text{ satisfies \eqref{eq:reldyn}},\\
& \exists s \in [0, \brd], \state_{\intr, j}(s) \in \brs^{\text{A}}_{j}(\brd, \iat),\\
& \state_{\intr, j}(t) = y\},
\end{aligned}
\end{equation}
where 
\begin{equation}
H^{\text{B}}_{j}(\state_{\intr, j}, \costate) = \min_{\substack{\ctrl_j \in \cset_j, \ctrl_\intr \in \cset_\intr, \\ \dstb_j \in \dset_j, \dstb_\intr \in \dset_\intr}} \costate \cdot f_r(\state_{\intr, j}, \ctrl_j, \ctrl_\intr, \dstb_j, \dstb_\intr)
\end{equation}

In absolute coordinates, we thus have that if the intruder starts outside $\tilde{\buff}_{ji}(t) = \boset_j(t) + \brs^{\text{B}}_{j}(0, \brd)$ at time $t$, then it cannot reach $\sep_j(\cdot)$ before time $t + \brd$. Finally, if we can ensure that the avoid region of $\veh_i$ at time $t$ is outside $\tilde{\buff}_{ji}(t)$, then $\state_{\intr, i}(\tsa) := \state_{\intr, i}(\tsa_i) \in \partial \brs^{\text{A}}_{i}(0, \iat)$ implies that $\tsa_j - \tsa_i \geq \brd$. Mathematically, if we define the set,  
\begin{equation} \label{eqn:buffRegion_case2}
\buff_{ji}(\tsa) = \boset_j(\tsa) + \brs^{\text{B}}_{j}(0, \brd) + \left(-\brs^{\text{A}}_{i}(0, \iat)\right),
\end{equation} 
then $(\tsa_j - \tsa_i) \geq \brd$ as long as $\state_i(\tsa) \in \left(\buff_{ji}(\tsa)\right)^C$. Thus, if $\state_i(\tsa) \in \left(\buff_{ji}(\tsa)\right)^C$, then the separation requirement \eqref{eqn:sep_cond} is satisfied for Case 2. Further, if $\state_i(\tsa) \in \left(\buff_{ji}(\tsa) \cup \buff_{ij}(\tsa)\right)^C$, then the separation requirement is satisfied regardless of \textit{any} intruder strategy. Note that we use $-\brs^{\text{A}}_{i}(0, \iat)$ instead of $\brs^{\text{A}}_{i}(0, \iat)$ in \eqref{eqn:buffRegion_case2} because $\brs^{\text{A}}_{i}(0, \iat)$ is computed using the relative state $\state_{\intr, i}$ and we are interested in finding the forbidden states for $\veh_i$ when the intruder is outside $\tilde{\buff}_{ji}(t)$.