% !TEX root = ../SPP_IoTjournal.tex
\subsection{Optimal Avoidance Controller} \label{sec:intruder_avoid}
In this section, our goal is to compute the control policy that a vehicle $\veh_i$ can use to avoid entering in the danger region $\dz_{i\intr}$. To compute the optimal avoidance control, we compute the set of states from which the joint states of $\veh_{\intr}$ and $\veh_i$ can enter danger zone $\dz_{i\intr}$ despite the best efforts of $\veh_i$ to avoid $\veh_{\intr}$. 

We define relative dynamics of the intruder $\veh_{\intr}$ with state $\state_\intr$ with respect to $\veh_i$ with state $\state_i$:
\begin{equation}
\label{eq:reldyn}
\begin{aligned}
\state_{\intr i} &= \state_\intr - \state_i \\
\dot \state_{\intr i} &= f_r(\state_{\intr i}, \ctrl_i, \ctrl_\intr, \dstb_i, \dstb_\intr)
\end{aligned}
\end{equation}
Given the relative dynamics, the set of states from which the joint states of $\veh_{\intr}$ and $\veh_{i}$ can enter danger zone $\dz_{i\intr}$ in a duration of $\iat$ despite the best efforts of $\veh_i$ to avoid $\veh_{\intr}$ is given by the backward reachable set $\brs^{\text{A}}_i(\tau, \iat),~ \tau \in [0, \iat]$:

\begin{equation} \label{eqn:avoidBRS}
\begin{aligned}
\brs^{\text{A}}_{i}(\tau, \iat) = & \{y: \forall \ctrl_i(\cdot) \in \cfset_i, \exists \ctrl_\intr(\cdot) \in \cfset_\intr, \exists \dstb_i(\cdot) \in \dfset_i, \\
& \exists \dstb_\intr(\cdot) \in \dfset_\intr, \state_{\intr i}(\cdot) \text{ satisfies \eqref{eq:reldyn}},\\
& \exists s \in [\tau, \iat], \state_{\intr i}(s) \in \targetset^{\text{A}}_{i}, \state_{\intr i}(\tau) = y\},\\
\targetset^{\text{A}}_{i} = & \{\state_{\intr i}: \|\pos_{\intr i}\|_2 \le \rc\}.
\end{aligned}
\end{equation}
The Hamiltonian to compute $\brs^{\text{A}}_{i}(\tau, \iat)$ is given as:
%\begin{equation}
%\begin{aligned}
%H^{\text{A}}_{i}(\state_{\intr i}, \costate) = & \max_{\ctrl_i \in \cset_i} \left( \right.\\
%&\left. \min_{\ctrl_\intr \in \cset_\intr, \dstb_i \in \dset_i, \dstb_\intr \in \dset_\intr} \costate \cdot f_r(\state_{\intr i}, \ctrl_i, \ctrl_\intr, \dstb_i, \dstb_\intr) \right)
%\end{aligned}
%\end{equation}
\begin{equation}
H^{\text{A}}_{i}(\state_{\intr i}, \costate) = \max_{\ctrl_i \in \cset_i} \min_{\substack{\ctrl_\intr \in \cset_\intr, \\ \dstb_\intr \in \dset_\intr, \\ \dstb_i \in \dset_i}} \costate \cdot f_r(\state_{\intr i}, \ctrl_i, \ctrl_\intr, \dstb_i, \dstb_\intr).
\end{equation}
We refer to $\brs^{\text{A}}_i(\tau, \iat)$ as \textit{avoid region} here on. The interpretation of $\brs^{\text{A}}_{i}(\tau, \iat), \tau < \iat$ is that if $\veh_i$ starts inside this set, i.e., $\state_{\intr i}(t) \in  \brs^{\text{A}}_{i}(\tau, \iat)$, then the intruder can force $\veh_i$ to enter the danger zone $\dz_{i\intr}$ within a duration of $(\iat-\tau)$, regardless of the control applied by the vehicle. If $\veh_i$ starts at the boundary of this set (denoted as $\partial \brs^{\text{A}}_{i}(\cdot, \iat)$), that is, $\state_{\intr i}(t) \in  \partial \brs^{\text{A}}_{i}(\tau, \iat)$, it can \textit{barely} successfully avoid the intruder for a duration of $(\iat-\tau)$ using the optimal avoidance control ${\ctrl^{\text{A}}_{i}}$ 
\begin{equation} \label{eqn:optAvoidCtrl}
{\ctrl^{\text{A}}_{i}} = \arg \max_{\ctrl_i \in \cset_i} \min_{\substack{\ctrl_\intr \in \cset_\intr, \\ \dstb_\intr \in \dset_\intr, \\ \dstb_i \in \dset_i}} \costate \cdot f_r(\state_{\intr i}, \ctrl_i, \ctrl_\intr, \dstb_i, \dstb_\intr).
\end{equation}
\noindent Finally, if $\veh_i$ starts outside this set, i.e., $\state_{\intr i}(t) \in \left( \brs^{\text{A}}_{i}(\tau, \iat)\right)^C$, then  $\veh_i$ and $\veh_{\intr}$ cannot instantaneously enter the danger zone $\dz_{i\intr}$, irrespective of the control applied by them at time $t$. In fact, $\veh_i$ can safely apply any control as long as it is outside the boundary of this set, but will have to apply the avoidance control in \eqref{eqn:optAvoidCtrl} to avoid the intruder once it reaches the boundary (also referred to as \textit{avoidance maneuver} here on).

In the worst case, $\veh_i$ might need to avoid the intruder for a duration of $\iat$ starting at $t = \tsa$; thus, the least we must have is that $\state_{\intr i}(\tsa) \in \left(\brs^{\text{A}}_{i}(0, \iat)\right)^C$ to ensure successful avoidance. Otherwise, regardless of what control a vehicle applies, the intruder can force it to enter the danger zone $\dz_{i\intr}$.
\begin{assumption}
\label{as:detection_range}
$\state_{\intr i}(\tsa) \in \left(\brs^{\text{A}}_{i}(0, \iat)\right)^C$ for all $i \in \{1, \ldots, \N\}$.
\end{assumption}

Intuitively, assumption \ref{as:detection_range} enforces a condition on the detection of the intruder by SPP vehicles. For example, if SPP vehicles are equipped with circular sensors, then assumption \ref{as:detection_range} implies that SPP vehicle must be able to detect a intruder that is within a distance of $\dsen$, where
\begin{equation} \label{eqn:sen_distance}
\dsen = \max\{ \|p_i\|: \exists \npos_i, (p_i, \npos_i) \in \brs^{\text{A}}_{i}(0, \iat) \};
\end{equation} 
otherwise, there exist an intruder control strategy such that $\veh_i$ and $\veh_{\intr}$ will collide irrespective of the control used by $\veh_i$. Thus, $\dsen$ is the \textit{minimum} detection range required by any path-planning algorithm to ensure a successful intruder avoidance for all intruder strategies. In general, assumption \ref{as:detection_range} is required to ensure that the intruder gives the SPP vehicles ``a chance" to react and avoid it. Hence, for analysis to follow, we assume that assumption \ref{as:detection_range} holds. 

Note that although \eqref{eqn:optAvoidCtrl} gives us a provably successful avoidnace control for avoiding the intruder if $\state_{\intr i}(\tsa) \in \left(\brs^{\text{A}}_{i}(0, \iat)\right)^C$, the vehicle may not be able to apply this control beacuse it may lead to a collision with other SPP vehicles. Thus, in general, assumption \ref{as:detection_range} is \textit{only necessary not sufficient} to guarantee intruder avoidance. However, we ensure that the SPP vehicles are always separated enough from each other so that they can apply avoidnace control in \eqref{eqn:optAvoidCtrl} if need be. Thus, for the proposed algorithm, assumption \ref{as:detection_range} is also sufficient for a successful intruder avoidance.
%Under normal circumstances when the intruder $\veh_{\intr}$ is far away, we have $\valfunc^{\text{S}}_{i}(0, \state_{\intr i}(t)) > 0$; as $\veh_{\intr}$ gets closer to $\veh_i$, $\valfunc^{\text{S}}_{i}(0, \state_{\intr i}(t))$ decreases. If $\veh_i$ applies the control ${\ctrl^{\text{S}}_{i}}$ when $\valfunc^{\text{S}}_{i}(0, \state_{\intr i}(t)) = 0$, then collision avoidance between $\veh_i$ and $\veh_{\intr}$ is guaranteed for a duration of $\iat$ under the worst-case intruder control strategy $\ctrl_\intr$.