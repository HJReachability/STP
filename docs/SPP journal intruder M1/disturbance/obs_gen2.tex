% !TEX root = ../SPP_journal.tex
\subsubsection{Least Restrictive Control \label{sec:lrc}}
Here, we again begin with the highest-priority vehicle $\veh_1$ planning its path by computing the BRS $\brs_i^\text{dstb}(t, \sta_i)$ in \eqref{eq:BRS}. However, if there is no centralized controller to enforce the control policy for higher-priority vehicles, weaker assumptions must be made by the lower-priority vehicles to ensure collision avoidance. One reasonable assumption is that all higher-priority vehicles follow the least restrictive control that would take them to their targets. This control would be given by 

\begin{equation}
\label{eq:lrctrl} % least restrictive control
\ctrl_j^\text{lrc}(t, \state_j)\in \begin{cases} \{\ctrl_j^\text{dstb}(t, \state_j) \text{ in } \eqref{eq:opt_ctrl_i}\} \text{ if } \state_j(t)\in \partial \brs_j^\text{dstb}(t, \sta_j), \\
\cset_j  \text{ otherwise}
\end{cases}
\end{equation}

Such a controller allows each higher-priority vehicle to use any controller it desires, except when it is on the boundary of the BRS, $\partial \brs_j^\text{dstb}(t, \sta_j)$, in which case the optimal control $\ctrl_j^\text{dstb}(t, \state_j)$ given by \eqref{eq:opt_ctrl_i} must be used to get to the target safely and on time. This assumption is the weakest assumption that could be made by lower-priority vehicles given that the higher-priority vehicles will get to their targets on time.

Suppose a lower-priority vehicle $\veh_i$ assumes that higher-priority vehicles $\veh_j, j < i$ use the least restrictive control strategy $\ctrl_j^\text{lrc}(t, \state_j)$ in \eqref{eq:lrctrl}. From the perspective of the lower-priority vehicle $\veh_i$, a higher-priority vehicle $\veh_j$ could be in any state that is reachable from $\veh_j$'s initial state $\state_j(\ldt_j) = \state_{j}^0$ and from which the target $\targetset_j$ can be reached. Mathematically, this is defined by the intersection of an FRS $\frs_j^\text{lrc}(\ldt_j, t)$ from the initial state $\state_j(\ldt_j)=\state_{j}^0$ and the BRS $\brs_j^\text{dstb}(t, \sta_j)$ defined in \eqref{eq:BRS} from the target set $\targetset_j$, $\frs_j^\text{lrc}(\ldt_j, t) \cap \brs_j^\text{dstb}(t, \sta_j)$. In this situation, since $\veh_j$ cannot be assumed to be using any particular feedback control, $\frs_j^\text{lrc}(\ldt_j, t)$ is defined as

\begin{equation}
\label{eq:FRS2}
\begin{aligned}
\frs_j^\text{lrc}(\ldt_j, t) = & \{y: \exists \ctrl_j(\cdot)\in\cfset_j, \exists \dstb_j(\cdot) \in \dfset_j, \\
& \state_j(\cdot) \text{ satisfies \eqref{eq:dyn}}, \state_j(\ldt_j) = \state_{j}^0,\\
& \state_j(t) = y\}.
\end{aligned}
\end{equation}

This FRS can be computed by solving \eqref{eq:HJIVI_FRS} with the Hamiltonian

\begin{equation}
\ham_j^\text{lrc}\left(\state_j, \lambda\right) = \max_{\ctrl_j \in \cset_j} \max_{\dstb_j \in \dset_j} \lambda \cdot \fdyn_j(\state_j, \ctrl_j, \dstb_j)
\end{equation}

In turn, the obstacle induced by a higher-priority $\veh_j$ for a lower-priority vehicle $\veh_i$ is as follows:

\begin{equation}
\label{eqn:lrcObs1}
\ioset_i^j(t) = \{\state_i: \exists y \in \pfrs_j(t), \|\pos_i - y\|_2 \le \rc \}, \text{ where}
\end{equation}

\begin{align}
\pfrs_j(t) & = \{p_j: \exists \npos_j, (p_j, \npos_j) \in \boset_j(t) \}, \label{eqn:lrcObs2}\\
\boset_j(t) & = \frs_j^\text{lrc}(\ldt_j, t) \cap \brs_j^\text{dstb}(t, \sta_j). \label{eqn:lrcObs3}
\end{align}

The least restrictive control method can be summarized as follows:
\begin{alg}
\label{alg:lrc}
\textbf{Least restrictive control algorithm}: Given initial conditions $\state_i^0$, vehicle dynamics \eqref{eq:dyn}, target set $\targetset_i$, and static obstacles $\soset_i, i = 1\ldots, \N$, for each $i$,
\begin{enumerate}
\item determine the total obstacle set $\obsset_i(t)$, given in \eqref{eq:obsseti}. In the case $i=1$, $\obsset_i(t) = \soset_i ~ \forall t$;
\item compute the BRS $\brs_i^\text{dstb}(t, \sta_i)$ defined in \eqref{eq:BRS}. The latest departure time $\ldt_i$ is then given by $\arg \sup_t \state_{i}^0 \in \brs_i^\text{dstb}(t, \sta_i)$;
\item compute the FRS $\frs_i^\text{lrc}(t)$ in \eqref{eq:FRS2}. Given $\frs_i^\text{lrc}(\ldt_i, t)$ and $\brs_i^\text{dstb}(t, \sta_i)$, compute the positions that the $\veh_i$ could be in. The set of these positions is given by \eqref{eqn:lrcObs2};
\item compute the induced obstacles $\ioset_k^i(t)$ for each $k>i$ using \eqref{eqn:lrcObs1}.
\end{enumerate}
\end{alg}

\begin{remark}
The centralized control method described in the previous section can be thought of as the ``most restrictive control'' method, in which all vehicles must use the optimal controller at all times, while the least restrictive control method allows vehicles to use any suboptimal controller that allows them to arrive at the target on time. These two methods can be considered two extremes of a spectrum in which varying degrees of optimality is assumed for higher-priority vehicles. Vehicles can also choose a control strategy in the middle of the two extremes, and for example use a control within some range around the optimal control, or use the optimal control unless some condition is met. The induced obstacles and the BRS can then be similarly computed using the corresponding control strategy.
\end{remark}