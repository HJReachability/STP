%%%%%%%%%%%%%%%%%%%%%%%%%%%%%%%%%%%%%%%%%%%%%%%%%%%%%%%%%%%%%%%%%%%%%%%%%%%%%%%%
%2345678901234567890123456789012345678901234567890123456789012345678901234567890
%        1         2         3         4         5         6         7         8

\documentclass[journal]{IEEEtran}  
%\documentclass[12pt, draftcls, onecolumn]{IEEEtran}      

\IEEEoverridecommandlockouts                              % This command is only
                                                          % needed if you want to
                                                          % use the \thanks command
%\overrideIEEEmargins
% See the \addtolength command later in the file to balance the column lengths
% on the last page of the document

\usepackage{mathtools}    % need for sub equations
\usepackage{amsfonts}
\usepackage{graphicx}   % need for figures
\usepackage{subcaption} 
\usepackage{epsfig} 
\usepackage{cancel}
\usepackage{amssymb}
\usepackage{color}
\usepackage{bm}
\usepackage[ruled,vlined,titlenotnumbered]{algorithm2e} 
\usepackage{todonotes} \setlength{\marginparwidth}{2.5cm} 
\usepackage{float}
\usepackage{cite}
\usepackage{enumitem}

\newcommand{\MCnote}{\textcolor{red}}
\newcommand{\SBnote}{\textcolor{red}}

\newcommand{\R}{\mathbb{R}} % Real number
\newcommand{\dist}{\text{dist}} % Distance
\newcommand{\rc}{R_c} % Capture radius
\newcommand{\cradius}{\rc}
\newcommand{\N}{N} % number of agents

\newcommand{\veh}{Q} % vehicle
\newcommand{\intr}{I} % Intruder index
\newcommand{\state}{x} % state
\newcommand{\ctrl}{u} % control
\newcommand{\dstb}{d} % disturbance
\newcommand{\pos}{p} % position
\newcommand{\npos}{h} % non-position states

\newcommand{\traj}{\zeta}
\newcommand{\errstate}{e}

\newcommand{\fdyn}{f} % full dynamics
\newcommand{\cset}{\mathcal{U}} % Control set
\newcommand{\cfset}{\mathbb{U}} % control function set
\newcommand{\dset}{\mathcal{D}} % disturbance
\newcommand{\dfset}{\mathbb{D}} % disturbance function set
\newcommand{\obsset}{\mathcal{G}} % Obstacle (the one used to solve PDE)
\newcommand{\dz}{\mathcal{Z}} % danger zone
\newcommand{\intobs}{\mathcal{M}} % Intermediate obstacles required for the intruder Method2

\newcommand{\valfunc}{V} % value function
\newcommand{\valfuncfwd}{W} % value function for forwards reachable set
\newcommand{\brs}{\mathcal{V}} % backwards reachable set
\newcommand{\frs}{\mathcal{W}} % forwards reachable set
\newcommand{\pfrs}{\mathcal{P}} % projected forwards reachable set
\newcommand{\targetset}{\mathcal{L}} % target set
\newcommand{\ham}{H} % Hamiltonian
\newcommand{\fc}{l} % Final condition
\newcommand{\ic}{l} % Initial condition
\newcommand{\obsfunc}{g} % Obstacle function
\newcommand{\costate}{\lambda}

\newcommand{\disckernel}{\Omega} % Discriminating kernel

\newcommand{\edt}{t^\text{EDT}} % earliest departure time
\newcommand{\ldt}{t^\text{LDT}} % latest departure time
\newcommand{\sta}{t^\text{STA}} % scheduled time of arrival
\newcommand{\ioset}{\mathcal{O}} % Induced obstacle
\newcommand{\boset}{\mathcal{B}} % Base obstacle
\newcommand{\sosetp}{\mathcal{S}} % static obstacle in position space
\newcommand{\soset}{\ioset^\text{static}} % static obstacle in state space
\newcommand{\iat}{t^\text{IAT}} % intruder avoidance time
\newcommand{\wcttr}{t^\text{WC}} % worst case TTR

\newcommand{\basicham}{\ham^\text{basic}}

\newcommand{\tsa}{\underline{t}} % time of start of avoidance
\newcommand{\tea}{\bar{t}} % time of end of avoidance

\newcommand{\errorbound}{\mathcal{E}} % Error ``bubble" between vehicle and tracking reference
\newcommand{\tracklaw}{\kappa} % Robust tracking law

\newtheorem{assumption}{Assumption}
\newtheorem{alg}{Algorithm}
\newtheorem{remark}{Remark}

\title{\LARGE \bf Robust Sequential Trajectory Planning Under Disturbances and Adversarial Intruder}

\author{Mo Chen$^1$, Somil Bansal$^2$, Jaime F. Fisac$^2$, Claire J. Tomlin$^2$
\thanks{This work has been supported in part by NSF under CPS:ActionWebs (CNS-931843), by ONR under the HUNT (N0014-08-0696) and SMARTS (N00014-09-1-1051) MURIs and by grant N00014-12-1-0609, by AFOSR under the CHASE MURI (FA9550-10-1-0567). The research of M. Chen and J. F. Fisac have received funding from the ``NSERC'' program and ``la Caixa" Foundation, respectively.}
\thanks{$^1$Department of Aeronautics and Astronautics, Stanford University. mochen72@stanford.edu}
\thanks{$^2$Department of Electrical Engineering and Computer Sciences, University of California, Berkeley. \{somil, jfisac, tomlin\}@eecs.berkeley.edu}
}

\begin{document}
\maketitle
\thispagestyle{empty}
\pagestyle{empty}

%%%
\begin{abstract}
%Provably safe and scalable multi-vehicle trajectory planning is an important and urgent problem due to the expected increase of automation in civilian airspace in the near future. Although this problem has been studied in the past, there has not been a method that guarantees both goal satisfaction and safety for vehicles with general nonlinear dynamics while taking into account disturbances and potential adversarial agents, to the best of our knowledge. Hamilton-Jacobi (HJ) reachability is the ideal tool for guaranteeing goal satisfaction and safety under such scenarios, and has been successfully applied to many small-scale problems. However, a direct application of HJ reachability in most cases becomes intractable when there are more than two vehicles due to the exponentially scaling computational complexity with respect to system dimension. In this paper, we take advantage of the guarantees HJ reachability provides, and eliminate the computation burden by assigning a strict priority ordering to the vehicles under consideration. Under this sequential trajectory planning (STP) scheme, vehicles reserve ``space-time'' portions in the airspace, and the space-time portions guarantee dynamic feasibility, collision avoidance, and optimality of the trajectories given the priority ordering. With a computation complexity that scales quadratically when accounting for both disturbances and an intruder, and \textit{linearly} when accounting for only disturbances, STP can tractably solve the multi-vehicle trajectory planning problem for vehicles with general nonlinear dynamics in a practical setting. We demonstrate our theory in representative simulations.
Provably safe and scalable multi-vehicle trajectory planning is an important and urgent problem. Although this problem has been studied in the past, there has not been a method that guarantees both goal satisfaction and safety for vehicles with general nonlinear dynamics while taking into account disturbances and potential adversarial agents, to the best of our knowledge. Hamilton-Jacobi (HJ) reachability is the ideal tool for guaranteeing goal satisfaction and safety under such scenarios, and has been successfully applied to many small-scale problems; however, its direct application in most cases becomes intractable when there are more than two vehicles due to the exponentially scaling computational complexity with respect to system dimension. In this paper, we take advantage of the guarantees HJ reachability provides, and eliminate the computation burden by assigning a strict priority ordering to vehicles under consideration. Under this sequential trajectory planning (STP) scheme, vehicles reserve ``space-time'' portions in the airspace. The space-time portions guarantee dynamic feasibility, collision avoidance, and optimality of trajectories given the priority ordering. With a computation complexity that scales quadratically when accounting for both disturbances and an intruder, and \textit{linearly} when accounting for only disturbances, STP can tractably solve the multi-vehicle trajectory planning problem for vehicles with general nonlinear dynamics in a practical setting. We demonstrate our theory in representative simulations. \vspace{-0.8cm}
\end{abstract}

%However, this problem is inherently challenging due to not only the complex interactions that exist between the vehicles under consideration, but also the presence of disturbances and potentially malicious intruders. 

% !TEX root = SPP2.tex
\section{Introduction}
Recently, there has been an immense surge of interest in using unmanned aerial vehicles (UAVs) for civil purposes. The applications of UAVs extend well beyond package delivery, and include aerial surveillance, disaster response, and other important tasks \cite{Tice91, Debusk10, Amazon16, AUVSI16, BBC16}. Many of these applications will involve UAVs flying in an urban environment, potentially in close proximity of humans. As a result, government agencies such as the Federal Aviation Administration (FAA) and National Aeronautics and Space Administration (NASA) of the United States are urgently trying to develop new scalable ways to organize an air space in which potentially thousands of UAVs can fly \cite{FAA13, NASA16}.

One essential problem that needs to be addressed is how a group of vehicles in the same vicinity can reach their destinations while avoiding collision with each other. Several previous studies have attempted to address this problem. In some of these studies, specific control strategies for the vehicles or moving entities are assumed, and approaches such as induced velocity obstacles have been used \cite{Fiorini98, Chasparis05, Vandenberg08}. Other researchers have used ideas involving virtual potential fields to maintain collision avoidance while maintaining a specific formation \cite{Saber02, Chuang07}. Although interesting results emerge from these previous studies, simultaneous trajectory planning and collision avoidance are not considered. 

In the past, trajectory planning and collision avoidance problems in safety-critical systems have been studied using reachability analysis, which provides guarantees on the success and safety of optimal system trajectories \cite{Barron90, Mitchell05, Bokanowski10, Margellos11, Fisac15}. In reachability analysis, one computes the reachable set, defined as the set of states from which the system can be driven to a target set. Reachability analysis has been successfully used in applications involving systems with no more than two vehicles, such as pairwise collision avoidance \cite{Mitchell05}, automated in-flight refueling \cite{Ding08}, two-player reach-avoid games \cite{Huang11}, and many others \cite{Bayen07}.

%In addition to the guarantees reachability theory provides and the evident flexibility of reachability theory for analyzing vastly different systems with nonlinear dynamics, many numerical tools for solving reachability problems are also available, making the approach practically appealing \cite{Mitchell05, Sethian96, Osher02, LSToolbox}.

Despite the advantages of reachability analysis, it cannot be directly applied to scenarios involving complex high dimensional systems such as multi-vehicle systems. The computation of reachable sets involves solving a Hamilton-Jacobi (HJ) partial differential equation (PDE) on a grid representing a discretization of the state space, causing an exponential scaling of computation complexity with respect to the dimension of the system, or roughly speaking, with the number of vehicles present.

In this paper, we build on the work in \cite{Chen15}, and assume a reasonable structure in the multi-vehicle path planning problem. In the sequential path planning (SPP) scheme, vehicles are assigned some priority. Higher-priority vehicles may ignore the lower-priority vehicles, who must take into account the presence of higher-priority vehicles by treating them as induced time-varying obstacles. Unlike the work in \cite{Chen15}, we incorporate disturbances for all vehicles and consider three different assumptions on the information each of the vehicles may have access to, making the sequential path planning substantially more practical. For each of the assumed information patterns, we propose a reachability-based method to compute the induced obstacles that would guarantee collision avoidance as well as successful transit to the destination. We demonstrate and compare our proposed methods through numerical simulations.

% !TEX root = nextUAVsched.tex
\section{Problem Formulation \label{sec:formulation}}
Consider $N$ vehicles $P_i,i=1\ldots,N$, each trying to reach one of $N$ target sets $\target_i,i=1\ldots,N$, while avoiding obstacles and collision with each other. Each vehicle $i$ has states $\x_i\in \R^{n_i}$ and travels on a domain $\amb=\obs \cup \free\in\R^p$, where $\obs$ represents the obstacles that each vehicle must avoid, and $\free$ represents all other states in the domain on which vehicles can move. Each vehicle $i = 1,2,\ldots,N$ moves with the following dynamics for $t\in[\tnow_i, \tf_i]$:

\begin{equation} \label{eq:dyn}
\dotx_i = f_i (t, \x_i, \ctrl_i), \quad\x_i(\ti_i) = \x_i^0 
\end{equation}

\noindent where $\x_i^0$ represents the initial condition of vehicle $i$, and $\ctrl_i(\cdot)$ represents the control function of vehicle $i$. In general, $f_i(\cdot,\cdot,\cdot)$ depends on the specific dynamic model of vehicle $i$, and need not be of the same form across different vehicles. Denote $\pos_i\in\R^p$ the subset of the states that represent the position of the vehicle. Given $\pos_i^0\in\free$, we define the admissible control function set for $P_i$ to be the set of all control functions such that $\pos_i(t) \in \free \forall t\ge \ti_i$. Denote the joint state space of all vehicles $\x \in \R^n$ where $n = \sum_i n_i$, and their joint control $\ctrl$.

We assume that the control functions $\ctrl_i(\cdot)$ are drawn from the set $\ctrlf_i := \{\ctrl_i: [\tnow_i, \tf_i] \rightarrow \ctrlin_i, \text{measurable}$\footnote{
A function $f:X\to Y$ between two measurable spaces $(X,\Sigma_X)$ and $(Y,\Sigma_Y)$ is said to be measurable if the preimage of a measurable set in $Y$ is a measurable set in $X$, that is: $\forall V\in\Sigma_Y, f^{-1}(V)\in\Sigma_X$, with $\Sigma_X,\Sigma_Y$ $\sigma$-algebras on $X$,$Y$.}\} where $\ctrlin_i \in \R^{n^\ctrl_i}$ is the set of allowed control inputs. Furthermore, we assume $f_i(t,\x_i, \ctrl_i)$ is bounded, Lipschitz continuous in $\x_i$ for any fixed $t,\ctrl_i$, and measurable in $t, \ctrl_i$ for each $\x_i$. Therefore given any initial state $\x_i^0$ and any control function $\ctrl_i(\cdot)$, there exists a unique, continuous trajectory $\x_i(\cdot)$ solving (\ref{eq:dyn}) \cite{coddington55}.

The goal of each vehicle $i$ is to arrive at $\target_i \subset \R^{n_i}$ at or before some scheduled time of arrival (STA) $\tf_i$ in minimum time, while avoiding obstacles and danger with all other vehicles. The target sets $\target_i$ can be used to represent desired kinematic quantities such as position and velocity and, in the case of non-holonomic systems, quantities such as heading angle.  $\tnow_i$ can be interpreted as the earliest start time (EST) of vehicle $i$, before which the vehicle may not depart from its initial state. Further, we define $\ti_i$, the latest (acceptable) start time (LST) for vehicle $i$. Our problem can now be thought of as determining the LST $\ti_i$ for each vehicle to get to $\target_i$ at or before the STA $\tf_i$, and finding a control to do this safely. If the LST is before the EST $\ti_i < \tnow_i$, then it is infeasible for vehicle $i$ to arrive at $\target_i$ at or before the STA $\tf_i$. Comparing $\ti_i$ and $\tnow_i$ is feasibility problem that may arise in practice; however, for simplicity of presentation, we will assume that $\tnow_i\le \ti_i \forall i$.

Danger is described by sets $\danger_{ij}(\x_j) \subset \amb$. In general, the definition of $\danger_{ij}$ depends on the conditions under which vehicles $i$ and $j$ are considered to be in an unsafe configuration, given the state of vehicle $j$. Here, we define danger to be the situation in which the two vehicles come within a certain radius $\Rc$ of each other: $\danger_{ij}(\x_j) = \{\x_i: \| \pos_i - \pos_j\|_2 \le \Rc \}$. Such a danger zone is also used by the FAA \cite{paglione99}. An illustration of the problem setup is shown in Figure \ref{fig:formulation}.

\begin{figure}
	\centering
	\includegraphics[width=0.35\textwidth]{"fig/formulation"}
	\caption{An illustration of the problem formulation with three vehicles. Each vehicle $P_i$ seeks to reach its target set $\target_i$ by time $t=\tf_i$, while avoiding physical obstacles $\obs$ and the danger zones of other vehicles.}
	\label{fig:formulation}
\end{figure}

In general, the above problem must be analyzed in the joint state space of all vehicles, making the solution intractable. In this paper, we will instead consider the problem of performing path planning of the vehicles in a sequential manner. Without loss of generality, we consider the problem of first fixing $i=1$ and determining the optimal control for vehicle $1$, the vehicle with the highest priority. The resulting optimal control $\ctrl_1$ sends vehicle $1$ to $\target_1$ in minimum time. 

Then, we plan the minimum time trajectory for each of the vehicles $2,\ldots,N$, in decreasing order of priority, given the previously-determined trajectories for higher-priority vehicles $1,\ldots,i-1$. We assume that all vehicles have complete information about the states and trajectories of higher-priority vehicles, and that all vehicles adhere to their planned trajectories. Thus, in planning its trajectory, vehicle $i$ treats higher-priority vehicles as known time-varying obstacles. 

With the above sequential path planning (SPP) protocol and assumptions, our problem now reduces to the following for vehicle $i$. Given $\x_j(\cdot), j=1,\ldots,i-1$, determine $\ctrl_i(\cdot)$ that maximizes $\ti_i$ and such that $x_i(\tau) \in \target_i, \tau\le \tf_i$.
% !TEX root = ../STP_IoTjournal.tex
\subsection{Time-Varying Reachability Background \label{sec:HJIVI}}
We will be using reachability analysis to compute a backward reachable set (BRS) $\brs$ given some target set $\targetset$, time-varying obstacle $\obsset(t)$, and the Hamiltonian function $\ham$ which captures the system dynamics as well as the roles of the control and disturbance. The BRS $\brs$ in a time interval $[t, t_f]$ will be denoted by

\begin{equation}
\brs(t, t_f) \quad \text{ (backward reachable set)}
\end{equation}

Several formulations of reachability are able to account for time-varying obstacles \cite{Bokanowski11, Fisac15} (or state constraints in general). For our application in STP, we utilize the time-varying formulation in \cite{Fisac15}, which accounts for the time-varying nature of systems without requiring augmentation of the state space with the time variable. In the formulation in \cite{Fisac15}, a BRS is computed by solving the following \textit{final value} double-obstacle HJ VI:

\begin{equation}
\label{eq:HJIVI_BRS}
\begin{aligned}
\max \Big\{ \min \{&D_t \valfunc(t, \state) + \ham(t, \state, \nabla \valfunc(t, \state)), \fc(\state) - \valfunc(t, \state) \}, \\
&-\obsfunc(t, \state) - \valfunc(t, \state) \Big\} = 0, \quad t \le t_f \\
&\valfunc(t_f, \state) = \max\{\fc(\state), -\obsfunc(t_f, \state)\}
\end{aligned}
\end{equation}

%In a similar fashion, the FRS is computed by solving the following \textit{initial value} HJ PDE:
%
%\begin{equation}
%\label{eq:HJIVI_FRS}
%\begin{aligned}
%D_t \valfuncfwd(t, \state) + &\ham(t, \state, \nabla \valfuncfwd(t, \state)) = 0 , \quad t \ge t_0  \\
%&\valfuncfwd(t_0, \state) = \max\{\fc(\state), -\obsfunc(t_0, \state)\}
%\end{aligned}
%\end{equation}
%
In \eqref{eq:HJIVI_BRS}, the function $\ic(\state)$ is the implicit surface function representing the target set $\targetset = \{\state: \ic(\state) \le 0\}$. Similarly, the function $\obsfunc(t, \state)$ is the implicit surface function representing the time-varying obstacles $\obsset(t) = \{\state: \obsfunc(t,\state)\le 0\}$. The BRS $\brs(t, t_f)$ is given by

%\begin{equation}
%\label{eq:implicitValfuncs}
%\begin{aligned}
%\brs(t, t_f) &= \{\state: \valfunc(t, \state) \le 0\} \\
%\frs(t_0, t) &= \{\state: \valfuncfwd(t, \state) \le 0 \}
%\end{aligned}
%\end{equation}
\begin{equation}
\label{eq:implicitValfuncs}
\brs(t, t_f) = \{\state: \valfunc(t, \state) \le 0\}
\end{equation}

Some of the reachability computations will not involve an obstacle set $\obsset(t)$, in which case we can simply set $\obsfunc(t, \state) \equiv \infty$ which effectively means that the outside maximum is ignored in \eqref{eq:HJIVI_BRS}.

The Hamiltonian, $\ham(t, \state, \nabla \valfunc(t,\state))$, depends on the system dynamics, and the role of control and disturbance. Whenever $\ham$ does not depend explicit on $t$, we will drop the argument. In addition, the Hamiltonian is an optimization that produces the optimal control $\ctrl^*(t, \state)$ and optimal disturbance $\dstb^*(t, \state)$, once $\valfunc$ is determined. For BRSs, whenever the existence of a control (``$\exists \ctrl$'') or disturbance is sought, the optimization is a minimum over the set of controls or disturbance. Whenever a BRS characterizes the behavior of the system for all controls (``$\forall \ctrl$'') or disturbances, the optimization is a maximum. We will introduce precise definitions of reachable sets, expressions for the Hamiltonian, expressions for the optimal controls as needed for the many different reachability calculations we use.
% !TEX root = STP_journal.tex
\section{STP Without Disturbances and With Perfect Information\label{sec:basic}}
In this section, we introduce the basic STP algorithm assuming that there is no disturbance affecting the vehicles, and that each vehicle knows the exact position of higher-priority vehicles. \SBnote{Although in practice, such assumptions do not hold, the description of the basic STP algorithm will introduce the notation needed for describing the subsequent, more realistic versions of STP.} We also show simulation results for the basic STP algorithm. The majority of the content in this section is taken from \cite{Chen15c}.

\subsection{Theory}
Recall that the STP vehicles $\veh_i, i=1,\ldots,N$, are each assigned a strict priority, with $\veh_j$ having a higher priority than $\veh_i$ if $j<i$. In the absence of disturbances, we can write the dynamics of the STP vehicles as

\begin{equation}
\label{eq:dyn_no_dstb}
\begin{aligned}
\dot\state_i &= \fdyn_i(\state_i, \ctrl_i), t \le \sta_i \\
\ctrl_i &\in \cset_i, \qquad i = 1 \ldots, \N
\end{aligned}
\end{equation}

%\noindent with trajectories denoted by $\traj_i(s; \state^0_i, \ldt, \ctrl_(\cdot))$.

In STP, each vehicle $\veh_i$ plans the trajectory to its target set $\targetset_i$ while avoiding static obstacles $\soset_i$ and the obstacles $\ioset_i^j(t)$ induced by higher-priority vehicles $\veh_j, j<i$. Path planning is done sequentially starting from the first vehicle and proceeding in descending priority, $\veh_1, \veh_2, \ldots, \veh_{\N}$ so that each of the trajectory planning problems can be done in the state space of only one vehicle. During its trajectory planning process, $\veh_i$ ignores the presence of lower-priority vehicles $\veh_k, k>i$, and induces the obstacles $\ioset_k^i(t)$ for $\veh_k, k>i$.

From the perspective of $\veh_i$, each of the higher-priority vehicles $\veh_j, j<i$ induces a time-varying obstacle denoted $\ioset_i^j(t)$ that $\veh_i$ needs to avoid\footnote{Note that the index $k$ in $\ioset_k^i$ denotes vehicles with lower priority than $\veh_i$, and the index $j$ in $\ioset_i^j(t)$ denotes vehicles with higher priority than $\veh_i$.}. Therefore, each vehicle $\veh_i$ must plan its trajectory to $\targetset_i$ while avoiding the union of all the induced obstacles as well as the static obstacles. Let $\obsset_i(t)$ be the union of all the obstacles that $\veh_i$ must avoid on its way to $\targetset_i$:

\begin{equation}
\label{eq:obsseti}
\obsset_i(t)  = \soset_i \cup \bigcup_{j=1}^{i-1} \ioset_i^j(t)
\end{equation}

With full position information of higher priority vehicles, the obstacle induced for $\veh_i$ by $\veh_j$ is simply

\begin{equation}
\label{eq:ioset}
\ioset_i^j(t) = \{\state_i: \|\pos_i - \pos_j(t)\|_2 \le \rc \}
\end{equation}

Each higher priority vehicle $\veh_j$ plans its trajectory while ignoring $\veh_i$. Since trajectory planning is done sequentially in descending order or priority, the vehicles $\veh_j, j<i$ would have planned their trajectories before $\veh_i$ does. Thus, in the absence of disturbances, $\pos_j(t)$ is \textit{a priori} known, and therefore $\ioset_i^j(t), j<i$ are known, deterministic moving obstacles, which means that $\obsset_i(t)$ is also known and deterministic. Therefore, the trajectory planning problem for $\veh_i$ can be solved by first computing the BRS $\brs_i^\text{basic}(t, \sta_i)$, defined as follows:
%
\begin{equation}
\label{eq:BRS_basic}
\begin{aligned}
\brs_i^\text{basic}(t, \sta_i) = & \{y: \exists \ctrl_i(\cdot) \in \cfset_i, \state_i(\cdot) \text{ satisfies \eqref{eq:dyn_no_dstb}}, \\
& \forall s \in [t, \sta_i],\state_i(s) \notin \obsset_i(s), \\
& \exists s \in [t, \sta_i], \state_i(s) \in \targetset_i, \state_i(t) = y\}
\end{aligned}
\end{equation}
%
The BRS $\brs(t, \sta_i)$ can be obtained by solving \eqref{eq:HJIVI_BRS} with $\targetset = \targetset_i$, $\obsset(t) = \obsset_i(t)$, and the Hamiltonian 
%
\begin{equation}
\label{eq:basicham}
\ham_i^\text{basic}(\state_i, \costate) = \min_{\ctrl_i\in\cset_i} \costate \cdot \fdyn_i(\state_i, \ctrl_i)
\end{equation}
%
\SBnote{Note that $\brs(t, \sta_i)$, by definition, does not contain any states from which it is inevitable to avoid the danger zone $\dz_{ij}$ (and $\obsset_i$ in general).} Given $\brs(t, \sta_i)$, the optimal control for reaching $\targetset_i$ while avoiding $\obsset_i(t)$ is then given by
%
\begin{equation}
\label{eq:basicOptCtrl}
\ctrl_i^\text{basic}(t, \state_i) = \arg \min_{\ctrl_i\in\cset_i} \costate \cdot \fdyn_i(\state_i, \ctrl_i)
\end{equation}
%
\noindent from which the trajectory $\state_i(\cdot)$ can be computed by integrating the system dynamics, which in this case are given by \eqref{eq:dyn_no_dstb}. In addition, the latest departure time $\ldt_i$ can be obtained from the BRS $\brs(t, \sta_i)$ as $\ldt_i = \arg \sup_t \{\state_i^0 \in \brs(t, \sta_i)\}$. In summary, the basic STP algorithm is given as follows:

\begin{alg}
\label{alg:basic}
\textbf{Basic STP algorithm}: Suppose we are given initial conditions $\state_i^0$, vehicle dynamics \eqref{eq:dyn_no_dstb}, target sets $\targetset_i$, and static obstacles $\soset_i, i = 1\ldots, \N$. For each $i$ in ascending order starting from $i=1$ (which corresponds to descending order of priority),
\begin{enumerate}
\item determine the total obstacle set $\obsset_i(t)$, given in \eqref{eq:obsseti}. In the case $i=1$, $\obsset_i(t) = \soset_i ~ \forall t$;
\item compute the BRS $\brs_i^\text{basic}(t, \sta_i)$ defined in \eqref{eq:BRS_basic}. The latest departure time $\ldt_i$ is then given by $\arg \sup_t \{\state^0_i \in \brs_i^\text{basic}(t, \sta_i)\}$;
\item determine the trajectory $\state_i(\cdot)$ using vehicle dynamics \eqref{eq:dyn_no_dstb}, with the optimal control  $\ctrl_i^\text{basic}(\cdot)$ given by \eqref{eq:basicOptCtrl};
\item given $\state_i(\cdot)$, compute the induced obstacles $\ioset_k^i(t)$ for each $k>i$. In the absence of disturbances, $\ioset_k^i(t)$ is given by \eqref{eq:ioset}.
\end{enumerate}
\end{alg}

\MCnote{Note that Step 1, which determines the total obstacle set, can be updated in a recursive manner by adding a new set of induced obstacles for each next vehicle: $\obsset_{i+1}(t) = \obsset_i(t) \cup \ioset_{i+1}^i(t)$. In addition, in implementation, Step 4 can be simplified by storing $\obsset_i(t)$ as a look-up table with the maximum dimensionality across all vehicle state spaces. When a vehicle plans its trajectory, irrelevant dimensions of $\obsset_i(t)$ can be ignored. This observation keeps the computational complexity of our algorithm linear with respect to the number of vehicles.}

\MCnote{As previously mentioned, the basic STP algorithm, as well as all subsequent variants of STP algorithms, will \textit{always} return a feasible trajectory that arrives at the target on time, as long as a feasible trajectory exists in the \textit{absence} of other vehicles. This is because a vehicle can simply depart early enough to avoid being blocked by higher-priority vehicles. In fact, the latest departure time $\ldt_i$ quantifies exactly when each vehicle needs to depart to arrive on time.}
% !TEX root = SPP_journal.tex
\section{Basic Results \label{sec:basic_results}}

% Disturbance files
% !TEX root = SPP2.tex
\section{Disturbances and Incomplete Information \label{sec:obs_gen}}
Disturbances and incomplete information significantly complicate the SPP scheme. The main difference is that the vehicle dynamics satisfy \eqref{eq:dyn} as opposed to \eqref{eq:dyn_no_dstb}. Committing to exact trajectories is therefore no longer possible, since the disturbance $d_i(\cdot)$ is a priori unknown. Thus, the induced obstacles $\ioset_i^j(t)$ are no longer just the danger zones centered around positions. We present three methods to address the above issues. The methods differ in terms of control policy information that is known to a lower priority vehicle, and have their relative advantages and disadvantages depending on the situation. The three methods are as follows:
\begin{itemize}
\item \textbf{Centralized control}: A specific control strategy is enforced upon a vehicle; this can be achieved, for example, by some central agent such as an air traffic controller.
\item \textbf{Least restrictive control}: A vehicle is required to arrive at its targets on time, but has no other restrictions on its control policy. When the control policy of a vehicle is unknown, but its timely arrive at its target can be assumed, the least restrictive control can be safely assumed by lower-priority vehicles.
\item \textbf{Robust trajectory tracking}: A vehicle declares a nominal trajectory which can be robustly tracked under disturbances.
\end{itemize}

In general, the above methods can be used in combination in a single path planning problem, with each vehicle independently having different control policies. Lower-priority vehicles would then plan their paths while taking into account the control policy of each higher-priority vehicle. For clarity, however, we will present each method as if all vehicles are using the same method of path planning.

In addition, for simplicity of explanation, we will assume that no static obstacles exist. In the situations where static obstacles do exist, the time-varying obstacles $\obsset_i(t)$ simply becomes the union of the induced obstacles $\ioset_i^j(t)$ in \eqref{eq:ioset} and the static obstacles.

\subsection{Method 1: Centralized Controller \label{sec:cc}}
The highest-priority vehicle $\veh_1$ first plans its path by computing the BRS (with $i=1$)
\vspace{-0.3em}
\begin{equation}
\label{eq:BRS}
\begin{aligned}
\brs_i(t) = \{x_i: &\exists u_i(\cdot) \in \cfset_i, \forall d_i(\cdot) \in \dfset_i, x_i(\cdot) \text{ satisfies \eqref{eq:dyn}}, \\
&\forall s \in [t, \sta_i], x_i(s) \notin \obsset_i(s), \\
&\exists s \in [t, \sta_i], x_i(s) \in \targetset_i\}
\end{aligned}
\end{equation}

Since we have assumed no static obstacles exist, we have that for $\veh_1, \obsset_1(s)=\emptyset ~ \forall s \le \sta_i$, and thus the above BRS is well-defined. This BRS can be computed by solving the HJ VI \eqref{eq:HJIVI} with the following Hamiltonian:
\vspace{-0.3em}
\begin{equation}
H_i\left(t, x_i, p\right) = \min_{u_i \in \cset_i} \max_{d_i \in \dset_i} p \cdot f_i(t, x_i, u_i, d_i)
\end{equation}

\noindent where $l_i(x_i), g_i(t,x_i),V_i(t,x_i)$ are implicit surface functions representing the target $\targetset_i, \obsset_i(t), \brs_i(t)$, respectively. From the BRS, we can obtain the optimal control
\vspace{-0.3em}
\begin{equation}
\label{eq:opt_ctrl_i}
u_i^*(t,x_i) =  \arg \min_{u_i \in \cset_i} \max_{d_i \in \dset_i} p \cdot f_i(t, x_i, u_i, d_i)
\end{equation}

The latest departure time $\ldt_i$ is then given by $\arg \inf_t x_{i0} \in \brs_i(t)$.

If there is a centralized controller directly controlling each of the $N$ vehicles, then the control law of each vehicle can be enforced. In this case, lower priority vehicles can safely assume that higher priority vehicles are applying the enforced control law. In particular, the optimal controller for getting to the target, $u^*_i(t, x_i)$ can be enforced. In this case, the dynamics of each vehicle becomes 
\vspace{-0.3em}
\begin{equation}
\label{eq:dyn_cc}
\begin{aligned}
\dot x_i &= f^*_i (t, x_i, d_i) = f_i(t, x_i, u^*_i(t,x_i), d_i) \\
d_i &\in \dset_i, \quad i = 1,\ldots, N, \quad t \in [\ldt_i, \sta_i]
\end{aligned}
\end{equation}

\noindent where $u_i$ no longer appears explicitly in the dynamics.

From the perspective of a lower-priority vehicle $\veh_i$, a higher-priority vehicle $\veh_j, j < i$ induces an time-varying obstacle that represents the positions that could possibly be within the capture radius $\cradius$ of $\veh_j$ under the dynamics $f^*_j(t, x_j, d_j)$. Determining this obstacle involves computing a forward reachable set (FRS) of $\veh_j$ starting from $x_j(\ldt) = x_{j0}$. The FRS $\frs_j(t)$ is defined as follows:
\vspace{-0.3em}
\begin{equation}
\label{eq:FRS1}
\begin{aligned}
&\frs_j(t) = \{y \in \R^{n_j}: \exists d_j(\cdot) \in \dfset_j, \\
&x_j(\cdot) \text{ satisfies \eqref{eq:dyn_cc}}, x_j(\ldt) = x_{j0}, x_j(t) = y\}
\end{aligned}
\end{equation}

Conveniently, the FRS can be computed using the following HJ VI:
\vspace{-0.4em}
\begin{equation}
\label{eq:FRS_HJIVI}
\begin{aligned}
&D_t W_j(t, x_j) + H_j\left(t, x_j, D_{x_j} W_j\right) = 0, t \in [\ldt_j, \sta_j]\\
&W_j(\ldt_j, x_j) = \bar l_j(x_j) \\
\end{aligned}
\end{equation}

\noindent with the following Hamiltonian
\begin{equation}
H_j\left(t, x_j, p\right) = \min_{d_j \in \dset_j} p \cdot f^*_j(t, x_j, d_j)
\end{equation}
\noindent where $\bar l$ is chosen to be\footnote{In practice, we define the target set to be a small region around the vehicle's initial state for computational reasons.} such that $\bar l (y) = 0 \Leftrightarrow y = x_j(\ldt)$.

The FRS $\frs_j(t)$ represents the set of possible states at time $t$ of a higher-priority vehicle $\veh_j$ given the worst case disturbance $d_j(\cdot)$ and given that $\veh_j$ uses the feedback controller $u_j^*(t, x_j)$. In order for a lower-priority vehicle $\veh_i$ to guarantee that it does not go within a distance of $\cradius$ to $\veh_j$, $\veh_i$ must stay a distance of at least $\cradius$ away from the set $\frs_j(t)$ for all possible values of the non-position states $\npos_j$. This gives the obstacle induced by a higher priority vehicle $\veh_j$ for a lower priority vehicle $\veh_i$ as follows:
\vspace{-0.4em}
\begin{equation}
\ioset_i^j(t) = \{x_i: \dist(\pos_i, \pfrs_j(t)) \le \cradius \}
\end{equation}

\noindent where the $\dist(\cdot, \cdot)$ function represents the minimum distance from a point to a set, and the set $\pfrs_j(t)$ is the set of states in the FRS $\frs_j(t)$ projected onto the states representing position $\pos_j$, and disregarding the non-position dimensions $\npos_j$:
\vspace{-0.4em}
\begin{equation}
\pfrs_j(t) = \{p: \exists \npos_j, (p, \npos_j) \in \frs_j(t)\}.
\end{equation}

Finally, taking the union of the induced obstacles $\ioset_i^j(t)$ as in \eqref{eq:ioset} gives us the time-varying obstacles $\obsset_i(t)$ needed to define and determine the BRS $\brs_i(t)$ in \eqref{eq:BRS}. Repeating this process, all vehicles will be able to plan paths that guarantee the vehicles' timely and safe arrival.

% !TEX root = SPP2.tex
\subsection{Method 2: Least Restrictive Control \label{sec:lrc}}
Here, we again begin with the highest vehicle $\veh_1$ planning its path by computing the BRS $\brs_i(t)$ in \eqref{eq:BRS}. However, if there is no centralized controller to enforce the control policy for higher-priority vehicles, weaker assumptions must be made by the lower-priority vehicles to ensure collision avoidance. One reasonable assumption that a lower-priority vehicle can make is that all higher-priority vehicles follow the least restrictive control that would take them to their targets. This control would be given by 
\vspace{-0.4em}
\begin{equation}
\label{eq:lrctrl} % least restrictive control
u_j(t, x_j)\in \begin{cases} \{u_j^*(t, x_j) \text{ given by } \eqref{eq:opt_ctrl_i}\} \text{ if } x_j(t)\in \partial \brs_j(t), \\
\cset_i  \text{ otherwise}
\end{cases}
\end{equation}

Such a controller allows each higher priority vehicle to use any controller it desires, except when it is on the boundary of the BRS, $\partial \brs_j(t)$, in which case the optimal control $u_j^*(t, x_j)$ given by \eqref{eq:opt_ctrl_i} must be used to get to the target on time. This assumption is the weakest assumption that could be made by lower priority vehicles given that the higher priority vehicles will get to their targets on time.

Suppose a lower-priority vehicle $\veh_i$ assumes that higher-priority vehicles $\veh_j, j < i$ use the least restrictive control strategy in \eqref{eq:lrctrl}. From the perspective of the lower-priority vehicle $\veh_i$, a higher-priority vehicle $\veh_j$ could be in any state that is reachable from $\veh_j$'s initial state $x_j(\ldt) = x_{j0}$ and from which the target $\targetset_j$ can be reached. Mathematically, this is defined by the intersection of a FRS from the initial state $x_j(\ldt)=x_{j0}$ and the BRS defined in \eqref{eq:BRS} from the target set $\targetset_j$, $\brs_j(t) \cap \frs_j(t)$. In this situation, since $\veh_j$ cannot be assumed to be using any particular feedback control, $\frs_j(t)$ is defined in \eqref{eq:FRS2}.
\vspace{-0.4em}
\MCnote{overloaded notation for FRS}
\begin{equation}
\label{eq:FRS2}
\begin{aligned}
\frs_j(t) &= \{y \in \R^{n_j}: \exists u_j(\cdot)\in\cfset_j, \exists d_j(\cdot) \in \dfset_j, \\
&x_j(\cdot) \text{ satisfies \eqref{eq:dyn}, }x_j(t) = y\}
\end{aligned}
\end{equation}

This FRS can be computed by solving \eqref{eq:FRS_HJIVI} with
\vspace{-0.4em}
\begin{equation}
H_j\left(t, x_j, p\right) = \min_{u_j \in \cset_j} \min_{d_j \in \dset_j} p \cdot f_j(t, x_j, u_j, d_j)
\end{equation}
\MCnote{mention without obstacles?}
In turn, the obstacle induced by a higher priority $\veh_j$ for a lower priority vehicle $\veh_i$ is as follows:
\vspace{-0.4em}
\begin{equation}
\begin{aligned}
\ioset_i^j(t) &= \{x_i: \dist(\pos_i, \pfrs_j(t)) \le \cradius \}, \text{ with} \\
\pfrs_j(t) &= \{p: \exists \npos_j, (p, \npos_j) \in \brs_j(t) \cap \frs_j(t)\}
\end{aligned}
\end{equation}

Note that the centralized controller method described in the previous section can be thought of as the ``most restrictive control'' method, in which all vehicles must use the optimal controller at all times, while the least restrictive control method allows vehicles to use any suboptimal controller that allows them to arrive at the target on time. These two methods can be considered two extremes of a spectrum in which varying degrees of optimality is assumed for higher-priority vehicles.
% !TEX root = SPP2.tex
\subsection{Method 3: Robust Trajectory Tracking\label{sec:rtt}}
Although it is impossible to commit to and track an exact trajectory in the presence of disturbances, it may still be possible to \textit{robustly} track a \textit{nominal} trajectory with a bounded error at all times. If this can be done, then the tracking error bound can be used to determine the induced obstacles. Here, computation is done in two phases: the \textit{planning phase} and the \textit{disturbance rejection phase}. In the planning phase, we compute a nominal trajectory $\state_{r,j}(\cdot)$ that is feasible in the absence of disturbances. In the disturbance rejection phase, we compute a bound on the tracking error.%\MCnote{don't need to explain where error comes bound, imo}%, caused by a vehicle's inability to exactly track the nominal trajectory in the presence of disturbances. 

In the planning phase, planning is done for a reduced control set $\cset^p\subset\cset$, as some margin is needed to reject unexpected disturbances while tracking the nominal trajectory. In the disturbance rejection phase, we determine the error bound independently of the nominal trajectory. Let $\state_j$ and $\state_{r,j}$ denote the states of the actual vehicle $\veh_j$ and the virtual evader, respectively, and define the tracking error $e_j=\state_j-\state_{r,j}$. When the error dynamics are independent of the absolute state as in \eqref{eq:edyn} (and also (7) in \cite{Mitchell05}), we can obtain error dynamics of the form
\begin{equation}
\label{eq:edyn} % Error dynamics
\begin{aligned}
\dot{e_j} &= \fdyn_{e_j}(e_j, \ctrl_j, \ctrl_{r,j},\dstb_j), \\
\ctrl_j &\in \cset_j, \ctrl_{r,j} \in \cset^p_j, \dstb_j \in \dset_j, \quad t \leq 0
\end{aligned}
\end{equation}

To obtain bounds on the tracking error, we first conservatively estimate the error bound around any reference state $\state_{r,j}$, denoted $\errorbound_j = \{e_j: \|\pos_{e_j}\|_2 \le R_{\text{EB}}\}$,
%\begin{equation} \label{eqn:err}
%\errorbound_j = \{e_j: \|\pos_{e_j}\|_2 \le R_{\text{EB}} \}, 
%\end{equation}
\noindent where $\pos_{e_j}$ denotes the position coordinates of $e_j$ and $R_{\text{EB}}$ is a design parameter. We next solve a reachability problem with its complement $\errorbound_j^c$, the set of tracking errors violating the error bound, as the target in the space of the error dynamics. From $\errorbound_j^c$, we compute the following BRS:
\begin{equation} \label{eqn:errBound}
\begin{aligned}
&\brs^{\text{EB}}_{j}(t, 0) = \{y: \forall \ctrl_j(\cdot) \in \cfset_j, \exists \ctrl_{r, j}(\cdot) \in \cfset^\pos_j, \exists \dstb_j(\cdot) \in \dfset_i, \\
& e_j(\cdot) \text{ satisfies \eqref{eq:edyn}}, e_j(t) = y, \exists s \in [t, 0], e_j(s) \in \errorbound_j^c\}, 
\end{aligned}
\end{equation}
where the Hamiltonian to compute the BRS is given by:
\begin{equation}
\begin{aligned}
H^{\text{EB}}_{j}(e_j, \costate) &= \max_{\ctrl_j \in \cset_j} \min_{\ctrl_r \in \cset^\pos_j, \dstb_j \in \dset_j} \costate \cdot \fdyn_{e_j}(e_j, \ctrl_j, \ctrl_{r,j}, \dstb_j).
\end{aligned}
\end{equation}

Letting $t \to -\infty$, we obtain the infinite-horizon control-invariant set $\disckernel_j := \lim_{t \to -\infty} \left( \brs^{\text{EB}}_{j}(t, 0) \right)^c$. If $\disckernel_j$ is nonempty, then the tracking error $e_j$ at flight time is guaranteed to remain within $\disckernel_j \subseteq \errorbound_j$ provided that the vehicle starts inside $\disckernel_j$ and subsequently applies the feedback control law
\begin{equation}
\label{eq:robust_tracking_law}
\tracklaw_j(e_j) = \arg\max_{\ctrl_j \in \cset_j} \min_{\ctrl_r \in\cset^\pos_j, \dstb_j \in \dset_j} \costate \cdot \fdyn_{e_j}(e_j,\ctrl_j,\ctrl_{r, j},\dstb_j).
\end{equation}

The induced obstacles by each higher-priority vehicle $\veh_j$ can thus be obtained by: 
\begin{equation} 
\label{eqn:rttObs}
\begin{aligned}
\ioset_i^j(t) &=  \{\state_i: \exists y \in \pfrs_j(t), \|\pos_i - y\|_2 \le \rc \} \\
\pfrs_j(t) &= \{\pos_j: \exists \npos_j, (\pos_j, \npos_j) \in \disckernel_j  + \state_{r,j}(t)\},
\end{aligned}
\end{equation}
\noindent where the ``$+$'' in \eqref{eqn:rttObs} denotes the Minkowski sum\footnote{The Minkowski sum of sets $A$ and $B$ is the set of all points that are the sum of any point in $A$ and $B$.}. Finally, we can obtain the total obstacle set $\obsset_i(t)$ using \eqref{eq:ioset}. %Intuitively, if $\veh_j$ is tracking $\state_{r,j}(t)$, then it will remain within the error bound $\disckernel_j$ around $\state_{r,j}(t) ~\forall t$. This is precisely the set $\pfrs_j(t)$. The induced obstacles can then be obtained by augmenting a danger zone around this set. Finally, we can obtain the total obstacle set $\obsset_i(t)$ using \eqref{eq:obsseti}.

Since each vehicle $\veh_j$, $j<i$, can only be guaranteed to stay within $\disckernel_j$, we must make sure during the path planning of $\veh_i$ that at any given time, the error bounds of $\veh_i$ and $\veh_j$, $\disckernel_i$ and $\disckernel_j$, do not intersect. This can be done by augmenting the total obstacle set by $\disckernel_i$:%This can be done by choosing the induced obstacle to be the Minkowski sum\footnote{The Minkowski sum of sets $A$ and $B$ is the set of all points that are the sum of any point in $A$ and $B$.} of the error bounds. Thus,

\begin{equation} 
\label{eqn:rttAugObs}
\tilde{\obsset}_i(t) = \obsset_i(t) + \disckernel_i.
\end{equation}

Finally, given $\disckernel_i$, we can guarantee that $\veh_i$ will reach its target $\targetset_i$ if $\disckernel_i \subseteq \targetset_i$; thus, in the path planning phase, we modify $\targetset_i$ to be $\tilde{\targetset}_i := \{\state_i: \disckernel_i + \state_i \subseteq \targetset_i\}$, and compute a BRS, with the control authority $\cset^\pos_i$, that contains the initial state of the vehicle. Mathematically,

\begin{equation}
\label{eq:rttBRS}
\begin{aligned}
\brs_i^\text{rtt}(t, \sta_i) = & \{y: \exists \ctrl_i(\cdot) \in \cfset^p_i, \state_i(\cdot) \text{ satisfies \eqref{eq:dyn_no_dstb}},\\
&\forall s \in [t, \sta_i], \state_i(s) \notin \tilde{\obsset}_i(t), \\
& \exists s \in [t, \sta_i], \state_i(s) \in \tilde{\targetset}_i, \state_i(t) = y\}
\end{aligned}
\end{equation}

The Hamiltonian to compute $\brs_i^\text{rtt}(t, \sta_i)$ and the optimal control for reaching $\tilde{\targetset}_i$ are given by \eqref{eq:basicSPPHam} and \eqref{eq:optCtrl} respectively.
%$\brs_i^\text{rtt}(t, \sta_i)$ can be obtained by solving \eqref{eq:HJIVI_BRS} using the Hamiltonian: 
%\begin{equation}
%\label{eq:RTTham}
%\ham_i^\text{rtt}(\state_i, \costate) = \min_{\ctrl_i \in \cset^\pos_i } \costate \cdot \fdyn_i(\state_i, \ctrl_i)
%\end{equation}
%
%The corresponding optimal control for reaching $\tilde{\targetset}_i$ is given by:
%\begin{equation}
%\label{eq:RTTOptCtrl}
%\ctrl_i^\text{rtt}(t) = \arg \min_{\ctrl_i \in \cset^\pos_i } \costate \cdot \fdyn_i(\state_i, \ctrl_i).
%\end{equation}
The nominal trajectory $\state_{r,i}(\cdot)$ can thus be obtained by using vehicle dynamics \eqref{eq:dyn_no_dstb}, with the optimal control  $\ctrl_i^\text{rtt}(\cdot)$. From the resulting nominal trajectory $\state_{r,i}(\cdot)$, the overall control policy to reach $\targetset_i$ can be obtained via \eqref{eq:robust_tracking_law}.
% !TEX root = ../SPP_journal.tex
\section{Numerical Simulations For Incomplete Information \label{sec:sim_dstb}}
We demonstrate our proposed methods using a four-vehicle example. Each vehicle has the following simple kinematics model:
\begin{equation}
\label{eq:dyn_i}
\begin{aligned}
\dot{\pos}_{x,i} &= v_i \cos \theta_i + d_{x,i} \\
\dot{\pos}_{y,i} &= v_i \sin \theta_i + d_{y,i}\\
\dot{\theta}_i &= \omega_i + d_{\theta,i}, \\
\underline{v} & \le v_i \le \bar{v}, |\omega_i| \le \bar{\omega},\\
\|(d_{x,i}, & d_{y,i}) \|_2 \le d_{r}, |d_{\theta,i}| \le \bar{d_{\theta}}
\end{aligned}
\end{equation}

\noindent where $p_i = (p_{x,i}, p_{y,i}), \theta_i, d = (d_{x,i}, d_{y,i}, d_{\theta,i})$ respectively represent $\veh_i$'s position, heading, and disturbances in the three states. The control of $\veh_i$ is $u_i = (v_i, \omega_i)$, where $v_i$ is the speed of $\veh_i$ and $\omega_i$ is the turn rate; both controls have a lower and upper bound. For illustration purposes, we choose $\underline{v} = 0.5, \bar{v} = 1, \bar\omega = 1$; however, our method can easily handle the case in which these inputs differ across vehicles and cases in which each vehicle has a different dynamic model. The disturbance bounds are chosen as $d_{r} = 0.1, \bar{d_{\theta}} = 0.2$, which correspond to a 10\% uncertainty in the dynamics. %The optimal control for vehicle $i$ can be obtained by optimizing the associated Hamiltonian, $H_i(t, D_{\bm{x}_i} V_i(\bm{x}_i,t), V_i(\bm{x}_i,t))$, and is given by:

%\begin{equation}
%\omega_i(t) = -\bar{\omega}_i \frac{D_{\theta_i}V_i(\bm{x}_i,t)}{\left| D_{\theta_i}V_i(\bm{x}_i,t) \right|},
%\end{equation}
%
%\begin{equation}
%v_i(t) =
%\left \{ 
%\begin{array}{ll}
%\underline{v} & \mbox{ if } D_{x_i}V_i(\bm{x}_i,t) \cos \theta_i + D_{y_i}V_i(\bm{x}_i,t) \sin \theta_i \geq 0 \\
%\bar{v} & \mbox{ otherwise } 
%\end{array}
%\right.
%\end{equation}

For this example, we have chosen scheduled times of arrival $\sta_i = 0~\forall i$ for simplicity. Each vehicle aims to get to a target set of the form \eqref{eq:target_sim} with target radius $r=0.1$. The vehicles have target centers $c_i$ and initial conditions $\state_i^0$ as follows:

\begin{equation}
\begin{aligned}
c_1 = (0.7, 0.2), \quad& \state_1^0 = (-0.5, 0, 0),\\
c_2 = (-0.7, 0.2), \quad& \state_2^0 = (0.5, 0, \pi), \\
c_3 = (0.7, -0.7), \quad& \state_3^0 = \left(-0.6, 0.6, 7\pi/4\right),\\
c_4 = (-0.7, -0.7), \quad & \state_4^0 = \left(0.6, 0.6, 5\pi/4\right),
\end{aligned}
\end{equation}

These parameters are the same as the example in Section \ref{sec:basic_results}, except for the $\sta_i$ values are the same, and that there are no static obstacles. The problem setup for this example is shown in Fig. \ref{fig:init_setup_dstb}.

With the above parameters, we obtain $\ldt_i, i=1,2,3,4$. Note that even though $\sta_i$ is assumed to be same for all vehicles in this example for simplicity, our method can easily handle the case in which $\sta_i$ is different for each vehicle as we have already shown in Section \ref{sec:basic_results}.

\begin{figure}
  \centering
  \includegraphics[width=0.40\textwidth]{"fig/init_setup"}
  \caption{Initial configuration of the four-vehicle example.}
  \label{fig:init_setup_dstb}
\end{figure}

For each proposed method of computing induced obstacles, we show the vehicles' entire trajectories (colored dotted lines), and overlay their positions (colored asterisks) and headings (arrows) at a point in time in which they are in relatively dense configuration. In all cases, the vehicles are able to avoid each other's danger zones (colored dashed circles) while getting to their target sets in minimum time. In addition, we show the evolution of the BRS over time for $\veh_3$ (green boundaries) as well as the obstacles induced by the higher-priority vehicles (black boundaries).

\subsection{Centralized Control}
\begin{figure}[H]
  \centering
  \includegraphics[width=0.50\textwidth]{"fig/cc_traj"}
  \caption{Simulated trajectories in the centralized control method. Since the higher priority vehicles induce relatively small obstacles in this case, vehicles do not deviate much from a straight line trajectory towards their respective targets.}
  \label{fig:cc_traj}
\end{figure}

Fig. \ref{fig:cc_traj} shows the simulated trajectories in the situation where a centralized controller enforces each vehicle to use the optimal controller $u^*_i(t, x_i)$ according to \eqref{eq:opt_ctrl_i}, as described in Section \ref{sec:cc}. In this case, vehicles appear to deviate slightly from a straight line trajectory towards their respective targets, just enough to avoid higher-priority vehicles. The deviation is small since the centralized controller is quite restrictive, making the possible positions of higher priority vehicles cover a small area. In the dense configuration at $t=-1.0$, the vehicles are close to each other but still outside each other's danger zones.

Fig. \ref{fig:cc_rs3} shows the evolution of the BRS for $\veh_3$ (green boundary), as well as the obstacles (black boundary) induced by the higher-priority vehicles $\veh_1$ (red) and $\veh_2$ (blue). The locations of the induced obstacles at different time points include the actual positions of $\veh_1$ and $\veh_2$ at those times, and the size of the obstacles remains relatively small. $\ldt_i$ numbers for the four vehicles (in order) in this case are $-1.35, -1.37, -1.94$ and $-2.04$. Numbers are relatively close for vehicles $\veh_1$, $\veh_2$ and $\veh_3$, $\veh_4$, because the obstacles generated by higher-priority vehicles are small and hence do not affect $\ldt$ of the lower-priority vehicles significantly. 

\begin{figure}[H]
  \centering
  \includegraphics[width=0.50\textwidth]{"fig/cc_rs3"}
  \caption{Evolution of the BRS and the obstacles induced by $\veh_1$ and $\veh_2$ for $\veh_3$ in the centralized control method. Since every vehicle is applying the optimal control at all times, the obstacle sizes are relatively small.}
  \label{fig:cc_rs3}
  \vspace{-1.2em}
\end{figure}
% !TEX root = SPP2.tex
\subsection{Least Restrictive Control}
Fig. \ref{fig:allTrajs} shows the simulated trajectories in the situation where each vehicle assumes that higher-priority vehicles use the least restrictive control to reach their targets, as described in \ref{sec:lrc}. Fig. \ref{fig:lrc_rs3} shows the BRS and induced obstacles for $\veh_3$.

$\veh_1$ (red) takes a relatively straight path to reach its target. From the perspective of all other vehicles, large obstacles are induced by $\veh_1$, since lower-priority vehicles make the weak assumption that higher-priority vehicles are using the least restrictive control. Because the obstacles induced by higher-priority vehicles are so large, it is faster for lower-priority vehicles to wait until higher-priority vehicles pass by than to move around the higher-priority vehicles. As a result, the vehicles never form a dense configuration, and their trajectories are all relatively straight, indicating that they end up taking a short path to the target after higher-priority vehicles pass by. This is also indicated by low $\ldt_i$ values for the four vehicles, which are $-1.35, -1.97, -2.66$ and $-3.39$, respectively. Compared to the centralized control method, $\ldt_i$'s decrease significantly for all vehicles, except $\veh_1$, the highest-priority vehicle, since it need not account for any moving obstacles. 

From $\veh_3$'s (green) perspective, the large obstacles induced by $\veh_1$ and $\veh_2$ are shown in Fig. \ref{fig:lrc_rs3} as the black boundary. As the BRS (green boundary) evolves over time, its growth gets inhibited by the large obstacles for a long time, as evident at $t=-0.89$. Eventually, the boundary of the BRS reaches the initial state of $\veh_3$ at $t = \ldt_3 = -2.66$.
%
%\begin{figure}[H]
%  \centering
%  \includegraphics[width=0.40\textwidth]{"fig/lrc_traj"}
%  \caption{Simulated trajectories in the least restrictive control method. All vehicles start moving before $\veh_1$ starts, because the large obstacles make it optimal to wait until higher priority vehicles pass by, leading to a smaller $\ldt$. }
%  \label{fig:lrc_traj}
%\end{figure}
%
\begin{figure}[H]
  \vspace{-1em}
  \centering
  \includegraphics[width=0.45\textwidth]{"fig/lrc_rs3"}
  \caption{Evolution of the BRS for $\veh_3$ in the least restrictive control method. $\ldt_3$ is significantly lower than that in the centralized control method ($-1.94$ vs. $-2.66$), reflecting the impact of larger induced obstacles.}
  \label{fig:lrc_rs3}
  \vspace{-1.5em}
\end{figure}
% !TEX root = SPP2.tex
\subsection{Robust Trajectory Tracking}
Fig. \ref{fig:rtt_traj} shows the vehicle trajectories in the situation where each vehicle robustly tracks a pre-specified trajectory and is guaranteed to stay inside a ``bubble" around the trajectory. Fig. \ref{fig:rtt_rs3} shows the evolution of BRS and induced obstacles for vehicle $\veh_3$. The obstacles induced by other vehicles inhibit the evolution of the BRS, carving out thin “channels,” which can be seen at $t = -2.59$, that separate the BRS into different “islands”. \MCnote{(deleted sentence)}%One can see how these channels and islands form by examining the time evolution of the BRS set.

\begin{figure}
  \centering
  \includegraphics[width=0.40\textwidth]{"fig/rtt_traj"}
  \caption{Simulated trajectories for the robust trajectory tracking method.}
  \label{fig:rtt_traj}
  \vspace{-1em}
\end{figure}

$\ldt_i$ values for the four vehicles in this case are $-1.61, -3.16, -3.57$ and $-2.47$ respectively. In this method, vehicles use reduced control authority for path planning towards a reduced-size effective target set. As a result, higher-priority vehicles tend to have lower $\ldt$ compared to the other two methods, as evident from $\ldt_1$. Because of this ``sacrifice" made by the higher-priority vehicles during the path planning phase, the $\ldt$'s of lower-priority vehicles may increase compared to those in the other methods, as evident from $\ldt_4$. Overall, it is unclear how $\ldt_i$ will change for a vehicle compared to the other methods, as the conservative path planning increases $\ldt_i$ for higher-priority vehicles and decreases $\ldt_i$ for lower-priority vehicles.

\begin{figure}[h]
  \centering
  \includegraphics[width=0.40\textwidth]{"fig/rtt_rs3"}
  \caption{Evolution of the BRS for $\veh_3$ in the robust trajectory tracking method. As the BRS grows in time, the induced obstacles carve out a channel. Note that a smaller target set is used to compute the BRS to ensure that the vehicle reaches the target set by $t=0$ for any allowed tracking error.}
  \label{fig:rtt_rs3}
  \vspace{-1em}
\end{figure}
\vspace{-0.2em}

%% !TEX root = SPP_journal.tex
\section{SPP Under Disturbances and Incomplete information \label{sec:incomp}}
In this section, we investigate SPP under the presence of disturbances and incomplete information about higher-priority vehicles' control policies. In the presence of disturbances, our joint system dynamics become in the form of \eqref{eq:dyn}, and the controls $\ctrl_i$ of the vehicles $\veh{i}$ must drive the state $\state_i$ into the target $\targetset_i$ while keeping all vehicles away from each other's danger zones despite the worst case disturbance.

In a practical setting, incomplete information may arise due to a few reasons. For example, the presence of disturbances implies that exact trajectories $\state_i(\cdot)$ cannot be completely known \textit{a priori}, since unknown disturbances will affect the evolution of the system trajectory. From the perspect of each vehicle $\veh{i}$, the effect of disturbances must be taken into account both in the dynamics the vehicle itself, and in the dynamics of other vehicles.

Even ignoring disturbances, each vehicle $\veh{i}$ may not have information about the control strategy of the other vehicles $\veh{j}, j\neq i$. For example, there may be many different control strategies for each vehicle to reach its target on time, and different vehicles may be using different control strategies. In this situation, lower-priority vehicles may need to plan their paths while avoiding danger zones of higher-priority vehicles without knowing what control strategies higher-priority vehicles may be using. Furthermore, each vehicle may even change control strategies on the fly.

We explore three different situations in which incomplete information may arise. In Section \ref{sec:incomp_optctrl}, we consider the situation in which all vehicles utilize a particular control strategy such as the optimal controller at all times. Such a scenario may occur if there is a centralized controller such as an air traffic controller controlling the vehicles. Next, in Section \ref{sec:incomp_LRctrl}, we assume that each vehicle only has information about the target sets and arrival times of other vehicles, and no information about other vehicles' control strategies. When the control policy is unknown, this is the weakest assumption that can be made by lower-priority vehicles, assuming that higher-priority vehicles get to their targets on time. Lastly, in Section \ref{sec:incomp_robust}, we consider the situation in which each vehicle divides its control input authority into a part that drives the vehicle towards the target, and a part that rejects disturbances. By reserving part of the control for disturbance rejection, the vehicles may declare nominal trajectories that can be robustly tracked.

To take into account disturbances and imperfect information, it turns out that we may still use Algorithm \ref{alg:basic}. In all three cases, each vehicle $\veh{i}$ induces a moving obstacle $\ioset_k^i(t)$ for the lower priority vehicles $\veh{k}, k>i$, just like in Algorithm \ref{alg:basic}. However, unlike in the basic SPP algorithm, computation of $\ioset_k^i(t)$ is no longer trival as in \eqref{eq:ioset}. We now describe how to compute $\ioset_k^i(t)$ in the three cases, which in turn will lead to different total obstacles $\obsset_i(t)$ that will be used to solve the HJ VI \eqref{eq:HJIVI_BRS}.

In general, the three methods can be used in combination in a single path planning problem, with each vehicle independently having different control policies. Lower-priority vehicles would then plan their paths while taking into account the control policy of each higher-priority vehicle. For clarity, however, we will present each method as if all vehicles are using the same method of path planning.

Notation:
\begin{itemize}
\item Augmented obstacles $\tilde\obsset(t)$
\item Base Obstacles $\boset_i^j(t)$
\end{itemize}

\subsection{Centralized Controller} \label{sec:incomp_optctrl}
Under the presence of disturbance, two modifications to Algorithm \ref{alg:basic} must be made: First, given the total obstacle set $\obsset_i(t)$, each vehicle $\veh{i}$ ensure that it gets to the target set $\targetset_i$ on time without entering any danger zones $\dz_{ij}$, despite the worst-case disturbance $\dstb_i(\cdot)$. Fortunately, disturbances can be easily accounted for by using \eqref{eq:HJIVI_BRS} to compute the BRS 

\begin{equation}
\label{eq:BRS_cc}
\begin{aligned}
\brs_i(t) = \{x_i: &\exists u_i(\cdot) \in \cfset_i, \forall d_i(\cdot) \in \dfset_i, x_i(\cdot) \text{ satisfies \eqref{eq:dyn}}, \\
&\forall s \in [t, \sta_i], x_i(s) \notin \obsset_i(s), \\
&\exists s \in [t, \sta_i], x_i(s) \in \targetset_i\}
\end{aligned}
\end{equation}


$\brs_i(t; \targetset_i, \obsset_i(\cdot), \ham_{\text{cc}})$, where the Hamiltonian is given by

\begin{equation}
\ham_i\left(\state_i, p\right) = \min_{\ctrl_i \in \cset_i} \max_{\dstb_i \in \dset_i} p \cdot \fdyn_i(\state_i, \ctrl_i, \dstb_i)
\end{equation}

\noindent from which we can obtain the optimal control $\ctrl_i^*(t, \state_i)$ can be obtained:

\begin{equation}
\label{eq:opt_ctrl_i}
\ctrl_i^*(t, \state_i) =  \arg \min_{\ctrl_i \in \cset_i} \max_{\dstb_i \in \dset_i} p \cdot \fdyn_i(\state_i, \ctrl_i, \dstb_i)
\end{equation}

If there is a centralized controller directly controlling each of the $N$ vehicles, then the control law of each vehicle can be enforced. In this case, lower priority vehicles can safely assume that higher priority vehicles are applying the enforced control law. In particular, the optimal controller for getting to the target, $u^*_i(t, x_i)$ can be enforced. In this case, the dynamics of each vehicle becomes 
\vspace{-0.3em}
\begin{equation}
\label{eq:dyn_cc}
\begin{aligned}
\dot \state_i &= \fdyn^\dstb*_i (\state_i, \dstb_i) = \fdyn_i(\state_i, \ctrl^*_i(t, \state_i), \dstb_i) \\
\dstb_i &\in \dset_i, \quad i = 1,\ldots, N, \quad t \in [\ldt_i, \sta_i]
\end{aligned}
\end{equation}

\noindent where $u_i$ no longer appears explicitly in the dynamics.

From the perspective of a lower-priority vehicle $\veh{i}$, a higher-priority vehicle $\veh{j}, j < i$ induces an time-varying obstacle that represents the positions that could possibly be within the capture radius $\rc$ of $\veh{j}$ under the dynamics $f^*_j(t, x_j, d_j)$. Determining this obstacle involves computing a forward reachable set (FRS) of $\veh{j}$ starting from $x_j(\ldt) = x_{j0}$. The FRS is denoted $\underline\frs_j(t; \underline\targetset_j, \emptyset, \ham_{cc*,j})$, and can be computed by solving \eqref{eq:HJIVI_FRS} without the inner minimization, and with the following Hamiltonian

\begin{equation}
\ham_{cc*,j}\left(\state_j, p\right) = \min_{\dstb_j \in \dset_j} p \cdot \fdyn^*_j(\state_j, \dstb_j)
\end{equation}

The target set for the FRS computation is chosen to be\footnote{In practice, we define the target set to be a small region around the vehicle's initial state for computational reasons.} $\underline \targetset = \{\state_j(\ldt)\}$.

The FRS $\underline\frs_j(t; \underline\targetset_j, \emptyset, \ham_{cc*,j})$ represents the set of possible states at time $t$ of a higher-priority vehicle $\veh{j}$ given the worst case disturbance $\dstb_j(\cdot)$ and given that $\veh{j}$ uses the feedback controller $\ctrl_j^*(t, \state_j)$. In order for a lower-priority vehicle $\veh{i}$ to guarantee that it does not go within a distance of $\rc$ to $\veh{j}$, $\veh{i}$ must stay a distance of at least $\rc$ away from the set $\frs_j(t)$ for all possible values of the non-position states $\npos_j$. This gives the obstacle induced by a higher priority vehicle $\veh{j}$ for a lower priority vehicle $\veh{i}$ as follows:

\begin{equation}
\ioset_i^j(t) = \{\state_i: \dist(\pos_i, \pfrs_j(t)) \le \rc \}
\end{equation}

\noindent where the $\dist(\cdot, \cdot)$ function here represents the minimum distance from a point to a set, and the set $\pfrs_j(t)$ is the set of states in the FRS $\underline\frs_j(t; \underline\targetset_j, \emptyset, \ham_{cc*,j})$ projected onto the states representing position $\pos_j$, and disregarding the non-position dimensions $\npos_j$:

\begin{equation}
\pfrs_j(t) = \{p: \exists \npos_j, (p, \npos_j) \in \underline\frs_j(t; \underline\targetset_j, \emptyset, \ham_{cc*,j})\}.
\end{equation}

Finally, the total time-varying obstacles $\obsset_i(t)$ for the next vehicle $\veh{i}$ can be determined using \eqref{eq:obsseti}. Afterwards, the BRS $\brs_i(t; \targetset_i, \obsset_i(\cdot), \ham_{\text{cc}})$. Repeating this process, all vehicles will be able to plan paths that guarantee the vehicles' timely and safe arrival.

\subsection{Least Restrictive Controller}  \label{sec:incomp_LRctrl}
Here, we again begin with the highest priority vehicle $\veh{1}$ planning its path by computing the BRS $\brs_i(t)$ in \eqref{eq:BRS}. However, if there is no centralized controller to enforce the control policy for higher-priority vehicles, weaker assumptions must be made by the lower-priority vehicles to ensure collision avoidance. One reasonable assumption that a lower-priority vehicle can make is that all higher-priority vehicles follow the least restrictive control that would take them to their targets. This control would be given by 
\vspace{-0.4em}
\begin{equation}
\label{eq:lrctrl} % least restrictive control
u_j(t, x_j)\in \begin{cases} \{u_j^*(t, x_j) \text{ given by } \eqref{eq:opt_ctrl_i}\} \text{ if } x_j(t)\in \partial \brs_j(t), \\
\cset_j  \text{ otherwise}
\end{cases}
\end{equation}

Such a controller allows each higher priority vehicle to use any controller it desires, except when it is on the boundary of the BRS, $\partial \brs_j(t)$, in which case the optimal control $u_j^*(t, x_j)$ given by \eqref{eq:opt_ctrl_i} must be used to get to the target safely and on time. This assumption is the weakest assumption that could be made by lower priority vehicles given that the higher priority vehicles will get to their targets on time.

Suppose a lower-priority vehicle $\veh{i}$ assumes that higher-priority vehicles $\veh{j}, j < i$ use the least restrictive control strategy in \eqref{eq:lrctrl}. From the perspective of the lower-priority vehicle $\veh{i}$, a higher-priority vehicle $\veh{j}$ could be in any state that is reachable from $\veh{j}$'s initial state $x_j(\ldt) = x_{j0}$ and from which the target $\targetset_j$ can be reached. Mathematically, this is defined by the intersection of a FRS from the initial state $x_j(\ldt)=x_{j0}$ and the BRS defined in \eqref{eq:BRS} from the target set $\targetset_j$, $\brs_j(t) \cap \frs_j(t)$. In this situation, since $\veh{j}$ cannot be assumed to be using any particular feedback control, $\frs_j(t)$ is defined in \eqref{eq:FRS2}.
\vspace{-0.4em}

\begin{equation}
\label{eq:FRS2}
\begin{aligned}
&\frs_j(t) = \{y \in \R^{n_j}: \exists u_j(\cdot)\in\cfset_j, \exists d_j(\cdot) \in \dfset_j, \\
&x_j(\cdot) \text{ satisfies \eqref{eq:dyn}}, x_j(\ldt) = x_{j0}, x_j(t) = y\}
\end{aligned}
\end{equation}

This FRS can be computed by solving \eqref{eq:FRS_HJIVI} without obstacles, and with
\vspace{-0.4em}
\begin{equation}
H_j\left(t, x_j, p\right) = \min_{u_j \in \cset_j} \min_{d_j \in \dset_j} p \cdot f_j(t, x_j, u_j, d_j)
\end{equation}

In turn, the obstacle induced by a higher priority $\veh{j}$ for a lower priority vehicle $\veh{i}$ is as follows:

\begin{equation}
\begin{aligned}
\ioset_i^j(t) &= \{x_i: \dist(\pos_i, \pfrs_j(t)) \le \rc \}, \text{ with} \\
\pfrs_j(t) &= \{p: \exists \npos_j, (p, \npos_j) \in \brs_j(t) \cap \frs_j(t)\}
\end{aligned}
\end{equation}

Note that the centralized controller method described in the previous section can be thought of as the ``most restrictive control'' method, in which all vehicles must use the optimal controller at all times, while the least restrictive control method allows vehicles to use any suboptimal controller that allows them to arrive at the target on time. These two methods can be considered two extremes of a spectrum in which varying degrees of optimality is assumed for higher-priority vehicles. Vehicles can also choose a control strategy in the middle of the two extremes and for example use a control within some percentage of the optimal control, or use the optimal control unless some condition is met. The induced obstacles and the BRS can then be similarly computed using the allowed control authority.

\subsection{Robust Tracking} \label{sec:incomp_robust}
Even though it is not possible to commit and track an exact trajectory in presence of disturbances unlike in \cite{Chen15}, it might still be possible to instead \textit{robustly} track a feasible nominal trajectory with a bounded error at all times. If this can be done, then the tracking error bound can be used to determine the induced obstacles. Here, computation is done in two phases: the planning phase and the disturbance rejection phase. In the planning phase, we compute a nominal trajectory $x_{r,j}(\cdot)$ that is feasible in the absence of disturbances. In the disturbance rejection phase, we then compute a bound on the tracking error.

It is important to note that the planning phase does not make full use of the vehicle's control authority, as some margin is needed to reject unexpected disturbances while tracking the nominal trajectory. Therefore, in this method, planning is done for a reduced control set $\cset^p\subset\cset$. The resulting trajectory reference will not utilize the vehicle's full maneuverability; replicating the nominal control is therefore always possible, with additional maneuverability available at execution time to counteract external disturbances.

In the disturbance rejection phase, we determine the error bound independently of the nominal trajectory. To compute this error bound, we wish to find a robust controlled-invariant set in the joint state space of the vehicle and a tracking reference that may ``maneuver" arbitrarily in the presence of an unknown bounded disturbance. Taking a worst-case approach, the tracking reference can be viewed as a virtual evader vehicle that is optimally avoiding the actual vehicle to enlarge the tracking error. We therefore can model trajectory tracking as a pursuit-evasion game in which the actual vehicle is playing against the coordinated worst-case action of the virtual vehicle and the disturbance. 

Let $x_j$ and $x_r$ represent the state of the actual vehicle $\veh{j}$ and the virtual evader, respectively, and define the tracking error $e_j=x_j-x_r$. In cases where the error dynamics are independent of the absolute state as in \eqref{eq:edyn} (and also (7) in \cite{Mitchell05}), we can obtain error dynamics of the form
\begin{equation}
\label{eq:edyn} % Error dynamics
\begin{aligned}
\dot{e_j} &= f_{e_j}(t, e_j, u_j, u_r,d_j), \\
u_j &\in \cset_j, u_r\in\cset^p_j, d_j \in \dset_j, \quad t \in [0, T],
\end{aligned}
\end{equation}

To obtain bounds on the tracking error, we first conservative estimate the error bound around any reference state $x_{r,j}$, denoted $\errorbound_j(x_{r,j})$, and solve a reachability problem with its complement, $\errorbound_j^c$ as the target in the space of the error dynamics; $\errorbound_j^c$ is the set of tracking errors violating the error bound. From $\errorbound_j^c$, we compute the backward reachable set using \eqref{eq:HJIVI} without obstacles, and with the Hamiltonian
\vspace{-0.5em}
\begin{equation}
\label{eq:HJIVI_track}
H_j\left(t, e_j, p\right) = \max_{u_j \in \cset_j} \min_{u_r \in\cset^p_j, d_j \in \dset_j} p \cdot f_{e_j}(t, e_j, u_j, u_r,d_j)
\end{equation}

Letting the time horizon tend to infinity, we obtain the infinite-horizon controlled-invariant set, which we denote by $\disckernel_j$. If this set is nonempty, then the tracking error $e_j$ at flight time is guaranteed to remain within $\errorbound_j$ provided that the vehicle starts inside $\disckernel_j$ and subsequently applies the feedback control law implicitly defined in \eqref{eq:HJIVI_track}:

\begin{equation}
\label{eq:robust_tracking_law}
\tracklaw_j(e_j) = \arg\max_{u_j \in \cset_j} \min_{u_r \in\cset^p_j, d_j \in \dset_j} p \cdot f_{e_j}(t,e_j,u_j,u_r,d_j).
\end{equation}

Given $\errorbound_j$, we can guarantee that $\veh{j}$ will reach its target $\targetset_j$ if $\errorbound_j \subset \targetset_j$; thus, in the path planning phase, we modify $\targetset_j$ to be $\{x_j: \errorbound_j(x_j) \subseteq \targetset_j\}$, and compute a BRS, with the control authority $\cset^p_j$, that contains the initial state of the vehicle. From the resulting nominal trajectory $x_{r,j}(\cdot)$, the overall control policy to reach the destination can be then obtained using \eqref{eq:robust_tracking_law}.

Finally, since each vehicle $\veh{j}$ can only be guaranteed to stay within $\errorbound_j(x_{r,j})$, we must make sure at any given time, the error bounds of $\veh{i}$ and $\veh{j}$, $\errorbound_i(x_{r,i})$ and $\errorbound_j(x_{r,j})$, do not intersect. This can be done by choosing the induced obstacle to be the Minkowski sum\footnote{The Minkowski sum of sets $A$ and $B$ is the set of all points that are the sum of any point in $A$ and $B$.} of the error bounds. Thus,
\vspace{-0.3em}
\begin{equation}
\begin{aligned}
\ioset_i^j(t) &= \{x_i: \dist(\pos_i, \pfrs_j(t)) \le \rc \} \\
\pfrs_j(t) &= \{p: \exists \npos_j, (p, \npos_j) \in \errorbound(0) + \errorbound(x_{r,j}(t)) \},
\end{aligned}
\end{equation}
\noindent where $0$ denotes the origin. 

%% !TEX root = SPP_journal.tex
\section{Incomplete Information Results \label{sec:basic_results}}

% Intruder files
% !TEX root = ../SPP_journal.tex
\section{Response to Intruders \label{sec:intruder}}
In Section \ref{sec:incomp}, we made the basic SPP algorithm more robust by taking into account disturbances and considering situations in which vehicles may not have complete information about the control strategy of the other vehicles. However, if a vehicle not in the set of SPP vehicles enters the system, or even worse, if this vehicle is an adversarial intruder, the old plan can lead to a collision. If vehicles do not plan with additional safety margin that takes intruders into account, a vehicle trying to avoid the intruder may cause an unexpected conflict with another SPP vehicle, and effectively becoming an intruder itself. This may lead to a domino effect, causing multiple conflicts. In this section, we propose a method to allow vehicles to avoid an intruder while maintaining the SPP structure.

In general, the effect of intruders on the vehicles in structured flight can be entirely unpredictable, since the intruders in principle could be adversarial in nature, and the number of intruders could be arbitrary. Therefore, for our analysis to produce reasonable results, some assumptions about the intruders must be made.

\begin{assumption}
\label{as:avoidOnce}
At most one intruder affects the SPP vehicles at any given time. The intruder exits the altitude level affecting the SPP vehicles after a duration of $\iat$.
\end{assumption}

Assumption \ref{as:avoidOnce} implies that any vehicle $\veh_i$ would need to avoid the intruder $\veh_{\intr}$ for a maximum duration of $\iat$. This assumption can be valid in situations where intruders are rare, and that some fail-safe or enforcement mechanism exists to force the intruder out of the altitude level affecting the SPP vehicles. Note that we do not make any assumptions about $\tsa$; however, we assume that once it appears, it stays for a maximum duration of $\iat$.
%in addition, after avoiding the intruder, Qi can safely assume that it would not need to avoid another intruder

\begin{assumption}
\label{as:dynKnown}
Dynamics of the intruder are known and given by $\dot\state_\intr = f_\intr(\state_\intr, \ctrl_\intr, \dstb_\intr)$
\end{assumption}

Assumption \ref{as:dynKnown} is required for HJ reachability analysis. In situations where the dynamics of the intruder are not known exactly, a conservative model of the intruder may be used instead.

Based on the above two assumptions, we present two different algorithms to successfully avoid intruders for a duration of $\iat$. In both methods, our aim is to design a control policy that avoids a collision with the intruder as well as with other SPP vehicles, and ensure a transit to the destination. However, depending on the initial state of the intruder, its control policy, and the disturbances in the dynamics of a vehicle and the intruder, a vehicle might end-up at different states after avoiding the intruder. Therefore, a control policy that ensures a successful transit to the destination needs to account for all such possible states, which is a path planning problem with multiple (infinite, to be precise) initial states and a single destination, and is hard to solve in general. In this work, we divide the intruder avoidance problem into two sub-problems: (i) we first design a control policy that will ensure a successful transit to the destination if intruder does not appear and will successfully avoid the intruder, if it does. (ii) after the intruder disappears, we re-solve a SPP problem (that is, we ``re-plan"), where the initial states of the vehicles are given by the states that they end-up in after avoiding the intruder. 
The proposed methods mainly differ in the number of vehicles involved in the re-planning. 

%Suppose some vehicle $\veh_i$ starts avoiding the intruder $\veh_{\intr}$ at some time $t = \tsa$, and stops avoiding at $t = \tea$. When $t < \tsa$, $\veh_i$ must plan its path taking into account the possibility that it may need to avoid an intruder $\veh_i$. Since $\veh_i$ may spend a duration of up to $\tsa$ performing avoidance, its induced obstacles $\ioset_k^i(t), k>i$ need to be computed in a way that reflects this possibility. The induced obstacles computation is discussed in Section \ref{sec:intruder_iocomp}.
%
%We must also ensure that while avoiding the intruder, $\veh_i$ does not collide with the total obstacle set $\obsset_i(t)$. This requires augmenting the total obstacle set to produce the augmented total obstacle $\tilde\obsset_i(t)$; the computation of $\tilde\obsset_i(t)$ and the controller that guarantees the avoidance of the augmented obstacles are discussed in Section \ref{sec:intruder_aocomp}.
%
%When $t > \tea$, $\veh_i$ no longer needs to take into account the possibility of any other intruders, and can simply avoid the unaugmented obstacles $\obsset_i(t)$ while getting to the target. The associated controller and the time-to-reach upper bound after intruder avoidance are discussed in Section \ref{sec:intruder_after}.
%
%Finally, Section \ref{sec:intruder_avoid} describes how $\veh_i$ can guarantee collision avoidance with the intruder.
%
%\MCnote{I wonder if the material below is needed}
%\begin{itemize}
%\item \textbf{Definitions of 3 different reachable sets (2 additional reachable sets $\frs(\iat, \brs(\ldt, \targetset))$, $\brs(\bar t, \targetset)$ where $\bar t$ is the time vehicle stops avoiding intruder)}.
%\end{itemize}
%
%We are now ready to develop a general theory that takes intruders into account. Our approach to a robust path planning can be summarized in the following steps:
%\begin{itemize}
%\item \textit{Step-1}: First, we compute the set of obstacles induced by the higher priority vehicles for the lower priority vehicles.  
%\item \textit{Step-2}: We then append \textit{all} obstacles (including static ones) by a forward reachable set (FRS) of duration $\iat$. This addendum ensures that if a vehicle starts outside this appended obstacle, then it cannot collide with the obstacle in $\iat$ seconds. Next, we compute a backwards reachable set (BRS) while avoiding these obstacles till it contains the initial state of the vehicle. This set will give us the controller that ensures liveness in the absence of intruders. 
%\item \textit{Step-3}: Compute a $\iat$-step FRS of the BRS computed in Step-2. This FRS is the free flight region that allows a vehicle to avoid any intruder for $\iat$ seconds. We then compute a BRS while avoding the obstacles induced by the higher priority vehicles (computed in Step-1) \textit{and} that contains the FRS calculated in Step-3. This BRS will give us a controller that guarantee intruder avoidance and liveness. 
%\item \textit{Step-4}: Compute the relative state space wherein we can successfully avoid the intruder for $\iat$ seconds. If a vehicle sense an intruder while it is still outside this region, the above algorithm will ensure safety at all times.  
%\end{itemize}

%The overall control policy for avoiding intruder and reaching target is thus given by:
%\begin{equation*}
%u^* = 
%\left \{ 
%\begin{array}{ll}
%\ctrl^*_\text{AO}(t, \state_i; \targetset_i, \tilde\obsset_i, \ham_\text{AO}) & t \leq \underbar{t}\\
%\ctrl^*_\text{CA}(t, \state_r; \targetset_\text{CA}, \obsset_\text{CA}, \ham_\text{CA}) & \underbar{t} \leq t \leq \bar{t} \\
%\ctrl^*(t, \state_i; \targetset_i, \obsset_i, \ham_\text{L}) & t \geq \bar{t}
%\end{array}
%\right.
%\end{equation*}

\textbf{To-Dos:}
\begin{itemize}
\item Something about the overall methodology of each method (in respective sections) before going into the technical details.\\ 
\item We don't have to re-plan for all vehicles in Method-1. We can in theory find out the vehicles that are impacted by intruders (using a reachability calculation) and just re-plan for them.\\
\item A remark about the single vehicle replanning property of Method-2. Moreover, Method-2 can, in theory, handle multiple intruders as long as they are affecting different vehicles. Though, we have to replan for several vehicles in that case.  \\
\item Once the replanning is complete, another intruder can appear in the system. So strictly speaking we are making an assumption that atmost one intruder is in the system \textit{at any given time} as opposed to throughout the trajectory. \\
\item Make sure the notation in this sesction is consistent with the rest of the paper.\\
\item For method-2 results, it may be helpful to include a figure which is showing the division of space among vehicles at some time (probably right before the intruder enters). 
\end{itemize}

% !TEX root = ../SPP_journal.tex
%Intuitively, the number of vehicles that are affected by an intruder depends on how close the vehicles are to each other. For example, if the vehicles are in a dense configuration, then multiple vehicles might get affected by the intruder simultaneously and hence re-planning is required at a larger scale. In this section, we propose a method that guarnatee intruder avoidance while allowing vehicles to be in a dense configuration, but may require a full re-planning of the system after the intruder disappears.     
Suppose some vehicle $\veh_i$ starts avoiding the intruder $\veh_{\intr}$ at some time $t = \tsa$, and stops avoiding at $t = \tea$. When $t < \tsa$, $\veh_i$ must plan its path taking into account the possibility that it may need to avoid an intruder $\veh_i$. Since $\veh_i$ may spend a duration of up to $\iat$ performing avoidance, its induced obstacles $\ioset_k^i(t), k>i$ need to be computed in a way that reflects this possibility. The induced obstacles computation is discussed in Section \ref{sec:intruder_iocomp}.

We must also ensure that while avoiding the intruder, $\veh_i$ does not collide with the total obstacle set $\obsset_i(t)$. This requires augmenting the total obstacle set to produce the augmented total obstacle $\tilde\obsset_i(t)$; the computation of $\tilde\obsset_i(t)$ and the controller that guarantees the avoidance of the augmented obstacles are discussed in Section \ref{sec:intruder_aocomp}.

In Section \ref{sec:intruder_avoid}, we describe how $\veh_i$ can guarantee collision avoidance with the intruder.

Finally, when $t > \tea$, $\veh_i$ has already successfuly avoided the intruder, but depending on the state it ends up in while avoiding the intruder, it might need to re-plan its trajectory to reach the target safely. The re-planning process is discussed in Section \ref{sec:replan_method1}.

%\begin{itemize}
%\item \textbf{Definitions of 3 different reachable sets (2 additional reachable sets $\frs(\iat, \brs(\ldt, \targetset))$, $\brs(\bar t, \targetset)$ where $\bar t$ is the time vehicle stops avoiding intruder)}.
%\end{itemize}
%
%We are now ready to develop a general theory that takes intruders into account. Our approach to a robust path planning can be summarized in the following steps:
%\begin{itemize}
%\item \textit{Step-1}: First, we compute the set of obstacles induced by the higher priority vehicles for the lower priority vehicles.  
%\item \textit{Step-2}: We then append \textit{all} obstacles (including static ones) by a forward reachable set (FRS) of duration $\iat$. This addendum ensures that if a vehicle starts outside this appended obstacle, then it cannot collide with the obstacle in $\iat$ seconds. Next, we compute a backwards reachable set (BRS) while avoiding these obstacles till it contains the initial state of the vehicle. This set will give us the controller that ensures liveness in the absence of intruders. 
%\item \textit{Step-3}: Compute a $\iat$-step FRS of the BRS computed in Step-2. This FRS is the free flight region that allows a vehicle to avoid any intruder for $\iat$ seconds. We then compute a BRS while avoding the obstacles induced by the higher priority vehicles (computed in Step-1) \textit{and} that contains the FRS calculated in Step-3. This BRS will give us a controller that guarantee intruder avoidance and liveness. 
%\item \textit{Step-4}: Compute the relative state space wherein we can successfully avoid the intruder for $\iat$ seconds. If a vehicle sense an intruder while it is still outside this region, the above algorithm will ensure safety at all times.  
%\end{itemize}

\subsubsection{Induced Obstacle Computation \label{sec:intruder_iocomp}}
The goal of this section is to compute, for each lower priority vehicle $\veh_i$, the time-varying obstacles induced by each higher priority vehicle $\veh_j, j < i$, denoted by $\ioset_i^j(t)$. As before, the total obstacle set $\obsset_i(t)$ can then be obtained using \eqref{eq:obsseti}. 

Depending on the information known to a lower priority vehicle about $\veh_j$'s control strategy, we can use one of the three methods described in Section \ref{sec:incomp} to compute the ``base" obstacles $\boset^j(t)$ , the obstacles that would have been induced by $\veh_j$ in the absence of an intruder. The induced obstacles, $\ioset_i^j(t)$, are then given by the states a vehicle can reach while avoiding the intruder, on top of the base obstacles. Since a vehicle avoids the intruder for a maximum of $\iat$, these states can be given by the $\iat$-horizon FRS of the base obstacles. Regardless of what control is used by $\veh_j$ for avoidance, it still remains within the FRS $\ioset_i^j(t) := \frs_{\mathcal{O}}(\iat, \boset^j(t-\iat), \emptyset, \ham_{\mathcal{O}})$, which is the set of all possible states that $\veh_j$ can reach after a duration of $\iat$ starting from inside $\boset^j(t-\iat)$. Here, the Hamiltonian is given by

\begin{equation}
\ham_{\mathcal{O}}(\state_j, p) = \max_{\ctrl_j \in \cset_j} \max_{\dstb_j \in \dset_j} p \cdot f_j (\state_j, \ctrl_j, \dstb_j)
\end{equation}
%
%Since there are no moving vehicle obstacles for the highest priority vehicle, $\obsset_1(t) = \soset$. 
%
%Computation of these base obstacles would requires information of a corresponding ``base" BRS of $\veh_j$; the process for computing this set is outlined in Step 2. In this section, we assume that the sequence of base obstacles, $\boset_i^j(t)$, is known. Given $\boset_i^j(t)$, we now show how to compute the obstacle set $\obsset_i(t)$. The induced obstacles are given by the states a vehicle can reach while avoiding the intruder, on top of the base obstacles. 

% !TEX root = ../SPP_journal.tex
\subsubsection{Augmented Obstacle Computation \label{sec:intruder_aocomp}}
We next need to ensure that $\veh_i$ doesn't collide with the obstacle set $\obsset_i(t)$ computed in Section \ref{sec:intruder_iocomp} even when it is avoiding the intruder. This can be achieved by ensuring that $\veh_i$ is sufficently far from $\obsset_i(t)$ such that regardless of the used avoidnace control, it cannot collide with $\obsset_i(t)$. In particular, we can compute a region around the obstacles $\obsset_i(\cdot)$ such that for all disturbances, $\veh_i$ can avoid colliding with obstacles for $\iat$ seconds, if $\veh_i$ starts outside this region. Augmenting this region to $\obsset_i(\cdot)$ will give us augmented obstacles, $\tilde\obsset_i(t)$, that can then be used during the path planning of $\veh_i$ to ensure collision avoidance with $\obsset_i(t)$.  

To ensure that a vehicle does not hit the obstacle $\obsset_i(t_1 + t')$ at time $t = t_1 + t'$ starting at $t = t_1$, even when it applies any control for the next $t'$ seconds, it suffices to avoid the $t'$-horizon BRS of $\obsset_i(t_1 + t')$. This argument applies for all obstacles to appear in the next $\iat$ seconds to ensure safety under any controller and disturbance for the next $\iat$ seconds. Mathematically,

\begin{equation} \label{eqn:inducedobs}
\tilde\obsset_i(t) = \bigcup_{\tau \in [0, \iat]} \brs_{\mathcal{G}}(\tau, \obsset_i(t+\tau), \emptyset, \ham_{\mathcal{G}})
\end{equation}
where $\brs_{\mathcal{G}}(\tau, \obsset_i(t+\tau), \emptyset, \ham_{\mathcal{G}})$ represents BRS of $\obsset_i(t+\tau)$ computed backwards for $\tau$ seconds. The Hamiltonian 
$\ham_{\mathcal{G}}$ is given by:

\begin{equation} \label{eqn:BRS_obsham}
\ham_{\mathcal{G}}(\state_i, p) = \min_{\ctrl_i \in \cset_i} \min_{\dstb_i \in \dset_i} p \cdot f_i (\state_i, \ctrl_i, \dstb_i)
\end{equation}

Finally, we compute a BRS that contains the initial state of $\veh_i$ while avoiding these augmented obstacles, $\brs_\text{AO}(t, \targetset_i, \bar\obsset_i, \ham_\text{AO})$, where $\ham_\text{AO}$ is given by:
\begin{equation} \label{eqn:BRSham}
\ham_\text{AO}(\state_i, p) = \min_{\ctrl_i \in \cset_i} \max_{\dstb_i \in \dset_i} p \cdot f_i (\state_i, \ctrl_i, \dstb_i)
\end{equation}

Note that $\brs_\text{AO}$ ensures liveness for $\veh_i$ in the absence of intruder. Moroever, if we start within $\brs_\text{AO}$, we are guaranteed to avoid collision for $\iat$ seconds, irrespective of the control and disturbance applied. 

\subsubsection{Optimal Avoidance Controller \label{sec:intruder_avoid}}
First, we define relative coordinates of the intruder $\veh_{\intr}$ with state $\state_\intr$ with respect to $\veh_i$ with state $\state_i$.

\begin{equation}
\label{eq:reldyn}
\begin{aligned}
\state_r &= \state_\intr - \state_i \\
\dot \state_r &= f_r(\state_r, \ctrl_i, \ctrl_\intr, \dstb_i, \dstb_\intr)
\end{aligned}
\end{equation}

Given the relative dynamics, we compute the set of states from which the joint states of $\veh_{\intr}$ and $\veh_i$ can enter danger zone $\dz_{i\intr}$ despite the best efforts of $\veh_i$ to avoid $\veh_{\intr}$. Under the relative dynamics \eqref{eq:reldyn}, this set of states is given by the backwards reachable set $\brs_\text{CA}(\iat, \targetset_\text{CA}, \obsset_\text{CA}, H_\text{CA})$, with

\begin{equation}
\begin{aligned}
\targetset_\text{CA} &= \{\state_r: \|\pos_r\|_2 \le \rc\},~\obsset_\text{CA} = \emptyset \\
H_\text{CA}(\state_r, p) &= \max_{\ctrl_i \in \cset_i} \min_{\ctrl_\intr \in \cset_\intr, \dstb_i \in \dset_i, \dstb_\intr \in \dset_\intr} p \cdot f_r(\state_r, \ctrl_i, \ctrl_\intr, \dstb_i, \dstb_\intr)
\end{aligned}
\end{equation}

Once the value function $\valfunc_\text{CA}(t, \state_r)$ representing $\brs_\text{CA}(\iat, \targetset_\text{CA}, \obsset_\text{CA}, H_\text{CA})$ is computed, the optimal avoidance control $\ctrl^*_\text{CA}$ can be obtained as:
\begin{equation}
\ctrl^*_\text{CA} = \arg \max_{\ctrl_i \in \cset_i} \min_{\ctrl_\intr \in \cset_\intr, \dstb_i \in \dset_i, \dstb_\intr \in \dset_\intr} p \cdot f_r(\state_r, \ctrl_i, \ctrl_\intr, \dstb_i, \dstb_\intr)
\end{equation}

Under normal circumstances when the intruder $\veh_{\intr}$ is far away, we have $\valfunc_\text{CA}(\iat, \state_r) > 0$; as the $\veh_{\intr}$ gets closer to $\veh_i$, $\valfunc_\text{CA}(\iat, \state_r)$ decreases. If $\veh_i$ applies the control $\ctrl^*_\text{CA}$ when $\valfunc_\text{CA}(\iat, \state_r) = 0$, then collision avoidance between $\veh_i$ and $\veh_{\intr}$ is guaranteed for a duration of $\iat$ under the worst-case intruder control strategy $\ctrl_\intr$.

In addition, obstacle augmentation ensures that during the avoidance maneuver $\veh_i$ will not collide with $\obsset_i(\cdot)$. %Therefore, applying $\ctrl_I^A$ for a duration of $\iat$ is still guaranteed to keep $\veh_i$ safe from all obstacles, and hence safe from collision with respect to all other vehicles $\veh_j, j \neq i$.
The overall control policy for avoiding intruder and collision with other vehicles is thus given by:
\begin{equation*}
\ctrl^*_{i, \text{A}}(t) = 
\left \{ 
\begin{array}{ll}
\ctrl^*_\text{AO}(t, \state_i; \targetset_i, \tilde\obsset_i, \ham_\text{AO}) & t \leq \underbar{t}\\
\ctrl^*_\text{CA}(t, \state_r; \targetset_\text{CA}, \obsset_\text{CA}, \ham_\text{CA}) & \underbar{t} \leq t \leq \bar{t}
\end{array}
\right.
\end{equation*}

\subsubsection{Re-planning after intruder avoidance\label{sec:replan_method1}} 
After the intruder disappears, a liveness controller to ensure that the vehicles reach their destinations can be obtained by solving a SPP problem as described in Section \ref{sec:incomp}, where the starting states of the vehicles are now given by the states they end up in after avoiding the intruder.  
Let the optimal control policy corresponding to this liveness controller is given by $\ctrl^*_{i, \text{L}}(t)$. The overall control policy that ensures intruder avoidance, collision avoidance with other vehicles and successful transition to the destination is given by:

\begin{equation*}
\ctrl_i^*(t) = 
\left \{ 
\begin{array}{ll}
\ctrl^*_{i, \text{A}}(t) & t \leq \bar{t}\\
\ctrl^*_{i, \text{L}}(t) & t \geq \bar{t}
\end{array}
\right.
\end{equation*}

\begin{remark}
Note that we only need to re-plan the trajectories of the vehicles that are affected by the intruder. In particular, if $\valfunc_\text{CA}(\iat, \state_r) > 0$ during the entire duration of $\iat$ for a vehicle, then the vehicle need not apply any avoidance control, and hence a re-planning is not required for this vehicle. 
\end{remark}

\begin{remark}
Note that even though we have presented the analysis for one intruder, the proposed method can handle multiple intruders as long as only one intruder is present \textit{at any given time}. 
\end{remark}

\SBnote{Summary algorithms with reference to basic SPP algorithm}
%% !TEX root = SPP_journal.tex
\subsection{Method 2: Single vehicle re-planning \label{sec:intruder_method2}}
As an alternative to Method 1, one can think about dividing the flight space of vehicles such that at any given time, any two vehicles are far enough from each other such that an intruder can only affect one vehicle in a duration of $\iat$ despite its best efforts. The advantage of this approach is that after the intruder disappears, we only have to re-plan the trajectory of a single vehicle regardless of the number of total vehicles in the system, which makes this approach particularly suitable for practical systems, as the re-planning needs to be done in real time.

In this method, we build upon this intuition and show that such a division of space is indeed possible. In Section \ref{sec:spaceDiv}, we show that regardless of the initial position of the intruder, atmost one vehicle needs to apply the avoidance maneuver under this space division. However, we still need to ensure that a vehicle do not collide with another vehicle while avoiding the intruder. The induces obstacles that reflect this possibility are computed in Section \ref{sec:intruder_iocomp_met2}. Intruder avoidance control and re-planning are discussed in Section \ref{sec:replan_method2}.

\subsubsection{Space division \label{sec:spaceDiv}}
Let $\state_r$ denotes the relative co-ordinates of $\veh{\intr}$ with respect to $\veh{j}$ as in \eqref{eq:reldyn}. Given the relative dynamics, we compute the set of states from which the joint states of $\veh{\intr}$ and $\veh{j}$ can enter danger zone $\dz_{j\intr}$ when both $\veh{j}$ and $\veh{\intr}$ are taking \textit{best actions to collide} with each other. Under the relative dynamics \eqref{eq:reldyn}, this set of states is given by the backwards reachable set $\brs_{\text{C},j}(\iat, \targetset_\text{C}, \emptyset, H_\text{C})$, with

\begin{equation}
\begin{aligned}
\targetset_{\text{C},j} &= \{\state_r: \|\pos_r\|_2 \le \rc\} \\
H_\text{C}(\state_r, p) &= \min_{\ctrl_j, \ctrl_\intr, \dstb_j, \dstb_\intr} p \cdot f_r(\state_r, \ctrl_j, \ctrl_\intr)
\end{aligned}
\end{equation}

The interpretation of set $\brs_{\text{C},j}$ is that if the $\veh{\intr}$ starts outside this set (in relative co-ordinates), then no matter what control $\veh{\intr}$ and $\veh{j}$ apply for the next $\iat$ seconds, they can't collide with each other. Thus, $\veh{j}$ need not avoid such an intruder and can apply any desired control. In other words, $\veh{j}$ should avoid the intruder if intruder starts inside $\brs_{\text{C},j}$, otherwise should not. We can similarly compute $\brs_{\text{C},i}$, which denotes the region around $\veh{i}$, where it needs to avoid the intruder. 

We now go back to the absolute coordinates and explain how we can achieve the desired space division. We first compute the base obstacles $\boset^j(t)$ induced by $\veh{j}$ as disucssed in Section \ref{sec:intruder_iocomp}. Once the base obstacles are computed, we augment every base obstacle with $\brs_{\text{C},j}$. This can be achieved by taking Minkowski sum of $\boset^j(t)$ and $\brs_{\text{C},j}$:
\begin{equation}
^1\intobs^j(t) = \boset^j(t) + \brs_{\text{C},j}.
\end{equation}
Note that if the initial state of the intruder $\state_{\intr0}$ is not in $^1\intobs^j(\underbar{t})$, then $\veh{j}$ need not avoid the intruder, where $\underbar{t}$ is the time at which intruder appears in the system.

However, to ensure that atmost one vehicle needs to apply the avoidance maneuver, we need to make sure that no other vehicle $\veh{i}, i \neq j$, needs to avoid the intruder if intruder starts inside $\brs_\text{C}$. This can  be achieved by ensuring that $\brs_{\text{C},i}$ and $\brs_{\text{C},j}$ do not intersect.
% !TEX root = ../SPP_journal.tex
\subsection{Intruder Results \label{sec:basic_results}}
To illustrate that our SPP method is robust with respect to disturbances as well as a single intruder that is present for a duration of $\iat$, we use a five-vehicle example in which one of the five vehicles is an intruder. We assume that each vehicle has the dynamics given in \eqref{eq:dyn_i}. For this example, we chose the parameters $\underline{v} = 0.1, \bar{v} = 1, \bar\omega = 1$, and disturbance bounds $d_{r} = 0.1, \bar{d_{\theta}} = 0.2$, which correspond to a 10\% uncertainty in the dynamics. 

The vehicles' initial states, scheduled times of arrival, and target sets are the same as those described in Section \ref{sec:sim_dstb}, except that in this example, we have increased the target radius to $r=0.15$. For illustrate purposes, we have chosen to use the robust trajectory tracking method described in Section \ref{sec:rtt} for disturbance rejection, and hence each vehicle tracks a nominal trajectory.

Fig. \ref{fig:intruder1_traj} shows the simulation at $t = -2.39$, which corresponds to the time at which the intruder ``disappears'' from the domain. This time is chosen to maximally highlight the impact of the intruder. Here, the intruder is shown in black, and the SPP vehicles are shown in the other different colors.

By the time $t = -2.39$, vehicle $\veh_2$ (red) and vehicle $\veh_3$ (green) have been avoiding the intruder for some time. This is evident from the amount of deviation between the actual positions of vehicles $\veh_2$ and $\veh_3$ and their nominal positions specified by the nominal trajectories they originally planned to track; these vehicles have abandoned nominal trajectory tracking in order to ensure safety with respect to the intruder. In contrast, $\veh_4$ (magenta), which has not needed to avoid the intruder, is tracking the nominal trajectory very closely.

One may notice that the SPP vehicles are rather far apart. This is because a large margin is needed to ensure that they maintain separation even when multiple vehicles need to avoid the intruder. In this example in particular, the lowest-priority vehicle $\veh_4$ needed to depart very early compared to $\veh_2$ and $\veh_3$ so that if an intruder were to arrive, $\veh_4$ does not impede the ability of the other vehicles to perform avoidance. The early departure of $\veh_4$ can be inferred from the fact that at $t=-2.39$, it is already nearly at its target.

For the same reason, the highest-priority vehicle $\veh_1$ has not departed from its initial state yet, and thus is not shown at $t=-2.39$. $\veh_2$ and $\veh_3$ needed to depart very early compared to $\veh_1$ to ensure sufficient margin for avoidance maneuvers.

Fig. \ref{fig:intruder1_diff} shows the nominal (black) and actual trajectories (red and green respectively) of vehicles $\veh_2$ (top subplot) and $\veh_3$ (bottom subplot). Specifically, the $x$ and $y$ positions over time are shown, and the black dotted vertical lines indicate the time interval in which the intruder is present. From Fig. \ref{fig:intruder1_diff}, one can clearly see that before the intruder was present, both vehicles are able to track their nominal trajectories closely. When the intruder appears, the vehicles deviate from their nominal trajectories significantly. After the intruder disappears, both vehicles replan new trajectories according to Section \ref{sec:replan_method1}, and at a later time, the resulting actual trajectories eventually arrive at the same location as the nominal trajectories.

\begin{figure}[H]
  \centering
  \includegraphics[width=\columnwidth]{"fig/intruder1_traj"}
  \caption{The positions of the SPP vehicles and the intruder at $t=-2.39$, the end of the intruder's appearance. The red and green vehicles $\veh_2, \veh_3$ have not been tracking their nominal trajectories for a while, and have been avoiding the intruder instead. Thus, their positions are far away from their nominal trajectories, indicated by the small colored circles. $\veh_4$ has not needed to avoid the intruder, and tracks its nominal trajectory closely. The nominal trajectory of $\veh_4$ allows it to stay far enough away from other vehicles so that all vehicles can remain safe in the presence of the intruder.}
  \label{fig:intruder1_traj}
\end{figure}

\begin{figure}[H]
  \centering
  \includegraphics[width=\columnwidth]{"fig/intruder1_diff"}
  \caption{The difference between the initially planned nominal trajectories and the actual trajectories for vehicles $\veh_2$ (top subplot) and $\veh_3$ (bottom subplot). which needed to perform avoidance with respect to the intruder during the time interval marked by the vertical black dotted lines. Before the intruder's presence, both vehicles track their nominal trajectories closely; however, both vehicles later deviate significantly from their nominal trajectories in order to avoid the intruder. After the intruder is gone, both vehicles replan their trajectories and arrive at their targets at a later time.}
  \label{fig:intruder1_diff}
\end{figure}

% !TEX root = ./SPP_IoTjournal.tex
\section{Conclusion}
Provably safe multi-vehicle path planning in an important problem that needs to be addressed to ensure that vehicles can fly in close proximity of each other. Recently, the SPP algorithm was proposed for multi-vehicle path planning problem that scales linearly with the number of vehicles. We illustrate the full potential of the algorithm by using it for large-scale multi-vehicle path planning problems under different flying conditions. We demonstrate how different types of space-time trajectories emerge naturally out of the algorithm for different disturbance conditions and other problem parameters. The reactivity of the obtained controller is also demonstrated under different wind conditions.
%%%%%%%%%%%%%%%%%%%%%%%%%%%%%%%%%%%%%%%%%%%%%%%%%%%%%%%%%%%%%%%%%%%%%%%%%%%%%%%%
%\addtolength{\textheight}{1cm}   % This command serves to balance the column lengths
                                  % on the last page of the document manually. It shortens
                                  % the textheight of the last page by a suitable amount.
                                  % This command does not take effect until the next page
                                  % so it should come on the page before the last. Make
                                  % sure that you do not shorten the textheight too much.

\bibliographystyle{IEEEtran}
\bibliography{IEEEabrv,references}
\end{document}
