%%%%%%%%%%%%%%%%%%%%%%%%%%%%%%%%%%%%%%%%%%%%%%%%%%%%%%%%%%%%%%%%%%%%%%%%%%%%%%%%
%2345678901234567890123456789012345678901234567890123456789012345678901234567890
%        1         2         3         4         5         6         7         8

%\documentclass[letterpaper, 10 pt, conference]{ieeeconf}  % Comment this line out
                                                          % if you need a4paper
\documentclass[letterpaper, 10pt, conference]{ieeeconf}      % Use this line for a4
                                                          % paper

 
\IEEEoverridecommandlockouts                              % This command is only
                                                          % needed if you want to
                                                          % use the \thanks command
\overrideIEEEmargins
% See the \addtolength command later in the file to balance the column lengths
% on the last page of the document

\usepackage{amsmath}    % need for sub equations
\usepackage{amsfonts}
\usepackage{graphicx}   % need for figures
\usepackage{subcaption}
\usepackage{epsfig} 
\usepackage{cancel}
\usepackage{amssymb}
\usepackage{color}
\usepackage{my_macros}
\usepackage[ruled,vlined,titlenotnumbered]{algorithm2e} 

\title{\LARGE \bf
Safe Sequential Path Planning of Multi-Vehicle Systems via Double-Obstacle Hamilton-Jacobi-Isaacs Variational Inequality}

\author{Mo Chen, Jaime F. Fisac, Shankar Sastry, Claire J. Tomlin
\thanks{This work has been supported in part by NSF under CPS:ActionWebs (CNS-931843), by ONR under the HUNT (N0014-08-0696) and SMARTS (N00014-09-1-1051) MURIs and by grant N00014-12-1-0609, by AFOSR under the CHASE MURI (FA9550-10-1-0567). The research of J.F. Fisac has received funding from the ``la Caixa" Foundation.}
\thanks{All authors are with the Department of Electrical Engineering and Computer Sciences, University of California, Berkeley. \{mochen72, jfisac, sastry, tomlin\}@eecs.berkeley.edu}
}

\begin{document}
\maketitle
\thispagestyle{empty}
\pagestyle{empty}

%%%
\begin{abstract}
We consider the problem of planning trajectories for a group of $N$ vehicles, each aiming to reach its own target set while avoiding danger zones of other vehicles. The analysis of problems like this is extremely important practically, especially given the growing interest in utilizing unmanned aircraft systems for civil purposes. The direct solution of this problem by solving a single-obstacle Hamilton-Jacobi-Isaacs (HJI) variational inequality (VI) is numerically intractable due to the exponential scaling of computation complexity with problem dimensionality. Furthermore, the single-obstacle HJI VI cannot directly handle situations in which vehicles do not have a common scheduled arrival time. Instead, we perform sequential path planning  by considering vehicles in order of priority, modeling higher-priority vehicles as time-varying obstacles for lower-priority vehicles. To do this, we solve a double-obstacle HJI VI which allows us to obtain the reach-avoid set, defined as the set of states from which a vehicle can reach its target while staying within a time-varying state constraint set. From the solution of the double-obstacle HJI VI, we can also extract the latest start time and the optimal control for each vehicle. This is a first application of the double-obstacle HJI VI which can handle systems with time-varying dynamics, target sets, and state constraint sets, and results in computation complexity that scales linearly, as opposed to exponentially, with the number of vehicles in consideration.
\end{abstract}

% !TEX root = SPP2.tex
\section{Introduction}
Recently, there has been an immense surge of interest in using unmanned aerial vehicles (UAVs) for civil purposes. The applications of UAVs extend well beyond package delivery, and include aerial surveillance, disaster response, and other important tasks \cite{Tice91, Debusk10, Amazon16, AUVSI16, BBC16}. Many of these applications will involve UAVs flying in an urban environment, potentially in close proximity of humans. As a result, government agencies such as the Federal Aviation Administration (FAA) and National Aeronautics and Space Administration (NASA) of the United States are urgently trying to develop new scalable ways to organize an air space in which potentially thousands of UAVs can fly \cite{FAA13, NASA16}.

One essential problem that needs to be addressed is how a group of vehicles in the same vicinity can reach their destinations while avoiding collision with each other. Several previous studies have attempted to address this problem. In some of these studies, specific control strategies for the vehicles or moving entities are assumed, and approaches such as induced velocity obstacles have been used \cite{Fiorini98, Chasparis05, Vandenberg08}. Other researchers have used ideas involving virtual potential fields to maintain collision avoidance while maintaining a specific formation \cite{Saber02, Chuang07}. Although interesting results emerge from these previous studies, simultaneous trajectory planning and collision avoidance are not considered. 

In the past, trajectory planning and collision avoidance problems in safety-critical systems have been studied using reachability analysis, which provides guarantees on the success and safety of optimal system trajectories \cite{Barron90, Mitchell05, Bokanowski10, Margellos11, Fisac15}. In reachability analysis, one computes the reachable set, defined as the set of states from which the system can be driven to a target set. Reachability analysis has been successfully used in applications involving systems with no more than two vehicles, such as pairwise collision avoidance \cite{Mitchell05}, automated in-flight refueling \cite{Ding08}, two-player reach-avoid games \cite{Huang11}, and many others \cite{Bayen07}.

%In addition to the guarantees reachability theory provides and the evident flexibility of reachability theory for analyzing vastly different systems with nonlinear dynamics, many numerical tools for solving reachability problems are also available, making the approach practically appealing \cite{Mitchell05, Sethian96, Osher02, LSToolbox}.

Despite the advantages of reachability analysis, it cannot be directly applied to scenarios involving complex high dimensional systems such as multi-vehicle systems. The computation of reachable sets involves solving a Hamilton-Jacobi (HJ) partial differential equation (PDE) on a grid representing a discretization of the state space, causing an exponential scaling of computation complexity with respect to the dimension of the system, or roughly speaking, with the number of vehicles present.

In this paper, we build on the work in \cite{Chen15}, and assume a reasonable structure in the multi-vehicle path planning problem. In the sequential path planning (SPP) scheme, vehicles are assigned some priority. Higher-priority vehicles may ignore the lower-priority vehicles, who must take into account the presence of higher-priority vehicles by treating them as induced time-varying obstacles. Unlike the work in \cite{Chen15}, we incorporate disturbances for all vehicles and consider three different assumptions on the information each of the vehicles may have access to, making the sequential path planning substantially more practical. For each of the assumed information patterns, we propose a reachability-based method to compute the induced obstacles that would guarantee collision avoidance as well as successful transit to the destination. We demonstrate and compare our proposed methods through numerical simulations.
% Introduction (1-1.5p)
%% Motivation
%% Related work
%% Summary of results

% !TEX root = nextUAVsched.tex
\section{Problem Formulation \label{sec:formulation}}
Consider $N$ vehicles $P_i,i=1\ldots,N$, each trying to reach one of $N$ target sets $\target_i,i=1\ldots,N$, while avoiding obstacles and collision with each other. Each vehicle $i$ has states $\x_i\in \R^{n_i}$ and travels on a domain $\amb=\obs \cup \free\in\R^p$, where $\obs$ represents the obstacles that each vehicle must avoid, and $\free$ represents all other states in the domain on which vehicles can move. Each vehicle $i = 1,2,\ldots,N$ moves with the following dynamics for $t\in[\tnow_i, \tf_i]$:

\begin{equation} \label{eq:dyn}
\dotx_i = f_i (t, \x_i, \ctrl_i), \quad\x_i(\ti_i) = \x_i^0 
\end{equation}

\noindent where $\x_i^0$ represents the initial condition of vehicle $i$, and $\ctrl_i(\cdot)$ represents the control function of vehicle $i$. In general, $f_i(\cdot,\cdot,\cdot)$ depends on the specific dynamic model of vehicle $i$, and need not be of the same form across different vehicles. Denote $\pos_i\in\R^p$ the subset of the states that represent the position of the vehicle. Given $\pos_i^0\in\free$, we define the admissible control function set for $P_i$ to be the set of all control functions such that $\pos_i(t) \in \free \forall t\ge \ti_i$. Denote the joint state space of all vehicles $\x \in \R^n$ where $n = \sum_i n_i$, and their joint control $\ctrl$.

We assume that the control functions $\ctrl_i(\cdot)$ are drawn from the set $\ctrlf_i := \{\ctrl_i: [\tnow_i, \tf_i] \rightarrow \ctrlin_i, \text{measurable}$\footnote{
A function $f:X\to Y$ between two measurable spaces $(X,\Sigma_X)$ and $(Y,\Sigma_Y)$ is said to be measurable if the preimage of a measurable set in $Y$ is a measurable set in $X$, that is: $\forall V\in\Sigma_Y, f^{-1}(V)\in\Sigma_X$, with $\Sigma_X,\Sigma_Y$ $\sigma$-algebras on $X$,$Y$.}\} where $\ctrlin_i \in \R^{n^\ctrl_i}$ is the set of allowed control inputs. Furthermore, we assume $f_i(t,\x_i, \ctrl_i)$ is bounded, Lipschitz continuous in $\x_i$ for any fixed $t,\ctrl_i$, and measurable in $t, \ctrl_i$ for each $\x_i$. Therefore given any initial state $\x_i^0$ and any control function $\ctrl_i(\cdot)$, there exists a unique, continuous trajectory $\x_i(\cdot)$ solving (\ref{eq:dyn}) \cite{coddington55}.

The goal of each vehicle $i$ is to arrive at $\target_i \subset \R^{n_i}$ at or before some scheduled time of arrival (STA) $\tf_i$ in minimum time, while avoiding obstacles and danger with all other vehicles. The target sets $\target_i$ can be used to represent desired kinematic quantities such as position and velocity and, in the case of non-holonomic systems, quantities such as heading angle.  $\tnow_i$ can be interpreted as the earliest start time (EST) of vehicle $i$, before which the vehicle may not depart from its initial state. Further, we define $\ti_i$, the latest (acceptable) start time (LST) for vehicle $i$. Our problem can now be thought of as determining the LST $\ti_i$ for each vehicle to get to $\target_i$ at or before the STA $\tf_i$, and finding a control to do this safely. If the LST is before the EST $\ti_i < \tnow_i$, then it is infeasible for vehicle $i$ to arrive at $\target_i$ at or before the STA $\tf_i$. Comparing $\ti_i$ and $\tnow_i$ is feasibility problem that may arise in practice; however, for simplicity of presentation, we will assume that $\tnow_i\le \ti_i \forall i$.

Danger is described by sets $\danger_{ij}(\x_j) \subset \amb$. In general, the definition of $\danger_{ij}$ depends on the conditions under which vehicles $i$ and $j$ are considered to be in an unsafe configuration, given the state of vehicle $j$. Here, we define danger to be the situation in which the two vehicles come within a certain radius $\Rc$ of each other: $\danger_{ij}(\x_j) = \{\x_i: \| \pos_i - \pos_j\|_2 \le \Rc \}$. Such a danger zone is also used by the FAA \cite{paglione99}. An illustration of the problem setup is shown in Figure \ref{fig:formulation}.

\begin{figure}
	\centering
	\includegraphics[width=0.35\textwidth]{"fig/formulation"}
	\caption{An illustration of the problem formulation with three vehicles. Each vehicle $P_i$ seeks to reach its target set $\target_i$ by time $t=\tf_i$, while avoiding physical obstacles $\obs$ and the danger zones of other vehicles.}
	\label{fig:formulation}
\end{figure}

In general, the above problem must be analyzed in the joint state space of all vehicles, making the solution intractable. In this paper, we will instead consider the problem of performing path planning of the vehicles in a sequential manner. Without loss of generality, we consider the problem of first fixing $i=1$ and determining the optimal control for vehicle $1$, the vehicle with the highest priority. The resulting optimal control $\ctrl_1$ sends vehicle $1$ to $\target_1$ in minimum time. 

Then, we plan the minimum time trajectory for each of the vehicles $2,\ldots,N$, in decreasing order of priority, given the previously-determined trajectories for higher-priority vehicles $1,\ldots,i-1$. We assume that all vehicles have complete information about the states and trajectories of higher-priority vehicles, and that all vehicles adhere to their planned trajectories. Thus, in planning its trajectory, vehicle $i$ treats higher-priority vehicles as known time-varying obstacles. 

With the above sequential path planning (SPP) protocol and assumptions, our problem now reduces to the following for vehicle $i$. Given $\x_j(\cdot), j=1,\ldots,i-1$, determine $\ctrl_i(\cdot)$ that maximizes $\ti_i$ and such that $x_i(\tau) \in \target_i, \tau\le \tf_i$.
% Problem formulation (1p)
%% A number of aircrafts aiming to reach a number of destinations respectively, on a certain schedule
%% How to guarantee they all get there and on time?

% !TEX root = SPP2.tex
\section{Background \label{sec:background}}
This section provides a brief summary of the work in \cite{Chen15}, in which SPP scheme is proposed under perfect information and absence of disturbance. Here, the dynamics of vehicle $\veh_i$ becomes

\begin{equation}
\label{eq:dyn_no_dstb}
\begin{aligned}
\dot{x}_i &= f_i(t, x_i, u_i), \quad t \in [\edt_i, \sta_i] \\
u_i &\in \cset_i \\
i &= 1,\ldots, N
\end{aligned}
\end{equation}

\noindent where the difference compared to \eqref{eq:dyn} is that the disturbance $d_i$ is no longer part of the dynamics.

In order to make the $N$-vehicle path planning problem safe and tractable, a reasonable structure is imposed to the problem: each vehicle is assigned a strict priority ordering. When planning its trajectory to its target, a higher-priority vehicle can disregard the presence of a lower priority vehicle. In contrast, a lower priority vehicle must take into account the presence of all higher priority vehicles, and plan its trajectory in a way that avoids the higher priority vehicles' danger zones. For convenience and without lost of generality, let vehicle $i$ have the $i$th highest priority and denote it as $\veh_i$. 

Under the above convention, each vehicle $\veh_i$ must take into account time-varying obstacles induced by vehicles $\veh_j, j<i$, denoted $\ioset_i^j(t)$. Optimal safe path planning of each lower-priority vehicle $\veh_i$ then consists of determining the optimal path that allows $\veh_i$ to each its target $\targetset_i$ while avoiding the moving obstacles $\obsset_j$, defined by

\begin{equation}
\obsset_i(t) = \bigcup_{j=1}^{i-1}\ioset_i^j(t)
\end{equation}

Such an optimal path planning problem can be solved by computing a backward reachable set (BRS) $\brs_i(t)$ from a target set $\targetset_i$ using formulations of HJ variational inequalities such as \cite{Bokanowski11, Fisac15}. In particular, we will utilize the formulation in \cite{Fisac15}, which does not require augmentation of the state space with the time variable.

Starting from the highest-priority vehicle $\veh_1$, one computes the BRS $\brs_1(t)$, from which the optimal control and optimal trajectory $x_1(\cdot)$ to the target $\targetset_1$ can be obtained. Under the absence of disturbances and perfect information, the obstacles induced by $\veh_1$ for lower-priority vehicle $\veh_i$ is simply the danger zone centered around the position of each point $p_1(\cdot)$ on the trajectory:

\begin{equation}
\ioset_i^1(t) = \{x_j: \|p_j - p_1(\cdot)\|\le\cradius\}
\end{equation}

Given $\ioset_i^j(t), j<i$, and continuing with $i = 2$, the optimal safe trajectories for each vehicle $\veh_i$ can be computed. All of the trajectories are optimal in the sense that given the requirement that $\veh_i$ must arrive at $\targetset_i$ at time $\sta_i$, the latest departure time $\ldt_i$ and the optimal control $u^*_i(\cdot)$ that guarantees arrival at $\sta_i$ can be obtained.
% Solution methodology (1.5-2p)
%% Variational inequality to be solved
%% How to treat first vehicle, how to treat previous vehicles
%% Numerical Implementation: discretization schemes and update rule

% !TEX root = ../SPP_IoTjournal.tex
\subsection{Results \label{sec:city_simResults}}

Focus on the following aspects:
\begin{itemize}
\item The technical details for the simulations, like RTT parameters, relative co-ordinate dynamics, rotation and translation of obstacles, union for obstacles, etc. 
\item Demonstration of theory (the vehicles avoid collision w/ other vehicles and reach their destinations).
\item Scaling of SPP.
\item Provide some more intuition about the solution that emerge out of theory-- Space-time separation, type of space-time trajectories (Almost straight line path w/ different starting times?), etc.
\item Reactivity of controller to the actual disturbance (Claire: be very detailed about explaining the setup of simulation)
\item Illustrate how the type of space-time trajectories change with change in disturbance bounds and STA
\end{itemize}
% Numerical Simuations (1-1.5p)
%% 2D + 2D example (what example, concretely?
%% Nx3D examples

% !TEX root = ./SPP_IoTjournal.tex
\section{Conclusion}
Provably safe multi-vehicle path planning in an important problem that needs to be addressed to ensure that vehicles can fly in close proximity of each other. Recently, the SPP algorithm was proposed for multi-vehicle path planning problem that scales linearly with the number of vehicles. We illustrate the full potential of the algorithm by using it for large-scale multi-vehicle path planning problems under different flying conditions. We demonstrate how different types of space-time trajectories emerge naturally out of the algorithm for different disturbance conditions and other problem parameters. The reactivity of the obtained controller is also demonstrated under different wind conditions.
% Conclusion (0.5p)

%%%%%%%%%%%%%%%%%%%%%%%%%%%%%%%%%%%%%%%%%%%%%%%%%%%%%%%%%%%%%%%%%%%%%%%%%%%%%%%%
%\addtolength{\textheight}{1cm}   % This command serves to balance the column lengths
                                  % on the last page of the document manually. It shortens
                                  % the textheight of the last page by a suitable amount.
                                  % This command does not take effect until the next page
                                  % so it should come on the page before the last. Make
                                  % sure that you do not shorten the textheight too much.

\bibliographystyle{IEEEtran}
\bibliography{references}
\end{document}
