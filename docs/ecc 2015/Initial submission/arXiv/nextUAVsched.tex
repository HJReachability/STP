%%%%%%%%%%%%%%%%%%%%%%%%%%%%%%%%%%%%%%%%%%%%%%%%%%%%%%%%%%%%%%%%%%%%%%%%%%%%%%%%
%2345678901234567890123456789012345678901234567890123456789012345678901234567890
%        1         2         3         4         5         6         7         8

%\documentclass[letterpaper, 10 pt, conference]{ieeeconf}  % Comment this line out
                                                          % if you need a4paper
\documentclass[letterpaper, 10pt, conference]{ieeeconf}      % Use this line for a4
                                                          % paper

 
\IEEEoverridecommandlockouts                              % This command is only
                                                          % needed if you want to
                                                          % use the \thanks command
\overrideIEEEmargins
% See the \addtolength command later in the file to balance the column lengths
% on the last page of the document

\usepackage{amsmath}    % need for sub equations
\usepackage{amsfonts}
\usepackage{graphicx}   % need for figures
\usepackage{subcaption}
\usepackage{epsfig} 
\usepackage{cancel}
\usepackage{amssymb}
\usepackage{color}
\usepackage{my_macros}
\usepackage[ruled,vlined,titlenotnumbered]{algorithm2e} 

\title{\LARGE \bf
Safe Sequential Path Planning of Multi-Vehicle Systems via Double-Obstacle Hamilton-Jacobi-Isaacs Variational Inequality}

\author{Mo Chen, Jaime F. Fisac, Shankar Sastry, Claire J. Tomlin
\thanks{This work has been supported in part by NSF under CPS:ActionWebs (CNS-931843), by ONR under the HUNT (N0014-08-0696) and SMARTS (N00014-09-1-1051) MURIs and by grant N00014-12-1-0609, by AFOSR under the CHASE MURI (FA9550-10-1-0567). The research of J.F. Fisac has received funding from the ``la Caixa" Foundation.}
\thanks{All authors are with the Department of Electrical Engineering and Computer Sciences, University of California, Berkeley. \{mochen72, jfisac, sastry, tomlin\}@eecs.berkeley.edu}
}

\begin{document}
\maketitle
\thispagestyle{empty}
\pagestyle{empty}

%%%
\begin{abstract}
We consider the problem of planning trajectories for a group of $N$ vehicles, each aiming to reach its own target set while avoiding danger zones of other vehicles. The analysis of problems like this is extremely important practically, especially given the growing interest in utilizing unmanned aircraft systems for civil purposes. The direct solution of this problem by solving a single-obstacle Hamilton-Jacobi-Isaacs (HJI) variational inequality (VI) is numerically intractable due to the exponential scaling of computation complexity with problem dimensionality. Furthermore, the single-obstacle HJI VI cannot directly handle situations in which vehicles do not have a common scheduled arrival time. Instead, we perform sequential path planning  by considering vehicles in order of priority, modeling higher-priority vehicles as time-varying obstacles for lower-priority vehicles. To do this, we solve a double-obstacle HJI VI which allows us to obtain the reach-avoid set, defined as the set of states from which a vehicle can reach its target while staying within a time-varying state constraint set. From the solution of the double-obstacle HJI VI, we can also extract the latest start time and the optimal control for each vehicle. This is a first application of the double-obstacle HJI VI which can handle systems with time-varying dynamics, target sets, and state constraint sets, and results in computation complexity that scales linearly, as opposed to exponentially, with the number of vehicles in consideration.
\end{abstract}

% % !TEX root = SPP2.tex
\section{Introduction}
Recently, there has been an immense surge of interest in using unmanned aerial vehicles (UAVs) for civil purposes. The applications of UAVs extend well beyond package delivery, and include aerial surveillance, disaster response, and other important tasks \cite{Tice91, Debusk10, Amazon16, AUVSI16, BBC16}. Many of these applications will involve UAVs flying in an urban environment, potentially in close proximity of humans. As a result, government agencies such as the Federal Aviation Administration (FAA) and National Aeronautics and Space Administration (NASA) of the United States are urgently trying to develop new scalable ways to organize an air space in which potentially thousands of UAVs can fly \cite{FAA13, NASA16}.

One essential problem that needs to be addressed is how a group of vehicles in the same vicinity can reach their destinations while avoiding collision with each other. Several previous studies have attempted to address this problem. In some of these studies, specific control strategies for the vehicles or moving entities are assumed, and approaches such as induced velocity obstacles have been used \cite{Fiorini98, Chasparis05, Vandenberg08}. Other researchers have used ideas involving virtual potential fields to maintain collision avoidance while maintaining a specific formation \cite{Saber02, Chuang07}. Although interesting results emerge from these previous studies, simultaneous trajectory planning and collision avoidance are not considered. 

In the past, trajectory planning and collision avoidance problems in safety-critical systems have been studied using reachability analysis, which provides guarantees on the success and safety of optimal system trajectories \cite{Barron90, Mitchell05, Bokanowski10, Margellos11, Fisac15}. In reachability analysis, one computes the reachable set, defined as the set of states from which the system can be driven to a target set. Reachability analysis has been successfully used in applications involving systems with no more than two vehicles, such as pairwise collision avoidance \cite{Mitchell05}, automated in-flight refueling \cite{Ding08}, two-player reach-avoid games \cite{Huang11}, and many others \cite{Bayen07}.

%In addition to the guarantees reachability theory provides and the evident flexibility of reachability theory for analyzing vastly different systems with nonlinear dynamics, many numerical tools for solving reachability problems are also available, making the approach practically appealing \cite{Mitchell05, Sethian96, Osher02, LSToolbox}.

Despite the advantages of reachability analysis, it cannot be directly applied to scenarios involving complex high dimensional systems such as multi-vehicle systems. The computation of reachable sets involves solving a Hamilton-Jacobi (HJ) partial differential equation (PDE) on a grid representing a discretization of the state space, causing an exponential scaling of computation complexity with respect to the dimension of the system, or roughly speaking, with the number of vehicles present.

In this paper, we build on the work in \cite{Chen15}, and assume a reasonable structure in the multi-vehicle path planning problem. In the sequential path planning (SPP) scheme, vehicles are assigned some priority. Higher-priority vehicles may ignore the lower-priority vehicles, who must take into account the presence of higher-priority vehicles by treating them as induced time-varying obstacles. Unlike the work in \cite{Chen15}, we incorporate disturbances for all vehicles and consider three different assumptions on the information each of the vehicles may have access to, making the sequential path planning substantially more practical. For each of the assumed information patterns, we propose a reachability-based method to compute the induced obstacles that would guarantee collision avoidance as well as successful transit to the destination. We demonstrate and compare our proposed methods through numerical simulations.
% Introduction (1-1.5p)
%% Motivation
%% Related work
%% Summary of results
\section{Introduction}
Consider a group of autonomous vehicles trying to perform a task or reach a goal which may be time-varying in their joint state space, while avoiding obstacles and other vehicles. Providing safety and performance guarantees for such a multi-agent autonomous system (MGAS) is very relevant practically. Unmanned aerial vehicles (UAVs), for example, have in the past been used mainly for military operations \cite{tice91}. However, recently, there has been a growing interest in using UAVs for civil applications, as companies like Amazon and Google are looking in the near future to send UAVs into the airspace to deliver packages \cite{primeAir,projectWing}. Government agencies such as the Federal Aviation Administration (FAA) and National Aeronautics and Space Administration (NASA) of the United States are also expressing growing interest in analyzing these problems in order to prevent airspace conflicts that could arise with the introduction of potentially many UAVs in an urban environment \cite{faa13}. In addition, UAVs can be used not only to deliver packages quickly, but in any situation where fast response is desired. For example, UAVs can provide emergency supplies to disaster-struck areas that are otherwise difficult to reach \cite{debusk10}.

In general, MGASs are difficult to analyze due to their inherent high dimensionality. For example, the joint state space of 10 vehicles would have 30 dimensions even if each vehicle is described by a simple model with three states. MGASs also often involve aspects of cooperation and asymmetric goals among the vehicles or teams of vehicles, making their analysis particularly interesting. Despite these difficulties, it is still important to analyze these systems because of their applications in robotics and aircraft safety.

MGASs have been explored extensively in the literature. Some researchers have done work on multi-vehicle path planning in the presence of other unknown vehicles or moving entities with assumptions on their specific control strategies \cite{chasparis05}. In a number of formulations for safe multi-vehicle navigation, these assumed strategies induce velocity obstacles that vehicles must avoid to maintain safety \cite{fiorini98, vandenberg08}. Researchers have also used potential functions to perform collision avoidance while maintaining formation given a predefined trajectory \cite{saber02,chuang07}. However, these bodies of work have not considered trajectory planning and collision avoidance simultaneously.

One well-known technique for optimal trajectory planning under disturbances or adversaries is reachability analysis, in which one computes the reach-avoid set, defined as the set of states from which the system can reach a target set while remaining within a state constraint set for all time. For reachability of systems of up to five dimensions, single-obstacle Hamilton-Jacobi-Isaacs (HJI) variational inequalities (VI) \cite{mitchell05,bokanowski10} have been used in situations where obstacles and target sets are static. Another HJI VI formulation \cite{barron89} is able to handle problems with moving target sets with no obstacles. 

A major practical appeal of the above approaches stems from the availability of modern numerical tools such as \cite{mitchell05, sethian96, osher02, LSToolbox}, which can efficiently solve HJI equations when the problem dimension is low. These numerical tools have been successfully used to solve a variety of differential games, path planning problems, and optimal control problems, including aircraft collision avoidance \cite{mitchell05}, automated in-flight refueling \cite{ding08}, and two-player reach-avoid games \cite{huang11}. The advantage of the HJI approaches is that they can be applied to a large variety of system dynamics, and provide guarantees on the system's safety and performance.

Despite the power of the previous HJI formulations, the approaches become numerically intractable very quickly as the number of vehicles in the system is increased. This is because the numerical computations are done on a grid in the joint state space of the system, resulting in an exponential scaling of computation complexity with respect to the dimensionality of the problem. Furthermore, state constraint sets, while useful for modeling unsafe vehicle configurations, are required to be time-invariant in \cite{mitchell05, bokanowski10, mitchell-thesis}. To solve problems involving time-varying state constraints, \cite{bokanowski11} proposed to augment the state space with time; however, this process introduces an extra state space dimension without addressing the added computation complexity.

Recently, \cite{fisac15} presented a double-obstacle HJI VI which handles problems in which the dynamics, target sets, and state constraint sets are all time-varying, and provided a numerical implementation based on well-known schemes. The formulation does not introduce any additional computation overhead compared to the above-mentioned techniques, yet it still maintains the same guarantees on the system's safety and performance. In this paper, we provide a first application of the theory presented in \cite{fisac15}. As a point of clarification, ``obstacles" in the context of HJI VIs refer to the effective constraints in the HJI VI, while obstacles in the state space represent physical obstacles that vehicles must avoid.

Our contributions are as follows. First, we formulate a multi-vehicle collision avoidance problem involving $N$ autonomous vehicles. Each vehicle seeks to get to its own target sets while avoiding obstacles and collision with all other vehicles. To reduce the problem complexity to make the problem tractable, we assign a priority to each vehicle, and model higher-priority vehicles as time-varying obstacles that need to be avoided. We then utilize the double-obstacle HJI VI proposed in \cite{fisac15} to compute reach-avoid sets to plan trajectories for vehicles in order of priority. This way, we are able to offer a tractable solution that scales linearly, as opposed to exponentially, with the number of vehicles. We compare our approach to the previous single-obstacle HJI VI approach involving static obstacles \cite{mitchell05, bokanowski10, mitchell-thesis} in a simple two-vehicle system, and demonstrate the scalability of our approach in a more complex four-vehicle system.

% % !TEX root = nextUAVsched.tex
\section{Problem Formulation \label{sec:formulation}}
Consider $N$ vehicles $P_i,i=1\ldots,N$, each trying to reach one of $N$ target sets $\target_i,i=1\ldots,N$, while avoiding obstacles and collision with each other. Each vehicle $i$ has states $\x_i\in \R^{n_i}$ and travels on a domain $\amb=\obs \cup \free\in\R^p$, where $\obs$ represents the obstacles that each vehicle must avoid, and $\free$ represents all other states in the domain on which vehicles can move. Each vehicle $i = 1,2,\ldots,N$ moves with the following dynamics for $t\in[\tnow_i, \tf_i]$:

\begin{equation} \label{eq:dyn}
\dotx_i = f_i (t, \x_i, \ctrl_i), \quad\x_i(\ti_i) = \x_i^0 
\end{equation}

\noindent where $\x_i^0$ represents the initial condition of vehicle $i$, and $\ctrl_i(\cdot)$ represents the control function of vehicle $i$. In general, $f_i(\cdot,\cdot,\cdot)$ depends on the specific dynamic model of vehicle $i$, and need not be of the same form across different vehicles. Denote $\pos_i\in\R^p$ the subset of the states that represent the position of the vehicle. Given $\pos_i^0\in\free$, we define the admissible control function set for $P_i$ to be the set of all control functions such that $\pos_i(t) \in \free \forall t\ge \ti_i$. Denote the joint state space of all vehicles $\x \in \R^n$ where $n = \sum_i n_i$, and their joint control $\ctrl$.

We assume that the control functions $\ctrl_i(\cdot)$ are drawn from the set $\ctrlf_i := \{\ctrl_i: [\tnow_i, \tf_i] \rightarrow \ctrlin_i, \text{measurable}$\footnote{
A function $f:X\to Y$ between two measurable spaces $(X,\Sigma_X)$ and $(Y,\Sigma_Y)$ is said to be measurable if the preimage of a measurable set in $Y$ is a measurable set in $X$, that is: $\forall V\in\Sigma_Y, f^{-1}(V)\in\Sigma_X$, with $\Sigma_X,\Sigma_Y$ $\sigma$-algebras on $X$,$Y$.}\} where $\ctrlin_i \in \R^{n^\ctrl_i}$ is the set of allowed control inputs. Furthermore, we assume $f_i(t,\x_i, \ctrl_i)$ is bounded, Lipschitz continuous in $\x_i$ for any fixed $t,\ctrl_i$, and measurable in $t, \ctrl_i$ for each $\x_i$. Therefore given any initial state $\x_i^0$ and any control function $\ctrl_i(\cdot)$, there exists a unique, continuous trajectory $\x_i(\cdot)$ solving (\ref{eq:dyn}) \cite{coddington55}.

The goal of each vehicle $i$ is to arrive at $\target_i \subset \R^{n_i}$ at or before some scheduled time of arrival (STA) $\tf_i$ in minimum time, while avoiding obstacles and danger with all other vehicles. The target sets $\target_i$ can be used to represent desired kinematic quantities such as position and velocity and, in the case of non-holonomic systems, quantities such as heading angle.  $\tnow_i$ can be interpreted as the earliest start time (EST) of vehicle $i$, before which the vehicle may not depart from its initial state. Further, we define $\ti_i$, the latest (acceptable) start time (LST) for vehicle $i$. Our problem can now be thought of as determining the LST $\ti_i$ for each vehicle to get to $\target_i$ at or before the STA $\tf_i$, and finding a control to do this safely. If the LST is before the EST $\ti_i < \tnow_i$, then it is infeasible for vehicle $i$ to arrive at $\target_i$ at or before the STA $\tf_i$. Comparing $\ti_i$ and $\tnow_i$ is feasibility problem that may arise in practice; however, for simplicity of presentation, we will assume that $\tnow_i\le \ti_i \forall i$.

Danger is described by sets $\danger_{ij}(\x_j) \subset \amb$. In general, the definition of $\danger_{ij}$ depends on the conditions under which vehicles $i$ and $j$ are considered to be in an unsafe configuration, given the state of vehicle $j$. Here, we define danger to be the situation in which the two vehicles come within a certain radius $\Rc$ of each other: $\danger_{ij}(\x_j) = \{\x_i: \| \pos_i - \pos_j\|_2 \le \Rc \}$. Such a danger zone is also used by the FAA \cite{paglione99}. An illustration of the problem setup is shown in Figure \ref{fig:formulation}.

\begin{figure}
	\centering
	\includegraphics[width=0.35\textwidth]{"fig/formulation"}
	\caption{An illustration of the problem formulation with three vehicles. Each vehicle $P_i$ seeks to reach its target set $\target_i$ by time $t=\tf_i$, while avoiding physical obstacles $\obs$ and the danger zones of other vehicles.}
	\label{fig:formulation}
\end{figure}

In general, the above problem must be analyzed in the joint state space of all vehicles, making the solution intractable. In this paper, we will instead consider the problem of performing path planning of the vehicles in a sequential manner. Without loss of generality, we consider the problem of first fixing $i=1$ and determining the optimal control for vehicle $1$, the vehicle with the highest priority. The resulting optimal control $\ctrl_1$ sends vehicle $1$ to $\target_1$ in minimum time. 

Then, we plan the minimum time trajectory for each of the vehicles $2,\ldots,N$, in decreasing order of priority, given the previously-determined trajectories for higher-priority vehicles $1,\ldots,i-1$. We assume that all vehicles have complete information about the states and trajectories of higher-priority vehicles, and that all vehicles adhere to their planned trajectories. Thus, in planning its trajectory, vehicle $i$ treats higher-priority vehicles as known time-varying obstacles. 

With the above sequential path planning (SPP) protocol and assumptions, our problem now reduces to the following for vehicle $i$. Given $\x_j(\cdot), j=1,\ldots,i-1$, determine $\ctrl_i(\cdot)$ that maximizes $\ti_i$ and such that $x_i(\tau) \in \target_i, \tau\le \tf_i$.
% Problem formulation (1p)
%% A number of aircrafts aiming to reach a number of destinations respectively, on a certain schedule
%% How to guarantee they all get there and on time?
\section{Problem Formulation \label{sec:formulation}}
Consider $N$ vehicles $P_i,i=1\ldots,N$, each trying to reach one of $N$ target sets $\target_i,i=1\ldots,N$, while avoiding obstacles and collision with each other. Each vehicle $i$ has states $\x_i\in \R^{n_i}$ and travels on a domain $\amb=\obs \cup \free\in\R^p$, where $\obs$ represents the obstacles that each vehicle must avoid, and $\free$ represents all other states in the domain on which vehicles can move. Each vehicle $i = 1,2,\ldots,N$ moves with the following dynamics for $t\in[\tnow_i, \tf_i]$:

\begin{equation} \label{eq:dyn}
\dotx_i = f_i (t, \x_i, \ctrl_i), \quad\x_i(\ti_i) = \x_i^0 
\end{equation}

\noindent where $\x_i^0$ represents the initial condition of vehicle $i$, and $\ctrl_i(\cdot)$ represents the control function of vehicle $i$. In general, $f_i(\cdot,\cdot,\cdot)$ depends on the specific dynamic model of vehicle $i$, and need not be of the same form across different vehicles. Denote $\pos_i\in\R^p$ the subset of the states that represent the position of the vehicle. Given $\pos_i^0\in\free$, we define the admissible control function set for $P_i$ to be the set of all control functions such that $\pos_i(t) \in \free \forall t\ge 0$. Denote the joint state space of all vehicles $\x \in \R^n$ where $n = \sum_i n_i$, and their joint control $\ctrl$.

We assume that the control functions $\ctrl_i(\cdot)$ are drawn from the set $\ctrlf_i := \{\ctrl_i: [\tnow_i, \tf_i] \rightarrow \ctrlin_i, \text{measurable}$\footnote{
A function $f:X\to Y$ between two measurable spaces $(X,\Sigma_X)$ and $(Y,\Sigma_Y)$ is said to be measurable if the preimage of a measurable set in $Y$ is a measurable set in $X$, that is: $\forall V\in\Sigma_Y, f^{-1}(V)\in\Sigma_X$, with $\Sigma_X,\Sigma_Y$ $\sigma$-algebras on $X$,$Y$.}\} where $\ctrlin_i \in \R^{n^\ctrl_i}$ is the set of allowed control inputs. Furthermore, we assume $f_i(t,\x_i, \ctrl_i)$ is bounded, Lipschitz continuous in $\x_i$ for any fixed $t,\ctrl_i$, and measurable in $t, \ctrl_i$ for each $\x_i$. Therefore given any initial state $\x_i^0$ and any control function $\ctrl_i(\cdot)$, there exists a unique, continuous trajectory $\x_i(\cdot)$ solving (\ref{eq:dyn}) \cite{coddington55}.

The goal of each vehicle $i$ is to arrive at $\target_i \subset \R^{n_i}$ at or before some scheduled time of arrival (STA) $\tf_i$ in minimum time, while avoiding obstacles and danger with all other vehicles. The target sets $\target_i$ can be used to represent desired kinematic quantities such as position and velocity and, in the case of non-holonomic systems, quantities such as heading angle.  $\tnow_i$ can be interpreted as the earliest start time (EST) of vehicle $i$, before which the vehicle may not depart from its initial state. Further, we define $\ti_i$, the latest (acceptable) start time (LST) for vehicle $i$. Our problem can now be thought of as determining the LST $\ti_i$ for each vehicle to get to $\target_i$ at or before the STA $\tf_i$, and finding a control to do this safely. If the LST is before the EST $\ti_i < \tnow_i$, then it is infeasible for vehicle $i$ to arrive at $\target_i$ at or before the STA $\tf_i$. Comparing $\ti_i$ and $\tnow_i$ is feasibility problem that may arise in practice; however, for simplicity of presentation, we will assume that $\tnow_i\le \ti_i \forall i$.

Danger is described by sets $\danger_{ij}(\x_j) \subset \amb$. In general, the definition of $\danger_{ij}$ depends on the conditions under which vehicles $i$ and $j$ are considered to be in an unsafe configuration, given the state of vehicle $j$. Here, we define danger to be the situation in which the two vehicles come within a certain radius $\Rc$ of each other: $\danger_{ij}(\x_j) = \{\x_i: \| \pos_i - \pos_j\|_2 \le \Rc \}$. Such a danger zone is also used by the FAA \cite{paglione99}. An illustration of the problem setup is shown in Figure \ref{fig:formulation}.

\begin{figure}
	\centering
	\includegraphics[width=0.3\textwidth]{"formulation"}
	\caption{An illustration of the problem formulation with three vehicles. Each vehicle $P_i$ seeks to reach its target set $\target_i$ by time $t=\tf_i$, while avoiding physical obstacles $\obs$ and the danger zones of other vehicles.}
	\label{fig:formulation}
\end{figure}

In general, the above problem must be analyzed in the joint state space of all vehicles, making the solution computationally intractable. In this paper, we will instead consider the problem of performing path planning of the vehicles in a sequential manner. Without loss of generality, we consider the problem of first fixing $i=1$ and determining the optimal control for vehicle $1$, the vehicle with the highest priority. The resulting optimal control $\ctrl_1$ sends vehicle $1$ to $\target_1$ in minimum time. 

Then, we plan the minimum time trajectory for each of the vehicles $2,\ldots,N$, in decreasing order of priority, given the previously-determined trajectories for higher-priority vehicles $1,\ldots,i-1$. We assume that all vehicles have complete information about the states and trajectories of higher-priority vehicles, and that all vehicles adhere to their planned trajectories. Thus, in planning its trajectory, vehicle $i$ treats higher-priority vehicles as known time-varying obstacles. 

With the above sequential path planning (SPP) protocol and assumptions, our problem now reduces to the following for vehicle $i$. Given $\x_j(\cdot), j=1,\ldots,i-1$, determine $\ctrl_i(\cdot)$ that maximizes $\ti_i$ and such that $x_i(\tau) \in \target_i, \tau\le \tf_i$.

% % !TEX root = SPP2.tex
\section{Background \label{sec:background}}
This section provides a brief summary of the work in \cite{Chen15}, in which SPP scheme is proposed under perfect information and absence of disturbance. Here, the dynamics of vehicle $\veh_i$ becomes

\begin{equation}
\label{eq:dyn_no_dstb}
\begin{aligned}
\dot{x}_i &= f_i(t, x_i, u_i), \quad t \in [\edt_i, \sta_i] \\
u_i &\in \cset_i \\
i &= 1,\ldots, N
\end{aligned}
\end{equation}

\noindent where the difference compared to \eqref{eq:dyn} is that the disturbance $d_i$ is no longer part of the dynamics.

In order to make the $N$-vehicle path planning problem safe and tractable, a reasonable structure is imposed to the problem: each vehicle is assigned a strict priority ordering. When planning its trajectory to its target, a higher-priority vehicle can disregard the presence of a lower priority vehicle. In contrast, a lower priority vehicle must take into account the presence of all higher priority vehicles, and plan its trajectory in a way that avoids the higher priority vehicles' danger zones. For convenience and without lost of generality, let vehicle $i$ have the $i$th highest priority and denote it as $\veh_i$. 

Under the above convention, each vehicle $\veh_i$ must take into account time-varying obstacles induced by vehicles $\veh_j, j<i$, denoted $\ioset_i^j(t)$. Optimal safe path planning of each lower-priority vehicle $\veh_i$ then consists of determining the optimal path that allows $\veh_i$ to each its target $\targetset_i$ while avoiding the moving obstacles $\obsset_j$, defined by

\begin{equation}
\obsset_i(t) = \bigcup_{j=1}^{i-1}\ioset_i^j(t)
\end{equation}

Such an optimal path planning problem can be solved by computing a backward reachable set (BRS) $\brs_i(t)$ from a target set $\targetset_i$ using formulations of HJ variational inequalities such as \cite{Bokanowski11, Fisac15}. In particular, we will utilize the formulation in \cite{Fisac15}, which does not require augmentation of the state space with the time variable.

Starting from the highest-priority vehicle $\veh_1$, one computes the BRS $\brs_1(t)$, from which the optimal control and optimal trajectory $x_1(\cdot)$ to the target $\targetset_1$ can be obtained. Under the absence of disturbances and perfect information, the obstacles induced by $\veh_1$ for lower-priority vehicle $\veh_i$ is simply the danger zone centered around the position of each point $p_1(\cdot)$ on the trajectory:

\begin{equation}
\ioset_i^1(t) = \{x_j: \|p_j - p_1(\cdot)\|\le\cradius\}
\end{equation}

Given $\ioset_i^j(t), j<i$, and continuing with $i = 2$, the optimal safe trajectories for each vehicle $\veh_i$ can be computed. All of the trajectories are optimal in the sense that given the requirement that $\veh_i$ must arrive at $\targetset_i$ at time $\sta_i$, the latest departure time $\ldt_i$ and the optimal control $u^*_i(\cdot)$ that guarantees arrival at $\sta_i$ can be obtained.
% Solution methodology (1.5-2p)
%% Variational inequality to be solved
%% How to treat first vehicle, how to treat previous vehicles
%% Numerical Implementation: discretization schemes and update rule
\section{Solution via double-obstacle HJI VI and SPP\label{sec:solution}}
One direct way of solving the problem formulated in Section \ref{sec:formulation} is by solving a corresponding single-obstacle HJI VI \cite{mitchell05,bokanowski10}. In this approach, one considers the joint time-invariant dynamics of the entire system, $f(\x,\ctrl)$, and defines the static goal set and the static avoid set in the joint state space of all vehicles. The goal set encodes the joint states representing all vehicles being at their target sets, and the avoid set encodes the joint states representing all unsafe configurations. These sets are defined as sub-zero level sets of appropriate implicit surface functions $\setf(\x)$ where $\x\in\set \Leftrightarrow \setf(\x)\le 0$. Having defined the implicit surface functions, the HJI VI (\ref{eq:HJIPDE}) is then solved backwards in time with the implicit surface function representing the terminal set $\goalf(\x)$ as the initial condition and the implicit surface function representing the avoid set $\avoidf(\x)$ as an effective constraint. From the solution, we obtain the reach-avoid set $\RA(t)$, which defines the set of states from which the system has a control to drive the state at time $t$ to the goal set $\goal$ at time $0$ while staying out of the avoid set $\avoid$ at all times. Note that the joint dynamics, goal set, and avoid set must be time-invariant. Time-varying dynamics and sets can be treated by augmenting the state space with time as an auxiliary state \cite{bokanowski11}; however, this state augmentation comes at a large computational expense.

\begin{equation}
\begin{aligned}
	\label{eq:HJIPDE}
	\max\big\{D_t\soln + \min \left[0, H\left(\x,D_{\x}\soln\right)\right], -\avoidf(\x)-\soln(\x,t) \big\}= 0,\\
\soln(\xj,0) = \goalf(\x)	 
\end{aligned}
\end{equation}
\noindent where the optimal Hamiltonian is given by
\begin{equation*}
H\left(\xj,p\right) = \min_{\ctrl \in \ctrlin} p \cdot f(\x,\ctrl).
\end{equation*}

The direct solution described above has been successfully used to solve a number of problems involving up to a pair of vehicles \cite{mitchell05, ding08, huang11, chen14}. However, since numerical methods for solving a PDE or a VI involve gridding up the state space, the computation complexity scales exponentially with the number of dimensions in the joint state. This makes the single-obstacle HJI VI inapplicable for problems involving three or more vehicles. Therefore, instead of solving a single-obstacle HJI VI in the joint state space in $\R^n=\R^{\sum_i n_i}$, we will consider the problem in in $\R^{n_i}$ and solve a sequence of \textit{double-obstacle} HJI VIs introduced in \cite{fisac15}. By doing so, we take advantage of the fact that time-varying targets, obstacles, and dynamics can be handled by the double-obstacle HJI VIs (but not by the single-obstacle HJI VI without incurring significant computational expense), making the analysis of the problem tractable. Furthermore, even if the dimensionality of the problem is sufficiently low for computing a numerical solution to the single-obstacle HJI VI, its inability to handle time-varying systems would still limit us to only consider problems in which the required time of arrival is common across all vehicles: $\tf_i = \tf \;\forall i$.

We first describe the framework for computing reach-avoid sets with arbitrary terrain, domain, moving obstacles, and moving target sets based on \cite{fisac15}. As with the single-obstacle HJI VI, sets are defined as sub-zero level sets of implicit surface functions; however, crucially, these implicit surface functions can be time-varying in the double-obstacle HJI VI without increasing computational complexity. Being able to compute reach-avoid sets with moving obstacles allows us to overcome the computational intractability described above by sequentially performing path planning for one vehicle at a time in order of priority, while treating higher-priority vehicles as moving obstacles. The target set is defined in the same way as in the single-obstacle HJI VI; the avoid set is by convention defined as the complement of the state constraint set in the double-obstacle HJI VI.

\subsection{Reachability via HJI VI}
We first state the result given in \cite{fisac15}, and then specialize the result to the problem formulation given in Section \ref{sec:formulation}. Consider a general nonlinear system describing the state evolution of two players in a differential game for $t\in[0,T]$.

\begin{equation}
\dot{x}(t) = f(t, x, u, d), \quad x(0) = x
\end{equation}

\noindent where $x$ is the joint state, $u$ is the control input for player 1, and $d$ is the control input for player 2. Their joint dynamics $f$ is assumed to be bounded, Lipschitz continuous in $x$ for any fixed $u,d$ and $t$, and measurable in $t,u,d$ for each $x$. Given control functions $u(\cdot), d(\cdot)$, there exists a unique trajectory $\phi_x^{u,d}((\tau),\tau)$ \cite{coddington55}. Player 1 wishes to minimize, and player 2 wishes to maximize the following cost functional:

\begin{equation}
\begin{aligned}
&\mathcal{V}\big(t,x,u(\cdot),d(\cdot)\big) \\
&\quad = \min_{\tau\in[t,T]}\max\big\{l(\phi_{x(0)}^{u,d}(\tau),\tau),\max_{s\in[t,\tau]} g(\phi_{x(0)}^{u,d}(s),s)\big\}
\end{aligned}
\end{equation}

The value of the game is thus given by

\begin{equation}
\begin{aligned}
\soln(x,t):=\sup_{u(\cdot)}\inf_{\delta[u](\cdot)}\mathcal{V}\big(t,x,u(\cdot),\delta[u](\cdot)\big)
\end{aligned}
\end{equation}

\noindent where player 2 chooses a nonanticipative strategy $d(\cdot) = \delta[u](\cdot)$, under which the control signal $d(t)$ is chosen in response to player 1's control function up to time $t$, $u(\tau),\tau\le t$ \cite{mitchell-thesis}. The value of the game characterizes reach-avoid set, or all the states from which player 1 can reach the target $\goal$ encoded by the implicit surface function $\goalf(x,t)$, while staying within some state constraint set $\constr$ encoded by the implicit surface function $\constrf(x,t)$, despite the adversarial actions of player 2. The value function is the unique viscosity solution \cite{crandall84} to the following single-obstacle HJI VI \cite{fisac15}:

\begin{equation}
\label{eq:HJIVI_full}
\begin{aligned}
\max\Big\{&\min\big\{D_t V + H\left(x, D_x V,t\right),l(x,t)-V(x, t)\big\}\\
& g(x,t)-V(x,t)\Big\}=0, \quad t\in[0,T], \quad x\in\R^n\\
&V(x,T) = \max\big\{l(x,T),g(x,T)\big\},  \quad x\in\R^n
\end{aligned}
\end{equation}

The proof is given in \cite{fisac15} and is based on viscosity solution theory \cite{evans84, barron90}.

Now consider the system with dynamics given by (\ref{eq:dyn}). Given a time-varying target set $\target_i(t)$ and obstacle $\avoid_i(t)$ that vehicle $i$ must avoid, we define implicit surface functions $\goalf(\x_i,t), \constrf(\x_i,t)$ such that $\x_i\in\target_i(t)\Leftrightarrow \goalf_i(\x_i,t)\le 0,\x_i\notin \avoid_i(t) \Leftrightarrow \constrf_i(x,t)\le 0$. Now, the problem formulated in Section \ref{sec:formulation} becomes one in which vehicle $i$ chooses a control function $\ctrl_i(\cdot)$ to minimize the following cost functional:

\begin{equation}
\label{eq:cost}
\begin{aligned}
&\mathcal{V}_i\big(t,\x_i, \ctrl_i(\cdot)\big) \\
&\qquad= \min_{\tau\in[t,T]}\max\big\{\goalf_i(\x_i(\tau),\tau),\max_{s\in[t,\tau]} \constrf_i(\x_i(s),s)\big\}
\end{aligned}
\end{equation}

Note here, we have an optimal control problem involving only one vehicle and no adversary, unlike in the case of the HJI VI (\ref{eq:HJIVI_full}). Now, specializing (\ref{eq:HJIVI_full}) to our optimal control problem, the value function that characterizes the reach-avoid set $\RA_i(t)$ is $\soln_i(\x_i, t)$, where $\x_i \in \RA_i(t) \Leftrightarrow \soln_i(\x_i, t) \le 0$. $\soln_i(\x_i, t)$ is the viscosity solution \cite{crandall84} of the HJI VI

\begin{equation}
\label{eq:HJI}
\begin{aligned}
\max\big\{\min\{D_t \soln_i + H_i\left(\x_i, D_{\x_i} \soln_i,t\right),\goalf_i(\x_i,t)-\soln_i(\x_i, t)\}\\
\quad \constrf_i(\x_i,t)-\soln_i(\x_i,t)\big\}=0, t\in[\tnow_i,\tf_i], \x_i\in\R^{n_i}\\
\soln_i(\x_i,\tf_i) = \max\left\{\goalf_i(\x_i,\tf_i),\constrf_i(\x_i,\tf_i)\right\}, \x_i\in\R^{n_i}
\end{aligned}
\end{equation}

\noindent where the Hamiltonian $H_i(t, \x_i, p)$ and optimal control $\ctrl_i$ are given by

\begin{equation}
\begin{aligned}
H_i(t,\x_i,p) &= \min_{\ctrl_i\in\ctrlin_i} p \cdot f_i(t,\x_i,\ctrl_i) \\
\ctrl^*_i &= \arg \min_{\ctrl_i} H_i(t,\x_i,p)
\end{aligned}
\end{equation}

\subsection{Sequential Path Planning}
In order to use (\ref{eq:HJI}) to perform SPP, we first define the moving obstacles induced by higher-priority vehicles. Specifically, for vehicle $i$, we define the moving obstacles $\mobs^i_j(t)$ induced by vehicles $j=1,\ldots,i-1$, given their known trajectories $\x_j (\cdot)$, to be

\bq
\mobs^i_j(t) := \{\x_i: \pos_i \in \danger_{ij}(\x_j(t)) \}
\eq

Each vehicle $i$ must avoid being in $\mobs^i_j(t)$ for each $j=1,\ldots,i-1$ and for all time $t$, as well as avoid being in static obstacles $\obs$ in the domain. Therefore, for the \ith vehicle, we compute the reach-avoid set with the following time-varying avoid set $\avoid_i(t)$ and goal set $\goal_i(t)$:

\bq
\begin{aligned}
\avoid_i(t) &:= \{\x_i: \pos_i \in \obs\} \cup \Big(\bigcup_{j=1,\ldots,i-1} \mobs^i_j(t)\Big)\\
\goal_i(t) &:= \target_i, t\le \tf_i
\end{aligned}
\eq

The goal set is represented by the implicit surface function $\goalf_i(\x,t)$, where $\goalf_i(\x_i,t)\le0\Leftrightarrow \x_i(t)\in \goal_i(t)$. The state constraint set in the HJI VI is defined as the complement of the avoid set, $\avoid_i^c(t)$, and is represented by the implicit surface function $\constrf(\x_i,t)$, where $\constrf(\x_i,t)\le0 \Leftrightarrow \x_i\notin \avoid_i(t)$. For both $\goalf_i(\x_i,t)$ and $\constrf(\x_i,t)$, we use the signed distance function (in $\x_i$) to the sets $\goal_i(t)$ and $\avoid_i^c(t)$, respectively.

Now, we can solve the double-obstacle HJI VI (\ref{eq:HJI}). The solution $\soln(\x_i,t)$ represents the reach-avoid set $\RA(t)$: $\soln(\x_i,t)\le0\Leftrightarrow \x_i(t)\in\RA(t)$. $\RA(t)$ is the set of states at starting time $t$ from which vehicle $i$ can arrive at $\target_i$ at or before time $\tf_i$ while avoiding obstacles and danger zones of all higher-priority vehicles $j=1,\ldots,i-1$. 

Alternatively, given an initial state $\x_i^0$, we can solve (\ref{eq:HJI}) to some $\ti_i = \inf\{t:\x_i^0 \in \RA(t)\}$. This represents the latest time that vehicle $i$ must depart from its initial position in order to reach $\target_i$ while avoiding obstacles and all danger zones of higher-priority vehicles $j=1,\ldots,i-1$.

The optimal control is given by

\bq
\label{eq:ctrl_syn}
\ctrl_i(t) = \arg \min \ham_i \left(t, D_{\x_i} \soln(\x_i, t), \soln(\x_i, t)\right)
\eq

Observe that since each vehicle $i$ is guaranteed to be safe with respect to higher priority vehicles $j=1,\ldots,i-1$, the safety of all vehicles, including lower-priority vehicles, can also be guaranteed.

% % !TEX root = ../SPP_IoTjournal.tex
\subsection{Results \label{sec:city_simResults}}

Focus on the following aspects:
\begin{itemize}
\item The technical details for the simulations, like RTT parameters, relative co-ordinate dynamics, rotation and translation of obstacles, union for obstacles, etc. 
\item Demonstration of theory (the vehicles avoid collision w/ other vehicles and reach their destinations).
\item Scaling of SPP.
\item Provide some more intuition about the solution that emerge out of theory-- Space-time separation, type of space-time trajectories (Almost straight line path w/ different starting times?), etc.
\item Reactivity of controller to the actual disturbance (Claire: be very detailed about explaining the setup of simulation)
\item Illustrate how the type of space-time trajectories change with change in disturbance bounds and STA
\end{itemize}
% Numerical Simuations (1-1.5p)
%% 2D + 2D example (what example, concretely?
%% Nx3D examples
\section{Numerical Implementation \label{sec:example}}
For the numerical examples in this paper, we use a numerical method provided in \cite{fisac15} which is based on methods in \cite{mitchell05, sethian96}. The numerical algorithm is shown in Algorithm \ref{alg:HJI}. Here, $\mathbf{i}$ represents the index for a particular grid cell, $I$ represents the set of grid indices, and $k$ represents the time step.  $\hat{V}$ represents the numerical approximation to $V$. $D^+_x\hat{V}, D^-_x\hat{V}$ represent the ``right" and ``left" approximations of spatial derivatives. For the numerical Hamiltonian $\hat{H}$, we used the well-known Lax-Friedrich approximation \cite{mitchell-thesis, osher91}.

For spatial derivatives $D^\pm_x\hat{V}$, we used a fifth-order accurate weighted essentially nonoscillatory scheme \cite{osher91,osher03}. For time derivative $D_t \hat{V}$, we used a third-order total variation diminishing Runge-Kutta scheme \cite{osher03, shu88}. For these derivative approximations, the implementation in \cite{LSToolbox} was used. For two-, three-, and four-dimensional (2D, 3D, 4D) computations, we used a $200^2, 71^3, 45^4$ grid, respectively.

\begin{algorithm}[h] 
\KwData{$\hat{l}(x_\mathbf{i},t_\mathbf{k}), \hat{g}(x_\mathbf{i},t_\mathbf{k})$}
 \KwResult{$\hat{V}(x_\mathbf{i},t_\mathbf{k})$}
 \BlankLine
 Initialization\DontPrintSemicolon\;\PrintSemicolon
   \For{$\mathbf{i}\in I$}{
     \nlset{Init}
     $\hat{V}(x_\mathbf{i},t_0) \leftarrow \max\{\hat{l}(x_\mathbf{i},t_0), \hat{g}(x_\mathbf{i},t_0)\}$\;
   }
   Value propagation\DontPrintSemicolon\;\PrintSemicolon
  \For{$k\leftarrow 1$ \KwTo $n$}{
     \For{$\mathbf{i}\in I$}{
     $\hat{V}(x_\mathbf{i},t_k) \leftarrow \hat{V}(x_\mathbf{i},t_{k-1})$ \DontPrintSemicolon\;\PrintSemicolon$\quad+\displaystyle\int_{t_k}^{t_{k-1}} \!\!\!\hat{H}\big(x_\mathbf{i}, D^+_x\hat{V}(x_\mathbf{i},\tau), D^-_x\hat{V}(x_\mathbf{i},\tau)\big)d\tau$\;
$\hat{V}(x_\mathbf{i},t_k) \leftarrow \min \left\{\hat{V}(x_\mathbf{i},t_k), l(x_\mathbf{i},t_k)\right\}$\;
$\hat{V}(x_\mathbf{i},t_\mathbf{k}) \leftarrow \max\left\{\hat{V}(x_\mathbf{i},t_k), g(x_\mathbf{i},t_k)\right\}$\;
     }
   }
 \caption{Numerical Double-Obstacle HJI Solution\label{alg:HJI}}
\end{algorithm}

Note that although the two examples we present have two and four vehicles, our method can be used for \textit{any} number of vehicles, as long as the state space of each vehicle is less than six dimensions. The computational complexity of our method scales linearly with the number of vehicles, allowing the possibility of performing trajectory planning for a very large number of vehicles.

\section{Two Vehicles with Kinematics Model \label{sec:2vek}}
Consider two vehicles $i = 1,2$ using the simple kinematics model with the following dynamics in $t\in[\tnow_i, \tf_i]$:

\begin{equation}
\begin{aligned}
\dotx_i &= v_i\ctrl_i(t), \ctrl_i(t) \in \ctrlin \\
\x_i(\tnow_i) &= \x_i^0 \\
\end{aligned}
\end{equation}

\noindent where $v_1=v_2=1$ are the maximum speeds of the vehicles and $\ctrlin$ is the unit disk. Under this model, each vehicle can move in any direction at some maximum speed. With the above dynamics, the Hamiltonian for each vehicle is

\bq
\ham_i(t, D_{\x_i}\soln_i(\x_i,t), \soln_i(\x_i,t)) = \min_{\ctrl_i} \{v_i\ctrl_i(t) \cdot D_{\x_i}\soln(\x_i, t)\}
\eq

\noindent giving the optimal control 
\bq
\ctrl_i(t) = -\frac{D_{\x_i}\soln_i(\x_i,t)}{\| D_{\x_i}\soln_i(\x_i,t) \|_2}
\eq

The vehicles have the following scheduled times of arrival from the following initial conditions:
\bq
\begin{aligned}
\x_1^0 &= (-0.5, 0), \x_2^0 = (0.5, 0)\\
\tf_1 &= \tf_2 = 0
\end{aligned}
\eq

The target sets of the vehicles are squares with side length $0.2$ on the opposite side of the domain, and the obstacles are rectangles near the middle of the domain. The system's initial conditions and domain are shown in Figure \ref{fig:kin_ic}.

For this system, we determine $\ti_1, \ti_2$, the latest acceptable times that vehicles $1,2$ must depart from their initial positions $\x_1^0, \x_2^0$ in order to reach their respective targets $\target_1, \target_2$ while avoiding obstacles and danger. We will do this by computing the reach-avoid sets from the target sets using two different methods. First, we perform SPP by solving the HJI VI (\ref{eq:HJI}) for the two vehicles as outlined in Section \ref{sec:solution}. Second, note that this system has a 4D joint state space, and thus the single-obstacle HJI VI (\ref{eq:HJIPDE}) would actually be numerically tractable. Therefore, we will also compute the reach-avoid set by solving (\ref{eq:HJIPDE}) in 4D for comparison.

\begin{figure}
	\centering
	\includegraphics[width=0.2\textwidth]{"kin_ic"}
	\caption{Initial configuration of the two-vehicle example.}
	\label{fig:kin_ic}
\end{figure}

\begin{figure}
	\centering
	\includegraphics[width=0.3\textwidth]{"kin_reach"}
	\caption{Evolution of reach-avoid set for vehicle $2$. The initial reach-avoid set at time 0 grows backwards in time unobstructed before it encounters obstacles (left top). Black arrows indicate direction of obstacle motion. When the time reaches $t=-0.61$, the growth of the reach-avoid set is inhibited by both the static obstacle $\obs$ and the time-varying obstacle induced by vehicle 1, $\mobs_1^2$. The evolution of the reach-avoid set is computed until $t=\ti_2=-1.13$, when the reach-avoid set contains vehicle $2$'s initial position.}
	\label{fig:kin_reach}
\end{figure}

\begin{figure}
	\centering
	\includegraphics[width=0.4\textwidth]{"kin_result"}
	\caption{A comparison between the single-obstacle and double-obstacle HJI VI solutions. With the double-obstacle HJI VI solution, vehicle $2$ optimally moves to $\target_2$ while avoiding vehicle $1$, which takes the shortest path to $\target_1$. With the single-obstacle HJI VI solution, both vehicles avoid each other along their way to the targets. The resulting reach-avoid sets at $t=\ti_2$ are very similar in both cases.}
	\label{fig:kin_result}
\end{figure}

\subsection{Solution via double-obstacle HJI VI and SPP}
With the HJI VI and SPP approach, we first determine the minimum time trajectory for vehicle $1$ from $\x_1^0$ to $\target_1$. Then, given this trajectory, we determine the optimal trajectory for vehicle $2$ that brings vehicle $2$ from $\x_2^0$ to $\target_2$ while avoiding the danger zone of vehicle $1$.

Figure \ref{fig:kin_reach} shows the reach-avoid set for vehicle $2$ at various times. We start at $t=\tf_2=0$, and propagate the reach-avoid set backwards in time until $t=\ti_2=-1.13$. Before the induced obstacle touches the reach-avoid set, the reach-avoid set grows from the target set in the same way as in a front propagation problem with uniform speed; this is shown in the left top subplot. Eventually, the obstacle inhibits the propagation of the reach-avoid set, shown in the next two subplots. Finally, the reach-avoid set grows to contain $\x_2^0$, and the computation is stopped at $\ti_2=-1.13$. The left top plot of Figure \ref{fig:kin_result} shows the resulting trajectory from applying the optimal control in Equation (\ref{eq:ctrl_syn}).

Computations were done on a $200^2$ grid. Trajectory planning for vehicle 1 took approximately 0.34 seconds using the fast marching method \cite{sethian96}. Trajectory planning via solving Equation (\ref{eq:HJI}) for vehicle $2$ given the trajectory for vehicle 1 took approximately 25 seconds. Computations were done on a Lenovo T420s laptop with a Core i7-2640M processor, and are orders of magnitude faster than doing a 4D HJI calculation, which took approximately 30 minutes.

\subsection{Solution via single-obstacle HJI VI}
To solve the single-obstacle HJI VI (\ref{eq:HJIPDE}), we define, in the joint state space of the vehicles, the \textit{static} joint target set

\bq
\target = \{(\x_1, \x_2)\in\R^4: \x_1 \in \target_1 \wedge \x_2 \in \target_2 \}
\eq

Next we define, also in the joint state space of the vehicles, the \textit{static} joint avoid set
\bq
\begin{aligned}
\avoid &= \{(\x_1, \x_2)\in\R^4: \x_1 \in \obs \vee \x_2 \in \obs \\
&\qquad \vee \|\x_1-\x_2\|_2\le\Rc \}
\end{aligned}
\eq

Now, we can solve the single-obstacle HJI VI (\ref{eq:HJIPDE}) with the terminal set  $\target\backslash\avoid$, and the avoid set $\avoid$.

The result of solving (\ref{eq:HJIPDE}) is shown in the top right and bottom left subplots of Figure \ref{fig:kin_result}. The top right subplot shows the resulting trajectory, in which the two vehicles cooperatively avoid collision. The bottom left plot compares the reach-avoid sets computed from solving (\ref{eq:HJIPDE}) and the double-obstacle HJI VI (\ref{eq:HJI}) at $t=\ti_2$. The two sets are quite similar. The discrepancy between the reach-avoid sets is due to the difference in control strategies derived from the two different approaches: with the single-obstacle HJI VI, we compute the joint optimal control for both vehicles, and with the double-obstacle HJI VI, we compute the optimal control for vehicle 2 given vehicle 1's optimal trajectory, which does not take into account vehicle 2's motion. 

For the latest start time, we obtained $\ti_2 = -1.15$ from the single-obstacle HJI VI (recall $\ti_2=-1.13$ from the double-obstacle HJI VI). This discrepancy is likely due to the grid resolution limitation when doing a 4D calculation. Computations were done on a $45^4$ grid, and took approximately 30 minutes.

\section{Four Vehicles with Constrained Turn Rate}
Consider four vehicles with states $\x_i = [x_i, y_i, \theta_i]^\top$ modeled using a horizontal kinematics model with the following dynamics for $t \in[\tnow_i, \tf_i],i=1,2,3,4$:

\begin{equation}
\begin{aligned}
\dot{x}_i &= v_i \cos(\theta_i) \\
\dot{y}_i &= v_i \sin(\theta_i) \\
\dot{\theta}_i &= \omega_i \\
\x_i(\tnow_i) &= \x_i^0 \\
|\omega_i| &\le \bar{\omega}_i \\
\end{aligned}
\end{equation}

\noindent where $(x_i, y_i)$ is the position of vehicle $i$, $\theta_i$ is the heading of vehicle $i$, and $v_i$ is the speed of vehicle $i$. The control input $\ctrl_i$ of vehicle $i$ is the turning rate $\omega_i$, whose absolute value is bounded by $\bar{\omega}_i$. For illustration, we chose $\bar{\omega}_i=1 \forall i$ and assume $v_i=1$ is constant; however, our method can easily handle the case in which $\bar{\omega}_i$ differ across vehicles and $v_i$ is a control input. The Hamiltonian associated with vehicle $i$ is

\bq
\begin{aligned}
\ham_i(t, &D_{\x_i}\soln_i(\x_i,t), \soln_i(\x_i,t)) \\
&= \min_i \big\{ v_i D_{x_i} \soln_i(\x_i, t) \cos(x_i(t)) \\
&\; + v_i D_{y_i} \soln_i(\x_i, t) \sin(y_i(t)) + D_{\theta_i} \soln_i(\x_i, t) \omega_i \big\}
\end{aligned}
\eq

\noindent giving the optimal control
\bq
\omega_i(t) = -\bar{\omega}_i\frac{D_{\theta_i}\soln_i(\x_i,t)}{\left| D_{\theta_i}\soln_i(\x_i,t) \right|}
\eq

The vehicles have initial conditions and STA as follows:
\bq
\begin{aligned}
\x_1^0 &= (-0.5, 0, 0), &\tf_1 &= 0\\
\x_2^0 &= (0.5, 0, \pi), &\tf_2 &= 0.2\\
\x_3^0 &= \left(-0.6, 0.6, \frac{7\pi}{4}\right), &\tf_3 &= 0.4\\
\x_4^0 &= \left(0.6, 0.6, \frac{5\pi}{4}\right), &\tf_4 &= 0.6\\
\end{aligned}
\eq

The target sets $\target_i$ of the vehicles are all 4 circles of radius $0.1$ in the domain. The centers of the target sets are at $(0.7, 0.2), (-0.7, 0.2), (0.7, -0.7), (-0.7, -0.7)$ for vehicles $i=1,2,3,4$, respectively. The domain $\amb$ and obstacles $\obs$ are the same as those of the example in Section \ref{sec:2vek}. The setup for this example is shown in Figure \ref{fig:dubins_ic}. 

The joint state space of this system is twelve-dimensional, intractable for analysis using the single-obstacle HJI VI (\ref{eq:HJIPDE}). Therefore, we will repeatedly solve the double-obstacle HJI VI (\ref{eq:HJI}) to compute the reach-avoid sets from targets $\target_i$ for vehicles $1,2,3,4$, in that order, with moving obstacles induced by vehicles $j=1,\ldots,i-1$. We will also obtain $\ti_i,i=1,2,3,4$, the LSTs for each vehicle in order to reach $\target_i$ by $\tf_i$.

Figures \ref{fig:dubins_reach_all}, \ref{fig:dubins_reach_3}, and \ref{fig:dubins_result} show the results. Since the state space of each vehicle is 3D, the reach-avoid set is also 3D. To visualize the results, we slice the reach-avoid sets at the initial heading angles $\theta_i^0$. Figure \ref{fig:dubins_reach_all} shows the 2D reach-avoid set slices for each vehicle at its LSTs $\ti_1=-1.12, \ti_2=-0.94,\ti_3=-1.48,\ti_4=-1.44$ determined from our method. The obstacles in the domain $\obs$ and the obstacles induced by other vehicles inhibit the evolution of the reach-avoid sets, carving out thin ``channels" that separate the reach-avoid set into different ``islands". One can see how these channels and islands form by looking at the time evolution of the reach-avoid set, shown in Figure \ref{fig:dubins_reach_3} for vehicle 3. 

Finally, Figure \ref{fig:dubins_result} shows the resulting trajectories of the four vehicles. The subplot labeled $t=-0.55$ shows all four vehicles in close proximity without collision: each vehicle is outside of the danger zone of all other vehicles. The actual arrival times of vehicles $i=1,2,3,4$ are $0, 0.19, 0.34, 0.31$, respectively. It is interesting to note that for some vehicles, the actual arrival times are earlier than the STAs $\tf_i, i=1,2,3,4$. This is because in order to arrive at the target by $\tf_i$, these vehicles must depart early enough to avoid major delays  resulting from the induced obstacles of other vehicles; these delays would have lead to a late arrival if vehicle $i$ departed after $\ti_i$.

\begin{figure}
	\centering
	\includegraphics[width=0.275\textwidth]{"dubins_ic"}
	\caption{Initial configuration of the four-vehicle example.}
	\label{fig:dubins_ic}
\end{figure}

\begin{figure}
	\centering
	\includegraphics[width=0.4\textwidth]{"dubins_reach_all"}
	\caption{Reach-avoid sets at $t=\ti_i$ for vehicles $1,2,3,4$, sliced at initial headings $\theta_i^0$. Black arrows indicate direction of obstacle motion. Due to the turn rate constraint, the presence of static obstacles $\obs$ and time-varying obstacles induced by higher-priority vehicles $\mobs^i_j(t)$ carves ``channels" in the reach-avoid set, dividing it up into multiple ``islands".}
	\label{fig:dubins_reach_all}
\end{figure}

\begin{figure}
	\centering
	\includegraphics[width=0.4\textwidth]{"dubins_reach_3"}
	\caption{Time evolution of the reach-avoid set for vehicle $3$, sliced at its initial heading $\theta_3^0=\frac{7\pi}{4}$. Black arrows indicate direction of obstacle motion. Initially, the reach-avoid set grows unobstructed by obstacles, as shown in the top subplots. Then, in the bottom subplots, the static obstacles $\obs$ and the induced obstacles of vehicles $1$ and $2$, $\mobs^3_1,\mobs^3_2$, carve out ``channels" in the reach-avoid set.}
	\label{fig:dubins_reach_3}
\end{figure}

\begin{figure}
	\centering
	\includegraphics[width=0.4\textwidth]{"dubins_result"}
	\caption{The planned trajectories of the four vehicles. In the left top subplot, only vehicles $3$ (green) and $4$ (purple) have started moving, showing $\ti_i$ is not common across the vehicles. Right top subplot: all vehicles have come within very close proximity, but none is in the danger zone another. Left bottom subplot: vehicle $1$ (blue) arrives at $\target_1$ at $t=0$. Right bottom subplot: all vehicles have reached their destination, some ahead of the STA $\tf_i$.}
	\label{fig:dubins_result}
\end{figure}

% % !TEX root = ./SPP_IoTjournal.tex
\section{Conclusion}
Provably safe multi-vehicle path planning in an important problem that needs to be addressed to ensure that vehicles can fly in close proximity of each other. Recently, the SPP algorithm was proposed for multi-vehicle path planning problem that scales linearly with the number of vehicles. We illustrate the full potential of the algorithm by using it for large-scale multi-vehicle path planning problems under different flying conditions. We demonstrate how different types of space-time trajectories emerge naturally out of the algorithm for different disturbance conditions and other problem parameters. The reactivity of the obtained controller is also demonstrated under different wind conditions.
% Conclusion (0.5p)
\section{Conclusions and Future Work}
We have presented a problem formulation that allows us to consider the multi-vehicle trajectory planning problem in a tractable way by planning trajectories for vehicles in order of priority. In order to do this, we modeled higher-priority vehicles as time-varying obstacles. We then solved a double-obstacle HJI VI to obtain the reach-avoid set for each vehicle. The reach-avoid set characterizes the region from which each vehicle is guaranteed to arrive at its target within a time horizon, while avoiding collision with obstacles and higher-priority vehicles. The solution also gives each vehicle a latest start time as well as the optimal control which guarantees that each vehicle safely reaches its target on time. This paper provides a first application of the double-obstacle HJI VI. Immediate future work includes investigating ways to relax assumptions about the knowledge of the trajectories of other vehicles, more sophisticated induced obstacles for modeling trajectory uncertainty, and problems involving adversarial agents, among many other possibilities.

%%%%%%%%%%%%%%%%%%%%%%%%%%%%%%%%%%%%%%%%%%%%%%%%%%%%%%%%%%%%%%%%%%%%%%%%%%%%%%%%
%\addtolength{\textheight}{1cm}   % This command serves to balance the column lengths
                                  % on the last page of the document manually. It shortens
                                  % the textheight of the last page by a suitable amount.
                                  % This command does not take effect until the next page
                                  % so it should come on the page before the last. Make
                                  % sure that you do not shorten the textheight too much.

\bibliographystyle{IEEEtran}
\bibliography{references}
\end{document}
