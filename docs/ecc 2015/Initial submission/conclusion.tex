\section{Conclusions and Future Work}
We have presented a problem formulation that allows us to consider the multi-vehicle trajectory planning problem in a tractable way by planning trajectories for vehicles in order of priority. In order to do this, we modeled higher-priority vehicles as time-varying obstacles. We then solved a double-obstacle HJI VI to obtain the reach-avoid set for each vehicle. The reach-avoid set characterizes the region from which each vehicle is guaranteed to arrive at its target within a time horizon, while avoiding collision with obstacles and higher-priority vehicles. The solution also gives each vehicle a latest start time as well as the optimal control which guarantees that each vehicle safely reaches its target on time. This paper provides a first application of the double-obstacle HJI VI. Immediate future work includes investigating ways to relax assumptions about the knowledge of the trajectories of other vehicles, more sophisticated induced obstacles for modeling trajectory uncertainty, and problems involving adversarial agents, among many other possibilities.