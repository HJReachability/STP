% !TEX root = SPP_journal.tex
\subsection{Incomplete Information Results \label{sec:basic_results}}
We demonstrate our proposed methods using a four-vehicle example. Each vehicle has the following simple kinematics model:

\begin{equation}
\label{eq:dyn_i}
\begin{aligned}
\dot{\pos}_{x,i} &= v_i \cos \theta_i + d_{x,i} \\
\dot{\pos}_{y,i} &= v_i \sin \theta_i + d_{y,i}\\
\dot{\theta}_i &= \omega_i + d_{\theta,i}, \\
\underline{v} & \le v_i \le \bar{v}, |\omega_i| \le \bar{\omega},\\
\|(d_{x,i}, & d_{y,i}) \|_2 \le d_{r}, |d_{\theta,i}| \le \bar{d_{\theta}}
\end{aligned}
\end{equation}

\noindent where $p_i = (p_{x,i}, p_{y,i}), \theta_i, d = (d_{x,i}, d_{y,i}, d_{\theta,i})$ respectively represent $\veh{i}$'s position, heading, and disturbances in the three states. The control of $\veh{i}$ is $u_i = (v_i, \omega_i)$, where $v_i$ is the speed of $\veh{i}$ and $\omega_i$ is the turn rate; both controls have a lower and upper bound. For illustration purposes, we chose $\underline{v} = 0.5, \bar{v} = 1, \bar\omega = 1$; however, our method can easily handle the case in which these inputs differ across vehicles and cases in which each vehicle has different dynamic model. The disturbance bounds are chosen as $d_{r} = 0.1, \bar{d_{\theta}} = 0.2$, which correspond to a 10\% uncertainty in the dynamics.

The initial states of the vehicles are given as follows:
\begin{equation}
\begin{aligned}
x_1^0 &= (-0.5, 0, 0), \quad &x_2^0 = (0.5, 0, \pi), \\
x_3^0 &= \left(-0.6, 0.6, 7\pi/4\right), \quad &x_4^0 = \left(0.6, 0.6, 5\pi/4\right).
\end{aligned}
\end{equation}

\noindent Each of the vehicles has a target set $\targetset_i$ that is circular in their position $\pos_i$ centered at $c_i = (c_{x,i}, c_{y,i})$ with radius $r$:
\vspace{-0.2em}
\begin{equation}
\targetset_i = \{x_i \in \R^3: \|p_i - c_i\| \le r\}
\end{equation}

\noindent For the example shown, we chose $c_1 = (0.7, 0.2), c_2 = (-0.7, 0.2), c_3 = (0.7, -0.7), c_4 = (-0.7, -0.7)$ and $r = 0.1$. The setup of the example is shown in Fig. \ref{fig:init_setup}.

Since the joint state space of this system is intractable for a direct application of HJ reachability theory, we repeatedly solve (\ref{eq:HJIVI}) to compute BRSs from the targets $\targetset_i, i =1,2,3,4$, in that order, with moving obstacles induced by vehicles $j=1,\ldots,i-1$. We also obtain $\ldt_i, i=1,2,3,4$ assuming $\sta_i=0$ without loss of generality. Note that even though $\sta_i$ is assumed to be same for all vehicles in this example for simplicity, our method can easily handle the case in which $\sta_i$ are different for each vehicle.

\begin{figure}
  \centering
  \includegraphics[width=0.30\textwidth]{"fig/init_setup"}
  \caption{Initial configuration of the four-vehicle example.}
  \label{fig:init_setup}
\end{figure}

For each proposed method of computing induced obstacles, we show the vehicles' entire trajectories (colored dotted lines), and overlay their positions (colored asterisks) and headings (arrows) at a point in time in which they are in relatively dense configuration. In all cases, the vehicles are able to avoid each other's danger zones (colored dashed circles) while getting to their target sets in minimum time. In addition, we show the evolution of the BRS over time for $\veh{3}$ (green boundaries) as well as the induced obstacles of the higher-priority vehicles (black boundaries).

\subsubsection{Centralized Controller}
Fig. \ref{fig:cc_traj} shows the simulated trajectories in the situation where a centralized controller enforces each vehicle to use the optimal controller $u^*_i(t, x_i)$ according to \eqref{eq:opt_ctrl_i}, as described in Section \ref{sec:cc}.

\begin{figure}
  \centering
  \includegraphics[width=0.40\textwidth]{"fig/cc_traj"}
  \caption{Simulated trajectories in the centralized controller method. Since the higher priority vehicles induce relatively small obstacles in this case, vehicles do not deviate much from a straight line trajectory towards their respective targets.}
  \label{fig:cc_traj}
\end{figure}

In this case, vehicles appear to deviate slightly from a straight line trajectory towards their respective targets, just enough to avoid higher-priority vehicles. The deviation is small since the centralized controller is quite restrictive, making the possible positions of higher priority vehicles cover a small area. In the dense configuration at $t=-1.0$, the vehicles are close to each other but still outside each other's danger zones.

\begin{figure}
  \centering
  \includegraphics[width=0.40\textwidth]{"fig/cc_rs3"}
  \caption{Evolution of the BRS and the obstacles induced by $\veh{1}$ and $\veh{2}$ for $\veh{3}$ in the centralized controller method. Since every vehicle is applying the optimal control at all times, the obstacle sizes are relatively small.}
  \label{fig:cc_rs3}
\end{figure}

Fig. \ref{fig:cc_rs3} shows the evolution of the BRS for $\veh{3}$ (green boundary), as well as the obstacles (black boundary) induced by the higher-priority vehicles $\veh{1}$ (blue) and $\veh{2}$ (red). The locations of the induced obstacles at different time points include the actual positions of $\veh{1}$ and $\veh{2}$ at those times, and the size of the obstacles remains relatively small. $\ldt_i$ numbers for the four vehicles (in order) in this case are $-1.35, -1.37, -1.94$ and $-2.04$. Numbers are relatively close for vehicles $\veh{1}$, $\veh{2}$ and $\veh{3}$, $\veh{4}$, because the obstacles generated by higher-priority vehicle are small and hence do not affect $\ldt$ of the lower-priority vehicles significantly. 

\subsection{Comparison of Proposed Methods}
This section briefly discusses the relative advantages and limitations of the proposed methods. Each method makes a trade-off between optimality (in terms of $\ldt_i$) and flexibility in control and disturbance rejection.

\subsubsection{Centralized Controller}
Given an order of priority, the vehicles will have the relatively high $\ldt_i$ in this method since a higher-priority vehicle maximizes its $\ldt_i$ as much as possible, while at the same time inducing a relatively small obstacle so as to minimize its impedance towards the lower-priority vehicles. A limitation of this method is that a centralized controller is likely required to ensure that the optimal control is being applied by the vehicles at all times, and hence ensure safety.

\subsubsection{Least Restrictive Control}
This method gives more control flexibility to the higher-priority vehicles, as long as the control does not push the vehicle out of its BRS. This flexibility, however, comes at the price of having larger induced obstacle, lowering $\ldt_i$ for the lower-priority vehicles.  

\subsubsection{Robust Trajectory Tracking}
Since the obstacle size is constant over time, this method is easier to implement from a practical standpoint. This method also aims at striking a balance between $\ldt_i$ across vehicles. In particular, the $\ldt$ of a higher-priority vehicle can be lower compared to the centralized controller method, so that a lower-priority vehicle can achieve a higher $\ldt$, making this method particularly suitable for the scenarios where there is no strong sense of priority among vehicles. This method, however, is computationally tractable only when the tracking error dynamics are independent of the absolute states, as it otherwise requires doing computation in the joint state space of system dynamics and virtual vehicle dynamics. 