% !TEX root = SPP_journal.tex
\subsection{Numerical Results \label{sec:basic_results}}
We now illustrate the basic SPP algorithm using a four-vehicle example. In this example and the examples in the later sections, we will use the following dynamics for each vehicle:

\begin{equation}
\begin{aligned}
\dot x_i &= v_i \cos(\theta_i) \\
\dot y_i &= v_i \sin(\theta_i) \\
\dot \theta_i &= \omega_i \\
\state_i(\edt_i) &= \state_i^0 = (x_i^0, y_i^0, \theta_i^0) \\
i &= 1, \ldots, \N
\end{aligned}
\end{equation}

\noindent where $\state_i = (x_i, y_i, \theta_i)$ is the state of vehicle $\veh{i}$, $\pos_i = (x_i, y_i)$ is the position of vehicle $\veh{i}$, $\theta_i$ is the heading of vehicle $\veh{i}$, $v_i$ is the speed of vehicle $\veh{i}$, and $\omega_i$ is the turn rate of vehicle $\veh{i}$. We have chosen these dynamics for clarity of illustration; in general, the SPP algorithm can handle more general systems of the form in which the vehicles have different control bounds and dynamics. 

In this particular example, we assume that the vehicles have constant speed $v_i = 1 ~ \forall i$, and that the control of each vehicle $\veh{i}$ is given by $\ctrl_i = \omega_i$ with $|\omega_i| \le \bar\omega = 1 ~ \forall i$. 

For this example, the target sets $\targetset_i$ of the vehicles are circles of radius $0.1$ in the position space; each vehicle is trying to reach some desired set of positions. In terms of the state space $\state_i$, the target set is defined as

\begin{equation}
\targetset_i = \{z_i \mid \dist(p_i, c_i) \le 0.1\}
\end{equation}

\noindent where $c_i$ are centers of the target circles.

The vehicles have target centers $c_i$, initial conditions $\state_i^0$, and scheduled times of arrivals $\sta_i$ as follows:

\begin{equation}
\begin{aligned}
c_1 = (0.7, 0.2), \quad& \state_1^0 = (-0.5, 0, 0), \quad & \sta_1 = 0 \\
c_2 = (-0.7, 0.2), \quad& \state_2^0 = (0.5, 0, \pi), \quad & \sta_2 = 0.2 \\
c_3 = (0.7, -0.7), \quad& \state_3^0 = \left(-0.6, 0.6, 7\pi/4\right), \quad & \sta_3 = 0.4\\
c_4 = (-0.7, -0.7), \quad & \state_4^0 = \left(0.6, 0.6, 5\pi/4\right), \quad & \sta_4 = 0.6
\end{aligned}
\end{equation}

The setup for this example is shown in Fig. \ref{fig:dubins_ic}. 

The joint state space of this four-vehicle system is twelve-dimensional (12D), making the joint path planning and collision avoidance problem intractable for direct analysis. Therefore, we apply the SPP algorithm described in Algorithm \ref{alg:basic} and repeatedly solve the double-obstacle HJ VI in \eqref{eq:HJIVI_BRS} to obtain the optimal control for each vehicle to reach its target while avoiding higher-priority vehicles. In addition, due to the flexibility of the HJ VI with respect to time-varying systems, the different scheduled times of arrival $\sta_i$ can be trivially incorporated. 

Fig. \ref{fig:dubins_reach_all}, \ref{fig:dubins_reach_3}, and \ref{fig:dubins_result} show the simulation results. Since the state space of each vehicle is 3D, each of the BRSs $\brs(t, \targetset_i, \obsset_i(t), \basicham)$ is also 3D. To visualize the results, we slice the BRSs at the initial heading angles $\theta_i^0$. Fig. \ref{fig:dubins_reach_all} shows the 2D BRS slices for each vehicle at its latest departure times $\ldt_1=-1.12, \ldt_2=-0.94,\ldt_3=-1.48,\ldt_4=-1.44$ determined from our method. The obstacles in the domain $\sosetp$ and the obstacles induced by other vehicles $\ioset_i^j(t)$ inhibit the evolution of the BRSs, carving out thin ``channels" that separate the BRSs into different ``islands". One can see how these channels and islands form by examining the time evolution of the BRS, shown in Figure \ref{fig:dubins_reach_3} for vehicle $\veh{3}$. 

Finally, Fig. \ref{fig:dubins_result} shows the resulting trajectories of the four vehicles. Most interestingly, the subplot labeled $t=-0.55$ shows all four vehicles in close proximity without collision: each vehicle is outside of the danger zone of all other vehicles (although the danger zones may overlap). This close proximity is an indication of the optimality of the basic SPP algorithm given the assigned priority ordering. Since no disturbances are present, getting as close to other vehicles' danger zones as possible without entering the danger zones intuitively results in short transit times.

The actual arrival times of vehicles $i=1,2,3,4$ are $0, 0.19, 0.34, 0.31$, respectively. It is interesting to note that for some vehicles, the actual arrival times are earlier than the scheduled times of arrivals $\sta_i, i=1,2,3,4$. This is because in order to arrive at the target by $\sta_i$, these vehicles must depart early enough to avoid major delays resulting from the induced obstacles of other vehicles; these delays would have lead to a late arrival if vehicle $i$ departed after $\sta_i$.

\begin{figure}
	\centering
	\includegraphics[width=0.5\textwidth]{"fig/dubins_ic"}
	\caption{Initial configuration of the four-vehicle example.}
	\label{fig:dubins_ic}
\end{figure}

\begin{figure}
	\centering
	\includegraphics[width=0.5\textwidth]{"fig/dubins_reach_all"}
	\caption{Reach-avoid sets at $t=\ldt_i$ for vehicles $1,2,3,4$, sliced at initial headings $\theta_i^0$. Black arrows indicate direction of obstacle motion. Due to the turn rate constraint, the presence of static obstacles $\sosetp$ and time-varying obstacles induced by higher-priority vehicles $\ioset_i^j(t)$ carves ``channels" in the reach-avoid set, dividing it up into multiple ``islands".}
	\label{fig:dubins_reach_all}
\end{figure}

\begin{figure}
	\centering
	\includegraphics[width=0.5\textwidth]{"fig/dubins_reach_3"}
	\caption{Time evolution of the reach-avoid set for vehicle $\veh{3}$, sliced at its initial heading $\theta_3^0=\frac{7\pi}{4}$. Black arrows indicate direction of obstacle motion. Initially, the reach-avoid set grows unobstructed by obstacles, as shown in the top subplots. Then, in the bottom subplots, the static obstacles $\sosetp$ and the induced obstacles $\ioset_3^1,\ioset_3^2$, carve out ``channels" in the reach-avoid set.}
	\label{fig:dubins_reach_3}
\end{figure}

\begin{figure}
	\centering
	\includegraphics[width=0.5\textwidth]{"fig/dubins_result"}
	\caption{The planned trajectories of the four vehicles. In the left top subplot, only vehicles $3$ (green) and $4$ (purple) have started moving, showing $\ldt_i$ is not common across the vehicles. Right top subplot: all vehicles have come within very close proximity, but none is in the danger zone another. Left bottom subplot: vehicle $1$ (blue) arrives at $\targetset_1$ at $t=0$. Right bottom subplot: all vehicles have reached their destination, some ahead of the STA $\sta_i$.}
	\label{fig:dubins_result}
\end{figure}