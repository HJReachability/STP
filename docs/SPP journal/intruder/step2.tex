% !TEX root = ../SPP_journal.tex
\subsection{Step 2: Augmented Obstacle Computation \label{sec:intruder_aocomp}}
The goal of this section is to compute a region around the obstacles $\obsset_i(\cdot)$ such that for all disturbances, $\veh{i}$ can avoid colliding with any obstacle for $\iat$ seconds if $\veh{i}$ starts outside this region. The utility of computing such a region will become clearer in Step 3. 

We let $\tilde\obsset_i(t)$ denote the region, which can be thought of as a set of augmented obstacles. To ensure that a vehicle does not hit the obstacle $\obsset_i(t_1 + t')$ at time $t = t_1 + t'$ starting at $t = t_1$, even when it applies any control for the next $t'$ seconds, it suffices to avoid the $t'$-horizon BRS of $\obsset_i(t_1 + t')$. This argument applies for all obstacles to appear in the next $\iat$ seconds to ensure safety under any controller and disturbance for the next $\iat$ seconds. Mathematically,

\begin{equation} \label{eqn:inducedobs}
\tilde\obsset_i(t) = \bigcup_{\tau \in [0, \iat]} \brs_{\mathcal{G}}(\tau, \obsset_i(t+\tau), \emptyset, \ham_{\mathcal{G}})
\end{equation}
where $\brs_{\mathcal{G}}(\tau, \obsset_i(t+\tau), \emptyset, \ham_{\mathcal{G}})$ represents BRS of $\obsset_i(t+\tau)$ computed backwards for $\tau$ seconds. The Hamiltonian 
$\ham_{\mathcal{G}}$ is given by:

\begin{equation} \label{eqn:BRSham}
\ham_{\mathcal{G}}(\state_i, p) = \min_{\ctrl_i} \min_{\dstb_i} p \cdot f_i (\state_i, \ctrl_i, \dstb_i)
\end{equation}

Finally, we compute a BRS that contains the initial state of $\veh{i}$ while these augmented obstacles, $\brs_\text{AO}(t, \targetset_i, \bar\obsset_i, \ham_\text{AO})$, where $\ham_\text{AO} = \ham_{\mathcal{G}}$ (\textbf{Somil}: this is not correct. Disturbance should maximize the Hamiltonian for this BRS.) Note that this BRS ensures liveness for $\veh{i}$ in the absence of intruder. Moroever, if we start from $\brs_\text{AO}$, we are guaranteed to avoid collision for $\iat$ seconds, irrespective of control and disturbance applied. 