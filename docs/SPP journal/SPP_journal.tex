%%%%%%%%%%%%%%%%%%%%%%%%%%%%%%%%%%%%%%%%%%%%%%%%%%%%%%%%%%%%%%%%%%%%%%%%%%%%%%%%
%2345678901234567890123456789012345678901234567890123456789012345678901234567890
%        1         2         3         4         5         6         7         8

\documentclass[journal]{IEEEtran}  
%\documentclass[12pt, draftcls, onecolumn]{IEEEtran}      

\IEEEoverridecommandlockouts                              % This command is only
                                                          % needed if you want to
                                                          % use the \thanks command
%\overrideIEEEmargins
% See the \addtolength command later in the file to balance the column lengths
% on the last page of the document

\usepackage{mathtools}    % need for sub equations
\usepackage{amsfonts}
\usepackage{graphicx}   % need for figures
\usepackage{subcaption}
\usepackage{epsfig} 
\usepackage{cancel}
\usepackage{amssymb}
\usepackage{color}
\usepackage{bm}
\usepackage[ruled,vlined,titlenotnumbered]{algorithm2e} 
\usepackage{todonotes} \setlength{\marginparwidth}{2.5cm} 
\usepackage{float}
\usepackage{cite}

\newcommand{\MCnote}{\textcolor{red}}
\newcommand{\SBnote}{\textcolor{blue}}

\newcommand{\R}{\mathbb{R}} % Real number
\newcommand{\dist}{\text{dist}} % Distance
\newcommand{\rc}{R_c} % Capture radius
\newcommand{\cradius}{\rc}
\newcommand{\N}{N} % number of agents

\newcommand{\veh}{Q} % vehicle
\newcommand{\intr}{I} % Intruder index
\newcommand{\state}{x} % state
\newcommand{\ctrl}{u} % control
\newcommand{\dstb}{d} % disturbance
\newcommand{\pos}{p} % position
\newcommand{\npos}{h} % non-position states

\newcommand{\traj}{\zeta}
\newcommand{\errstate}{e}

\newcommand{\fdyn}{f} % full dynamics
\newcommand{\cset}{\mathcal{U}} % Control set
\newcommand{\cfset}{\mathbb{U}} % control function set
\newcommand{\dset}{\mathcal{D}} % disturbance
\newcommand{\dfset}{\mathbb{D}} % disturbance function set
\newcommand{\obsset}{\mathcal{G}} % Obstacle (the one used to solve PDE)
\newcommand{\dz}{\mathcal{Z}} % danger zone
\newcommand{\intobs}{\mathcal{M}} % Intermediate obstacles required for the intruder Method2

\newcommand{\valfunc}{V} % value function
\newcommand{\valfuncfwd}{W} % value function for forwards reachable set
\newcommand{\brs}{\mathcal{V}} % backwards reachable set
\newcommand{\frs}{\mathcal{W}} % forwards reachable set
\newcommand{\pfrs}{\mathcal{P}} % projected forwards reachable set
\newcommand{\targetset}{\mathcal{L}} % target set
\newcommand{\ham}{H} % Hamiltonian
\newcommand{\fc}{l} % Final condition
\newcommand{\ic}{l} % Initial condition
\newcommand{\obsfunc}{g} % Obstacle function
\newcommand{\costate}{\lambda}

\newcommand{\disckernel}{\Omega} % Discriminating kernel

\newcommand{\edt}{t^\text{EDT}} % earliest departure time
\newcommand{\ldt}{t^\text{LDT}} % latest departure time
\newcommand{\sta}{t^\text{STA}} % scheduled time of arrival
\newcommand{\ioset}{\mathcal{O}} % Induced obstacle
\newcommand{\boset}{\mathcal{B}} % Base obstacle
\newcommand{\sosetp}{\mathcal{S}} % static obstacle in position space
\newcommand{\soset}{\ioset^\text{static}} % static obstacle in state space
\newcommand{\iat}{t^\text{IAT}} % intruder avoidance time
\newcommand{\wcttr}{t^\text{WC}} % worst case TTR

\newcommand{\basicham}{\ham^\text{basic}}

\newcommand{\tsa}{\underline{t}} % time of start of avoidance
\newcommand{\tea}{\bar{t}} % time of end of avoidance

\newcommand{\errorbound}{\mathcal{E}} % Error ``bubble" between vehicle and tracking reference
\newcommand{\tracklaw}{\kappa} % Robust tracking law

\newtheorem{assumption}{Assumption}
\newtheorem{alg}{Algorithm}
\newtheorem{remark}{Remark}

\title{\LARGE \bf Robust Sequential Path Planning Under Disturbances and Adversarial Intruder}

\author{Mo Chen, Somil Bansal, Jaime F. Fisac, and Claire J. Tomlin
\thanks{This work has been supported in part by NSF under CPS:ActionWebs (CNS-931843), by ONR under the HUNT (N0014-08-0696) and SMARTS (N00014-09-1-1051) MURIs and by grant N00014-12-1-0609, by AFOSR under the CHASE MURI (FA9550-10-1-0567). The research of M. Chen and J. F. Fisac have received funding from the ``NSERC'' program and ``la Caixa" Foundation, respectively.}
\thanks{All authors are with the Department of Electrical Engineering and Computer Sciences, University of California, Berkeley. \{mochen72, somil, jfisac, tomlin\}@eecs.berkeley.edu}
}

\begin{document}
\maketitle
\thispagestyle{empty}
\pagestyle{empty}

%%%
\begin{abstract}
Provably safe and scalable multi-vehicle path planning is an important and urgent problem due to the expected increase of automation in civilian airspace in the near future. Although this problem has been studied in the past, there has not been a method that guarantees both liveness and safety for vehicles with general non-linear dynamics while taking into account disturbances and potential intruders, to the best of our knowledge. Hamilton-Jacobi (HJ) reachability is the ideal tool for guaranteeing liveness and safety under such scenarios, and has been successfully applied to many small-scale problems. However, a direct application of HJ reachability in most cases becomes intractable when there are more than two vehicles due to the exponentially scaling computation complexity with respect to system dimension. In this paper, we take advantage of the guarantees HJ reachability provides, and eliminate the computation burden by assigning a strict priority ordering to the vehicles under consideration. Under this sequential path planning (SPP) scheme, higher-priority vehicles can plan their paths freely, while lower-priority vehicles treat higher-priority vehicles as moving obstacles. Our proposed method guarantees collision avoidance and optimality of the paths given the priority ordering. With a computation complexity that scales only quadratically when accounting for both disturbances and an intruder, and \textit{linearly} when accounting for only disturbances, SPP can tractably solve the multi-vehicle path planning problem for vehicles with general non-linear dynamics in a practical setting. 
\end{abstract}

%However, this problem is inherently challenging due to not only the complex interactions that exist between the vehicles under consideration, but also the presence of disturbances and potentially malicious intruders. 

% !TEX root = ../SPP_IoTjournal.tex
\section{Introduction \label{sec:introduction}}
Recently, there has been an immense surge of interest in the use of unmanned aerial systems (UASs) for civil applications. The applications include package delivery, aerial surveillance, disaster response, among many others \cite{Tice91, Debusk10, Amazon16, AUVSI16, BBC16}. These civil applications will involve unmanned aerial vehicles (UAVs) flying in urban environments, potentially in close proximity to humans, other UAVs, and other important assets. As a result, government agencies such as the Federal Aviation Administration (FAA) and National Aeronautics and Space Administration (NASA) of the United States are urgently trying to develop new scalable ways to organize an airspace in which potentially thousands of UAVs can fly together \cite{FAA13, Kopardekar16}.

One essential problem that needs to be addressed for this endeavor to be successful is that of trajectory planning: how a group of vehicles in the same vicinity can reach their destinations while avoiding situations which are considered dangerous, such as collisions. Many previous studies address this problem under different assumptions. In some studies, specific control strategies for the vehicles are assumed, and approaches such as those involving induced velocity obstacles \cite{Fiorini98, Chasparis05, Vandenberg08,Wu2012} and involving virtual potential fields to maintain collision avoidance \cite{Olfati-Saber2002, Chuang07} have been used. Methods have also been proposed for real-time trajectory generation \cite{Feng-LiLian2002}, for path planning for vehicles with linear dynamics in the presence of obstacles with known motion \cite{Ahmadzadeh2009}, and for cooperative path planning via waypoints which do not account for vehicle dynamics \cite{Bellingham}. Other related work is in the collision avoidance problem without path planning. These results include those that assume the system has a linear model \cite{Beard2003, Schouwenaars2004, Stipanovic2007}, rely on a linearization of the system model \cite{Massink2001, Althoff2011}, assume a simple positional state space \cite{Lin2015}, and many others \cite{Lalish2008, Hoffmann2008, Chen2016}.

However, to make sure that a dense group of UAVs can safely fly in close vicinity of each other, we need the capability to flexibly plan provably safe and dynamically feasible trajectories without making strong assumptions on the vehicles' dynamics and other vehicles' motion. Moreover, any trajectory planning scheme that addresses collision avoidance must also guarantee both goal satisfaction and safety of UAVs despite disturbances caused by wind and communication faults \cite{Kopardekar16}. Furthermore, unexpected scenarios such as UAV malfunctions or even UAVs with malicious intent need to be accounted for. Finally, the proposed scheme should scale well with the number of vehicles.

The problem of trajectory planning and collision avoidance under disturbances in safety-critical systems has been studied using Hamilton-Jacobi (HJ) reachability analysis, which provides guarantees on goal satisfaction and safety of optimal system trajectories \cite{Barron90, Mitchell05, Bokanowski10, Bokanowski11, Margellos11, Fisac15}. Reachability-based methods are particularly suitable in the context of UAVs because of the formal guarantees that are provided. In reachability analysis, one computes the reach-avoid set, defined as the set of states from which the system can be driven to a target set while satisfying time-varying state constraints at all times. A major practical appeal of this approach stems from the availability of modern numerical tools, which can compute various definitions of reachable sets \cite{Sethian96, Osher02, Mitchell02, Mitchell07b}. These numerical tools, for example, have been successfully used to solve a variety of differential games, trajectory planning problems, and optimal control problems. Concrete practical applications include aircraft auto-landing \cite{Bayen07}, automated aerial refueling \cite{Ding08}, MPC control of quadrotors \cite{Bouffard12}, and multiplayer reach-avoid games \cite{Huang11}. Despite its power, the approach becomes numerically intractable as the state space dimension increases. In particular, reachable set computations involve solving a HJ partial differential equation (PDE) or variational inequality (VI) on a grid representing a discretization of the state space, resulting in an \textit{exponential} scaling of computational complexity with respect to the dimensionality of the problem. Therefore, as such, dynamic programming-based approaches such as reachability analysis are not suitable for managing the next generation airspace, which is a large-scale system with a high-dimensional joint state space because of the possible high density of vehicles that needs to be accommodated \cite{Kopardekar16}.  

To overcome this problem, the Sequential Path Planning (SPP) method has been proposed \cite{Chen15c}, in which vehicles are assigned a strict priority ordering. Higher-priority vehicles plan their trajectories without taking into account the lower-priority vehicles. Lower-priority vehicles treat higher-priority vehicles as moving obstacles. Under this assumption, time-varying formulations of reachability \cite{Bokanowski11, Fisac15} can be used to obtain the optimal and provably safe trajectories for each vehicle, starting from the highest-priority vehicle. Thus, the curse of dimensionality is overcome for the multi-vehicle trajectory planning problem at the cost of a structural assumption, under which the computation complexity scales just \textit{linearly} with the number of vehicles. In addition, a structure like this has the potential to flexibly divide up the airspace for the use of many UAVs and allows a tractable multi-vehicle trajectory-planning. In general, different economic mechanisms can be used to come up with a priority order. One example could be first-come-first-serve mechanism, as highlighted in NASA’s concept of operations for UAS traffic management \cite{Kopardekar16}. \SBnote{Mo, could you please read this paragraph again?}

The authors in \cite{Bansal2017} extend the SPP method to the scenarios where disturbances, such as wind, are present in the system, resolving some of the practical challenges associated with the basic SPP algorithm in \cite{Chen15c}. However, if a vehicle not in the set of SPP vehicles enters the system, or even worse, if this vehicle has malicious intent, the original plan can lead to vehicles entering into another vehicle’s danger zone. Thus, if vehicles do not plan with an additional safety margin that takes a potential intruder into account, a vehicle trying to avoid the intruder may effectively become an intruder itself, leading to a domino effect, causing the entire SPP structure to collapse. 

The authors in \cite{chen2016robust} propose an SPP algorithm that accounts for such a potential intruder. However, a new full-scale trajectory planning problem is required to be solved in real time to ensure safe transit of the vehicles to their respective destinations. Since the replanning must be done in real-time, the proposed algorithm is intractable for large-scale systems even with the SPP structure, rendering the method unsuitable for practical implementation in these cases. In this work, we propose a novel intruder avoidance algorithm, which will need to replan trajectories only for a \textit{fixed number of vehicles} if the intruder appears in the system, irrespective of the total number of SPP vehicles. Moreover, this number is a design parameter, which can be chosen beforehand based on the computational resources available for replanning during the run time, thus overcoming the limitations of the algorithm in \cite{chen2016robust}. 

Intuitively, for every vehicle, we compute a \textit{separation region} such that the vehicle needs to account for the intruder if and only if the intruder is inside this separation region. We then compute a \textit{buffer region} between the separation regions of any two vehicles, and ensure that this buffer is maintained as vehicles are traveling to their destinations. Thus, to intrude with any additional vehicle, the intruder will have to travel through this buffer region. In fact, we can affect this traveling time based on the size of the buffer region. Thus, for a given time duration, we can design the buffer region size such that the intruder can affect atmost a desirable number of vehicles. A high-level overview of the proposed algorithm is provided in Algorithm \ref{alg:basic_idea}.    
%
\begin{algorithm}[tb]
\SetKwInOut{Input}{input}
\SetKwInOut{Output}{output}
	\DontPrintSemicolon
	\caption{Overview of the proposed intruder avoidance algorithm (planning phase)}
	\label{alg:basic_idea}
	\Input{Set of vehicles $\veh_i, i = 1, \ldots, \N$ in the descending priority order;\newline
	Vehicle dynamics and initial states;\newline
	Vehicle destinations and any obstacles to avoid;\newline
	Intruder dynamics;\newline
	$\nva$: Maximum number of vehicles allowed to re-plan their trajectories.}
    \Output{Provably safe vehicle trajectories to respective destinations despite disturbance and intruder;\newline 
    Intruder avoidance and goal-satisfaction controller.}
	\For{\text{$i=1:N$}}{
			compute the separation region of $\veh_i$;\;
			compute the required buffer region based on $\nva$;\;
			use SPP algorithm for trajectory planning of $\veh_i$ such that the buffer region is maintained between $\veh_i$ and $\veh_j$ for all $j<i$;\;
			output the trajectory and optimal controller for $\veh_i$.\;
		}
\end{algorithm}
%

The rest of the paper is organized as follows: in Section \ref{sec:formulation}, we formally present the SPP problem in the presence of disturbances and adversarial intruders. In Section \ref{sec:background}, we present a brief review of time-varying reachability and the basic SPP algorithms proposed in \cite{Chen15c}, \cite{Bansal2017}. In Section \ref{sec:intruder}, we explain the proposed algorithm to account for intruders. Finally, we illustrate this algorithm through a fifty-vehicle simulation in an urban environment in Section \ref{sec:simulations}. All running notations in this paper are summarized in Table \ref{table:notation}.

% !TEX root = ../SPP_IoTjournal.tex
\section{Sequential Path Planning Problem \label{sec:formulation}}
Consider $\N$ vehicles $\veh_i, i = 1, \ldots, \N$ (also denoted as \textit{SPP vehicles}) which participate in the SPP process. We assume their dynamics are given by

\begin{equation}
\label{eq:dyn}
\begin{aligned}
\dot\state_i &= \fdyn_i(\state_i, \ctrl_i, \dstb_i), t \le \sta_i \\
\ctrl_i &\in \cset_i, \dstb_i \in \dset_i, i = 1 \ldots, \N
\end{aligned}
\end{equation}

\noindent where $\state_i \in \R^{n_i}$, $\ctrl_i \in \cset_i$ and $\dstb_i \in \dset_i$, respectively, represent the state, control and disturbance experienced by vehicle $\veh_i$. We partition the state $\state_i$ into the position component $\pos_i \in \R^{n_\pos}$ and the non-position component $\npos_i \in \R^{n_i - n_\pos}$: $\state_i = (\pos_i, \npos_i)$. %We assume that the control functions $\ctrl_i(\cdot), \dstb_i(\cdot)$ are drawn from the set of measurable functions\footnote{A function $f:X\to Y$ between two measurable spaces $(X,\Sigma_X)$ and $(Y,\Sigma_Y)$ is said to be measurable if the preimage of a measurable set in $Y$ is a measurable set in $X$, that is: $\forall V\in\Sigma_Y, f^{-1}(V)\in\Sigma_X$, with $\Sigma_X,\Sigma_Y$ $\sigma$-algebras on $X$,$Y$.}. For convenience, 
We will use the sets $\cfset_i, \dfset_i$ to respectively denote the set of functions from which the control and disturbance functions $\ctrl_i(\cdot), \dstb_i(\cdot)$ are drawn.

% We further assume that the flow field $\fdyn_i: \R^{n_i}\times\cset_i\times\dset_i \rightarrow \R^{n_i}$ is uniformly continuous, bounded, and Lipschitz continuous in $\state_i$ for fixed $\ctrl_i$ and $\dstb_i$. With this assumption, given $\ctrl_i(\cdot) \in \cfset_i, \dstb_i(\cdot) \in \dfset_i$, there exists a unique trajectory solving \eqref{eq:dyn} \cite{EarlA.Coddington1955}. %We will denote trajectories of \eqref{eq:dyn} starting from state $\state^0_i$ at time $t_0$ under control $\ctrl_i(\cdot)$ and disturbance $\dstb_i(\cdot)$ as $\traj_i(t; \state^0_i, t_0, \ctrl_i(\cdot))$. Trajectories satisfy an initial condition and the differential equation \eqref{eq:dyn} almost everywhere:

%\begin{equation}
%\begin{aligned}
%\frac{d}{dt}\traj_i(t; \state^0_i, t_0, \ctrl_i(\cdot)) &= \fdyn(\state^0_i, \ctrl_i, \dstb_i) \\
%\traj_i(t_0; \state^0_i, t_0, \ctrl_i(\cdot)) &= \state^0_i
%\end{aligned}
%\end{equation}

%In addition, we assume that the disturbances $\dstb_i(\cdot)$ are drawn the set of non-anticipative strategies \cite{Mitchell05} $\Gamma$, defined as follows:
%\begin{equation}
%\begin{aligned}
%& \Gamma := \{\mathcal{N}: \cfset_i \rightarrow \dfset_i:  \ctrl_i(r) = \hat{\ctrl}_i(r) \text{ a. e. } r\in[t,s] \\
%& \Rightarrow \mathcal{N}[\ctrl_i](r) = \mathcal{N}[\hat{\ctrl}_i](r) \text{ a. e. } r\in[t,s]\}
%\end{aligned}
%\end{equation}

Each vehicle $\veh_i$ has initial state $\state^0_i$, and aims to reach its target $\targetset_i$ by some scheduled time of arrival $\sta_i$. The target in general represents some set of desirable states, for example the destination of $\veh_i$. %For most of the paper we will make the assumption that $\edt_i$ is early enough for $\veh_i$ to feasibly get to $\targetset_i$ on time; this can be done by letting $\edt_i \rightarrow -\infty$. The assumption on $\edt_i$ is merely for convenience: in situations where $\edt_i$ is $-\infty$. In some situations, we may find that it is infeasible for $\veh_i$ to get to $\targetset_i$ at or before $\sta_i$. Whenever unsure, we may first determine the earliest feasible $\sta_i$ as described in Section \ref{sec:intruder}.
On its way to $\targetset_i$, $\veh_i$ must avoid a set of static obstacles $\soset_i \subset \R^{n_i}$. The interpretation of $\soset_i$ could be any set of states that are forbidden for each SPP vehicle such as a tall building. Each vehicle $\veh_i$ must also avoid the danger zones with respect to every other vehicle $\veh_j, j\neq i$. The danger zones in general can represent any joint configurations between $\veh_i$ and $\veh_j$ that are considered to be unsafe. We define the danger zone of $\veh_i$ with respect to $\veh_j$ to be
\begin{equation} \label{eqn:danger_zone_defn}
\dz_{ij} = \{(\state_i, \state_j): \|\pos_i - \pos_j\|_2 \le \rc\}
\end{equation}
\noindent whose interpretation is that $\veh_i$ and $\veh_j$ are considered to be in an unsafe configuration when they are within a distance of $\rc$ of each other. In particular, $\veh_i$ and $\veh_j$ are said to have collided, if $(\state_i, \state_j) \in \dz_{ij}$.

In addition to the obstacles and danger zones, an intruder vehicle can also appear in the system. An intruder vehicle may have malicious intent or it can be a non-participating vehicle, which does not have malicious intent, but can accidentally cause a collision with other vehicles since it may not follow the SPP structure. This general definition of intruder allows us to develop algorithms that can also account for vehicles who are not communicating with the SPP vehicles or do not know about the SPP structure. \SBnote{Mo, could you please read this paragraph again? Also, do you think we should expand on this idea of ``non-participating" vehicles in the introduction?}

In general, the effect of an intruder on the vehicles in structured flight can be entirely unpredictable, since the intruder in principle could be adversarial in nature, and the number of intruders could be arbitrary. In particular, if the number of intruders in the system is arbitrary, a collision avoidance problem must to be solved for each SPP vehicle in the joint state-space of all intruders and the vehicle, even with the SPP structure. Therefore, to make our analysis intractable, we make the following two assumptions: \SBnote{Mo, could you please read this paragraph again?}
\begin{assumption}
\label{as:avoidOnce}
At most one intruder (denoted as $\veh_I$ here on) affects the SPP vehicles at any given time. The intruder is removed after a duration of $\iat$. 
\end{assumption}    
This assumption can be valid in situations where intruders are rare, and that some fail-safe or enforcement mechanism exists to force the intruder out of the planning space affecting the SPP vehicles. For example, when SPP vehicles are flying at a particular altitude level, the removal of the intruder can be achieved by exiting the altitude level.
 
Let the time at which intruder appears in the system be $\tsa$ and the time at which it disappears be $\tea$. Assumption \ref{as:avoidOnce} implies that $\tea \leq \tsa + \iat$. Thus, any vehicle $\veh_i$ would need to avoid the intruder $\veh_{\intr}$ for a maximum duration of $\iat$. After a duration of $\iat$, the intruder is no longer present in the system. Note that we do not pose any restriction on $\tsa$; we only assume that once the intruder appears, it stays for a maximum duration of $\iat$.
\begin{assumption}
\label{as:dynKnown}
The dynamics of the intruder are known and given by $\dot\state_\intr = f_\intr(\state_\intr, \ctrl_\intr, \dstb_\intr)$.
\end{assumption}
Assumption \ref{as:dynKnown} is required for HJ reachability analysis. In situations where the dynamics of the intruder are not known exactly, a conservative model of the intruder may be used instead. We also denote the initial state of the intruder as $\state_{\intr}^0.$ Note that $\state_{\intr}^0$ is unknown.

Given the set of SPP vehicles, their targets $\targetset_i$, the static obstacles $\soset_i$, the vehicles' danger zones with respect to each other $\dz_{ij}$, and the intruder dynamics $f_\intr(\cdot)$, our goal is as follows. For each vehicle $\veh_i$, synthesize a controller which guarantees that $\veh_i$ reaches its target $\targetset_i$ at or before the scheduled time of arrival $\sta_i$, while avoiding the static obstacles $\soset_i$, the danger zones with respect to all other vehicles $\dz_{ij}, j\neq i$, and the intruder vehicle $\veh_{\intr}$, irrespective of the control strategy of the intruder. In addition, we would like to obtain the latest departure time $\ldt_i$ such that $\veh_i$ can still arrive at $\targetset_i$ on time.

In general, the above optimal path planning problem must be solved in the joint space of all $\N$ SPP vehicles and the intruder vehicle. However, due to the high joint dimensionality, a direct dynamic programming-based solution is intractable. Therefore, the authors in \cite{Chen15c} proposed to assign a priority to each vehicle, and perform SPP given the assigned priorities. Without loss of generality, let $\veh_j$ have a higher priority than $\veh_i$ if $j<i$. Under the SPP scheme, higher-priority vehicles can ignore the presence of lower-priority vehicles, and perform path planning without taking into account the lower-priority vehicles' danger zones. A lower-priority vehicle $\veh_i$, on the other hand, must ensure that it does not enter the danger zones of the higher-priority vehicles $\veh_j, j<i$ or the intruder vehicle $\veh_{\intr}$; each higher-priority vehicle $\veh_j$ induces a set of time-varying obstacles $\ioset_i^j(t)$, which represents the possible states of $\veh_i$ such that a collision between $\veh_i$ and $\veh_j$ or $\veh_i$ and $\veh_{\intr}$ could occur.

It is straight-forward to see that if each vehicle $\veh_i$ is able to plan a trajectory that takes it to $\targetset_i$ while avoiding the static obstacles $\soset_i$, the danger zones of \textit{higher-priority vehicles} $\veh_j, j<i$, and the danger zone of the \textit{intruder} $\veh_{\intr}$ irrespective of the intruder's control policy, then the set of SPP vehicles $\veh_i, i=1,\ldots,\N$ would all be able to reach their targets safely. Under the SPP scheme, path planning can be done sequentially in descending order of vehicle priority in the state space of only a single vehicle. Thus, SPP provides a solution whose complexity scales linearly with the number of vehicles, as opposed to exponentially, with a direct application of dynamic programming approaches. 

However, when an intruder appears in the system, depending on the initial state of the intruder and its control policy, a vehicle may arrive at different states after avoiding the intruder. Therefore, a control policy that ensures a successful transit to the destination needs to account for all such possible states, which is a path planning problem with multiple initial states and a single destination, and is hard to solve in general. Thus, we divide the intruder avoidance problem into two sub-problems: (i) we first design a control policy that ensures a successful transit to the destination if no intruder appears and that successfully avoid the intruder, if it does (Algorithm \ref{alg:basic_idea}). (ii) after the intruder disappears at $\tea$, we replan the trajectories of the affected vehicles. Following the same theme and assumptions, the authors in \cite{chen2016robust} present an algorithm to avoid an intruder in SPP formulation; however, in the worst-case, the algorithm might need to replan the trajectories for \textit{all} SPP vehicles. Since the replanning is done in real-time, the method in \cite{chen2016robust} is unsuitable for practical implementation for large multi-vehicle systems. Our goal in this work is to present an algorithm that ensures that only a \textit{small and fixed} number of vehicles need to replan their trajectories, regardless of the total number of vehicles. Thus, the replanning time is constant and can be done in real time. In particular, we answer the following inter-dependent questions:
\begin{enumerate}
\item How can each vehicle guarantee that it will reach its target set without getting into any danger zones, despite no knowledge of the intruder initial state, the time at which it appears, and its control strategy?
\item How can we ensure that replanning only needs to be done for at most a fixed maximum number of vehicles after the intruder disappears from the system? \label{question2}
\item Can we choose the maximum number of the vehicles in question \ref{question2} above?
\end{enumerate}

\begin{table*}
    \caption{Different mathematical notations and their interpretation (in the alphabetical order of symbols).}    
    \resizebox{\hsize}{!}{
    \begin{tabular}{ |>{\centering\arraybackslash}m{1.5cm}| m{5cm} | m{3cm} | m{\columnwidth} |}
 %   \begin{tabular}{ | l | l | p{11cm} |}
    \hline
    \textbf{Notation} & \textbf{Description} & \textbf{Location} & \textbf{Interpretation} \\ \hline
    
    %%% Letter D %%% 
    % Disturbance
    $\dstb_i$ & Disturbance in the dynamics of vehicle $i$ & Beginning of Section \ref{sec:formulation} & -    \\ \hline   
    $\dstb_{\intr}$ & Disturbance in the dynamics of the intruder & Assumption \ref{as:dynKnown} & -    \\ \hline   
    
    %%% Letter F %%%
    % Dynamics
    $\fdyn_i$ & Dynamics of vehicle $i$ & Beginning of Section \ref{sec:formulation} & -    \\ \hline
    $f_\intr$ & Dynamics of the intruder & Assumption \ref{as:dynKnown} & -    \\ \hline
    $f_r$ & Relatiove dynamics between two vehicles & Equation \eqref{eq:reldyn} & - \\ \hline
    
    %%% Letter G %%%
    % Obstacle
    $\obsset_i(t)$ & The overall obstacle for vehicle $i$ & Equation \eqref{eq:obsseti} & The set of states that vehicle $i$ must avoid on its way to the destination. \\ \hline
    
    %%% Letter H %%%
    % Non-position state components of the vehicle
    $\npos_i$ & Non-position state component of vehicle $i$ & Beginning of Section \ref{sec:formulation} & -    \\ \hline
    
    %%% Letter K %%%
    % Number of vehicles to avoid
    $\nva$ & - & Beginning of Section \ref{sec:intruder} & The maximum number of vehicles that should apply the avoidnace maneuver or the maximum number of vehicles that we can replan trajectories for in real-time.    \\ \hline    
    
    %%% Letter L %%%
    % Target set
    $\targetset_i$ & Target set of vehicle $i$ & Beginning of Section \ref{sec:formulation} & The destination of vehicle $i$.    \\ \hline
    
    %%% Letter M %%%    
    % Base obstacles
    $\boset_j(t)$ & Base obstacle induced by vehicle $j$ at time $t$ & Equations (25), (31) and (37) in \cite{chen2016robust} & The set of all possible states that vehicle $j$ can be in at time $t$ if the intruder does not appear in the system till time $t$. \\ \hline    
    
    %%% Letter N %%%
    % Number of vehicles
    $\N$ & Number of SPP vehicles & Beginning of Section \ref{sec:formulation} & -    \\ \hline
    
    %%% Letter O %%%
    % Obstacles
	$\ioset_i^j(t)$ & Induced obstacle by vehicle $j$ for vehicle $i$ & After Assumption \ref{as:dynKnown} in Section \ref{sec:formulation} & The possible states of vehicle $i$ such that a collision between vehicle $i$ and vehicle $j$ or vehicle $i$ and the intruder vehicle (if present) could occur.    \\ \hline 
	% Static Obstacles   
    $\soset_i$ & Static obstacle for vehicle $i$ & Beginning of Section \ref{sec:formulation} & Obstacles that vehicle $i$ needs to avoid on its way to destination, e.g, tall buildings. \\ \hline    
    
    %%% Letter P %%%
    % Position state components of the vehicle
    $\pos_i$ & Position of vehicle $i$ & Beginning of Section \ref{sec:formulation} & -    \\ \hline
     
    %%% Letter Q %%%
    % SPP vehicle
    $\veh_{i}$ & $i$th SPP vehicle & Beginning of Section \ref{sec:formulation} & -  \\ \hline
    % Intruder vehicle
    $\veh_{\intr}$ & The intruder vehicle & Assumption \ref{as:avoidOnce} & -  \\ \hline
    
    %%% Letter R %%%
    % Unsafe distance
    $\rc$ & Danger zone radius & Equation \eqref{eqn:danger_zone_defn} & The closest distance between vehicle $i$ and vehicle $j$ that is considered to be safe. \\ \hline  
    
    %%% Letter S %%%
    % Separation Region
    $\sep_j(t)$ & Separation region of vehicle $j$ at time $t$ & Beginning of Section \ref{sec:sepRegion_case1} & The set of all states of intruder at time $t$ for which vehicle $j$ is forced to apply an avoidance maneuver. \\ \hline        
    
    %%% Letter T %%%
    % Avoid start time
    $\tsa_i$ & Avoid start time of vehicle $i$ & Equation \eqref{eqn:avoidStartTime2} & The first time at which vehicle $i$ is forced to apply an avoidance maneuver by the intruder vehicle. Defined to be $\infty$ if vehicle $i$ never applies an avoidance maneuver.\\ \hline    
    % Buffer region duration
    $\brd$ & Buffer region travel duration & Beginning of Section \ref{sec:intruder} & The minimum time required for the intruder to travel through the buffer region between any pair of vehicles. \\ \hline
    % Intruder avoidance time
    $\iat$ & Intruder avoidance time & Assumption \ref{as:avoidOnce} & The maximum duration for which the intruder is present in the system. \\ \hline
    % Intruder appearance time
    $\tsa$ & Intruder appearance time & After Assumption \ref{as:avoidOnce} & The time at which the intruder appears in the system. \\ \hline
    % Intruder avoidance time
    $\tea$ & Intruder disappearance time & After Assumption \ref{as:avoidOnce} & The time at which the intruder disappears from the system. \\ \hline
    % Latest Departure Time
    $\ldt_i$ & Latest departure time of vehicle $i$ & End of Section \ref{sec:formulation} & The latest departure time for vehicle $i$ such that it safely reaches its destination by the scheduled time of arrival. \\ \hline
    % Scheduled time of arrival
    $\sta_i$ & Scheduled time of arrival (STA) of vehicle $i$ & Beginning of Section \ref{sec:formulation} & The time by which vehicle $i$ is required to reach its destination.    \\ \hline
    
 
	%%% Letter U %%% 
    % control
    $\ctrl_i$ & Control of vehicle $i$ & Beginning of Section \ref{sec:formulation} & -    \\ \hline   
    $\ctrl_{\intr}$ & Control of the intruder & Assumption \ref{as:dynKnown} & -    \\ \hline    
    ${\ctrl^{\text{A}}_{i}}$ & Optimal avoidance control of vehicle $i$ & Equation \eqref{eqn:optAvoidCtrl} & The control that vehicle $i$ need to apply to successfully avoid the intruder once the relative state between vehicle $i$ and the intruder reaches the boundary of the avoid region of vehicle $i$.  \\ \hline    

    %%% Letter V %%%
    % Avoid region
    $\brs^{\text{A}}_{i}(\tau, \iat)$ & Avoid region of vehicle $i$ & Equation \eqref{eqn:avoidBRS} & The set of relative states $\state_{\intr i}$ for which the intruder can force vehicle $i$ to enter in the danger zone $\dz_{i \intr}$ within a duration of $(\iat-\tau)$. \\ \hline

 	%%% Letter X %%%    
    % State of the vehicle
    $\state_i$ & State of vehicle $i$ & Beginning of Section \ref{sec:formulation} & - \\ \hline
    % State of the intruder
    $\state_{\intr}$ & State of the intruder vehicle & Assumption \ref{as:dynKnown} & - \\ \hline
    % State of the vehicle
    $\state_i^0$ & Initial state of vehicle $i$ & Beginning of Section \ref{sec:formulation} & - \\ \hline
    % State of the intruder
    $\state_{\intr}^0$ & Initial state of the intruder vehicle & Assumption \ref{as:dynKnown} & - \\ \hline
    % Relative State
    $\state_{\intr i}$ & Relative state between the intruder and vehicle $i$ & Equation \eqref{eq:reldyn} & - \\ \hline
    
   
    %%% Letter Z %%%    
    % Danger Zone
    $\dz_{ij}$ & Danger zone between vehicle $i$ and vehicle $j$ & Equation \eqref{eqn:danger_zone_defn} & Set of all states of vehicle $i$ and vehicle $j$ which are within unsafe distance of each other. The vehicles are said to have collided if their states belong to $\dz_{ij}$. \\ \hline
    
    \end{tabular}
    }
    \label{table:notation}
\end{table*}


% !TEX root = SPP_journal.tex
\section{Double-Obstacle Hamilton-Jacobi Variational Inequality \label{sec:HJIVI}}
Notation:
\begin{itemize}
\item Backwards reachable set from target set $\targetset$: $\brs(t, \targetset)$ (second argument is target set)
\end{itemize}
% !TEX root = SPP_journal.tex
\section{The Basic SPP Algorithm\label{sec:basic}}
\MCnote{Need ``optimal'' example from SPP1 paper initial submission?}
In this section, we introduce the basic SPP algorithm assuming that there is no disturbance affecting the vehicles, and that each vehicle has the exact position of higher-priority vehicles. Although in practice, such assumptions do not hold, the basic SPP algorithm can serve as a useful approximation in certain situations. In addition, the description of the basic SPP algorithm will introduce the notation needed for describing the subsequent, more realistic versions of SPP. We also show simulation results for the basic SPP algorithm. The majority of the content in this section is taken from \cite{Chen15}.

\subsection{SPP Without Disturbances and With Perfect Information}
Recall that the SPP vehicles $\veh_i, i=1,\ldots,N$, are each assigned a strict priority, with $\veh_j$ having a higher priority than $\veh_i$ if $j<i$. In the absence of disturbances, we can write the dynamics the SPP vehicles as

\begin{equation}
\label{eq:dyn_no_dstb}
\begin{aligned}
\dot\state_i &= \fdyn_i(\state_i, \ctrl_i), t \in [\edt, \sta] \\
\ctrl_i &\in \cset_i, \qquad i = 1 \ldots, \N
\end{aligned}
\end{equation}

\noindent with trajectories denoted by $\traj_i(s; \state^0_i, \ldt, \ctrl_(\cdot))$.

In SPP, each vehicle $\veh_i$ plans the path to its target set $\targetset_i$ while avoiding static obstacles $\soset$ and the obstacles $\ioset_i^j(t)$ induced by higher priority vehicles $\veh_j, j<i$. Path planning is done sequentially starting from the first vehicle and proceeding in descending priority, $\veh{1}, \veh{2}, \ldots, \veh{\N}$ so that each of the path planning problems can be done in the state space of only one vehicle. In its path planning process, $\veh_i$ ignores the presence of the lower priority vehicles $\veh{k}, k>i$, and induces the obstacles $\ioset_k^i(t), k>i$.

From the perspective of $\veh_i$, each of the higher priority vehicles $\veh_j, j<i$ induces a time-varying obstacle denoted $\ioset_i^j(t)$. Therefore, each vehicle $\veh_i$ must plan its path to $\targetset_i$ while avoiding the union of all the induced obstacles as well as the static obstacles. Let $\obsset_i(t)$ be the union of all the obstacles that $\veh_i$ must avoid on its way to $\targetset_i$:

\begin{equation}
\label{eq:obsseti}
\obsset_i(t)  = \soset \cup \bigcup_{j=1}^{i-1} \ioset_i^j(t)
\end{equation}

With full position information of higher priority vehicles, the obstacle induced for $\veh_i$ by $\veh_j$ is simply

\begin{equation}
\label{eq:ioset}
\ioset_i^j(t) = \{\state_i: (\state_i, \state_j(t)) \in \dz_{ij}\}
\end{equation}

Each higher priority vehicle $\veh_j$ plans its path while ignoring $\veh_i$. Since path planning is done sequentially in descending order or priority, the vehicles $\veh_j, j<i$ would have planned their paths before $\veh_i$ does. Thus, in the absence of disturbances, $\pos_j(t)$ is \textit{a priori} known, and therefore $\ioset_i^j(t), j<i$ are known, deterministic moving obstacles, which means that $\obsset_i(t)$ is also known and deterministic. Therefore, the path planning problem for $\veh_i$ can be solved by first computing the BRS $\brs_i(t)$, defined as follows:

\begin{equation}
\label{eq:BRS_basic}
\begin{aligned}
&\brs_i(t) = \{\state_i: \exists \ctrl_i(\cdot) \in \cfset_i, \\
&\quad\forall s \in [t, \sta_i], \traj_i(s; \state^0_i, \ldt, \ctrl_(\cdot)) \notin \obsset_i(s), \\
&\quad\exists s \in [t, \sta_i], \traj_i(s; \state^0_i, \ldt, \ctrl_(\cdot)) \in \targetset_i\}
\end{aligned}
\end{equation}

The BRS $\brs(t)$ can be obtained by solving \eqref{eq:HJIVI_BRS} with $\targetset = \targetset_i$, $\obsset = \obsset_i$, and the Hamiltonian 

\begin{equation}
\label{eq:basicham}
\ham_i(t, \state_i, p) = \min_{\ctrl_i} p \cdot \fdyn_i(\state_i, \ctrl_i)
\end{equation}

The optimal control for reaching $\targetset_i$ while avoiding $\obsset_i(t)$ is then given by

\begin{equation}
\label{eq:basicOptCtrl}
\ctrl_i^*(t) = \arg \min_{\ctrl_i} p \cdot \fdyn_i(\state_i, \ctrl_i)
\end{equation}

In summary, the basic SPP algorithm is given as follows:

\begin{alg}
\label{alg:basic}
\textbf{Basic SPP algorithm}: Given initial conditions $\state_i^0$, vehicle dynamics \eqref{eq:dyn_no_dstb}, target set $\targetset_i$, and static obstacles $\soset_i, i = 1\ldots, \N$, for each $i$ in ascending order starting from $i=1$ (which corresponds to descending order or priority),
\begin{enumerate}
\item Determine the total obstacle set $\obsset_i(t)$, given in \eqref{eq:obsseti}. In the case $i=1$, $\obsset_i(t) = \soset_i ~ \forall t$.
\item Compute the BRS $\brs_i(t)$ defined in \eqref{eq:BRS_basic}. The latest departure time $\ldt_i$ is then given by $\arg \sup_t \state^0_i \in \brs_i(t)$.
\item Determine the trajectory $\traj_i(\cdot; \state_i^0, \ldt, \ctrl^*(\cdot))$ using vehicle dynamics \eqref{eq:dyn_no_dstb}, with the optimal control  $\ctrl_i^*(\cdot)$ given by \eqref{eq:basicOptCtrl}.
\item Given $\traj_i(\dot; \state_i^0, \ldt, \ctrl^*(\cdot))$, compute the induced obstacles $\ioset_k^i(t)$ for each $k>i$. In the absence of disturbances, $\ioset_k^i(t)$ is given by \eqref{eq:ioset} with $\state_i(t) = \traj_i(t; \state_i^0, \ldt, \ctrl^*(\cdot))$.
\end{enumerate}
\end{alg}
% !TEX root = SPP_journal.tex
\subsection{Basic SPP Numerical Results \label{sec:basic_results}}
\SBnote{There are a few things that we need to work on to make things consistent across the paper:
\begin{itemize}
\item Make sure the obstacle sets are defined in the similar fashion everywhere (currently it is different across the basic SPP method and SPP with disturbance method.)
\item Let's use $\lambda$ for co-state everywhere. Also, let's explain what $\lambda$ is the very first time we introduce this notation.
\item Let's use $x$ for state everywhere in the paper.
\item Do we really need $t_i^{EDT}$ in this paper. We are not using it anywhere. We can just use the notation that we used in the SPP paper.
\item Once we write the FRS and BRS definitions, we need to make sure that we are using those notations in all the following sections.
\end{itemize}}

We now illustrate the basic SPP algorithm using a four-vehicle example. In this example, we will use the following dynamics for each vehicle:

\begin{equation}
\begin{aligned}
\dot x_i &= v_i \cos(\theta_i) \\
\dot y_i &= v_i \sin(\theta_i) \\
\dot \theta_i &= \omega_i \\
\state_i(\edt_i) &= \state_i^0 = (x_i^0, y_i^0, \theta_i^0) \\
i &= 1, \ldots, \N
\end{aligned}
\end{equation}

\noindent where $\state_i = (x_i, y_i, \theta_i)$ is the state of vehicle $\veh{i}$, $\pos_i = (x_i, y_i)$ is the position of vehicle $\veh{i}$, $\theta_i$ is the heading of vehicle $\veh{i}$, $v_i$ is the speed of vehicle $\veh{i}$, and $\omega_i$ is the turn rate of vehicle $\veh{i}$. We have chosen these dynamics for clarity of illustration; in general, the SPP algorithm can handle more general systems of the form in which the vehicles have different control bounds and dynamics. 

In this particular example, we assume that the vehicles have constant speed $v_i = 1 ~ \forall i$, and that the control of each vehicle $\veh{i}$ is given by $\ctrl_i = \omega_i$ with $|\omega_i| \le \bar\omega = 1 ~ \forall i$. 

For this example, the target sets $\targetset_i$ of the vehicles are circles of radius $r$ in the position space; each vehicle is trying to reach some desired set of positions. In terms of the state space $\state_i$, the target set is defined as

\begin{equation}
\label{eq:target_sim}
\targetset_i = \{z_i: \dist(p_i, c_i) \le r\}
\end{equation}

\noindent where $c_i$ are centers of the target circles. For the simulation of the basic SPP algorithm, we used r = 0.1.

The vehicles have target centers $c_i$, initial conditions $\state_i^0$, and scheduled times of arrivals $\sta_i$ as follows:

\begin{equation}
\begin{aligned}
c_1 = (0.7, 0.2), \quad& \state_1^0 = (-0.5, 0, 0), \quad & \sta_1 = 0 \\
c_2 = (-0.7, 0.2), \quad& \state_2^0 = (0.5, 0, \pi), \quad & \sta_2 = 0.2 \\
c_3 = (0.7, -0.7), \quad& \state_3^0 = \left(-0.6, 0.6, 7\pi/4\right), \quad & \sta_3 = 0.4\\
c_4 = (-0.7, -0.7), \quad & \state_4^0 = \left(0.6, 0.6, 5\pi/4\right), \quad & \sta_4 = 0.6
\end{aligned}
\end{equation}

The setup for this example is shown in Fig. \ref{fig:dubins_ic}, which also shows the static obstacles as the black rectangles around the center of the domain.

The joint state space of this four-vehicle system is twelve-dimensional (12D), making the joint path planning and collision avoidance problem intractable for direct analysis. Therefore, we apply the SPP algorithm described in Algorithm \ref{alg:basic} and repeatedly solve the double-obstacle HJ VI in \eqref{eq:HJIVI_BRS} to obtain the optimal control for each vehicle to reach its target while avoiding higher-priority vehicles. In addition, due to the flexibility of the HJ VI with respect to time-varying systems, the different scheduled times of arrival $\sta_i$ can be trivially incorporated. 

Fig. \ref{fig:dubins_reach_all}, \ref{fig:dubins_reach_3}, and \ref{fig:dubins_result} show the simulation results. Since the state space of each vehicle is 3D, each of the BRSs $\brs(t, \targetset_i, \obsset_i(t), \basicham)$ is also 3D. To visualize the results, we slice the BRSs at the initial heading angles $\theta_i^0$. Fig. \ref{fig:dubins_reach_all} shows the 2D BRS slices for each vehicle at its latest departure times $\ldt_1=-1.12, \ldt_2=-0.94,\ldt_3=-1.48,\ldt_4=-1.44$ determined from our method. The obstacles in the domain $\sosetp$ and the obstacles induced by other vehicles $\ioset_i^j(t)$ inhibit the evolution of the BRSs, carving out thin ``channels" that separate the BRSs into different ``islands". One can see how these channels and islands form by examining the time evolution of the BRS, shown in Figure \ref{fig:dubins_reach_3} for vehicle $\veh_3$. 

Finally, Fig. \ref{fig:dubins_result} shows the resulting trajectories of the four vehicles. Most interestingly, the subplot labeled $t=-0.55$ shows all four vehicles in close proximity without collision: each vehicle is outside of the danger zone of all other vehicles (although the danger zones may overlap). This close proximity is an indication of the optimality of the basic SPP algorithm given the assigned priority ordering. Since no disturbances are present, getting as close to other vehicles' danger zones as possible without entering the danger zones intuitively results in short transit times.

The actual arrival times of vehicles $i=1,2,3,4$ are $0, 0.19, 0.34, 0.31$, respectively. It is interesting to note that for some vehicles, the actual arrival times are earlier than the scheduled times of arrivals $\sta_i, i=1,2,3,4$. This is because in order to arrive at the target by $\sta_i$, these vehicles must depart early enough to avoid major delays resulting from the induced obstacles of other vehicles; these delays would have lead to a late arrival if vehicle $i$ departed after $\sta_i$.

\begin{figure}
	\centering
	\includegraphics[width=0.5\textwidth]{"fig/dubins_ic"}
	\caption{Initial configuration of the four-vehicle example.}
	\label{fig:dubins_ic}
\end{figure}

\begin{figure}
	\centering
	\includegraphics[width=0.5\textwidth]{"fig/dubins_reach_all"}
	\caption{BRSs at $t=\ldt_i$ for vehicles $1,2,3,4$, sliced at initial headings $\theta_i^0$. Black arrows indicate direction of obstacle motion. Due to the turn rate constraint, the presence of static obstacles $\sosetp$ and time-varying obstacles induced by higher-priority vehicles $\ioset_i^j(t)$ carve ``channels" in the BRS, dividing it up into multiple ``islands".}
	\label{fig:dubins_reach_all}
\end{figure}

\begin{figure}
	\centering
	\includegraphics[width=0.5\textwidth]{"fig/dubins_reach_3"}
	\caption{Time evolution of the BRS for vehicle $\veh_3$, sliced at its initial heading $\theta_3^0=\frac{7\pi}{4}$. Black arrows indicate direction of obstacle motion. Top row: the BRS grows unobstructed by obstacles. Bottom row: the static obstacles $\sosetp$ and the induced obstacles $\ioset_3^1,\ioset_3^2$, carve out ``channels" in the BRS.}
	\label{fig:dubins_reach_3}
\end{figure}

\begin{figure}
	\centering
	\includegraphics[width=0.5\textwidth]{"fig/dubins_result"}
	\caption{The planned trajectories of the four vehicles. Top left: only vehicles $\veh_3$ (green) and $\veh_4$ (purple) have started moving, showing $\ldt_i$ is not common across the vehicles. Top right: all vehicles have come within very close proximity, but none is in the danger zone another. Bottom left: vehicle $\veh_1$ (blue) arrives at $\targetset_1$ at $t=0$. Bottom right: all vehicles have reached their destination, some ahead of the STA $\sta_i$.}
	\label{fig:dubins_result}
\end{figure}

% Disturbance files
% !TEX root = ../SPP_journal.tex
\section{SPP With Disturbances and Incomplete Information \label{sec:incomp}}
Disturbances and incomplete information significantly complicate the SPP scheme. The main difference is that the vehicle dynamics satisfy \eqref{eq:dyn} as opposed to \eqref{eq:dyn_no_dstb}. Committing to exact trajectories is therefore no longer possible, since the disturbance $\dstb_i(\cdot)$ is \textit{a priori} unknown. Thus, the induced obstacles $\ioset_i^j(t)$ are no longer just the danger zones centered around positions. 

\subsection{Theory}
We present three methods to address the above issues. The methods differ in terms of control policy information that is known to a lower-priority vehicle, and have their relative advantages and disadvantages depending on the situation. The three methods are as follows:
\begin{itemize}
\item \textbf{Centralized control}: A specific control strategy is enforced upon a vehicle; this can be achieved, for example, by some central agent such as an air traffic controller.
\item \textbf{Least restrictive control}: A vehicle is required to arrive at its target on time, but has no other restrictions on its control policy. When the control policy of a vehicle is unknown, but its timely arrive at its target can be assumed, the least restrictive control can be safely assumed by lower-priority vehicles.
\item \textbf{Robust trajectory tracking}: A vehicle declares a nominal trajectory which can be robustly tracked under disturbances.
\end{itemize}

In general, the above methods can be used in combination in a single path planning problem, with each vehicle independently having different control policies. Lower-priority vehicles would then plan their paths while taking into account the control policy information known for each higher-priority vehicle. For clarity, we will present each method as if all vehicles are using the same method of path planning.

In addition, for simplicity of explanation, we will assume that no static obstacles exist. In the situations where static obstacles do exist, the time-varying obstacles $\obsset_i(t)$ simply become the union of the induced obstacles $\ioset_i^j(t)$ in \eqref{eq:ioset} and the static obstacles. The material in this section is taken partially from \cite{Bansal2017}.

\subsubsection{Centralized Control\label{sec:cc}}
The highest-priority vehicle $\veh_1$ first plans its path by computing the BRS (with $i=1$)
\begin{equation}
\label{eq:BRS}
\begin{aligned}
\brs_i^\text{dstb}(t, \sta_i) = & \{y: \exists \ctrl_i(\cdot) \in \cfset_i, \forall \dstb_i(\cdot) \in \dfset_i, \state_i(\cdot) \text{ satisfies \eqref{eq:dyn}},\\
& \forall s \in [t, \sta_i], \state_i(s) \notin \obsset_i(s), \state_i(t) = y\\
& \exists s \in [t, \sta_i], \state_i(s) \in \targetset_i\}
\end{aligned}
\end{equation}

Since we have assumed no static obstacles exist, we have that for $\veh_1, \obsset_1(s)=\emptyset ~ \forall s \le \sta_i$, and thus the above BRS is well-defined. This BRS can be computed by solving the HJ VI \eqref{eq:HJIVI_BRS} with the following Hamiltonian:

\begin{equation}
\ham_i^\text{dstb}\left(\state_i, \costate\right) = \min_{\ctrl_i \in \cset_i} \max_{\dstb_i \in \dset_i} \costate \cdot \fdyn_i(\state_i, \ctrl_i, \dstb_i)
\end{equation}

From the BRS, we can obtain the optimal control

\begin{equation}
\label{eq:opt_ctrl_i}
\ctrl_i^\text{dstb}(t,\state_i) =  \arg \min_{\ctrl_i \in \cset_i} \max_{\dstb_i \in \dset_i} \costate \cdot \fdyn_i(\state_i, \ctrl_i, \dstb_i)
\end{equation}

Here, as well as in the other two methods, the latest departure time $\ldt_i$ is then given by $\arg \sup_t \state_{i}^0 \in \brs_i^\text{dstb}(t, \sta_i)$.

If there is a central agent directly controlling each of the $N$ vehicles, then the control law of each vehicle can be enforced. In this case, lower-priority vehicles can safely assume that higher-priority vehicles are applying the enforced control law. In particular, the optimal controller for getting to the target, $\ctrl^\text{dstb}_i(t, \state_i)$, can be enforced. In this case, the dynamics of each vehicle becomes 

\begin{equation}
\label{eq:dyn_cc}
\begin{aligned}
\dot \state_i &= \fdyn^\text{cc}_i (t, \state_i, \dstb_i) = \fdyn_i(\state_i, \ctrl^\text{dstb}_i(t,\state_i), \dstb_i) \\
\dstb_i &\in \dset_i, \quad i = 1,\ldots, N, \quad t \in [\ldt_i, \sta_i]
\end{aligned}
\end{equation}

\noindent where $\ctrl_i$ no longer appears explicitly in the dynamics.

From the perspective of a lower-priority vehicle $\veh_i$, a higher-priority vehicle $\veh_j, j < i$ induces a time-varying obstacle that represents the positions that could possibly be within the collision radius $\rc$ of $\veh_j$ under the dynamics $\fdyn^\text{cc}_j(t, \state_j, \dstb_j)$. Determining this obstacle involves computing an FRS of $\veh_j$ starting from\footnote{In practice, we define the target set to be a small region around the vehicle's initial state for computational reasons.} $\state_j(\ldt_j) = \state_{j}^0$. The FRS $\frs_j^\text{cc}(\ldt_j, t)$ is defined as follows:

\begin{equation}
\label{eq:FRS1}
\begin{aligned}
\frs_j^\text{cc}(\ldt_j, t) = & \{y: \exists \dstb_j(\cdot) \in \dfset_j, \state_j(\cdot) \text{ satisfies \eqref{eq:dyn_cc}},\\
& \state_j(\ldt_j) = \state_{j}^0, \state_j(t) = y\}.
\end{aligned}
\end{equation}

This FRS can be computed using \eqref{eq:HJIVI_FRS} with the Hamiltonian

\begin{equation}
\ham_j^\text{cc}\left(t, \state_j, \costate\right) = \max_{\dstb_j \in \dset_j} \costate \cdot f^\text{cc}_j(t, \state_j, \dstb_j)
\end{equation}

The FRS $\frs_j^\text{cc}(\ldt_j, t)$ represents the set of possible states at time $t$ of a higher-priority vehicle $\veh_j$ given all possible disturbances $\dstb_j(\cdot)$ and given that $\veh_j$ uses the feedback controller $\ctrl_j^\text{dstb}(t, \state_j)$. In order for a lower-priority vehicle $\veh_i$ to guarantee that it does not go within a distance of $\rc$ to $\veh_j$, $\veh_i$ must stay a distance of at least $\rc$ away from the FRS $\frs_j^\text{cc}(\ldt_j, t)$ for all possible values of the non-position states $\npos_j$. This gives the obstacle induced by a higher-priority vehicle $\veh_j$ for a lower-priority vehicle $\veh_i$ as follows:

\begin{equation} \label{eqn:ccObs}
\ioset_i^j(t) = \{\state_i: \exists y \in \pfrs_j(t), \|\pos_i - y\|_2 \le \rc \}
\end{equation}

\noindent where the set $\pfrs_j(t)$ is the set of states in the FRS $\frs_j^\text{cc}(\ldt_j, t)$ projected onto the states representing position $\pos_j$, and disregarding the non-position dimensions $\npos_j$:

\begin{align} 
\pfrs_j(t) & = \{p_j: \exists \npos_j, (p_j, \npos_j) \in \boset_j(t) \}, \label{eqn:ccObs_help1}\\
\boset_j(t) & = \frs_j^\text{cc}(\ldt_j, t). \label{eqn:ccObs_help2}
\end{align}

Finally, taking the union of the induced obstacles $\ioset_i^j(t)$ as in \eqref{eq:obsseti} gives us the time-varying obstacles $\obsset_i(t)$ needed to define and determine the BRS $\brs_i^\text{dstb}(t, \sta_i)$ in \eqref{eq:BRS}. Repeating this process, all vehicles will be able to plan paths that guarantee the vehicles' timely and safe arrival. The centralized control algorithm can be summarized as follows:
\begin{alg}
\label{alg:cc}
\textbf{Centralized control algorithm}: Given initial conditions $\state_i^0$, vehicle dynamics \eqref{eq:dyn}, target set $\targetset_i$, and static obstacles $\soset_i, i = 1\ldots, \N$, for each $i$,
\begin{enumerate}
\item determine the total obstacle set $\obsset_i(t)$, given in \eqref{eq:obsseti}. In the case $i=1$, $\obsset_i(t) = \soset_i ~ \forall t$;
\item compute the BRS $\brs_i^\text{dstb}(t, \sta_i)$ defined in \eqref{eq:BRS}. The latest departure time $\ldt_i$ is then given by $\arg \sup_t \state^0_i \in \brs_i^\text{dstb}(t, \sta_i)$;
\item compute the optimal control $\ctrl_i^\text{dstb}(t,\state_i)$ corresponding to $\brs_i^\text{dstb}(t, \sta_i)$ given by \eqref{eq:opt_ctrl_i}. Given $\ctrl_i^\text{dstb}(t,\state_i)$, compute the FRS $\frs_i^\text{cc}(\ldt_i, t)$ in \eqref{eq:FRS1};
\item finally, compute the induced obstacles $\ioset_k^i(t)$ for each $k>i$. In the centralized control method, $\ioset_k^i(t)$ is computed using \eqref{eqn:ccObs} where $\pfrs_i(t)$ is given by \eqref{eqn:ccObs_help1}.
\end{enumerate}
\end{alg}
% !TEX root = SPP2.tex
\subsection{Method 2: Least Restrictive Control \label{sec:lrc}}
If there is no centralized controller to enforce the control policy for higher priority vehicles, weaker assumptions must be made by the lower priority vehicles to ensure collision avoidance. One reasonable assumption that a lower priority vehicle can make is that all higher priority vehicles follow the least restrictive control that would take them to their targets. This control would be given by 

\begin{equation}
\label{eq:lrctrl} % least restrictive control
u_j(t, x_j)\in \begin{cases} \{u_j^*(t, x_j) \text{ given by } \eqref{eq:opt_ctrl_i}\} \text{ if } x_j\in \partial \brs_j(t), \\
\cset_i  \text{ otherwise}
\end{cases}
\end{equation}

Such a controller allows each higher priority vehicle to use any controller it desires, except when it is on the boundary of the BRS, $\partial \brs_j$, in which case the optimal control $u_j^*(t, x_j)$ given by \eqref{eq:opt_ctrl_i} must be used to get to the target on time. This assumption is the weakest assumption that could be made by lower priority vehicles given that the higher priority vehicles will get to their targets on time.

Suppose a lower priority vehicle $\veh_i$ assumes that higher priority vehicles $\veh_j, j < i$ use the least restrictive control strategy \eqref{eq:lrctrl}. From the perspective of the lower priority vehicle $\veh_i$, a higher priority vehicle $\veh_j$ could be in any state that is reachable from $\veh_j$'s initial state $x_j(\edt)$ and from which the target $\targetset_j$ can be reached. Mathematically, this is defined by $\veh_j$ is the intersection of the FRS from the initial state $x_j(\edt)$ and the BRS defined in \eqref{eq:brs} from the target set $\targetset_j$, $\brs_j(t) \cap \frs_j(t)$. In this situation, since $\veh_j$ cannot be assumed to be using any particular feedback control, $\frs_j(t)$ is defined in \eqref{eq:frs2} and can also be computed by solving \eqref{eq:FRS_j}.

\SBnote{traj.-based definition}
\begin{equation}
\label{eq:frs2}
\begin{aligned}
&\frs_j(t) = \{y \in \R^{n_j}: \exists u \in \cfset, \exists d \in \dfset, \\
& \quad \dot{x}_j = f_j(x_j, u_j, d_j) \Rightarrow, x_j(t) = y\}
\end{aligned}
\end{equation}

In turn, the obstacle induced by a higher priority $\veh_j$ for a lower priority vehicle $\veh_i$ is as follows:

\begin{equation}
\ioset_i^j(t) = \{x_i: \dist(\pos_i, \pfrs_j(t)) \le \cradius \}
\end{equation}

\noindent where $\pfrs_j(t)$ is given by

\begin{equation}
\pfrs_j(t) = \{p: \exists \npos_j, (p, \npos_j) \in \brs_j(t) \cap \frs_j(t)\}
\end{equation}
% !TEX root = ../STP_IoTjournal.tex
\subsection{Robust Trajectory Tracking (RTT) \label{sec:rtt}}
In the basic STP algorithm, lower priority vehicles know the trajectories of all higher priority vehicles. The region that a lower priority vehicle needs to avoid is thus simply given by the danger zones around these trajectories; however, disturbances and incomplete information significantly complicate the STP scheme. Committing to exact trajectories is no longer possible, since the disturbance $\dstb_i(\cdot)$ is \textit{a priori} unknown. Thus, the induced obstacles $\ioset_i^j(t)$ are no longer just the danger zones centered around positions. In this section, we provide an overview of the RTT algorithm that can overcome these issues. For simplicity of explanation, we will assume that no static obstacles exist, but method can be generalized even when static obstacles do exist. The material in this section is taken partially from \cite{Bansal2017}. Note that other algorithms have been developed in \cite{Bansal2017} to account for the disturbances, we use RTT algorithm for the simulations in this paper and only present RTT algorithm here. Interested readers are referred to \cite{Bansal2017} for the other algorithms. 

Even though it is impossible to commit to and track an exact trajectory in the presence of disturbances, it may still be possible to instead \textit{robustly} track a feasible \textit{nominal} trajectory with a bounded error at all times. If this can be done, then the tracking error bound can be used to determine the induced obstacles. Here, computation is done in two phases: the \textit{planning phase} and the \textit{disturbance rejection phase}. 

In the planning phase, a nominal trajectory $\state_{r,j}(\cdot)$ is computed that is feasible in the absence of disturbances. This planning is done for a reduced control set $\cset^p\subset\cset$, as some margin is needed to reject unexpected disturbances while tracking the nominal trajectory.

In the disturbance rejection phase, we compute a bound on the tracking error, independently of the nominal trajectory. To compute this error bound, we find a robust controlled-invariant set in the joint state space of the vehicle and a tracking reference that may ``maneuver" arbitrarily in the presence of an unknown bounded disturbance. Taking a worst-case approach, the tracking reference can be viewed as a virtual evader vehicle that is optimally avoiding the actual vehicle to enlarge the tracking error. We therefore can model trajectory tracking as a pursuit-evasion game in which the actual vehicle is playing against the coordinated worst-case action of the virtual vehicle and the disturbance. %In general, this game will be governed by dynamics of the form:

Let $\state_j$ and $\state_{r,j}$ denote the states of the actual vehicle $\veh_j$ and the virtual evader, respectively, and define the tracking error $e_j=\state_j-\state_{r,j}$. When the error dynamics are independent of the absolute state as in \eqref{eq:edyn} (and also (7) in \cite{Mitchell05}), we can obtain error dynamics of the form

\begin{equation}
\label{eq:edyn} % Error dynamics
\begin{aligned}
\dot{e_j} &= \fdyn_{e_j}(e_j, \ctrl_j, \ctrl_{r,j},\dstb_j), \\
\ctrl_j &\in \cset_j, \ctrl_{r,j} \in \cset^p_j, \dstb_j \in \dset_j, \quad t \leq 0
\end{aligned}
\end{equation}

To obtain bounds on the tracking error, we first conservatively estimate the error bound around any reference state $\state_{r,j}$, denoted $\errorbound_j$:

\begin{equation} \label{eqn:err}
\errorbound_j = \{e_j: \|\pos_{e_j}\|_2 \le R_{\text{EB}} \}, 
\end{equation}

\noindent where $\pos_{e_j}$ denotes the position coordinates of $e_j$ and $R_{\text{EB}}$ is a design parameter. We next solve a reachability problem with its complement $\errorbound_j^c$, the set of tracking errors violating the error bound, as the target in the space of the error dynamics. From $\errorbound_j^c$, we compute the following BRS:

\begin{equation} \label{eqn:errBound}
\begin{aligned}
\brs^{\text{EB}}_{j}(t, 0) = & \{y: \forall \ctrl_j(\cdot) \in \cfset_j, \exists \ctrl_{r, j}(\cdot) \in \cfset^\pos_j, \exists \dstb_j(\cdot) \in \dfset_i, \\
& e_j(\cdot) \text{ satisfies \eqref{eq:edyn}}, e_j(t) = y, \\
& \exists s \in [t, 0], e_j(s) \in \errorbound_j^c\}, 
\end{aligned}
\end{equation}
where the Hamiltonian to compute the BRS is given by:
\begin{equation}
\begin{aligned}
H^{\text{EB}}_{j}(e_j, \costate) &= \max_{\ctrl_j \in \cset_j} \min_{\ctrl_r \in \cset^\pos_j, \dstb_j \in \dset_j} \costate \cdot \fdyn_{e_j}(e_j, \ctrl_j, \ctrl_{r,j}, \dstb_j).
\end{aligned}
\end{equation}

Letting $t \to -\infty$, we obtain the infinite-horizon control-invariant set $\disckernel_j := \lim_{t \to -\infty} \left( \brs^{\text{EB}}_{j}(t, 0) \right)^c$. If $\disckernel_j$ is nonempty, then the tracking error $e_j$ at flight time is guaranteed to remain within $\disckernel_j \subseteq \errorbound_j$ provided that the vehicle starts inside $\disckernel_j$ and subsequently applies the feedback control law

\begin{equation}
\label{eq:robust_tracking_law}
\tracklaw_j(e_j) = \arg\max_{\ctrl_j \in \cset_j} \min_{\ctrl_r \in\cset^\pos_j, \dstb_j \in \dset_j} \costate \cdot \fdyn_{e_j}(e_j,\ctrl_j,\ctrl_{r, j},\dstb_j).
\end{equation}

The induced obstacles by each higher-priority vehicle $\veh_j$ can thus be obtained by: 
\begin{equation} 
\label{eqn:rttObs}
\begin{aligned}
\ioset_i^j(t) &=  \{\state_i: \exists y \in \pfrs_j(t), \|\pos_i - y\|_2 \le \rc \} \\
\pfrs_j(t) &= \{\pos_j: \exists \npos_j, (\pos_j, \npos_j) \in \boset_j(t)\} \\
\boset_j(t) &= \disckernel_j  + \state_{r,j}(t),
\end{aligned}
\end{equation}

\noindent where the ``$+$'' in \eqref{eqn:rttObs} denotes the Minkowski sum\footnote{The Minkowski sum of sets $A$ and $B$ is the set of all points that are the sum of any point in $A$ and $B$.}. Intuitively, if $\veh_j$ is tracking $\state_{r,j}(t)$, then it will remain within the error bound $\disckernel_j$ around $\state_{r,j}(t) ~\forall t$. This is precisely the set $\pfrs_j(t)$. The induced obstacles can then be obtained by augmenting a danger zone around this set. Finally, we can obtain the total obstacle set $\obsset_i(t)$ using:
\begin{equation}
\label{eq:obsseti}
\obsset_i(t)  = \soset_i \cup \bigcup_{j=1}^{i-1} \ioset_i^j(t)
\end{equation} 

Since each vehicle $\veh_j$, $j<i$, can only be guaranteed to stay within $\disckernel_j$, we must make sure during the trajectory planning of $\veh_i$ that at any given time, the error bounds of $\veh_i$ and $\veh_j$, $\disckernel_i$ and $\disckernel_j$, do not intersect. This can be done by augmenting the total obstacle set by $\disckernel_i$:%This can be done by choosing the induced obstacle to be the Minkowski sum\footnote{The Minkowski sum of sets $A$ and $B$ is the set of all points that are the sum of any point in $A$ and $B$.} of the error bounds. Thus,

\begin{equation} 
\label{eqn:rttAugObs}
\tilde{\obsset}_i(t) = \obsset_i(t) + \disckernel_i.
\end{equation}

Finally, given $\disckernel_i$, we can guarantee that $\veh_i$ will reach its target $\targetset_i$ if $\disckernel_i \subseteq \targetset_i$; thus, in the trajectory planning phase, we modify $\targetset_i$ to be $\tilde{\targetset}_i := \{\state_i: \disckernel_i + \state_i \subseteq \targetset_i\}$, and compute a BRS, with the control authority $\cset^\pos_i$, that contains the initial state of the vehicle. Mathematically,

\begin{equation}
\label{eq:rttBRS}
\begin{aligned}
\brs_i^\text{rtt}(t, \sta_i) = & \{y: \exists \ctrl_i(\cdot) \in \cfset^p_i, \state_i(\cdot) \text{ satisfies \eqref{eq:dyn_no_dstb}},\\
&\forall s \in [t, \sta_i], \state_i(s) \notin \tilde{\obsset}_i(t), \\
& \exists s \in [t, \sta_i], \state_i(s) \in \tilde{\targetset}_i, \state_i(t) = y\}
\end{aligned}
\end{equation}

The BRS $\brs_i^\text{rtt}(t, \sta_i)$ can be obtained by solving \eqref{eq:HJIVI_BRS} using the Hamiltonian: 
\begin{equation}
\label{eq:RTTham}
\ham_i^\text{rtt}(\state_i, \costate) = \min_{\ctrl_i \in \cset^\pos_i } \costate \cdot \fdyn_i(\state_i, \ctrl_i)
\end{equation}

The corresponding optimal control for reaching $\tilde{\targetset}_i$ is given by:
\begin{equation}
\label{eq:RTTOptCtrl}
\ctrl_i^\text{rtt}(t) = \arg \min_{\ctrl_i \in \cset^\pos_i } \costate \cdot \fdyn_i(\state_i, \ctrl_i).
\end{equation}

The nominal trajectory $\state_{r,i}(\cdot)$ can thus be obtained by using vehicle dynamics in the absence of disturbances (given by \eqref{eq:dyn_no_dstb}) with the optimal control  $\ctrl_i^\text{rtt}(\cdot)$ given by \eqref{eq:RTTOptCtrl}. From the resulting nominal trajectory $\state_{r,i}(\cdot)$, the overall control policy to reach $\targetset_i$ can be obtained via \eqref{eq:robust_tracking_law}. The robust trajectory tracking method can be summarized as follows:

\begin{algorithm}[tb!]
\SetKwInOut{Input}{input}
\SetKwInOut{Output}{output}
	\DontPrintSemicolon
	\caption{Robust trajectory tracking algorithm (STP algorithm in the presence of disturbances)}
	\label{alg:rtt}
	\Input{Set of vehicles $\veh_i, i = 1, \ldots, \N$ in the descending priority order;\newline
	Vehicle dynamics \eqref{eq:dyn} and initial states $\state_i^0$;\newline
	Vehicle destinations $\targetset_i$ and static obstacles $\soset_i$.}
    \Output{The nominal trajectories to respective destinations for all vehicles;\newline 
    Goal-satisfaction controller for all vehicles.}
	\For{\text{$i=1:N$}}{
			\textbf{Computation of tracking error bound for $\veh_{i}$} \;
			obtain the error dynamics given by \eqref{eq:edyn}; \;
			decide on a reduced control authority $\cset^\pos_i$  for the planning phase, and choose a parameter $R_{\text{EB}}$ to conservatively bound the tracking error; \;
			compute the BRS $\brs^{\text{EB}}_{i}(t, 0)$ using \eqref{eqn:errBound};\;
			compute the $\disckernel_i$ using the converged BRS $\brs^{\text{EB}}_{i}$; \;
			\If{$\disckernel_i \neq \emptyset$}{ 
				the error bound on the tracking error is given by $\disckernel_i$; \;}
			\Else{
				recompute the tracking error bound using a larger $R_{\text{EB}}$; \;}
			\textbf{Computation of obstacles for $\veh_{i}$} \;	
			determine the total obstacle set $\obsset_i(t)$, given in \eqref{eq:obsseti}. In the case $i=1$, $\obsset_i(t) = \soset_i ~ \forall t$; \;
			using $\disckernel_i$, determine the augmented obstacle set $\tilde{\obsset}_i(t)$, given in \eqref{eqn:rttAugObs}.\;
			\textbf{Trajectory planning for $\veh_{i}$} \;
			compute the reduced target set $\tilde{\targetset}_i$; \;
			compute the BRS $\brs_i^\text{rtt}(t, \sta_i)$ defined in \eqref{eq:rttBRS}.\;
			\textbf{Trajectory and controller of $\veh_{i}$} \;
			compute the nominal controller $\ctrl_i^\text{rtt}(\cdot)$ given by \eqref{eq:RTTOptCtrl};\;
			determine the nominal trajectory $\state_{r,i}(\cdot)$ using vehicle dynamics \eqref{eq:dyn_no_dstb} and the control $\ctrl_i^\text{rtt}(\cdot)$; \;
			the overall goal satisfaction controller for $\veh_i$ is given by \eqref{eq:robust_tracking_law}; \;
			output the nominal trajectory and the optimal tracking controller for $\veh_i$. \;
			\textbf{Induced obstacles by $\veh_{i}$} \;
			given the trajectory $\state_{r,i}(\cdot)$ and the tracking error bound $\disckernel_i$, compute the induced obstacles $\ioset_k^i(t)$ given by \eqref{eqn:rttObs} for all $k>i$.
		}
\end{algorithm}
% !TEX root = ../STP_journal.tex
\subsection{Numerical Simulations \label{sec:sim_dstb}}
We demonstrate our proposed methods for accounting for disturbances and incomplete information using a four-vehicle example. Each vehicle has the simple kinematics model in \eqref{eqn:NumSimpleDyn} but with disturbances added to the evolution of each state:
\begin{equation}
\label{eq:dyn_i}
\begin{aligned}
\dot{\pos}_{x,i} &= v_i \cos \theta_i + d_{x,i} \\
\dot{\pos}_{y,i} &= v_i \sin \theta_i + d_{y,i}\\
\dot{\theta}_i &= \omega_i + d_{\theta,i}, \\
\underline{v} & \le v_i \le \bar{v}, |\omega_i| \le \bar{\omega},\\
\|(d_{x,i}, & d_{y,i}) \|_2 \le d_{r}, |d_{\theta,i}| \le \bar{d_{\theta}}
\end{aligned}
\end{equation}

\noindent where $d = (d_{x,i}, d_{y,i}, d_{\theta,i})$ represents $\veh_i$'s disturbances in the three states. The control of $\veh_i$ is $u_i = (v_i, \omega_i)$, where $v_i$ is the speed of $\veh_i$ and $\omega_i$ is the turn rate; both controls have a lower and upper bound. For illustration purposes, we choose $\underline{v} = 0.5, \bar{v} = 1, \bar\omega = 1$; however, our method can easily handle the case in which these inputs differ across vehicles. The disturbance bounds are chosen as $d_r = 0.1, \bar{d_{\theta}} = 0.2$, which correspond to a 10\% uncertainty in the dynamics. %The optimal control for vehicle $i$ can be obtained by optimizing the associated Hamiltonian, $H_i(t, D_{\bm{x}_i} V_i(\bm{x}_i,t), V_i(\bm{x}_i,t))$, and is given by:

%\begin{equation}
%\omega_i(t) = -\bar{\omega}_i \frac{D_{\theta_i}V_i(\bm{x}_i,t)}{\left| D_{\theta_i}V_i(\bm{x}_i,t) \right|},
%\end{equation}
%
%\begin{equation}
%v_i(t) =
%\left \{ 
%\begin{array}{ll}
%\underline{v} & \mbox{ if } D_{x_i}V_i(\bm{x}_i,t) \cos \theta_i + D_{y_i}V_i(\bm{x}_i,t) \sin \theta_i \geq 0 \\
%\bar{v} & \mbox{ otherwise } 
%\end{array}
%\right.
%\end{equation}

\begin{figure}[H]
  \centering
  \includegraphics[width=\columnwidth]{"fig/init_setup"}
  \caption{Initial configuration of the four-vehicle example in the presence of disturbances.}
  \label{fig:init_setup_dstb}
\end{figure}

For this example, we have chosen scheduled times of arrival $\sta_i = 0~\forall i$ for simplicity. Each vehicle aims to get to a target set of the form \eqref{eq:target_sim} with target radius $r=0.1$. The vehicles' target centers $c_i$ and initial conditions $\state_i^0$ are given by \eqref{eqn:NumIC}.

These parameters are the same as the example in Section \ref{sec:basic_results}, except that the $\sta_i$ values are the same for all vehicles, and that there are no static obstacles. The problem setup for this example is shown in Fig. \ref{fig:init_setup_dstb}.

With the above parameters, we obtain $\ldt_i, i=1,2,3,4$. Note that even though $\sta_i$ is assumed to be same for all vehicles in this example for simplicity, our method can easily handle the case in which $\sta_i$ is different for each vehicle as we have already shown in Section \ref{sec:basic_results}.

For each proposed method of computing induced obstacles, we show the vehicles' entire trajectories (colored dotted lines), and overlay their positions (colored asterisks) and headings (arrows) at a point in time in which they are in a relatively dense configuration. In all cases, the vehicles are able to avoid each other's danger zones (colored dashed circles) while getting to their target sets in minimum time. In addition, we show the evolution of the BRS over time for $\veh_3$ (green boundaries) as well as the obstacles induced by the higher-priority vehicles (black boundaries).

\begin{figure}[H]
  \centering
  \includegraphics[width=\columnwidth]{"fig/cc_traj"}
  \caption{Simulated trajectories in the centralized control method. Since the higher priority vehicles induce relatively small obstacles in this case, vehicles do not deviate much from a straight line trajectory towards their respective targets, and arrive at a dense configuration similar to that in Fig. \ref{fig:dubins_result}.}
  \label{fig:cc_traj}
\end{figure}

\subsubsection{Centralized Control}
Fig. \ref{fig:cc_traj} shows the simulated trajectories in the situation where a centralized controller enforces each vehicle to use the optimal controller $\ctrl^\text{dstb}_i(t, \state_i)$ according to \eqref{eq:opt_ctrl_i}, as described in Section \ref{sec:cc}. In this case, vehicles appear to deviate slightly from a straight line trajectory towards their respective targets, just enough to avoid higher-priority vehicles. The deviation is small since the centralized controller is quite restrictive, making the possible positions of higher-priority vehicles cover a small area. In the dense configuration at $t=-1.0$, the vehicles are close to each other but still outside each other's danger zones.

\begin{figure}[H]
  \centering
  \includegraphics[width=\columnwidth]{"fig/cc_rs3"}
  \caption{Evolution of the BRS and the obstacles induced by $\veh_1$ and $\veh_2$ for $\veh_3$ in the centralized control method. Since vehicles apply the optimal control at all times, the obstacle sizes are only slightly bigger than those in Fig. \ref{fig:dubins_reach_all} and \ref{fig:dubins_reach_3}.}
  \label{fig:cc_rs3}
\end{figure}

Fig. \ref{fig:cc_rs3} shows the evolution of the BRS for $\veh_3$ (green boundary), as well as the obstacles (black boundary) induced by the higher-priority vehicles $\veh_1$ and $\veh_2$. The locations of the induced obstacles at different time points include the actual positions of $\veh_1$ and $\veh_2$ at those times, and the sizes of obstacles remain relatively small. The $\ldt_i$ values for the four vehicles (in order) in this case are $-1.35, -1.37, -1.94$ and $-2.04$, relatively close for vehicles pairs $(\veh_1, \veh_2)$ and $(\veh_3, \veh_4)$, because the obstacles generated by higher-priority vehicles are small and hence do not affect the $\ldt_i$ of lower-priority vehicles significantly.
% !TEX root = ../SPP_journal.tex
\subsubsection{Least Restrictive Control}
Fig. \ref{fig:lrc_traj} shows the simulated trajectories in the situation where each vehicle assumes that higher-priority vehicles use the least restrictive control to reach their targets, as described in \ref{sec:lrc}. Fig. \ref{fig:lrc_rs3} shows the BRS and induced obstacles for $\veh_3$.

\begin{figure}[H]
  \centering
  \includegraphics[width=\columnwidth]{"fig/lrc_traj"}
  \caption{Simulated trajectories in the least restrictive control method. All vehicles start moving before $\veh_1$ starts, because the large obstacles make it optimal to wait until higher priority vehicles pass by, leading to earlier $\ldt_i$'s. }
  \label{fig:lrc_traj}
\end{figure}

$\veh_1$ (red) takes a relatively straight path to reach its target. From the perspective of all other vehicles, large obstacles are induced by $\veh_1$, since lower-priority vehicles make the weak assumption that higher-priority vehicles are using the least restrictive control. Because the obstacles induced by higher-priority vehicles are so large, it is faster for lower-priority vehicles to wait until higher-priority vehicles pass by than to move around the higher-priority vehicles. As a result, the vehicles never form a dense configuration, and their trajectories are all relatively straight, indicating that they end up taking a short path to the target after higher-priority vehicles pass by. This is also indicated by the early $\ldt_i$ values for the four vehicles, $-1.35, -1.97, -2.66$ and $-3.39$, respectively. Compared to the centralized control method, $\ldt_i$'s are significantly earlier for all vehicles, except $\veh_1$, the highest-priority vehicle, since it need not account for any moving obstacles. 

From $\veh_3$'s (green) perspective, the large obstacles induced by $\veh_1$ and $\veh_2$ are shown in Fig. \ref{fig:lrc_rs3} as the black boundary. As the BRS (green boundary) evolves over time, its growth gets inhibited by the large obstacles for a long time, from $t=-0.89$ to $t=-1.39$. Eventually, the boundary of the BRS reaches the initial state of $\veh_3$ at $t = \ldt_3 = -2.66$.

\begin{figure}[H]
  \centering
  \includegraphics[width=\columnwidth]{"fig/lrc_rs3"}
  \caption{Evolution of the BRS for $\veh_3$ in the least restrictive control method. In this case, $\ldt_3=-2.66$, significantly earlier than that in the centralized control method ($-1.94$), reflecting the impact of larger induced obstacles.}
  \label{fig:lrc_rs3}
\end{figure}
% !TEX root = ../SPP_Journal.tex
\subsection{Robust Trajectory Tracking}
In the planning phase, we reduced the maximum turn rate of the vehicles from $1$ to $0.6$, and the speed range from $[0.5, 1]$ to exactly $0.75$ (constant speed). With these reduced control authorities, we determined from the disturbance rejection phase that a nominal trajectory from the planning phase can be robustly tracked within a distance of $R_{\text{EB}} = 0.075$.

Fig. \ref{fig:rtt_traj} shows the vehicle trajectories in the situation where each vehicle robustly tracks a pre-specified trajectory and is guaranteed to stay inside a ``bubble" around the trajectory. Fig. \ref{fig:rtt_rs3} shows the evolution of BRS and induced obstacles for vehicle $\veh_3$. The obstacles induced by other vehicles inhibit the evolution of the BRS, carving out thin “channels,” which can be seen at $t = -2.59$, that separate the BRS into different “islands”. %One can see how these channels and islands form by examining the time evolution of the BRS set.

\begin{figure}[H]
  \centering
  \includegraphics[width=\columnwidth]{"fig/rtt_traj"}
  \caption{Simulated trajectories for the robust trajectory tracking method.}
  \label{fig:rtt_traj}
  \vspace{-1em}
\end{figure}

\begin{figure}[H]
  \centering
  \includegraphics[width=\columnwidth]{"fig/rtt_rs3"}
  \caption{Evolution of the BRS for $\veh_3$ in the robust trajectory tracking method. As the BRS grows in time, the induced obstacles carve out a channel. Note that a smaller target set is used to compute the BRS to ensure that the vehicle reaches the target set by $t=0$ for any allowed tracking error.}
  \label{fig:rtt_rs3}
  \vspace{-1em}
\end{figure}
\vspace{-0.2em}

In this case, the $\ldt_i$ values for the four vehicles are $-1.61, -3.16, -3.57$ and $-2.47$ respectively. In this method, vehicles use reduced control authority for path planning towards a reduced-size effective target set. As a result, higher-priority vehicles tend to have lower $\ldt$ compared to the other two methods, as evident from $\ldt_1$. Because of this ``sacrifice" made by the higher-priority vehicles during the path planning phase, the $\ldt$'s of lower-priority vehicles may increase compared to those in the other methods, as evident from $\ldt_4$. Overall, it is unclear how $\ldt_i$ will change for a vehicle compared to the other methods, as the conservative path planning increases $\ldt_i$ for higher-priority vehicles and decreases $\ldt_i$ for lower-priority vehicles.

%% !TEX root = SPP_journal.tex
\section{SPP Under Disturbances and Incomplete information \label{sec:incomp}}
In this section, we investigate SPP under the presence of disturbances and incomplete information about higher-priority vehicles' control policies. In the presence of disturbances, our joint system dynamics become in the form of \eqref{eq:dyn}, and the controls $\ctrl_i$ of the vehicles $\veh{i}$ must drive the state $\state_i$ into the target $\targetset_i$ while keeping all vehicles away from each other's danger zones despite the worst case disturbance.

In a practical setting, incomplete information may arise due to a few reasons. For example, the presence of disturbances implies that exact trajectories $\traj_i(\cdot; \state_i^0, \ldt, \ctrl(\cdot), \dstb(\cdot))$ cannot be completely known \textit{a priori}, since $\dstb(\cdot)$ is unknown. The effect of disturbances is compounded by the fact that from the perspective of each vehicle $\veh{i}$, the effect of disturbances must be taken into account both in the dynamics the vehicle itself, and in the dynamics of other vehicles.

In addition to the presence of disturbances, each vehicle $\veh{i}$ may not have information about the control strategy of the other vehicles $\veh{j}, j\neq i$. For example, there may be many different control strategies for each vehicle to reach its target on time, and different vehicles may be using different control strategies. In this situation, lower-priority vehicles may need to plan their paths while avoiding danger zones of higher-priority vehicles without knowing what control strategies higher-priority vehicles may be using. Furthermore, each vehicle may even change control strategies on-the-fly.

We explore three different situations in which incomplete information may arise along with the presence of disturbances. In Section \ref{sec:incomp_optctrl}, we consider the situation in which all vehicles utilize a particular control strategy such as the optimal controller at all times. Such a scenario may occur if there is a centralized controller such as an air traffic controller controlling the vehicles. Next, in Section \ref{sec:incomp_LRctrl}, we assume that each vehicle only has information about the target sets and arrival times of other vehicles, and no information about other vehicles' control strategies. When the control policy is unknown, this is the weakest assumption that can be made by lower-priority vehicles, assuming that higher-priority vehicles get to their targets on time. Lastly, in Section \ref{sec:incomp_robust}, we consider the situation in which each vehicle divides its control input authority into a part that drives the vehicle towards the target, and a part that rejects disturbances. By reserving part of the control for disturbance rejection, the vehicles may declare nominal trajectories that can be robustly tracked.

To take into account disturbances and imperfect information, it turns out that we may still use Algorithm \ref{alg:basic}. In all three cases, each vehicle $\veh{i}$ induces a moving obstacle $\ioset_k^i(t)$ for the lower priority vehicles $\veh{k}, k>i$, just like in Algorithm \ref{alg:basic}. However, unlike in the basic SPP algorithm, computation of $\ioset_k^i(t)$ is no longer trival as in \eqref{eq:ioset}. We now describe how to compute $\ioset_k^i(t)$ in the three cases, which in turn will lead to different total obstacles $\obsset_i(t)$ that will be used to solve the HJ VI \eqref{eq:HJIVI_BRS}.

In general, the three methods can be used in combination in a single path planning problem, with each vehicle independently having different control policies. Lower-priority vehicles would then plan their paths while taking into account the control policy of each higher-priority vehicle. For clarity, however, we will present each method as if all vehicles are using the same method of path planning.

\subsection{Centralized Controller} \label{sec:incomp_optctrl}
Under the presence of disturbance, two modifications to Algorithm \ref{alg:basic} must be made: First, given the total obstacle set $\obsset_i(t)$, each vehicle $\veh{i}$ ensure that it gets to the target set $\targetset_i$ on time without entering any danger zones $\dz_{ij}$, despite the worst-case disturbance $\dstb_i(\cdot)$. Fortunately, disturbances can be accounted for by using \eqref{eq:HJIVI_BRS} to compute the BRS 

\begin{equation}
\label{eq:BRS_cc}
\begin{aligned}
&\brs_i(t) = \{\state_i: \exists \ctrl_i(\cdot) \in \cfset_i, \forall \dstb_i(\cdot) \in \dfset_i, \\
&\quad\forall s \in [t, \sta_i], \traj_i(s; \state^0_i, \ldt, \ctrl_i(\cdot), \dstb_i(\cdot)) \notin \obsset_i(s), \\
&\quad\exists s \in [t, \sta_i], \traj_i(s; \state^0_i, \ldt, \ctrl_i(\cdot), \dstb_i(\cdot)) \in \targetset_i\}
\end{aligned}
\end{equation}

The above BRS can be obtained by solving \eqref{eq:HJIVI_BRS} with the Hamiltonian

\begin{equation}
\ham_i\left(\state_i, p\right) = \min_{\ctrl_i \in \cset_i} \max_{\dstb_i \in \dset_i} p \cdot \fdyn_i(\state_i, \ctrl_i, \dstb_i)
\end{equation}

\noindent from which we can obtain the optimal control $\ctrl_i^*(t, \state_i)$

\begin{equation}
\label{eq:opt_ctrl_i}
\ctrl_i^*(t, \state_i) =  \arg \min_{\ctrl_i \in \cset_i} \max_{\dstb_i \in \dset_i} p \cdot \fdyn_i(\state_i, \ctrl_i, \dstb_i)
\end{equation}

If there is a centralized controller directly controlling each of the $N$ vehicles, then the control law of each vehicle can be enforced. In this case, lower priority vehicles can safely assume that higher priority vehicles are applying the enforced control law. In particular, the optimal controller for getting to the target, $u^*_i(t, x_i)$ can be enforced. In this case, the dynamics of each vehicle becomes 

\begin{equation}
\label{eq:dyn_cc}
\begin{aligned}
\dot \state_i &= \fdyn^*_i (\state_i, \dstb_i) \coloneqq \fdyn_i(\state_i, \ctrl^*_i(t, \state_i), \dstb_i) \\
\dstb_i &\in \dset_i, \quad i = 1,\ldots, N, \quad t \in [\ldt_i, \sta_i]
\end{aligned}
\end{equation}

\noindent where $u_i$ no longer appears explicitly in the dynamics.

From the perspective of a lower-priority vehicle $\veh{i}$, a higher-priority vehicle $\veh{j}, j < i$ induces an time-varying obstacle that represents the positions that could possibly be within the capture radius $\rc$ of $\veh{j}$ under the dynamics \eqref{eq:dyn_cc}. Determining this obstacle involves computing a FRS $\frs_j(t)$ of $\veh{j}$ starting from $x_j(\ldt) = x_j^0$, defined as 

\begin{equation}
\label{eq:FRS_cc}
\begin{aligned}
\frs_j(t) = \{&y \in \R^{n_j}: \exists d_j(\cdot) \in \dfset_j, \\
&\traj_j(t; \state^0_j, \ldt, \ctrl^*_i(\cdot), \dstb_i(\cdot) )= y\}
\end{aligned}
\end{equation}

\noindent where the target set for the FRS computation is chosen to be\footnote{In practice, we define the target set to be a small region around the vehicle's initial state for computational reasons.} the initial state $\{\state_j(\ldt)\}$. $\frs_j(t)$ can be computed by solving \eqref{eq:HJIVI_FRS} with the following Hamiltonian

\begin{equation}
\ham_j\left(\state_j, \costate\right) = \max_{\dstb_j \in \dset_j} \costate \cdot \fdyn^*_j(\state_j, \dstb_j)
\end{equation}

The FRS $\frs_j(t)$ represents the set of possible states at time $t$ of a higher-priority vehicle $\veh{j}$ given all possible disturbances $\dstb_j(\cdot)$ and given that $\veh{j}$ uses the feedback controller $\ctrl_j^*(t, \state_j)$. In order for a lower-priority vehicle $\veh{i}$ to guarantee that it does not go within a distance of $\rc$ to $\veh{j}$, $\veh{i}$ must stay a distance of at least $\rc$ away from the set $\frs_j(t)$ for all possible values of the non-position states $\npos_j$. This gives the obstacle induced by a higher priority vehicle $\veh{j}$ for a lower priority vehicle $\veh{i}$ as follows:

\begin{equation}
\ioset_i^j(t) = \{\state_i: \dist(\pos_i, \pfrs_j(t)) \le \rc \}
\end{equation}

\noindent where the $\dist(\cdot, \cdot)$ function here represents the minimum distance from a point to a set, and the set $\pfrs_j(t)$ is the set of states in the FRS $\\frs_j(t)$ projected onto the states representing position $\pos_j$, and disregarding the non-position dimensions $\npos_j$:

\begin{equation}
\pfrs_j(t) = \{\pos_j: \exists \npos_j, \state_j = (\pos_j, \npos_j) \in \frs_j(t)\}
\end{equation}

Finally, the total time-varying obstacles $\obsset_i(t)$ for the next vehicle $\veh{i}$ can be determined using \eqref{eq:obsseti}. Afterwards, the BRS $\brs_i(t)$ in \eqref{eq:BRS_cc} can be computed to obtain the optimal control. Repeating this process, all vehicles will be able to plan paths that guarantee the vehicles' timely and safe arrival.

\subsection{Least Restrictive Controller}
\label{sec:incomp_LRctrl}
Here, for each vehicle we again compute the BRS in \eqref{eq:BRS_cc}. However, if there is no centralized controller to enforce the control policy for higher-priority vehicles, weaker assumptions must be made by the lower-priority vehicles to ensure collision avoidance. One reasonable assumption that a lower-priority vehicle can make is that all higher-priority vehicles follow the least restrictive control that would take them to their targets. This control would be given by 

\begin{equation}
\label{eq:lrctrl} % least restrictive control
\ctrl_j(t, \state_j)\in \begin{cases} \{\ctrl_j^*(t) \text{ in \eqref{eq:opt_ctrl_i}}\} \text{ if } \state_j(t)\in \partial \brs_j(t) \text{ in \eqref{eq:BRS_cc}}, \\
\cset_j  \text{ otherwise}
\end{cases}
\end{equation}

Such a controller allows each higher-priority vehicle to use any controller it desires, except when it is on the boundary of the BRS, $\partial \brs_j(t)$, in which case the optimal control $\ctrl_j^*(t)$ given by \eqref{eq:opt_ctrl_i} must be used to get to the target safely and on time. This assumption is the weakest assumption that could be made by lower priority vehicles given that the higher priority vehicles will get to their targets on time.

Suppose a lower-priority vehicle $\veh{i}$ assumes that higher-priority vehicles $\veh{j}, j < i$ use the least restrictive control strategy in \eqref{eq:lrctrl}. From the perspective of the lower-priority vehicle $\veh{i}$, a higher-priority vehicle $\veh{j}$ could be in any state that is reachable from $\veh{j}$'s initial state $\state_j(\ldt) =\state_j^0$ and from which the target $\targetset_j$ can be reached. Mathematically, this is defined by $\brs_j(t) \cap \frs_j(t)$ where $\brs_j(t)$ is given by \eqref{eq:BRS_cc}. In this situation, since $\veh{j}$ cannot be assumed to be using any particular feedback control, $\frs_j(t)$ is defined by

\begin{equation}
\label{eq:FRS_lrc}
\begin{aligned}
\frs_j(t) = \{&y \in \R^{n_j}:\exists \ctrl_j(\cdot) \in \cfset_j, \exists \dstb_j(\cdot) \in \dfset_j, \\
&\traj_j(t; \state^0_j, \ldt, \ctrl_j(\cdot), \dstb_j(\cdot))= y\}
\end{aligned}
\end{equation}

Compared to \eqref{eq:FRS_cc}, the FRS defined by \eqref{eq:FRS_lrc} has an additional quantifier $\exists \ctrl_j(\cdot)$ for the control, which also appears in the trajectory $\traj_j(t; \state^0_j, \ldt, \ctrl_j(\cdot), \dstb_j(\cdot))$. This FRS can be computed by solving \eqref{eq:HJIVI_FRS} without obstacles, and with

\begin{equation}
H_j\left(t, \state_j, p\right) = \max_{\ctrl_j \in \cset_j} \max_{\dstb_j \in \dset_j} p \cdot \fdyn_j(t, \state_j, \ctrl_j, \dstb_j)
\end{equation}

In turn, the obstacle induced by a higher priority $\veh{j}$ for a lower priority vehicle $\veh{i}$ is as follows:

\begin{equation}
\begin{aligned}
\ioset_i^j(t) &= \{x_i: \dist(\pos_i, \pfrs_j(t)) \le \rc \}, \text{ with} \\
\pfrs_j(t) &= \{\pos_j: \exists \npos_j, \state_j = (p_j, \npos_j) \in \brs_j(t) \cap \frs_j(t)\}
\end{aligned}
\end{equation}

Note that the centralized controller method described in the previous section can be thought of as the ``most restrictive control'' method, in which all vehicles must use the optimal controller at all times, while the least restrictive control method allows vehicles to use any suboptimal controller that allows them to arrive at the target on time. These two methods can be considered two extremes of a spectrum in which varying degrees of optimality is assumed for higher-priority vehicles. Vehicles can also choose a control strategy in the middle of the two extremes and for example use a control within some percentage of the optimal control, or use the optimal control unless some condition is met. The induced obstacles and the BRS can then be similarly computed using the allowed control authority.

\subsection{Robust Tracking} \label{sec:incomp_robust}
Even though it is not possible to commit and track an exact trajectory in presence of disturbances, it might still be possible to instead \textit{robustly} track a feasible nominal trajectory with a bounded error at all times. If this can be done, then the tracking error bound can be used to determine the induced obstacles. Here, computation is done in two phases: the planning phase and the disturbance rejection phase. In the planning phase, we compute a nominal trajectory that is feasible in the absence of disturbances. In the disturbance rejection phase, we then compute a bound on the tracking error.

In the planning phase, the disturbance is ignored, and our system reduces to 

\begin{equation}
\label{eq:plan_dyn}
\begin{aligned}
\dot\state_j^P &= \fdyn_j^P(\state_j^P, \ctrl_j^P), t \in [\edt, \sta] \\
\ctrl_j^P &\in \cset_j^P, \qquad j = 1 \ldots, \N
\end{aligned}
\end{equation}

\noindent where the superscript $P$ indicates that the dynamics are used in the planning phase to produce a nominal trajectory. Note that the planning phase does not make full use of the vehicle's control authority, as some margin is needed to reject unexpected disturbances while tracking the nominal trajectory. Therefore, in this method, planning is done for a reduced control set $\cset^p\subset\cset$ to produce the nominal trajectory

\begin{equation}
\traj_j^P(\cdot; \state^0_j, \ldt, \ctrl_j^{P*}(\cdot))
\end{equation}

\noindent which does not utilize the vehicle's full maneuverability: $\ctrl_j^{P*}(\cdot) \in \cfset_j^P \subset \cfset_j$. Replicating the nominal control is therefore always possible, with additional maneuverability available at execution time to counteract external disturbances.

In the disturbance rejection phase, we determine the error bound independently of the nominal trajectory. To compute this error bound, we wish to find a robust controlled-invariant set in the joint state space of the vehicle and a tracking reference that may ``maneuver" arbitrarily in the presence of an unknown bounded disturbance. Taking a worst-case approach, the tracking reference can be viewed as a virtual ``evader'' vehicle that is optimally avoiding the actual vehicle to enlarge the tracking error. We therefore can model trajectory tracking as a pursuit-evasion game in which the actual vehicle is playing against the coordinated worst-case action of the virtual vehicle and the disturbance. 

Let $\state_j$ and $\state_j^P$ represent the state of the actual vehicle $\veh{j}$ and the virtual evader, respectively, and define the tracking error $\errstate_j=\state_j-\state_j^P$. In cases where the error dynamics are independent of the absolute state as in \eqref{eq:edyn} and (7) in \cite{Mitchell05}, we can obtain error dynamics of the form
\begin{equation}
\label{eq:edyn} % Error dynamics
\begin{aligned}
\dot{\errstate_j} &= \fdyn_j^\errstate(\errstate_j, \ctrl_j, \ctrl_j^P, \dstb_j), \\
\ctrl_j &\in \cset_j, \ctrl_j^P\in\cset^P_j, \dstb_j \in \dset_j, \quad t \in [0, T],
\end{aligned}
\end{equation}

To obtain bounds on the tracking error, we first conservatively estimate the error bound around any reference state $\state_j^P$, denoted $\errorbound_j(\state_j^P)$, and solve a reachability problem with its complement, $\errorbound_j^c$ as the target in the space of the error dynamics; $\errorbound_j^c$ is the set of tracking errors violating the error bound. From $\errorbound_j^c$, we compute the backward reachable set 

\begin{equation}
\label{eq:BRS_rtt}
\begin{aligned}
&\brs(t) = \{\errstate_i: \forall \ctrl_j(\cdot) \in \cfset_j, \exists \ctrl_j^P \in \cfset_j^P, \exists \dstb_j(\cdot) \in \dfset_j, \\
&\quad\exists s \in [t, \sta_i], \traj_j^P(\cdot; \state^0_j, \ldt, \ctrl_j^P(\cdot)) \in \errorbound_j^c\}
\end{aligned}
\end{equation}

\noindent using \eqref{eq:HJIVI_BRS} without obstacles, and with the Hamiltonian

\begin{equation}
\label{eq:HJIVI_track}
\ham_j(\errstate_j, \costate) = \max_{\ctrl_j \in \cset_j} \min_{\ctrl_j^P \in\cset^P_j, \dstb_j \in \dset_j} \costate \cdot \fdyn^\errstate_j(\errstate_j, \ctrl_j, \ctrl_j^P, \dstb_j)
\end{equation}

Letting the time horizon tend to infinity, we obtain the infinite-horizon controlled-invariant set, which we denote by $\disckernel_j$. If this set is nonempty, then the tracking error $\errstate_j$ at flight time is guaranteed to remain within $\errorbound_j$ provided that the vehicle starts inside $\disckernel_j$ and subsequently applies the feedback control law implicitly defined in \eqref{eq:HJIVI_track}:

\begin{equation}
\label{eq:robust_tracking_law}
\tracklaw_j(\errstate_j) = \arg\max_{\ctrl_j \in \cset_j} \min_{\ctrl_j^P \in\cset^P_j, \dstb_j \in \dset_j} \costate \cdot \fdyn^\errstate_j(\errstate_j, \ctrl_j, \ctrl_j^P, \dstb_j)
\end{equation}

Given $\errorbound_i$, we can guarantee that $\veh{i}$ will reach its target $\targetset_i$ if the state of a nominal trajectory is such that $\errorbound_i(\state_j^P) \subset \targetset_i$; thus, in the path planning phase, we define $\targetset_i^P = \{\state_i: \errorbound_i(\state_i) \subseteq \targetset_i\}$, and compute the following BRS: 

\begin{equation}
\label{eq:BRS_rttplan}
\begin{aligned}
&\brs_i(t) = \{\state_i: \exists \ctrl_i^P(\cdot) \in \cfset_i, \\
&\quad\forall s \in [t, \sta_i], \traj_i^P(s; \state^0_i, \ldt, \ctrl_i^P(\cdot)) \notin \obsset_i(s), \\
&\quad\exists s \in [t, \sta_i], \traj_i^P(s; \state^0_i, \ldt, \ctrl_i^P(\cdot)) \in \targetset_i^P\}
\end{aligned}
\end{equation}

This BRS can be computed using \eqref{eq:HJIVI_BRS} with the Hamiltonian

\begin{equation}
\label{eq:rttham}
\ham_i(t, \state_i^P, \costate) = \min_{\ctrl_i^P \in \cset_i^P} \costate \cdot \fdyn_i^P(\state_i^P, \ctrl_i^P)
\end{equation}

The nominal trajectory $\traj_j^P(\cdot; \state^0_j, \ldt, \ctrl_j^{P*}(\cdot))$ can then be obtained from the optimal control

\begin{equation}
\label{eq:rttOptCtrl}
\ctrl_i^{P*}(t) = \arg \min_{\ctrl_i^P \in \cset_i^P} \costate \cdot \fdyn_i^P(\state_i^P, \ctrl_i^P)
\end{equation}

From the resulting nominal trajectory, the overall control policy to reach the destination can be then obtained using \eqref{eq:robust_tracking_law}.

Finally, since each vehicle $\veh{j}$ can only be guaranteed to stay within $\errorbound_j(\state_j^P)$, we must make sure at any given time, the error bounds of $\veh{i}$ and $\veh{j}$, $\errorbound_i(\state_i^P)$ and $\errorbound_j(\state_j^P)$, do not intersect. This can be done by choosing the induced obstacle to be the Minkowski sum\footnote{The Minkowski sum of sets $A$ and $B$ is the set of all points that are the sum of any point in $A$ and $B$.} of the error bounds. Thus,
\vspace{-0.3em}
\begin{equation}
\begin{aligned}
\ioset_i^j(t) &= \{\state_i: \dist(\pos_i, \pfrs_j(t)) \le \rc \} \\
\pfrs_j(t) &= \{\pos_j: \exists \npos_j, \state_j = (\pos_j, \npos_j) \in \errorbound(0) + \errorbound(\state_j^P(t)) \},
\end{aligned}
\end{equation}
\noindent where $0$ denotes the origin. 

%% !TEX root = SPP_journal.tex
\subsection{Incomplete Information Results \label{sec:basic_results}}
We demonstrate our proposed methods using a four-vehicle example. Each vehicle has the following simple kinematics model:

\begin{equation}
\label{eq:dyn_i}
\begin{aligned}
\dot{\pos}_{x,i} &= v_i \cos \theta_i + d_{x,i} \\
\dot{\pos}_{y,i} &= v_i \sin \theta_i + d_{y,i}\\
\dot{\theta}_i &= \omega_i + d_{\theta,i}, \\
\underline{v} & \le v_i \le \bar{v}, |\omega_i| \le \bar{\omega},\\
\|(d_{x,i}, & d_{y,i}) \|_2 \le d_{r}, |d_{\theta,i}| \le \bar{d_{\theta}}
\end{aligned}
\end{equation}

\noindent where $p_i = (p_{x,i}, p_{y,i}), \theta_i, d = (d_{x,i}, d_{y,i}, d_{\theta,i})$ respectively represent $\veh{i}$'s position, heading, and disturbances in the three states. The control of $\veh{i}$ is $u_i = (v_i, \omega_i)$, where $v_i$ is the speed of $\veh{i}$ and $\omega_i$ is the turn rate; both controls have a lower and upper bound. For illustration purposes, we chose $\underline{v} = 0.5, \bar{v} = 1, \bar\omega = 1$; however, our method can easily handle the case in which these inputs differ across vehicles and cases in which each vehicle has different dynamic model. The disturbance bounds are chosen as $d_{r} = 0.1, \bar{d_{\theta}} = 0.2$, which correspond to a 10\% uncertainty in the dynamics.

The initial states of the vehicles are given as follows:
\begin{equation}
\begin{aligned}
x_1^0 &= (-0.5, 0, 0), \quad &x_2^0 = (0.5, 0, \pi), \\
x_3^0 &= \left(-0.6, 0.6, 7\pi/4\right), \quad &x_4^0 = \left(0.6, 0.6, 5\pi/4\right).
\end{aligned}
\end{equation}

\noindent Each of the vehicles has a target set $\targetset_i$ that is circular in their position $\pos_i$ centered at $c_i = (c_{x,i}, c_{y,i})$ with radius $r$:
\vspace{-0.2em}
\begin{equation}
\targetset_i = \{x_i \in \R^3: \|p_i - c_i\| \le r\}
\end{equation}

\noindent For the example shown, we chose $c_1 = (0.7, 0.2), c_2 = (-0.7, 0.2), c_3 = (0.7, -0.7), c_4 = (-0.7, -0.7)$ and $r = 0.1$. The setup of the example is shown in Fig. \ref{fig:init_setup}.

Since the joint state space of this system is intractable for a direct application of HJ reachability theory, we repeatedly solve (\ref{eq:HJIVI}) to compute BRSs from the targets $\targetset_i, i =1,2,3,4$, in that order, with moving obstacles induced by vehicles $j=1,\ldots,i-1$. We also obtain $\ldt_i, i=1,2,3,4$ assuming $\sta_i=0$ without loss of generality. Note that even though $\sta_i$ is assumed to be same for all vehicles in this example for simplicity, our method can easily handle the case in which $\sta_i$ are different for each vehicle.

\begin{figure}
  \centering
  \includegraphics[width=0.30\textwidth]{"fig/init_setup"}
  \caption{Initial configuration of the four-vehicle example.}
  \label{fig:init_setup}
  \vspace{-1.5em}
\end{figure}

For each proposed method of computing induced obstacles, we show the vehicles' entire trajectories (colored dotted lines), and overlay their positions (colored asterisks) and headings (arrows) at a point in time in which they are in relatively dense configuration. In all cases, the vehicles are able to avoid each other's danger zones (colored dashed circles) while getting to their target sets in minimum time. In addition, we show the evolution of the BRS over time for $\veh{3}$ (green boundaries) as well as the induced obstacles of the higher-priority vehicles (black boundaries).

\subsubsection{Centralized Controller}
Fig. \ref{fig:cc_traj} shows the simulated trajectories in the situation where a centralized controller enforces each vehicle to use the optimal controller $u^*_i(t, x_i)$ according to \eqref{eq:opt_ctrl_i}, as described in Section \ref{sec:cc}.

\begin{figure}
  \centering
  \includegraphics[width=0.40\textwidth]{"fig/cc_traj"}
  \caption{Simulated trajectories in the centralized controller method. Since the higher priority vehicles induce relatively small obstacles in this case, vehicles do not deviate much from a straight line trajectory towards their respective targets.}
  \label{fig:cc_traj}
  \vspace{-1.4em}
\end{figure}

In this case, vehicles appear to deviate slightly from a straight line trajectory towards their respective targets, just enough to avoid higher-priority vehicles. The deviation is small since the centralized controller is quite restrictive, making the possible positions of higher priority vehicles cover a small area. In the dense configuration at $t=-1.0$, the vehicles are close to each other but still outside each other's danger zones.

\begin{figure}[h]
  \centering
  \includegraphics[width=0.40\textwidth]{"fig/cc_rs3"}
  \caption{Evolution of the BRS and the obstacles induced by $\veh{1}$ and $\veh{2}$ for $\veh{3}$ in the centralized controller method. Since every vehicle is applying the optimal control at all times, the obstacle sizes are relatively small.}
  \label{fig:cc_rs3}
  \vspace{-1.2em}
\end{figure}

Fig. \ref{fig:cc_rs3} shows the evolution of the BRS for $\veh{3}$ (green boundary), as well as the obstacles (black boundary) induced by the higher-priority vehicles $\veh{1}$ (blue) and $\veh{2}$ (red). The locations of the induced obstacles at different time points include the actual positions of $\veh{1}$ and $\veh{2}$ at those times, and the size of the obstacles remains relatively small. $\ldt_i$ numbers for the four vehicles (in order) in this case are $-1.35, -1.37, -1.94$ and $-2.04$. Numbers are relatively close for vehicles $\veh{1}$, $\veh{2}$ and $\veh{3}$, $\veh{4}$, because the obstacles generated by higher-priority vehicle are small and hence do not affect $\ldt$ of the lower-priority vehicles significantly. 

\subsection{Comparison of Proposed Methods}
This section briefly discusses the relative advantages and limitations of the proposed methods. Each method makes a trade-off between optimality (in terms of $\ldt_i$) and flexibility in control and disturbance rejection.

\subsubsection{Centralized Controller}
Given an order of priority, the vehicles will have the relatively high $\ldt_i$ in this method since a higher-priority vehicle maximizes its $\ldt_i$ as much as possible, while at the same time inducing a relatively small obstacle so as to minimize its impedance towards the lower-priority vehicles. A limitation of this method is that a centralized controller is likely required to ensure that the optimal control is being applied by the vehicles at all times, and hence ensure safety.

\subsubsection{Least Restrictive Control}
This method gives more control flexibility to the higher-priority vehicles, as long as the control does not push the vehicle out of its BRS. This flexibility, however, comes at the price of having larger induced obstacle, lowering $\ldt_i$ for the lower-priority vehicles.  

\subsubsection{Robust Trajectory Tracking}
Since the obstacle size is constant over time, this method is easier to implement from a practical standpoint. This method also aims at striking a balance between $\ldt_i$ across vehicles. In particular, the $\ldt$ of a higher-priority vehicle can be lower compared to the centralized controller method, so that a lower-priority vehicle can achieve a higher $\ldt$, making this method particularly suitable for the scenarios where there is no strong sense of priority among vehicles. This method, however, is computationally tractable only when the tracking error dynamics are independent of the absolute states, as it otherwise requires doing computation in the joint state space of system dynamics and virtual vehicle dynamics. 

% Intruder files
% !TEX root = ../SPP_IoTjournal.tex
\section{Response to Intruders \label{sec:intruder}}

\SBnote{Most of the assumption/notation content in this section will be moved to the problem formulation section.}
In Section \ref{sec:incomp}, we made the basic SPP algorithm more robust by taking into account disturbances and considering situations in which vehicles may not have complete information about the control strategy of the other vehicles. However, if a vehicle not in the set of SPP vehicles enters the system, or even worse, if this vehicle is an adversarial intruder, the original plan can lead to vehicles entering into each other's danger zones. If vehicles do not plan with an additional safety margin that takes a potential intruder into account, a vehicle trying to avoid the intruder may effectively become an intruder itself, leading to a domino effect. In this section, we propose a method to allow vehicles to avoid an intruder while maintaining the SPP structure.

In general, the effect of an intruder on the vehicles in structured flight can be entirely unpredictable, since the intruder in principle could be adversarial in nature, and the number of intruders could be arbitrary. Therefore, for our analysis to produce reasonable results, two assumptions about the intruders must be made.

\begin{assumption}
\label{as:avoidOnce}
At most one intruder (denoted as $\veh_I$ here on) affects the SPP vehicles at any given time. The intruder exits the altitude level affecting the SPP vehicles after a duration of $\iat$. 
\end{assumption}

Let the time at which intruder appears in the system be $\tsa$ and the time at which it disappears be $\tea$. Assumption \ref{as:avoidOnce} implies that $\tea \leq \tsa + \iat$. Thus, any vehicle $\veh_i$ would need to avoid the intruder $\veh_{\intr}$ for a maximum duration of $\iat$. This assumption can be valid in situations where intruders are rare, and that some fail-safe or enforcement mechanism exists to force the intruder out of the altitude level affecting the SPP vehicles. Note that we do not make any assumptions about $\tsa$; however, we assume that once it appears, it stays for a maximum duration of $\iat$.
%in addition, after avoiding the intruder, Qi can safely assume that it would not need to avoid another intruder

\begin{assumption}
\label{as:dynKnown}
The dynamics of the intruder are known and given by $\dot\state_\intr = f_\intr(\state_\intr, \ctrl_\intr, \dstb_\intr)$. The initial state of the intruder is given by $\state_{\intr}^0.$
\end{assumption}

Assumption \ref{as:dynKnown} is required for HJ reachability analysis. In situations where the dynamics of the intruder are not known exactly, a conservative model of the intruder may be used instead.

Based on the above assumptions, we aim to design a control policy that ensures separation with the intruder and with other SPP vehicles, and ensures a successful transit to the destination. However, depending on the initial state of the intruder, its control policy, and the disturbances in the dynamics of a vehicle and the intruder, a vehicle may arrive at different states after avoiding the intruder. Therefore, a control policy that ensures a successful transit to the destination needs to account for all such possible states, which is a path planning problem with multiple (infinite, to be precise) initial states and a single destination, and is hard to solve in general. Thus, we divide the intruder avoidance problem into two sub-problems: (i) we first design a control policy that ensures a successful transit to the destination if no intruder appears and that successfully avoid the intruder, if it does. (ii) after the intruder disappears at $\tea$, we replan the trajectories of the affected vehicles. 

Since the replanning is done in real-time, it should be fast and scalable with the number of SPP vehicles. Intuitively, one can think about dividing the flight space of vehicles such that at any given time, any two vehicles are far enough from each other so that an intruder can only affect one vehicle in a duration of $\iat$ despite its best efforts. The advantage of this approach is that after the intruder disappears, we only have to replan the trajectory of a single vehicle regardless of the number of total vehicles in the system, which makes this approach particularly suitable for practical systems. In this method, we build upon this intuition and show that such a division of space is indeed possible. Thus the proposed method guarantees that \textit{atmost one} vehicle is affected by the presence of intruder, regardless of the number of SPP vehicles, and hence the replanning can be efficiently done in real-time. 

In Sections \ref{sec:sepRegion} and \ref{sec:buffRegion}, we compute a space division of state-space such that atmost one vehicle needs to apply the avoidance maneuver regardless of the initial state of the intruder. However, we still need to ensure that a vehicle do not collide with another vehicle while avoiding the intruder. The induced obstacles that reflect this possibility are computed in Section \ref{sec:intruderObs}. Intruder avoidance control and re-planning are discussed in Section \ref{sec:replan}.

\subsection{Separation Region} \label{sec:sepRegion}
Depending on the information known to a lower-priority vehicle $\veh_i$ about $\veh_j$'s control strategy, we can use one of the three methods described in Section 5 in \SBnote{First journal paper should be cited here} to compute the ``base" obstacles $\boset_j(t)$, the obstacles that would have been induced by $\veh_j$ in the absence of an intruder.

Given $\boset^j(t)$, we want to compute the set of all initial states of the intruder for which vehicle $\veh_j$ may have to apply an avoidnace maneuver. We refer to this set as \textit{separation region} here on, and denote it as $\sep_j(t)$. The significance of $\sep_j(.)$ is that $\veh_j$ can apply any control even in the presence of intruder, if the intruder appears outside $\sep_j(\tsa)$, that is $\state_{\intr}^0 \in \left(\sep_j(\tsa)\right)^c$. $\sep_j(t)$ can be conveniently computed using the relative dynamics between $\veh_j$ and $\veh_{\intr}$. 

We define relative dynamics of the intruder $\veh_{\intr}$ with state $\state_\intr$ with respect to $\veh_i$ with state $\state_i$:
\begin{equation}
\label{eq:reldyn}
\begin{aligned}
\state_{\intr, i} &= \state_\intr - \state_i \\
\dot \state_{\intr, i} &= f_r(\state_{\intr, i}, \ctrl_i, \ctrl_\intr, \dstb_i, \dstb_\intr)
\end{aligned}
\end{equation}
Given the relative dynamics, we compute the set of states from which the joint states of $\veh_{\intr}$ and $\veh_{j}$ can enter danger zone $\dz_{j\intr}$ when both $\veh_{j}$ and $\veh_{\intr}$ are using \textit{optimal control to collide} with each other for a duration of $\iat$. Under the relative dynamics \eqref{eq:reldyn}, this set of states is given by the backwards reachable set $\brs^{\text{S}}_j(\iat, \targetset_\text{C}, \emptyset, H_\text{C})$, with

\begin{equation} \label{eqn:optAvoid}
\begin{aligned}
\brs^{\text{S}}_{j}(t, \iat) = & \{y: \exists \ctrl_j(\cdot) \in \cfset_j, \ctrl_\intr(\cdot) \in \cfset_\intr, \dstb_j(\cdot) \in \dfset_j, \\
& \dstb_\intr(\cdot) \in \dfset_\intr, \state_{\intr, j}(\cdot) \text{ satisfies \eqref{eq:reldyn}},\\
& \exists s \in [t, \iat], \state_{\intr, j}(s) \in \targetset^{\text{S}}_{j}, \state_{\intr, j}(t) = y\},
\end{aligned}
\end{equation}
where 
\begin{equation}
\begin{aligned}
\targetset^{\text{S}}_{j} &= \{\state_{\intr, j}: \|\pos_{\intr, j}\|_2 \le \rc\} \\
H^{\text{S}}_{j}(\state_{\intr, j}, \costate) &= \min_{\ctrl_j \in \cset_j, \ctrl_\intr \in \cset_\intr, \dstb_i \in \dset_i, \dstb_\intr \in \dset_\intr} \costate \cdot f_r(\state_{\intr, j}, \ctrl_j, \ctrl_\intr, \dstb_j, \dstb_\intr)
\end{aligned}
\end{equation}

The interpretation of set $\brs^{\text{S}}_{j}(0, \iat)$ is that if the $\veh_{\intr}$ starts outside this set (in relative co-ordinates), then no matter what control $\veh_{\intr}$ and $\veh_{j}$ apply for the next $\iat$ seconds, they cannot enter the danger zone $\dz_{j\intr}$. Thus, $\veh_{j}$ need not avoid the intruder and can apply any desired control. The set $\sep_j(t)$ is thus given by:
\begin{equation} \label{eqn:sepRegion}
\sep_j(t) = \boset_j(t) + \brs^{\text{S}}_{j}(0, \iat),
\end{equation}
where the ``$+$'' in \eqref{eqn:sepRegion} denotes the Minkowski sum\footnote{The Minkowski sum of sets $A$ and $B$ is the set of all points that are the sum of any point in $A$ and $B$.}.

\subsection{Buffer Region} \label{sec:buffRegion}
In section \ref{sec:sepRegion}, we computed sets $\sep_j(\cdot)$ such that $\veh_j$ avoids the intruder only if $\state_{\intr}^0 \in sep_j(\tsa)$. But to ensure that atmost one vehicle avoids the intruder, we also need to make sure that no other vehicle $\veh_i, i>j$, avoids the intruder if it appears inside $sep_j(\tsa)$. This can be achieved by ensuring that vehicle $\veh_i$ is far enough from $\sep_j(\cdot)$ (that is, there is a ``buffer" region between $\veh_i$ and $\sep_j(\cdot)$), such that intruder appearing inside $sep_j(\cdot)$ cannot enter the danger zone $\dz_{i\intr}$. We denote this buffer region as $\buff^i(t)$, and the separation region augmented with the buffer region as $\tilde{\sep}^j_i(t)$.

$\buff^i(t)$ can be computed using the relative dynamics 

However, to ensure that atmost one vehicle needs to apply the avoidance maneuver, we need to make sure that no other vehicle $\veh{i}, i \neq j$, needs to avoid the intruder if intruder starts inside $\brs_\text{C}$. This can  be achieved by ensuring that $\brs_{\text{C},i}$ and $\brs_{\text{C},j}$ do not intersect.







\textbf{To-Dos:}
\begin{itemize}
\item A remark about the single vehicle replanning property of Method-2. Moreover, Method-2 can, in theory, handle multiple intruders as long as they are affecting different vehicles. Though, we have to replan for several vehicles in that case. 
\item Once the replanning is complete, another intruder can appear in the system. So strictly speaking we are making an assumption that atmost one intruder is in the system \textit{at any given time} as opposed to throughout the trajectory.
\item For method-2 results, it may be helpful to include a figure which is showing the division of space among vehicles at some time (probably right before the intruder enters). 
\end{itemize}

% !TEX root = ../STP_journal.tex
%Intuitively, the number of vehicles that are affected by an intruder depends on how close the vehicles are to each other. For example, if the vehicles are in a dense configuration, then multiple vehicles might get affected by the intruder simultaneously and hence re-planning is required at a larger scale. In this section, we propose a method that guarantee intruder avoidance while allowing vehicles to be in a dense configuration, but may require a full re-planning of the system after the intruder disappears.     
Suppose some vehicle $\veh_i$ starts avoiding the intruder $\veh_{\intr}$ at some time $t = \tsa$, and stops avoiding at $t = \tea$. When $t < \tsa$, $\veh_i$ must plan its trajectory taking into account the possibility that it may need to avoid an intruder $\veh_\intr$. Since $\veh_i$ may spend a duration of up to $\iat$ performing avoidance, its induced obstacles $\ioset_k^i(t), k>i$ need to be computed in a way that reflects this possibility. The induced obstacles computation is discussed in Section \ref{sec:intruder_iocomp}.

We must also ensure that while avoiding the intruder, $\veh_i$ does not collide with the total obstacle set $\obsset_i(t)$. This requires computing the augmented total obstacle $\tilde\obsset_i(t)$; the computation of $\tilde\obsset_i(t)$ and the controller that guarantees the avoidance of the augmented obstacles are discussed in Section \ref{sec:intruder_aocomp}.

In Section \ref{sec:intruder_avoid}, we describe how $\veh_i$ can guarantee collision avoidance with the intruder. A pairwise collision avoidance problem such as this has been solved in isolation in \cite{Mitchell05}.

Finally, when $t > \tea$, $\veh_i$ has already successfully avoided the intruder, but depending on the state it happens to arrive at after avoiding the intruder, it may need to re-plan its trajectory to reach the target safely. The re-planning process is discussed in Section \ref{sec:re-plan_method1}.

\subsubsection{Induced Obstacle Computation} \label{sec:intruder_iocomp}
The goal of this section is to compute, for each lower-priority vehicle $\veh_i$, the time-varying obstacles induced by each higher-priority vehicle $\veh_j, j < i$, denoted by $\ioset_i^j(t)$. As before, the total obstacle set $\obsset_i(t)$ can then be obtained using \eqref{eq:obsseti}. To compute the obstacle that $\veh_i$ needs to avoid at time $t$, it is sufficient to consider the scenarios where $\tsa \in [t-\iat, t]$. This is because if $\tsa < t - \iat$, then the STP vehicles would already be in the re-planning phase at time $t$ and hence cannot be in conflict. 

Depending on the information known to a lower-priority vehicle $\veh_i$ about $\veh_j$'s control strategy, we can use one of the three methods described in Section \ref{sec:incomp} to compute the ``base" obstacles $\boset_j(t)$; these are the obstacles that would have been induced by $\veh_j$ in the absence of an intruder. The base obstacles are respectively given by \eqref{eqn:ccObs_help2}, \eqref{eqn:lrcObs3} and \eqref{eqn:rttObs} for the centralized control, least restrictive control and robust trajectory tracking methods.

The induced obstacles, $\ioset_i^j(t)$, are then given by the states that $\veh_j$ can reach while avoiding the intruder, starting from some state in $\boset_j(\tsa), \tsa \in [t-\iat, t]$. These states can be obtained by computing an FRS from the base obstacles.
\begin{equation} \label{eq:FRS_intObs1}
\begin{aligned}
\frs_{j}^{\mathcal{O}}(t-\tau, t) = & \{y: \exists \ctrl_j(\cdot) \in \cfset_j, \exists \dstb_j(\cdot) \in \dfset_j, \\
& \state_j(\cdot) \text{ satisfies \eqref{eq:dyn}}, \state_j(t-\tau) \in \boset_j(t-\tau), \\
& \state_j(t) = y\}.
\end{aligned}
\end{equation}
$\frs_{j}^{\mathcal{O}}(t-\tau, t)$ represents the set of all possible states that $\veh_j$ can reach after a duration of $\tau$ starting from inside $\boset_j(t-\tau)$. This FRS can be obtained by solving the HJ VI in \eqref{eq:HJIVI_FRS} with the following Hamiltonian:
\begin{equation}
\ham_{j}^{\mathcal{O}}(\state_j, \costate) = \max_{\ctrl_j \in \cset_j} \max_{\dstb_j \in \dset_j} \costate \cdot f_j (\state_j, \ctrl_j, \dstb_j) \label{intobs2}.
\end{equation} 
Since $\tau \in [0, \iat]$, the induced obstacles can be obtained as:
\begin{equation} \label{eq:intObs}
\begin{aligned}
\ioset_i^j(t) & = \{\state_i: \exists y \in \pfrs_j(t), \|\pos_i - y\|_2 \le \rc \}\\
\pfrs_j(t) & = \{p_j: \exists \npos_j, (p_j, \npos_j) \in \bigcup_{\tau \in [0, \iat]} \frs_{j}^{\mathcal{O}}(t-\tau, t) \}
\end{aligned}
\end{equation}

Note that by the definition of base obstacles, $\boset_j(t+\tau_2) \subseteq \frs_{j}^{\text{BO}}(t+\tau_1, t+\tau_2) ~\forall t, \tau_2 > \tau_1$, where $\frs_{j}^{\text{BO}}(t+\tau_1, t+\tau_2)$ denotes the FRS of $\boset_j(t+\tau_1)$ computed for a duration of $\tau_2-\tau_1$. Therefore, we have that $\frs_{j}^{\mathcal{O}}(t-\tau, t) \subseteq \frs_{j}^{\mathcal{O}}(t-\iat, \iat) ~\forall \tau \in [0, \iat]$. Thus, $\pfrs_j(t)$ in \eqref{eq:intObs} can be equivalently written as
\begin{equation} \label{eq:intObs_help1}
\pfrs_j(t) = \{p_j: \exists \npos_j, (p_j, \npos_j) \in \frs_{j}^{\mathcal{O}}(t-\iat, t) \}.
\end{equation}

%
%Since there are no moving vehicle obstacles for the highest priority vehicle, $\obsset_1(t) = \soset$. 
%
%Computation of these base obstacles would requires information of a corresponding ``base" BRS of $\veh_j$; the process for computing this set is outlined in Step 2. In this section, we assume that the sequence of base obstacles, $\boset_i^j(t)$, is known. Given $\boset_i^j(t)$, we now show how to compute the obstacle set $\obsset_i(t)$. The induced obstacles are given by the states a vehicle can reach while avoiding the intruder, on top of the base obstacles. 

\subsubsection{Augmented Obstacle Computation} \label{sec:intruder_aocomp}
We next need to ensure that $\veh_i$ doesn't collide with the obstacles $\obsset_i(\cdot)$ computed in Section \ref{sec:intruder_iocomp} even when it is avoiding the intruder. In particular, we can compute a region around the obstacles $\obsset_i(\cdot)$ such that for all disturbances, $\veh_i$ can avoid colliding with obstacles for $\iat$ seconds regardless of its avoidance control, if $\veh_i$ starts outside this region. Augmenting $\obsset_i(\cdot)$ with this region gives us the augmented obstacles, $\tilde\obsset_i(\cdot)$, that can then be used during the trajectory planning of $\veh_i$ to ensure collision avoidance with $\obsset_i(\cdot)$.  

Suppose that the intruder appears in the system at some time time $\tsa = t - \iat + \tau, \tau \in [0, \iat]$. In this case, we need to ensure that $\veh_i$ does not collide with the obstacle $\obsset_i(t + \tau)$ at time $t + \tau$, regardless of its control $\ctrl_i(s)$ and disturbance $\dstb_i(s)$ for the time interval $s \in [\tsa, t + \tau]$. It is, therefore, sufficient to avoid the $\tau$-horizon BRS of $\obsset_i(t + \tau)$ at time $t$. This argument applies for all $\tau \in [0, \iat]$. Mathematically,

%\begin{equation} \label{eqn:inducedobs}
%\tilde\obsset_i(t) = \bigcup_{\tau \in [0, \iat]} \brs_{\mathcal{G}}(\tau, \obsset_i(t+\tau), \emptyset, \ham_{\mathcal{G}})
%\end{equation}
%where $\brs_{\mathcal{G}}(\tau, \obsset_i(t+\tau), \emptyset, \ham_{\mathcal{G}})$ represents BRS of $\obsset_i(t+\tau)$ computed backwards for $\tau$ seconds. The Hamiltonian 
%$\ham_{\mathcal{G}}$ is given by:
\begin{equation} \label{eqn:inducedobs}
\tilde\obsset_i(t) = \bigcup_{\tau \in [0, \iat]} \brs^{\mathcal{G}}_{i}(t, t+\tau)
\end{equation}
where $\brs^{\mathcal{G}}_{i}(t, t+\tau)$ represents BRS of $\obsset_i(t+\tau)$ computed backwards for $\tau$ seconds. Formally, 
\begin{equation} \label{eqn:inducedobs_help1}
\begin{aligned}
\brs^{\mathcal{G}}_{i}(t, t+\tau) = & \{y: \exists \ctrl_i(\cdot) \in \cfset_i, \exists \dstb_i(\cdot) \in \dfset_i, \\
& \state_i(\cdot) \text{ satisfies \eqref{eq:dyn}}, \state_i(t) = y, \\
& \exists s \in [t, t+\tau], \state_i(s) \in \obsset_i(s)\}.
\end{aligned}
\end{equation}

The Hamiltonian $\ham^{\mathcal{G}}_{i}$ to compute $\brs^{\mathcal{G}}_{i}(\cdot)$ is given by:
\begin{equation} \label{eqn:BRS_obsham}
\ham^{\mathcal{G}}_{i}(\state_i, \costate) = \min_{\ctrl_i \in \cset_i} \min_{\dstb_i \in \dset_i} \costate \cdot f_i (\state_i, \ctrl_i, \dstb_i)
\end{equation}

\begin{remark}
Note that if we use the robust trajectory tracking method to compute the base obstacles, we would need to augment the obstacles in \eqref{eqn:inducedobs} by the error bound of $\veh_i$, $\disckernel_i$, as discussed in section \ref{sec:rtt}.
\end{remark}

Finally, we compute a BRS $\brs^{\text{AO}}_{i}(t, \sta_i)$ for trajectory planning that contains the initial state of $\veh_i$ while avoiding these augmented obstacles:
\begin{equation} \label{eqn:intrBRS1}
\begin{aligned}
\brs^{\text{AO}}_{i}(t, \sta_i) = & \{y: \exists \ctrl_i(\cdot) \in \cfset_i, \forall \dstb_i(\cdot) \in \dfset_i, \\
& \state_i(\cdot) \text{ satisfies \eqref{eq:dyn}}, \forall s \in [t, \sta_i], \state_i(s) \notin \tilde\obsset_i(s), \\
& \exists s \in [t, \sta_i], \state_i(s) \in \targetset_i, \state_i(t) = y \}.
\end{aligned}
\end{equation}
The Hamiltonian $\ham^{\text{AO}}_{i}$ to compute BRS in \eqref{eqn:intrBRS1} is given by:
\begin{equation} \label{eqn:BRSham}
\ham^{\text{AO}}_{i}(\state_i, \costate) = \min_{\ctrl_i \in \cset_i} \max_{\dstb_i \in \dset_i} \costate \cdot f_i (\state_i, \ctrl_i, \dstb_i)
\end{equation}

Note that $\brs^{\text{AO}}_{i}(\cdot)$ ensures goal satisfaction for $\veh_i$ in the absence of intruder. The goal satisfaction controller is given by:
\begin{equation}
{\ctrl^{\text{AO}}_{i}}(t, \state_i) = \arg \min_{\ctrl_i \in \cset_i} \max_{\dstb_i \in \dset_i} \costate \cdot f_i (\state_i, \ctrl_i, \dstb_i)
\end{equation}
Moreover, if $\veh_i$ starts within $\brs^{\text{AO}}_{i}$, it is guaranteed to avoid collision for a duration of $\iat$, starting at any $\tsa < \sta_i$, irrespective of the control and disturbance applied during this time period. 

\subsubsection{Optimal Avoidance Controller} \label{sec:intruder_avoid}
First, we define relative dynamics of the intruder $\veh_{\intr}$ with state $\state_\intr$ with respect to $\veh_i$ with state $\state_i$.

\begin{equation}
\label{eq:reldyn}
\begin{aligned}
\state_{\intr, i} &= \state_\intr - \state_i \\
\dot \state_{\intr, i} &= f_r(\state_{\intr, i}, \ctrl_i, \ctrl_\intr, \dstb_i, \dstb_\intr)
\end{aligned}
\end{equation}

Given the relative dynamics, we compute the set of states from which the joint states of $\veh_{\intr}$ and $\veh_i$ can enter danger zone $\dz_{i\intr}$ despite the best efforts of $\veh_i$ to avoid $\veh_{\intr}$. This set of states is given by the BRS $\brs^{\text{CA}}(t, \iat),~ t \in [0, \iat]$:%(\iat, \targetset_\text{CA}, \obsset_\text{CA}, H_\text{CA})$, with

\begin{equation} \label{eqn:optAvoid}
\begin{aligned}
\brs^{\text{CA}}_{i}(t, \iat) = & \{y: \forall \ctrl_i(\cdot) \in \cfset_i, \exists \ctrl_\intr(\cdot) \in \cfset_\intr, \exists \dstb_i(\cdot) \in \dfset_i, \\
& \exists \dstb_\intr(\cdot) \in \dfset_\intr, \state_{\intr, i}(\cdot) \text{ satisfies \eqref{eq:reldyn}},\\
& \exists s \in [t, \iat], \state_{\intr, i}(s) \in \targetset^{\text{CA}}_{i}, \state_{\intr, i}(t) = y\},
\end{aligned}
\end{equation}
where $\targetset^{\text{CA}}_{i} = \{\state_{\intr, i}: \|\pos_{\intr, i}\|_2 \le \rc\}$, and the Hamiltonian for computing this BRS is given by

\begin{equation*}
\begin{aligned}
&H^{\text{CA}}_{i}(\state_{\intr, i}, \costate) = \max_{\ctrl_i \in \cset_i} \Big( \\
&\qquad \qquad \qquad \min_{\ctrl_\intr \in \cset_\intr, \dstb_i \in \dset_i, \dstb_\intr \in \dset_\intr} \costate \cdot f_r(\state_{\intr, i}, \ctrl_i, \ctrl_\intr, \dstb_i, \dstb_\intr) \Big)
\end{aligned}
\end{equation*}

Once the value function $\valfunc^{\text{CA}}_{i}(t, \state_{\intr, i})$ corresponding to the BRS $\brs^{\text{CA}}_{i}(t, \iat)$ is computed, the optimal avoidance control ${\ctrl^{\text{CA}}_{i}}$ can be obtained as:
\begin{equation} \label{eqn:optAvoidCtrl}
\begin{aligned}
&{\ctrl^{\text{CA}}_{i}}(t, \state_i, \state_\intr)  = \arg \max_{\ctrl_i \in \cset_i} \Big( \\
&\qquad \qquad \qquad \min_{\ctrl_\intr \in \cset_\intr, \dstb_i \in \dset_i, \dstb_\intr \in \dset_\intr} \costate \cdot f_r(\state_{\intr, i}, \ctrl_i, \ctrl_\intr, \dstb_i, \dstb_\intr) \Big)
\end{aligned}
\end{equation}

Under normal circumstances when the intruder $\veh_{\intr}$ is far away, we have $\valfunc^{\text{CA}}_{i}(0, \state_{\intr, i}) > 0$; as $\veh_{\intr}$ gets closer to $\veh_i$, $\valfunc^{\text{CA}}_{i}(0, \state_{\intr, i})$ decreases. If $\veh_i$ applies the control ${\ctrl^{\text{CA}}_{i}}$ when $\valfunc^{\text{CA}}_{i}(0, \state_{\intr, i}) = 0$, then collision avoidance between $\veh_i$ and $\veh_{\intr}$ is guaranteed for a duration of $\iat$ under the worst-case intruder control strategy.

In addition, obstacle augmentation \eqref{eqn:inducedobs} ensures that $\veh_i$ does not collide with $\obsset_i(\cdot)$ during the avoidance maneuver. %Therefore, applying $\ctrl_I^A$ for a duration of $\iat$ is still guaranteed to keep $\veh_i$ safe from all obstacles, and hence safe from collision with respect to all other vehicles $\veh_j, j \neq i$.
The overall control policy for avoiding the intruder and collision with other vehicles is thus given by:
\begin{equation*}
{\ctrl^{\text{A}}_{i}}(t) = 
\left \{ 
\begin{array}{ll}
{\ctrl^{\text{AO}}_{i}}(t, \state_i) & t \leq \tsa\\
{\ctrl^{\text{CA}}_{i}}(t, \state_i, \state_\intr) & \tsa\leq t \leq \tea
\end{array}
\right.
\end{equation*}

\subsubsection{Replanning after intruder avoidance\label{sec:re-plan_method1}} 
After the intruder disappears, goal satisfaction controllers which ensure that the vehicles reach their destinations can be obtained by solving an STP problem as described in Section \ref{sec:incomp}, where the starting states of the vehicles are now given by the states they end up in, denoted $\tilde{\state}_j^0$, after avoiding the intruder. Let the optimal control policy corresponding to this goal satisfaction controller be denoted ${\ctrl^{\text{L}}_{i}}(t, \state_i)$. The overall control policy that ensures intruder avoidance, collision avoidance with other vehicles, and successful transition to the destination is given by:

\begin{equation*}
\ctrl_i^*(t) = 
\left \{ 
\begin{array}{ll}
{\ctrl^{\text{A}}_{i}}(t, \state_i) & t \leq \tea\\
{\ctrl^{\text{L}}_{i}}(t, \state_i) & t > \tea
\end{array}
\right.
\end{equation*}

Note that in order to re-plan using a STP method, we need to determine feasible $\sta_i$ for all vehicles. This can be done by computing an FRS:
\begin{equation} \label{eq:re-planFRS}
\begin{aligned} 
\frs_i^{\text{RP}}(\tea, t) = & \{y \in \R^{n_i}: \exists \ctrl_i(\cdot) \in \cfset_i, \forall \dstb_i(\cdot) \in \dfset_i, \\
& \state_i(\cdot) \text{ satisfies \eqref{eq:dyn}}, \state_i(\tea) = \tilde{\state}_i^0, \\
& \state_i(t) = y, \forall s \in [\tea, t], \state_i(s) \notin \obsset_i^{\text{RP}}(s) \},
\end{aligned}
\end{equation}
\noindent where $\tilde{\state}_i^0$ represents the state of $\veh_i$ at $t = \tea$; $\obsset_i^{\text{RP}}(\cdot)$ takes into account the fact that $\veh_i$ needs to avoid higher-priority vehicles $\veh_j, j<i$ and is defined in an way analogous to \eqref{eq:obsseti}.
%\begin{equation} 
%\obsset_j^{\text{RP}}(t) = \bigcup_{k<j} \frs_k^{\text{RP}}(\tea, t), ~~~t > \tea.
%\end{equation}

The FRS in \eqref{eq:re-planFRS} can be obtained by solving %the HJ VI in \eqref{eq:HJIVI_FRS} with the following Hamiltonian:

\begin{equation}
\begin{aligned}
\max \Big\{&D_t \valfuncfwd^{\text{RP}}(t, \state_i) + \ham_i^{\text{RP}}(t, \state_i, \nabla \valfuncfwd^{\text{RP}}(t, \state_i)), \\
&\qquad - \obsfunc^{\text{RP}}(t, \state_i) - \valfuncfwd^{\text{RP}}(t, \state_i) \Big\} = 0\\
&\valfuncfwd^{\text{RP}}(\tsa, \state_i) = \max\{\fc^{\text{RP}}(\state_i), -\obsfunc^{\text{RP}}(\tsa, \state_i)\} \\
&\ham_i^{\text{RP}}(\state_i, \costate) = \max_{\ctrl_i \in \cset_i} \min_{\dstb_i \in \dset_i} \costate \cdot f_i (\state_i, \ctrl_i, \dstb_i)
\end{aligned}
\end{equation} 

\noindent where $\valfuncfwd^{\text{RP}}, \obsfunc^{\text{RP}}, \fc^{\text{RP}}$ represent the FRS, obstacles during re-planning, and the initial state of $\veh_i$, respectively. The new $\sta$ of $\veh_j$ is now given by the earliest time at which $\frs_j^{\text{RP}}(\tea, t)$ intersects the target set $\targetset_j$, $\sta_j := \arg \inf_t \{ \frs_j^{\text{RP}}(\tea, t) \cap \targetset_j \neq \emptyset \}$. Intuitively, this means that there exists a control policy which will steer the vehicle to its destination by that time, despite the worst case disturbance it might experience.

\begin{remark}
Note that we only need to re-plan the trajectories of the vehicles that are affected by the intruder. In particular, if $\valfunc^{\text{CA}}(0, \state_{\intr, i}(t)) > 0$ during the entire duration $t \in [\tsa, \tea]$ for a vehicle, then the vehicle would need not to apply any avoidance control, and hence re-planning would not be required for this vehicle. 
\end{remark}

\begin{remark}
In general, an intruder can be present in the system for much longer than $\iat$, as long as it is not affecting the STP vehicles. $\tsa$ thus really corresponds to the time an intruder starts affecting a STP vehicle.
\end{remark}

\begin{remark}
Note that even though we have presented the analysis for one intruder, the proposed method can handle multiple intruders as long as only one intruder is present \textit{at any given time}. 
\end{remark}

We conclude this section with the overall STP algorithm that takes into account an intruder that may appear for a duration of $\iat$: 
\begin{alg}
\label{alg:intruder}
\textbf{Intruder Avoidance algorithm (offline planning)}: Given initial conditions $\state_i^0$, vehicle dynamics \eqref{eq:dyn}, intruder dynamics in Assumption \ref{as:dynKnown}, target sets $\targetset_i$, and static obstacles $\soset_i, i = 1\ldots, \N$, for each $i$,
\begin{enumerate}
\item determine the total obstacle set $\obsset_i(t)$, given in \eqref{eq:obsseti}. In the case $i=1$, $\obsset_i(t) = \soset_i ~ \forall t$;
\item compute the augmented obstacle set $\tilde\obsset_i(t)$ given by \eqref{eqn:inducedobs}, where $\brs^{\mathcal{G}}_{i}(0, \tau)$ is given by \eqref{eqn:inducedobs_help1};
\item given $\tilde\obsset_i(t)$, compute the BRS $\brs^{\text{AO}}_{i}(t, \sta_i)$ defined in \eqref{eqn:intrBRS1};
\item the optimal control to avoid the intruder can be obtained by computing $\brs^{\text{CA}}_{i}(t, \iat)$ in \eqref{eqn:optAvoid} and using \eqref{eqn:optAvoidCtrl};
\item the induced obstacles $\ioset_k^i(t)$ for each $k>i$ can be computed using \eqref{eq:intObs}.
\end{enumerate}

\textbf{Intruder Avoidance algorithm (online re-planning)}: For each vehicle $i$ which performed avoidance in response to the intruder,
\begin{enumerate}
\item compute $\frs_i^{\text{RP}}(\tea, t)$ using $\eqref{eq:re-planFRS}$. The new $\sta_i$ for $\veh_i$ is given by $\arg \inf_t \{ \frs_j^{\text{RP}}(\tea, t) \cap \targetset_j \neq \emptyset \}$;
\item given $\sta_i$, $\tilde{\state}_i^0$, vehicle dynamics \eqref{eq:dyn}, target set $\targetset_i$, and static obstacles $\soset_i, i = 1\ldots, \N$, use any of the three STP methods discussed in Section \ref{sec:incomp} for re-planning. 
\end{enumerate}
\end{alg}
%% !TEX root = SPP_journal.tex
\subsection{Method 2: Single vehicle re-planning \label{sec:intruder_method2}}
As an alternative to Method 1, one can think about dividing the flight space of vehicles such that at any given time, any two vehicles are far enough from each other such that an intruder can only affect one vehicle in a duration of $\iat$ despite its best efforts. The advantage of this approach is that after the intruder disappears, we only have to re-plan the trajectory of a single vehicle regardless of the number of total vehicles in the system, which makes this approach particularly suitable for practical systems, as the re-planning needs to be done in real time.

In this method, we build upon this intuition and show that such a division of space is indeed possible. In Section \ref{sec:spaceDiv}, we show that regardless of the initial position of the intruder, atmost one vehicle needs to apply the avoidance maneuver under this space division. However, we still need to ensure that a vehicle do not collide with another vehicle while avoiding the intruder. The induces obstacles that reflect this possibility are computed in Section \ref{sec:intruder_iocomp_met2}. Intruder avoidance control and re-planning are discussed in Section \ref{sec:replan_method2}.

\subsubsection{Space division \label{sec:spaceDiv}}
Let $\state_r$ denotes the relative co-ordinates of $\veh{\intr}$ with respect to $\veh{j}$ as in \eqref{eq:reldyn}. Given the relative dynamics, we compute the set of states from which the joint states of $\veh{\intr}$ and $\veh{j}$ can enter danger zone $\dz_{j\intr}$ when both $\veh{j}$ and $\veh{\intr}$ are taking \textit{best actions to collide} with each other. Under the relative dynamics \eqref{eq:reldyn}, this set of states is given by the backwards reachable set $\brs_{\text{C},j}(\iat, \targetset_\text{C}, \emptyset, H_\text{C})$, with

\begin{equation}
\begin{aligned}
\targetset_{\text{C},j} &= \{\state_r: \|\pos_r\|_2 \le \rc\} \\
H_\text{C}(\state_r, p) &= \min_{\ctrl_j, \ctrl_\intr, \dstb_j, \dstb_\intr} p \cdot f_r(\state_r, \ctrl_j, \ctrl_\intr)
\end{aligned}
\end{equation}

The interpretation of set $\brs_{\text{C},j}$ is that if the $\veh{\intr}$ starts outside this set (in relative co-ordinates), then no matter what control $\veh{\intr}$ and $\veh{j}$ apply for the next $\iat$ seconds, they can't collide with each other. Thus, $\veh{j}$ need not avoid such an intruder and can apply any desired control. In other words, $\veh{j}$ should avoid the intruder if intruder starts inside $\brs_{\text{C},j}$, otherwise should not. We can similarly compute $\brs_{\text{C},i}$, which denotes the region around $\veh{i}$, where it needs to avoid the intruder. 

We now go back to the absolute coordinates and explain how we can achieve the desired space division. We first compute the base obstacles $\boset^j(t)$ induced by $\veh{j}$ as disucssed in Section \ref{sec:intruder_iocomp}. Once the base obstacles are computed, we augment every base obstacle with $\brs_{\text{C},j}$. This can be achieved by taking Minkowski sum of $\boset^j(t)$ and $\brs_{\text{C},j}$:
\begin{equation}
^1\intobs^j(t) = \boset^j(t) + \brs_{\text{C},j}.
\end{equation}


However, to ensure that atmost one vehicle needs to apply the avoidance maneuver, we need to make sure that no other vehicle $\veh{i}, i \neq j$, needs to avoid the intruder if intruder starts inside $\brs_\text{C}$. This can  be achieved by ensuring that $\brs_{\text{C},i}$ and $\brs_{\text{C},j}$ do not intersect.
% !TEX root = ../SPP_journal.tex
\subsection{Intruder Results \label{sec:basic_results}}
To illustrate that our SPP method is robust with respect to disturbances as well as a single intruder that is present for $\iat$, we use a five-vehicle example in which one of the five vehicles is an intruder. We assume that each vehicles have the dynamics given in \eqref{eq:dyn_i}. For this example, we chose the parameters $\underline{v} = 0.1, \bar{v} = 1, \bar\omega = 1$, and disturbance bounds $d_{r} = 0.1, \bar{d_{\theta}} = 0.2$, which correspond to a 10\% uncertainty in the dynamics. 

The vehicles' initial states, scheduled times of arrival, and target sets are the same as those described in Section \ref{sec:sim_dstb}, except for this example, we have increased the target radius to $r=0.15$. For illustrate purposes, we have chosen to use the robust trajectory tracking method described in Section \ref{sec:rtt} for disturbance rejection, and hence each vehicle tracks a nominal trajectory.

Fig. \ref{fig:intruder1_traj} shows the simulation at $t = -2.39$, which corresponds to the time at which the intruder ``disappears'' from the domain. This time is chosen to highlight the maximum impact of the intruder. Here, the intruder is shown in black, and the SPP vehicles are shown in the other different colors.

By this time instant $t = -2.39$, vehicle $\veh_2$ (red) and vehicle $\veh_3$ (green) have been avoiding the intruder for some time. This is evident from the amount of deviation between the actual positions of vehicles $\veh_2$ and $\veh_3$ and their nominal position specified by the nominal trajectories they were suppose to track in the absence of the intruder; these vehicles have abandoned nominal trajectory tracking in order to ensure safety with respect to the intruder. In contrast, $\veh_4$, which has not needed to avoid the intruder, is tracking the nominal trajectory very closely.

One may notice that the SPP vehicles are rather far apart. This is because a large margin is needed to ensure their separation even when multiple vehicles need to avoid the intruder. In this example in particular, the lowest-priority vehicle $\veh_4$ needed to depart very early compared to $\veh_2$ and $\veh_3$ so that in the situation in which an intruder arrives, $\veh_4$ does not impede the ability of the other vehicles to perform avoidance. The early departure of $\veh_4$ can be inferred from the fact that at $t=-2.39$, it is already nearly at its target.

For the same reason, the position of highest-priority vehicle $\veh_1$ has not departed from its initial state yet, and thus is not shown at $t=-2.39$. $\veh_2$ and $\veh_3$ needed to depart very early compared to $\veh_1$ to ensure sufficient margin for avoidance maneuvers.

Fig. \ref{fig:intruder1_diff} shows the nominal (black) and actual trajectories (red and green respectively) of vehicles $\veh_2$ (top subplot) and $\veh_3$ (bottom subplot). Specifically, the $x$ and $y$ positions over time are shown, and the black dotted vertical lines indicate the time interval in which the intruder is present. From Fig. \ref{fig:intruder1_diff}, one can clearly see that before the intruder was present, both vehicles are tracking their nominal trajectories closely. When the intruder appeared, the vehicles began deviating from their nominal trajectories significantly. After the intruder disappeared, both vehicles replanned new trajectories according to Section \ref{sec:replan_method1}, and the resulting actual trajectories eventually arrive at the same location as the nominal trajectories at a later time.

\begin{figure}[H]
  \centering
  \includegraphics[width=\columnwidth]{"fig/intruder1_traj"}
  \caption{The positions of the SPP vehicles and the intruder at $t=-2.39$, the end of the intruder's appearance. The red and green vehicles $\veh_2, \veh_3$ have not been tracking their nominal trajectories for a while, and are avoiding the intruder instead. Thus, their positions are far away from their nominal trajectories. $\veh_4$ has not needed to avoid the intruder, and tracks its nominal trajectory closely. The nominal trajectory of $\veh_4$ allows it to stay far enough away from other vehicles so that all vehicles are safe in the presence of the intruder.}
  \label{fig:intruder1_traj}
\end{figure}

\begin{figure}[H]
  \centering
  \includegraphics[width=\columnwidth]{"fig/intruder1_diff"}
  \caption{The difference between the initially planned nominal trajectories and the actual trajectories for vehicles $\veh_2$ (top subplot) and $\veh_3$ (bottom subplot). which needed to perform avoidance with respect to the intruder during the time interval marked by the vertical black dotted lines. Before the intruder's presence, both vehicles track their nominal trajectories closely; however, both vehicles later deviate significantly from their nominal trajectories in order to avoid the intruder. After the intruder is gone, both vehicles replan their trajectories and arrive at their targets at a later time.}
  \label{fig:intruder1_diff}
\end{figure}

% !TEX root = STP_journal.tex
\section{Conclusions and Future Work}
Guaranteed-safe multi-vehicle trajectory planning is a challenging problem, and previous analyses often either require strong assumptions on the motion of the vehicles or result in a large degree of conservatism. Differential game techniques such as Hamilton-Jacobi (HJ) reachability are ideally suited for guaranteeing goal satisfaction and safety under disturbances, but become intractable for even a small number of vehicles.

Our robust sequential trajectory planning (STP) method assigns a strict priority ordering to vehicles to offer a tractable and practical approach to the multi-vehicle trajectory planning problem. Under the proposed method, a portion of ``space-time'' is reserved for vehicles in the airspace in descending priority order to allow for dense vehicle configurations. Unlike previous priority-based methods, our approach accounts for disturbances and an adversarial intruder. STP reduces the scaling of HJ reachability's computational complexity from exponential to linear with respect to the number of vehicles, while maintaining hard guarantees on goal satisfaction and safety under disturbances. In the presence of a single intruder vehicle, STP still guarantees goal satisfaction and safety with a quadratically scaling computational complexity.

In the future, we plan to investigate ways of guaranteeing a maximum number of vehicles that need to re-plan, combine reachability analysis with other trajectory and path planning methods to improve computation speed, and to better understand the scenarios under which the STP scheme is the most useful by running large-scale simulations.
\vspace{-0.2cm}
%%%%%%%%%%%%%%%%%%%%%%%%%%%%%%%%%%%%%%%%%%%%%%%%%%%%%%%%%%%%%%%%%%%%%%%%%%%%%%%%
%\addtolength{\textheight}{1cm}   % This command serves to balance the column lengths
                                  % on the last page of the document manually. It shortens
                                  % the textheight of the last page by a suitable amount.
                                  % This command does not take effect until the next page
                                  % so it should come on the page before the last. Make
                                  % sure that you do not shorten the textheight too much.

\bibliographystyle{IEEEtran}
\bibliography{IEEEabrv,references}
\end{document}
