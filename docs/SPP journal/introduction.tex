% !TEX root = SPP_journal.tex
\section{Introduction}
Recently, there has been an immense surge of interest in using unmanned aerial systems (UASs) for civil purposes. UASs have great potential for applications such as package delivery, aerial surveillance, disaster response, and many other important tasks \cite{Tice91, Debusk10, Amazon16, AUVSI16, BBC16}. Unlike previous uses of UASs for military purposes, civil applications of UASs will involve unmanned aerial vehicles (UAVs) flying in urban environments, potentially in close proximity of humans and other important assets. As a result, government agencies such as the Federal Aviation Administration (FAA) and National Aeronautics and Space Administration (NASA) of the United States are urgently trying to develop new scalable ways to organize an air space in which potentially thousands of UAVs can fly together \cite{FAA13, NASA16, Kopardekar16}.

One essential problem that needs to be addressed is how a group of vehicles in the same vicinity can reach their destinations while avoiding situations which are considered dangerous. In some previous studies that address this problem, specific control strategies for the vehicles are assumed, and approaches such as induced velocity obstacles have been used \cite{Fiorini98, Chasparis05, Vandenberg08}. Other researchers have used ideas involving virtual potential fields to maintain collision avoidance while maintaining a specific formation \cite{Saber02, Chuang07}. 

Although interesting results emerge from the studies above, simultaneous trajectory planning and collision avoidance were not considered. The capability to flexibly plan provably safe trajectories without making strong assumptions on other vehicles' motion is essential for dense groups of UAVs flying in each other's vicinity. In addition, in a practical setting, any path planning and collision avoidance scheme must guarantee the liveness and safety of UAVs despite disturbances such as weather effects and communication faults \cite{} \MCnote{Check PK's UTM paper}. Furthermore, unexpected scenarios such as UAV malfunctions or even UAVs with malicious intent need to be accounted for.

The problem of trajectory planning and collision avoidance under disturbances in safety-critical systems have been studied using reachability analysis, which provides guarantees on the success and safety of optimal system trajectories \cite{Barron90, Mitchell05, Bokanowski10, Bokanowski11, Margellos11, Fisac15}. Reachability-based methods are particular suitable for in the context of UAVs because of the hard guarantees that are provided. In reachability analysis, one computes the reachable set, defined as the set of states from which the system can be driven to a target set. Reachability analysis has been successfully used in applications involving systems with no more than two vehicles, such as pairwise collision avoidance \cite{Mitchell05}, automated in-flight refueling \cite{Ding08}, and many others \cite{Huang11, Bayen07}. 

One of the main challenges of managing the next generation of airspace \MCnote{check PK's paper for terminalogy}is the density of vehicles that needs to be accommodated. Such a large-scale has a high-dimensional joint state space, making dynamic programming-based approaches such as reachability analysis intractable. In particular, reachable set computations involve solving a Hamilton-Jacobi (HJ) partial differential equation (PDE) on a grid representing a discretization of the state space, causing computation complexity to scale exponentially with system dimension. 

In this paper, we propose the sequential path planning (SPP) method to tackle the multi-vehicle path planning problem. Our method scales linearly with the number of vehicles, and provides hard guarantees on both the safety and liveness of all vehicles. These guarantees are maintained even under the presence of disturbances and a single intruder vehicle that could potentially be malicious. Thus, the curse of dimensionality is partially overcome for the multi-vehicle path planning problem. 

On a high level, the SPP method assigns a strict priority ordering to the vehicles under consideration. Higher-priority vehicles plan their paths without taking into account the lower-priority vehicles. Lower-priority vehicles treat higher-priority vehicles as moving obstacles. Under this assumption, time-varying formulations of reachability \cite{} can be used to obtain the optimal and provably safe paths for each vehicle, starting from the highest-priority vehicle. In a sense, the SPP method reserves a portion of the ``space-time'' in the airspace for each vehicle. The reservation is designed in a way such that each vehicle can robustly remain inside their reserved ``space-time'' under disturbances and the presence of a single intruder, arrive at some set of goal states on time, and be guaranteed to not enter any other reserved space-time.

We will present the SPP method as follows:
\begin{itemize}
\item In Sections \ref{sec:formulation} and \ref{sec:HJIVI}, we formally present the path planning problem and definitions specific to the SPP method. In addition, we provide a brief overview of a time-varying formulation of reachability that we use in the SPP method.
\item In Section \ref{sec:basic}, we present the basic SPP method while assuming that each vehicle has perfect information about other vehicles' control strategies, and that no disturbances are present. We illustrate the basic SPP method in a simulation.
\item In Section \ref{sec:obs_gen}, we make the basic SPP method robust to the presence of disturbances and incomplete information about other vehicles' control strategies. Here, we propose three different methods to provide this robustness. The three methods differ in their assumptions on the information available to each vehicle. In addition, we also provide simulation results.
\item Finally, in Section \ref{sec:intruder}, we show how the SPP method can be made robust against an intruder on top of disturbances. We prove that assuming the intruder may only be present for a maximum duration, all vehicles can still safely reach their targets on time. We provide a simulation results validating our theory.
\end{itemize}
