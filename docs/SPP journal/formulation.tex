% !TEX root = SPP_journal.tex
\section{Problem Formulation \label{sec:formulation}}
Consider $\N$ vehicles which participate in the SPP process and denote these \textit{SPP vehicles} $\veh{i}, i = 1, \ldots, \N$. We assume their dynamics are given by

\begin{equation}
\begin{aligned}
\dot\state_i &= \fdyn_i(\state_i, \ctrl_i, \dstb_i), t \in [\edt, \sta] \\
\state_i(\edt) &= \state^0_i \\
\ctrl_i &\in \cset_i \\
\dstb_i &\in \dset_i \\
i &= 1 \ldots, \N
\end{aligned}
\end{equation}

\noindent where $\state_i \in \R^{n_i}$ represents the state of vehicle $\veh{i}$, $\ctrl_i \in \cset_i$ the control of $\veh{i}$, and $\dstb_i \in \dset_i$ the disturbance experienced by $\veh{i}$. For convenience, we partition the state $\state_i$ into the position component $\pos_i \in \R^{n_\pos}$ and the non-position component $\npos_i \in \R^{n_i - n_\pos}$: $\state_i = (\pos_i, \npos_i)$.

Each vehicle $\veh{i}$ has initial state $\state^0_i$, and departs after $\edt$, its earliest departure time. $\veh{i}$ aims to reach its target $\targetset_i$ by some scheduled time of arrival $\sta_i$. The target in general represents some set of desirable states, for example the destination of $\veh{i}$. On its way to $\targetset_i$, $\veh{i}$ must avoid a set of static obstacles $\sosetp \subset \R^{n_\pos}$. The interpretation of $\sosetp$ could be a tall building, a region around an airport, or any set of positions that are forbidden for the set of SPP vehicles. In terms of the state space $\state_i = (\pos_i, \npos_i)$ of vehicle $\veh{i}$, the forbidden set induced by the static obstacles $\sosetp$ is given by

\begin{equation}
\begin{aligned}
\soset_i = \{\state_i \mid p_i \in \sosetp\}
\end{aligned}
\end{equation}

In addition to the static obstacles, each vehicle $\veh{i}$ must also avoid the danger zones with respect to each of the other vehicle $\veh{j}, j\neq i$. The danger zones in general can represent any joint configuration between $\veh{i}$ and $\veh{j}$ that is considered to be unsafe. In this paper, we define the danger zone of $\veh{i}$ with respect to $\veh{j}$ to be

\begin{equation}
\dz_{ij} = \{(\state_i, \state_j): \dist(\pos_i, \pos_j) \le \rc\}
\end{equation}

\noindent whose interpretation is that $\veh{i}$ and $\veh{j}$ are considered to be in an unsafe configuration when they are within a distance of $\rc$ of each other.

The control of each vehicle $\veh{i}$ is constrained

Compute:
\begin{itemize}
\item See older SPP papers
\item $\ldt_i$
\end{itemize}