% !TEX root = SPP_journal.tex
\section{Response to Intruders \label{sec:HJIVI}}
\begin{assumption}
\label{as:avoidOnce}
Only one intruder, avoid for a duration of $\iat$
\end{assumption}

\MCnote{BRSs and FRSs probably also need to be specified using subscripts to avoid having to write all the arguments all the time.}

\begin{itemize}
\item Definitions of 3 different reachable sets (2 additional reachable sets $\frs(\iat, \brs(\ldt, \targetset))$, $\brs(\bar t, \targetset)$ where $\bar t$ is the time vehicle stops avoiding intruder)

\item Steps involved in computing the reach-avoid sets for vehicles:
\begin{itemize}
\item \textit{Step-1}: Compute the set of obstacles induced by the higher priority vehicles. 
\item \textit{Step-2}: Append \textit{all} obstacles (including static ones) by a $\iat$-step FRS. Compute the BRS with these obstacles till it contains the initial state of the vehicle. This set will give us the controller that ensures liveness in the absence of intruders. 
\item \textit{Step-3}: Compute the $\iat$-step FRS for the BRS computed in Step-2. This FRS is the free flight region that will allow us to avoid any intruder for $\iat$ seconds. 
\item \textit{Step-4}: Compute the BRS taking into account the obstacles induced by the higher priority vehicles (computed in Step-1) \textit{and} that contains the FRS calculated in Step-3. This BRS will give us a controller that guarantee intruder avoidance and liveness. 
\item \textit{Step-5}: Compute the relative state space wherein we can successfully avoid the intruder for $\iat$ seconds. 
\end{itemize}
\end{itemize}

%\subsection{Obstacle Augmentation}
%\begin{itemize}
%\item For path planning
%\item For lower priority vehicles
%\end{itemize}
%
%\subsection{Avoidance Controller}
%\begin{itemize}
%\item Reachable set computation
%\item Controller
%\end{itemize}
%
%\subsection{Liveness Controller}
%\begin{itemize}
%\item Path planning using augmented obstacles
%\item Overall controller
%\end{itemize}

% !TEX root = ../SPP_journal.tex
\subsection{Step 1}
Backwards reachable set: $\brs(t, \targetset_i, \obsset_i, \ham_\text{nom})$
% !TEX root = ../SPP_journal.tex
\subsection{Step 2: Augmented Obstacle Computation}
The goal of this section is to compute a ``safety region" around the obstacles $\obsset_i(.)$ such that no matter what $\veh{i}$ and disturbance do, it doesn't collide with any obstacle for $\iat$ seconds, if it starts outside this safety region. The utility of computing such a region will become clearer in Step 3. 

We let $\tilde\obsset_i(t)$ the obstacles augmented by this region. To ensure that a vehicle does not hit the obstacle $\obsset_i(t_1 + t')$ at time $t = t_1 + t'$ starting at $t = t_1$, even when it applies any control for the next $t'$ seconds, it suffices to avoid the BRS of $\obsset_i(t_1 + t')$ computed backwards for $t'$ seconds. This argument applies for all obstacles to appear in the next $\iat$ seconds to ensure safety under any controller and disturbance for the next $\iat$ seconds. Mathematically,

\begin{equation} \label{eqn:inducedobs}
\tilde\obsset_i(t) = \bigcup_{\tau \in [0, \iat]} \brs_{\mathcal{G}}(\tau, \obsset_i(t+\tau), \emptyset, \ham_{\mathcal{G}})
\end{equation}
where $\brs_{\mathcal{G}}(\tau, \obsset_i(t+\tau), \emptyset, \ham_{\mathcal{G}})$ represents BRS of $\obsset_i(t+\tau)$ computed backwards for $\tau$ seconds. The Hamiltonian 
$\ham_{\mathcal{G}}$ is given by:

\begin{equation} \label{eqn:BRSham}
\ham_{\mathcal{G}}(\state_i, p) = \min_{\ctrl_i} \max_{\dstb_i} p \cdot f_i (\state_i, \ctrl_i, \dstb_i)
\end{equation}

Finally, we compute a backwards reachable set that contains the initial state of $\veh{i}$ while these augmented obstacles, $\brs_\text{AO}(t, \targetset_i, \bar\obsset_i, \ham_\text{AO})$, where $\ham_\text{AO} = \ham_{\mathcal{G}}$. Note that this BRS ensures liveness for $\veh{i}$ in the absence of intruder. Moroever, if we start from $\brs_\text{AO}$, we are guaranteed to avoid collision for $\iat$ seconds, irrespective of control and disturbance applied. 
%% !TEX root = ../SPP_journal.tex
\subsubsection{Step 3: Intruder Avoidance \label{sec:intruder_after}}
\MCnote{This section doesn't seem like intruder avoidance}
Under Assumption \ref{as:avoidOnce}, a vehicle $\veh{i}$ does not need to avoid an intruder more than once. Thus, after performing an avoidance maneuver, $\veh{i}$ can simply resume moving towards its destination given by $\targetset_i$ while avoiding the (un-augmented) obstacles $\obsset_i(t)$ in \eqref{eq:obsseti}\MCnote{Add equation reference}. The optimal controller for this task is obtained from the BRS $\brs_\text{L}(t, \targetset_i, \obsset_i, \ham_\text{liveness}$) given in Section \ref{sec:basic}.

\subsubsection{Time-to-reach bound after intruder avoidance}
Before avoiding the intruder $\veh{\intr}$, vehicle $\veh{i}$ uses the controller $\ctrl^*_\text{AO}(t, \state_i; \targetset_i, \tilde\obsset_i, \ham_\text{AO})$, which by construction keeps its state $\state_i$ within the BRS $\brs_\text{AO}(t, \targetset_i, \tilde\obsset_i, \ham_\text{AO})$. When $\veh{\intr}$ interferes, $\veh{i}$ is forced to perform an avoidance maneuver for a duration of up to $\iat$ in order to avoid collisions. 

Regardless of what control is used for avoidance, $\veh{i}$ still remains within the FRS 

\begin{equation}
\frs_\text{APS}(\iat, \brs(t, \targetset_i, \obsset_i, \ham_\text{nom}), \emptyset, \ham_\text{APS}), 
\end{equation}

\noindent which is the set of all possible states that $\veh{i}$ can reach starting from inside $\brs_\text{AO}(t)$. Here, the Hamiltonian is given by

\begin{equation}
\ham_\text{APS}(\state_i, p) = \min_{\ctrl_i} \min_{\dstb_i} p \cdot f_i (\state_i, \ctrl_i, \dstb_i)
\end{equation}

Let $\wcttr$ be the smallest time horizon for the BRS $\brs(\wcttr, \targetset_i, \obsset_i, \ham_\text{L})$ such that it contains $\frs(\iat, \brs(t, \targetset_i, \obsset_i, \ham_\text{nom}), \emptyset, \ham_\text{APS})$ entirely. Then, $\wcttr$ would be an upper bound on the time that $\veh{i}$ needs to safely arrive at $\targetset_i$ after avoiding the intruder $\veh{\intr}$.
% !TEX root = ../SPP_journal.tex
\subsection{Step 4}
Under Assumption \ref{as:avoidOnce}, a vehicle $\veh{i}$ does not need to avoid an intruder more than once. Thus, after performing an avoidance maneuver, $\veh{i}$ can simply resume moving towards its destination given by $\targetset_i$ while avoiding the (un-augmented) obstacles $\obsset_i(t)$ in \eqref{}. The optimal controller for this task is obtained from the backwards reachable set $\brs(t, \targetset_i, \obsset_i, H_\text{liveness}$) where $H_\text{liveness}(t, x, p)$ is given in Section \ref{sec:basic}.

\subsubsection{Time-to-reach bound after intruder avoidance}

% !TEX root = ../SPP_journal.tex
\subsection{Step 5: Optimal Avoidance Controller}
First, we define relative coordinates of the intruder $\veh{\intr}$ with state $\state_\intr$ with respect to $\veh{i}$ with state $\state_i$.

\begin{equation}
\label{eq:reldyn}
\begin{aligned}
\state_r &= \state_\intr - \state_i \\
\dot \state_r &= f_r(\state_r, \ctrl_i, \ctrl_\intr)
\end{aligned}
\end{equation}

Given the relative dynamics, we compute the set of states from which the joint states of $\veh{\intr}$ and $\veh{i}$ can enter danger zone $\dz_{i\intr}$ despite the best efforts of $\veh{i}$ to avoid $\veh{\intr}$. Under the relative dynamics \eqref{eq:reldyn}, this set of states is given by the backwards reachable set $\brs(\iat, \targetset_\text{CA}, \obsset_\text{CA}, H_\text{CA})$, with

\begin{equation}
\begin{aligned}
\targetset_\text{CA} &= \{\state_r: \|\pos_r\|_2 \le \rc\} \\
\obsset_\text{CA} &= \emptyset \\
H_\text{CA}(\state_r, p) &= \max_{\ctrl_i} \min_{\ctrl_\intr} p \cdot f_r(\state_r, \ctrl_i, \ctrl_\intr)
\end{aligned}
\end{equation}

Once the value function $\valfunc_\text{CA}(t, \state_r)$ representing $\brs_\text{CA}(\iat, \targetset_\text{CA}, \obsset_\text{CA}, H_\text{CA})$ is computed, the optimal avoidance control $\ctrl^*_\text{CA}$ can be derived from \eqref{eq:opt_ctrl}.

Under normal circumstances when the intruder $\veh{\intr}$ is far away, we have $\valfunc_\text{CA}(\iat, \state_r) > 0$; as the $\veh{\intr}$ gets closer to $\veh{i}$, $\valfunc_\text{CA}(\iat, \state_r)$ decreases. If $\veh{i}$ applies the control $\ctrl^*_\text{CA}$ when $\valfunc_\text{CA}(\iat, \state_r) = 0$, then collision avoidance between $\veh{i}$ and $\veh{\intr}$ is guaranteed for a duration of $\iat$ under the worst-case intruder control strategy $\ctrl_\intr$.

In addition, under Assumption \ref{as:avoidOnce}, we have $\state_i \in \brs(\bar t, \targetset_i, \bar\obsset(t))$. Therefore, applying $\ctrl_I^A$ for a duration of $\iat$ is still guaranteed to keep $\veh{i}$ safe from all obstacles, and hence safe from collision with respect to all other vehicles $\veh{j}, j \neq i$.
