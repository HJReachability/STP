% !TEX root = SPP_journal.tex
\section{Response to Intruders \label{sec:HJIVI}}
\begin{assumption}
\label{as:avoidOnce}
Only one intruder, avoid for a duration of $\iat$
\end{assumption}

\begin{itemize}
\item Definitions of 3 different reachable sets (2 additional reachable sets $\frs(\iat, \brs(\ldt, \targetset))$, $\brs(\bar t, \targetset)$ where $\bar t$ is the time vehicle stops avoiding intruder)

\item Steps involved in computing the reach-avoid sets for vehicles:
\begin{itemize}
\item \textit{Step-1}: Compute the set of obstacles induced by the higher priority vehicles. 
\item \textit{Step-2}: Append \textit{all} obstacles (including static ones) by a $\iat$-step FRS. Compute the BRS with these obstacles till it contains the initial state of the vehicle. This set will give us the controller that ensures liveness in the absence of intruders. 
\item \textit{Step-3}: Compute the $\iat$-step FRS for the BRS computed in Step-2. This FRS is the free flight region that will allow us to avoid any intruder for $\iat$ seconds. 
\item \textit{Step-4}: Compute the BRS taking into account the obstacles induced by the higher priority vehicles (computed in Step-1) \textit{and} that contains the FRS calculated in Step-3. This BRS will give us a controller that guarantee intruder avoidance and liveness. 
\item \textit{Step-5}: Compute the relative state space wherein we can successfully avoid the intruder for $\iat$ seconds. 
\end{itemize}
\end{itemize}

%\subsection{Obstacle Augmentation}
%\begin{itemize}
%\item For path planning
%\item For lower priority vehicles
%\end{itemize}
%
%\subsection{Avoidance Controller}
%\begin{itemize}
%\item Reachable set computation
%\item Controller
%\end{itemize}
%
%\subsection{Liveness Controller}
%\begin{itemize}
%\item Path planning using augmented obstacles
%\item Overall controller
%\end{itemize}

% !TEX root = ../SPP_journal.tex
\subsection{Step 1: Induced Obstacle Computation \label{sec:intruder_iocomp}}
The goal of this section is to compute, for each lower priority vehicle $\veh{i}$, the time-varying obstacle induced by each higher priority vehicle $\veh{j}, j < i$, denoted by $\ioset_i^j(t)$. As before, once $\ioset_i^j(t)$ is computed, one can then solve \eqref{eq:HJIVI_i} with the union of all obstacles induced by higher priority vehicles as the total obstacle set $\obsset_i(t)$. 

Note that since there are no moving vehicle obstacles for the highest priority vehicle, $\obsset_1(t) = \soset$. To compute the obstacle set $\ioset_i^j(t)$ where $i> 1$, we first compute the ``base" obstacles using any of the three methods outlined in Section \ref{sec:incomp}. Base obstacles correspond to the states which a vehicle can reach when it is not avoiding an intruder. Computation of these base obstacles would requires information of a corresponding ``base" BRS of $\veh{j}$; the process for computing this set is outlined in Step 2. In this section, we assume that the sequence of base obstacles, $\boset_i^j(t)$, is known. Given $\boset_i^j(t)$, we now show how to compute the obstacle set $\obsset_i(t)$. 

The induced obstacles are given by the states a vehicle can reach while avoiding the intruder, on top of the base obstacles. Since a vehicle avoids the intruder for a maximum of $\iat$, these states can be given by the $\iat$-horizon FRS of the base obstacle. Regardless of what control is used by $\veh{j}$ for avoidance, it still remains within the FRS $\ioset_i^j(t) := \frs_{\mathcal{O}}(\iat, \boset_i^j(t-\iat), \emptyset, \ham_{\mathcal{O}})$, which is the set of all possible states that $\veh{j}$ can reach after a duration of $\iat$ starting from inside $\boset_i^j(t-\iat)$. Here, the Hamiltonian is given by

\begin{equation}
\ham_{\mathcal{O}}(\state_j, p) = \min_{\ctrl_j} \min_{\dstb_j} p \cdot f_j (\state_j, \ctrl_j, \dstb_j)
\end{equation}

% !TEX root = ../SPP_journal.tex
\subsection{Step 2}
%\input{./intruder/step3}
% !TEX root = ../SPP_journal.tex
\subsection{Step 4}
Under Assumption \ref{as:avoidOnce}, a vehicle $\veh{i}$ does not need to avoid an intruder more than once. Thus, after performing an avoidance maneuver, $\veh{i}$ can simply resume moving towards its destination given by $\targetset_i$ while avoiding the (un-augmented) obstacles $\obsset_i(t)$ in \eqref{}\MCnote{Add equation reference}. The optimal controller for this task is obtained from the backwards reachable set $\brs(t, \targetset_i, \obsset_i, \ham_\text{liveness}$) where $H_\text{liveness}(t, x, p)$ is given in Section \ref{sec:basic}.

\subsubsection{Time-to-reach bound after intruder avoidance}
Before avoiding the intruder $\veh{\intr}$, vehicle $\veh{i}$ uses the controller $\ctrl^*(t, \state; \targetset_i, \obsset_i, \ham_\text{nom})$, which by construction keeps its state $\state_i$ within the backwards reachable set $\brs(t, \targetset_i, \obsset_i, \ham_\text{nom})$ by construction. When $\veh{\intr}$ interferes, $\veh{i}$ is forced to perform an avoidance maneuver for a duration of up to $\iat$ in order to avoid collisions. 

Regardless of what control is used for avoidance, $\veh{i}$ still remains within $\frs(\iat, \brs(t, \targetset_i, \obsset_i, \ham_\text{nom}), \emptyset, \ham_\text{APS})$, where

\begin{equation}
\ham_\text{APS}(\state_i, p) = \min_{\ctrl_i} \min_{\dstb_i} p \cdot f_i (\state_i, \ctrl_i, \dstb_i)
\end{equation}

Let $\wcttr$ be the smallest time horizon for the BRS $\brs(\wcttr, \targetset_i, \obsset_i, \ham_\text{liveness}$ such that it contains $\frs(\iat, \brs(t, \targetset_i, \obsset_i, \ham_\text{nom}), \emptyset, \ham_\text{APS})$ entirely. Then, $\wcttr$ would be an upper bound on the time that $\veh{i}$ needs to safely arrive at $\targetset_i$.
% !TEX root = ../SPP_journal.tex
\subsection{Step 5: Optimal Avoidance Controller}
First, we define relative coordinates of the intruder $\veh{\intr}$ with state $\state_\intr$ with respect to $\veh{i}$ with state $\state_i$.

\begin{equation}
\label{eq:reldyn}
\begin{aligned}
\state_r &= \state_\intr - \state_i \\
\dot \state_r &= f_r(\state_r, \ctrl_i, \ctrl_\intr)
\end{aligned}
\end{equation}

Given the relative dynamics, we compute the set of states from which the joint states of $\veh{\intr}$ and $\veh{i}$ can enter danger zone $\dz_{i\intr}$ despite the best efforts of $\veh{i}$ to avoid $\veh{\intr}$. Under the relative dynamics \eqref{eq:reldyn}, this set of states is given by the backwards reachable set $\brs(\iat, \targetset_\text{CA}, \obsset_\text{CA}, H_\text{CA})$, with

\begin{equation}
\begin{aligned}
\targetset_\text{CA} &= \{\state_r: \|\pos_r\|_2 \le \rc\} \\
\obsset_\text{CA} &= \emptyset \\
H_\text{CA}(\state_r, p) &= \max_{\ctrl_i} \min_{\ctrl_\intr} p \cdot f_r(\state_r, \ctrl_i, \ctrl_\intr)
\end{aligned}
\end{equation}

Once the value function $\valfunc_\text{CA}(t, \state_r)$ representing $\brs(\iat, \targetset_\text{CA}, \obsset_\text{CA}, H_\text{CA})$ is computed, the optimal avoidance control can be obtained as 

\begin{equation}
\ctrl_i^A = \arg \max_{\ctrl_i} \min_{\ctrl_\intr} \nabla\valfunc_\text{CA}(t,\state_r) \cdot f_r(\state_r, \ctrl_i, \ctrl_\intr)
\end{equation}

(``Factor'' optimal control to an earlier section?? This would be useful because it's difficult to specify the $t$ in $\valfunc$)

(Decide on notation for $\ctrl_i^A$)

If $\veh{i}$ applies the control $\ctrl_i^A$ when $\valfunc(\iat, \state_r) \le 0$, then collision avoidance between $\veh{i}$ and $\veh{\intr}$ is guaranteed for a duration of $\iat$ under the worst-case intruder control strategy $\ctrl_\intr$.

In addition, under Assumption \ref{as:avoidOnce}, we have $\state_i \in \brs(\bar t, \targetset_i, \bar\obsset(t))$. Therefore, applying $\ctrl_I^A$ for a duration of $\iat$ is still guaranteed to keep $\veh{i}$ safe from all obstacles, and hence safe from collision with respect to all other vehicles $\veh{j}, j \neq i$.
