% !TEX root = SPP_journal.tex
\section{Response to Intruders \label{sec:HJIVI}}
The sequential path planning algorithms presented in the previous sections guarantees safety only for the scenarios when no other vehicles are present in the system. If a new vehicle enters the system, or even worse if this vehicle is intruder, the old plan can lead to a collision. In this section, our goal is to design a trajectory plan that is robust to the presence of intruders. To keep pur analysis tractable, we make the following assumption:  

\begin{assumption} \label{as:avoidOnce}
Only one intruder is present in the system for a maximum duration of $\iat$. Moreover, dynamics of the intruder are known. 
\end{assumption}

Note that we do not assume anything about the time at which the intruder appears in the system; however, we assume that once it appears, it stays for a maximum duration of $\iat$. 

\begin{itemize}
\item \textbf{Definitions of 3 different reachable sets (2 additional reachable sets $\frs(\iat, \brs(\ldt, \targetset))$, $\brs(\bar t, \targetset)$ where $\bar t$ is the time vehicle stops avoiding intruder)}.
\end{itemize}

We are now ready to develop a general theory that takes intruders into account. Our approach to a robust path planning can be summarized in the following steps:
\begin{itemize}
\item \textit{Step-1}: First, we compute the set of obstacles induced by the higher priority vehicles for the lower priority vehicles.  
\item \textit{Step-2}: We then append \textit{all} obstacles (including static ones) by a forward reachable set (FRS) of duration $\iat$. This addendum ensures that if a vehicle starts outside this appended obstacle, then it cannot collide with the obstacle in $\iat$ seconds. Next, we compute a backwards reachable set (BRS) while avoiding these obstacles till it contains the initial state of the vehicle. This set will give us the controller that ensures liveness in the absence of intruders. 
\item \textit{Step-3}: Compute a $\iat$-step FRS of the BRS computed in Step-2. This FRS is the free flight region that allows a vehicle to avoid any intruder for $\iat$ seconds. We then compute a BRS while avoding the obstacles induced by the higher priority vehicles (computed in Step-1) \textit{and} that contains the FRS calculated in Step-3. This BRS will give us a controller that guarantee intruder avoidance and liveness. 
\item \textit{Step-4}: Compute the relative state space wherein we can successfully avoid the intruder for $\iat$ seconds. If a vehicle sense an intruder while it is still outside this region, the above algorithm will ensure safety at all times.  
\end{itemize}

%\subsection{Obstacle Augmentation}
%\begin{itemize}
%\item For path planning
%\item For lower priority vehicles
%\end{itemize}
%
%\subsection{Avoidance Controller}
%\begin{itemize}
%\item Reachable set computation
%\item Controller
%\end{itemize}
%
%\subsection{Liveness Controller}
%\begin{itemize}
%\item Path planning using augmented obstacles
%\item Overall controller
%\end{itemize}

% !TEX root = ../SPP_journal.tex
\subsection{Step 1: Induced Obstacle Computation \label{sec:intruder_iocomp}}
The goal of this section is to compute, for each lower priority vehicle $\veh{i}$, the time-varying obstacle induced by each higher priority vehicle $\veh{j}, j < i$, denoted by $\ioset_i^j(t)$. As before, once $\ioset_i^j(t)$ is computed, one can then solve \eqref{eq:HJIVI_i} with the union of all obstacles induced by higher priority vehicles as the total obstacle set $\obsset_i(t)$. 

Note that since there are no moving vehicle obstacles for the highest priority vehicle, $\obsset_1(t) = \soset$. To compute the obstacle set $\ioset_i^j(t)$ where $i> 1$, we first compute the ``base" obstacles using any of the three methods outlined in Section \ref{sec:incomp}. Base obstacles correspond to the states which a vehicle can reach when it is not avoiding an intruder. Computation of these base obstacles would requires information of a corresponding ``base" BRS of $\veh{j}$; the process for computing this set is outlined in Step 2. In this section, we assume that the sequence of base obstacles, $\boset_i^j(t)$, is known. Given $\boset_i^j(t)$, we now show how to compute the obstacle set $\obsset_i(t)$. 

The induced obstacles are given by the states a vehicle can reach while avoiding the intruder, on top of the base obstacles. Since a vehicle avoids the intruder for a maximum of $\iat$, these states can be given by the $\iat$-horizon FRS of the base obstacle. Regardless of what control is used by $\veh{j}$ for avoidance, it still remains within the FRS $\ioset_i^j(t) := \frs_{\mathcal{O}}(\iat, \boset_i^j(t-\iat), \emptyset, \ham_{\mathcal{O}})$, which is the set of all possible states that $\veh{j}$ can reach after a duration of $\iat$ starting from inside $\boset_i^j(t-\iat)$. Here, the Hamiltonian is given by

\begin{equation}
\ham_{\mathcal{O}}(\state_j, p) = \min_{\ctrl_j} \min_{\dstb_j} p \cdot f_j (\state_j, \ctrl_j, \dstb_j)
\end{equation}

% !TEX root = ../SPP_journal.tex
\subsection{Step 2}
\input{./intruder/step3}
% !TEX root = ../SPP_journal.tex
\subsection{Step 4}
Under Assumption \ref{as:avoidOnce}, a vehicle $\veh{i}$ does not need to avoid an intruder more than once. Thus, after performing an avoidance maneuver, $\veh{i}$ can simply resume moving towards its destination given by $\targetset_i$ while avoiding the (un-augmented) obstacles $\obsset_i(t)$ in \eqref{}\MCnote{Add equation reference}. The optimal controller for this task is obtained from the backwards reachable set $\brs(t, \targetset_i, \obsset_i, \ham_\text{liveness}$) where $H_\text{liveness}(t, x, p)$ is given in Section \ref{sec:basic}.

\subsubsection{Time-to-reach bound after intruder avoidance}
Before avoiding the intruder $\veh{\intr}$, vehicle $\veh{i}$ uses the controller $\ctrl^*(t, \state; \targetset_i, \obsset_i, \ham_\text{nom})$, which by construction keeps its state $\state_i$ within the backwards reachable set $\brs(t, \targetset_i, \obsset_i, \ham_\text{nom})$ by construction. When $\veh{\intr}$ interferes, $\veh{i}$ is forced to perform an avoidance maneuver for a duration of up to $\iat$ in order to avoid collisions. 

Regardless of what control is used for avoidance, $\veh{i}$ still remains within $\frs(\iat, \brs(t, \targetset_i, \obsset_i, \ham_\text{nom}), \emptyset, \ham_\text{APS})$, where

\begin{equation}
\ham_\text{APS}(\state_i, p) = \min_{\ctrl_i} \min_{\dstb_i} p \cdot f_i (\state_i, \ctrl_i, \dstb_i)
\end{equation}

Let $\wcttr$ be the smallest time horizon for the BRS $\brs(\wcttr, \targetset_i, \obsset_i, \ham_\text{liveness}$ such that it contains $\frs(\iat, \brs(t, \targetset_i, \obsset_i, \ham_\text{nom}), \emptyset, \ham_\text{APS})$ entirely. Then, $\wcttr$ would be an upper bound on the time that $\veh{i}$ needs to safely arrive at $\targetset_i$.

The overall control policy for avoiding intruder and reaching target is thus given by:
\begin{equation*}
u^* = 
\left \{ 
\begin{array}{ll}
\ctrl^*_\text{AO}(t, \state_i; \targetset_i, \tilde\obsset_i, \ham_\text{AO}) & t \leq \underbar{t}\\
\ctrl^*_\text{CA}(t, \state_r; \targetset_\text{CA}, \obsset_\text{CA}, \ham_\text{CA}) & \underbar{t} \leq t \leq \bar{t} \\
\ctrl^*(t, \state_i; \targetset_i, \obsset_i, \ham_\text{L}) & t \geq \bar{t}
\end{array}
\right.
\end{equation*}

\begin{itemize}
\item \textbf{Explain computation of the base obstacles}
\end{itemize}
