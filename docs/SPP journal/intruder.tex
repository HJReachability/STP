% !TEX root = SPP_journal.tex
\section{Response to Intruders \label{sec:HJIVI}}
In Section \ref{}, we made the basic SPP algorithm more robust by taking into account disturbances and considering situations in which vehicles may not have complete information about the control strategy of the other vehicles. 

If a vehicle not in the set of SPP vehicles enters the system, or even worse, if this vehicle is an adversarial intruder, the old plan can lead to a collision. If vehicles do not plan with additional safety margin that takes intruders into account, a vehicle trying to avoid the intruder may cause an unexpected conflict with another SPP vehicle, and effectively becoming an intruder itself. This may lead to a domino effect, causing multiple conflicts. In this section, we propose a method to allow vehicles to avoid an intruder while maintaining the SPP structure.

In general, the effect of intruders on the vehicles in structured flight can be entirely unpredictable, since the intruders in principle could be adversarial in nature, and the number of intruders could be arbitrary. Therefore, for our analysis to produce reasonable results, some assumptions about the intruders must be made.

\begin{assumption}
\label{as:avoidOnce}
At most one intruder affects the SPP vehicles.  The intruder exits the altitude level affecting the SPP vehicles after a duration of $\iat$.
\end{assumption}

Assumption \ref{as:avoidOnce} implies that any vehicle $\veh{i}$ would need to avoid the intruder $\veh{\intr}$ for a maximum duration of $\iat$; in addition, after avoiding the intruder, $\veh{i}$ can safely assume that it would not need to avoid another intruder. This assumption can be valid in situations where intruders are rare, and that some fail-safe or enforcement mechanism exists to force the intruder out of the altitude level affecting the SPP vehicles. Note that we make any assumptions about $\tsa$; however, we assume that once it appears, it stays for a maximum duration of $\iat$.

\begin{assumption}
\label{as:dynKnown}
Dynamics of the intruder are known and given by $\dot\state_\intr = f_\intr(\state_\intr, \ctrl_\intr, \dstb_\intr)$
\end{assumption}

Assumption \ref{as:dynKnown} is required for HJ reachability analysis. In situations where the dynamics of the intruder are not known exactly, a conservative model of the intruder may be used instead.

Suppose some vehicle $\veh{i}$ starts avoiding the intruder $\veh{\intr}$ at some time $t = \tsa$, and stops avoiding at $t = \tea$. When $t < \tsa$, $\veh{i}$ must plan its path taking into account the possibility that it may need to avoid an intruder $\veh{i}$. Since $\veh{i}$ may spend a duration of up to $\tsa$ performing avoidance, its induced obstacles $\ioset_k^i(t), k>i$ need to be computed in a way that reflects this possibility. The induced obstacles computation is discussed in Section \ref{sec:intruder_iocomp}.

We must also ensure that while avoiding the intruder, $\veh{i}$ does not collide with the total obstacle set $\obsset_i(t)$. This requires augmenting the total obstacle set to produce the augmented total obstacle $\tilde\obsset_i(t)$; the computation of $\tilde\obsset_i(t)$ and the controller that guarantees the avoidance of the augmented obstacles are discussed in Section \ref{sec:intruder_aocomp}.

When $t > \tea$, $\veh{i}$ no longer needs to take into account the possibility of any other intruders, and can simply avoid the unaugmented obstacles $\obsset_i(t)$ while getting to the target. The associated controller and the time-to-reach upper bound after intruder avoidance are discussed in Section \ref{sec:intruder_after}.

Finally, Section \ref{sec:intruder_avoid} describes how $\veh{i}$ can guarantee collision avoidance with the intruder.

\MCnote{I wonder if the material below is needed}
\begin{itemize}
\item \textbf{Definitions of 3 different reachable sets (2 additional reachable sets $\frs(\iat, \brs(\ldt, \targetset))$, $\brs(\bar t, \targetset)$ where $\bar t$ is the time vehicle stops avoiding intruder)}.
\end{itemize}

We are now ready to develop a general theory that takes intruders into account. Our approach to a robust path planning can be summarized in the following steps:
\begin{itemize}
\item \textit{Step-1}: First, we compute the set of obstacles induced by the higher priority vehicles for the lower priority vehicles.  
\item \textit{Step-2}: We then append \textit{all} obstacles (including static ones) by a forward reachable set (FRS) of duration $\iat$. This addendum ensures that if a vehicle starts outside this appended obstacle, then it cannot collide with the obstacle in $\iat$ seconds. Next, we compute a backwards reachable set (BRS) while avoiding these obstacles till it contains the initial state of the vehicle. This set will give us the controller that ensures liveness in the absence of intruders. 
\item \textit{Step-3}: Compute a $\iat$-step FRS of the BRS computed in Step-2. This FRS is the free flight region that allows a vehicle to avoid any intruder for $\iat$ seconds. We then compute a BRS while avoding the obstacles induced by the higher priority vehicles (computed in Step-1) \textit{and} that contains the FRS calculated in Step-3. This BRS will give us a controller that guarantee intruder avoidance and liveness. 
\item \textit{Step-4}: Compute the relative state space wherein we can successfully avoid the intruder for $\iat$ seconds. If a vehicle sense an intruder while it is still outside this region, the above algorithm will ensure safety at all times.  
\end{itemize}

%\subsection{Obstacle Augmentation}
%\begin{itemize}
%\item For path planning
%\item For lower priority vehicles
%\end{itemize}
%
%\subsection{Avoidance Controller}
%\begin{itemize}
%\item Reachable set computation
%\item Controller
%\end{itemize}
%
%\subsection{Liveness Controller}
%\begin{itemize}
%\item Path planning using augmented obstacles
%\item Overall controller
%\end{itemize}

% !TEX root = ../SPP_journal.tex
\subsection{Step 1}
Backwards reachable set: $\brs(t, \targetset_i, \obsset_i, \ham_\text{nom})$
% !TEX root = ../SPP_journal.tex
\subsection{Step 2: Augmented Obstacle Computation}
The goal of this section is to compute a ``safety region" around the obstacles $\obsset_i(.)$ such that no matter what $\veh{i}$ and disturbance do, it doesn't collide with any obstacle for $\iat$ seconds, if it starts outside this safety region. The utility of computing such a region will become clearer in Step 3. 

We let $\tilde\obsset_i(t)$ the obstacles augmented by this region. To ensure that a vehicle does not hit the obstacle $\obsset_i(t_1 + t')$ at time $t = t_1 + t'$ starting at $t = t_1$, even when it applies any control for the next $t'$ seconds, it suffices to avoid the BRS of $\obsset_i(t_1 + t')$ computed backwards for $t'$ seconds. This argument applies for all obstacles to appear in the next $\iat$ seconds to ensure safety under any controller and disturbance for the next $\iat$ seconds. Mathematically,

\begin{equation} \label{eqn:inducedobs}
\tilde\obsset_i(t) = \bigcup_{\tau \in [0, \iat]} \brs_{\mathcal{G}}(\tau, \obsset_i(t+\tau), \emptyset, \ham_{\mathcal{G}})
\end{equation}
where $\brs_{\mathcal{G}}(\tau, \obsset_i(t+\tau), \emptyset, \ham_{\mathcal{G}})$ represents BRS of $\obsset_i(t+\tau)$ computed backwards for $\tau$ seconds. The Hamiltonian 
$\ham_{\mathcal{G}}$ is given by:

\begin{equation} \label{eqn:BRSham}
\ham_{\mathcal{G}}(\state_i, p) = \min_{\ctrl_i} \max_{\dstb_i} p \cdot f_i (\state_i, \ctrl_i, \dstb_i)
\end{equation}

Finally, we compute a backwards reachable set that contains the initial state of $\veh{i}$ while these augmented obstacles, $\brs_\text{AO}(t, \targetset_i, \bar\obsset_i, \ham_\text{AO})$, where $\ham_\text{AO} = \ham_{\mathcal{G}}$. Note that this BRS ensures liveness for $\veh{i}$ in the absence of intruder. Moroever, if we start from $\brs_\text{AO}$, we are guaranteed to avoid collision for $\iat$ seconds, irrespective of control and disturbance applied. 
% !TEX root = ../SPP_journal.tex
\subsubsection{Step 3: Intruder Avoidance \label{sec:intruder_after}}
\MCnote{This section doesn't seem like intruder avoidance}
Under Assumption \ref{as:avoidOnce}, a vehicle $\veh{i}$ does not need to avoid an intruder more than once. Thus, after performing an avoidance maneuver, $\veh{i}$ can simply resume moving towards its destination given by $\targetset_i$ while avoiding the (un-augmented) obstacles $\obsset_i(t)$ in \eqref{eq:obsseti}\MCnote{Add equation reference}. The optimal controller for this task is obtained from the BRS $\brs_\text{L}(t, \targetset_i, \obsset_i, \ham_\text{liveness}$) given in Section \ref{sec:basic}.

\subsubsection{Time-to-reach bound after intruder avoidance}
Before avoiding the intruder $\veh{\intr}$, vehicle $\veh{i}$ uses the controller $\ctrl^*_\text{AO}(t, \state_i; \targetset_i, \tilde\obsset_i, \ham_\text{AO})$, which by construction keeps its state $\state_i$ within the BRS $\brs_\text{AO}(t, \targetset_i, \tilde\obsset_i, \ham_\text{AO})$. When $\veh{\intr}$ interferes, $\veh{i}$ is forced to perform an avoidance maneuver for a duration of up to $\iat$ in order to avoid collisions. 

Regardless of what control is used for avoidance, $\veh{i}$ still remains within the FRS 

\begin{equation}
\frs_\text{APS}(\iat, \brs(t, \targetset_i, \obsset_i, \ham_\text{nom}), \emptyset, \ham_\text{APS}), 
\end{equation}

\noindent which is the set of all possible states that $\veh{i}$ can reach starting from inside $\brs_\text{AO}(t)$. Here, the Hamiltonian is given by

\begin{equation}
\ham_\text{APS}(\state_i, p) = \min_{\ctrl_i} \min_{\dstb_i} p \cdot f_i (\state_i, \ctrl_i, \dstb_i)
\end{equation}

Let $\wcttr$ be the smallest time horizon for the BRS $\brs(\wcttr, \targetset_i, \obsset_i, \ham_\text{L})$ such that it contains $\frs(\iat, \brs(t, \targetset_i, \obsset_i, \ham_\text{nom}), \emptyset, \ham_\text{APS})$ entirely. Then, $\wcttr$ would be an upper bound on the time that $\veh{i}$ needs to safely arrive at $\targetset_i$ after avoiding the intruder $\veh{\intr}$.
% !TEX root = ../SPP_journal.tex
\subsection{Step 4}
Under Assumption \ref{as:avoidOnce}, a vehicle $\veh{i}$ does not need to avoid an intruder more than once. Thus, after performing an avoidance maneuver, $\veh{i}$ can simply resume moving towards its destination given by $\targetset_i$ while avoiding the (un-augmented) obstacles $\obsset_i(t)$ in \eqref{}. The optimal controller for this task is obtained from the backwards reachable set $\brs(t, \targetset_i, \obsset_i, H_\text{liveness}$) where $H_\text{liveness}(t, x, p)$ is given in Section \ref{sec:basic}.

\subsubsection{Time-to-reach bound after intruder avoidance}


The overall control policy for avoiding intruder and reaching target is thus given by:
\begin{equation*}
u^* = 
\left \{ 
\begin{array}{ll}
\ctrl^*_\text{AO}(t, \state_i; \targetset_i, \tilde\obsset_i, \ham_\text{AO}) & t \leq \underbar{t}\\
\ctrl^*_\text{CA}(t, \state_r; \targetset_\text{CA}, \obsset_\text{CA}, \ham_\text{CA}) & \underbar{t} \leq t \leq \bar{t} \\
\ctrl^*(t, \state_i; \targetset_i, \obsset_i, \ham_\text{L}) & t \geq \bar{t}
\end{array}
\right.
\end{equation*}

\begin{itemize}
\item \textbf{Explain computation of the base obstacles}
\end{itemize}
