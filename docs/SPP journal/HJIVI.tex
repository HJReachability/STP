% !TEX root = SPP_journal.tex
\section{Double-Obstacle Hamilton-Jacobi Variational Inequality \label{sec:HJIVI}}
The goal of reachability analysis is to compute either a backward reachable set (BRS) $\brs$, a forward reachable set (FRS) $\frs$, or a sequence of BRSs and FRSs, given some time-varying target set $\targetset(t)$, time-varying obstacle $\obsset(t)$, and the Hamiltonian function $\ham$ which captures the system dynamics as well as the roles of the control and disturbance. The BRS $\brs$ or FRS $\frs$ with given parameters $\targetset(\cdot), \obsset(\cdot), \ham$ will be denoted by

\begin{equation}
\begin{aligned}
\brs(t; \targetset(\cdot), \obsset(\cdot), \ham) &\quad \text{ (backward reachable set)}\\
\frs(t; \targetset(\cdot), \obsset(\cdot), \ham) &\quad \text{ (forward reachable set)}
\end{aligned}
\end{equation}

In the SPP scheme, a lower priority vehicle must avoid a set of moving obstacles on its way to the target. Reachability theory is a powerful tool for performing optimal path planning with hard guarantees on safety and performance under disturbances. For our application in SPP, we utilize the time-varying formulation in \cite{}, which accounts for the time-varying nature of systems without requiring augmentation of the state space with the time variable. In the formulation in \cite{}, a BRS is computed by solving the following \textit{final value} double-obstacle HJ VI:

\begin{equation}
\label{eq:HJIVI_BRS}
\begin{aligned}
\max \{ \min \{D_t \valfunc + \ham(t, \state, \nabla \valfunc(t, \state)), \fc(t, \state) - \valfunc(t, \state) \}, -\obsfunc(t, \state) - \valfunc(t, \state) \}, &\qquad t \in [0, T] \\
\valfunc(T, \state) = \max\{\fc(T, \state), -\obsfunc(T, \state)\}&
\end{aligned}
\end{equation}

In a similar fashion, the FRS is computed by solving the following \textit{initial value} double-obstacle HJ VI:

\begin{equation}
\label{eq:HJIVI_FRS}
\begin{aligned}
\max \{ \min \{D_t \valfuncfwd + \ham(t, \state, \nabla \valfuncfwd(t, \state)), \ic(t, \state) - \valfuncfwd(t, \state) \}, -\obsfunc(t, \state) - \valfuncfwd(t, \state) \}, &\qquad t \in [0, T] \\
\valfuncfwd(0, \state) = \max\{\fc(0, \state), -\obsfunc(0, \state)\}&
\end{aligned}
\end{equation}

In both \eqref{eq:HJIVI_BRS} and \eqref{eq:HJIVI_FRS}, the function \MCnote{This sentence is only correct if the notations for initial and final conditions are the same} $\ic(t, \state)$ is the implicit surface function representing the target set $\targetset(t) = \{\state \mid \ic(t, \state) \le 0\}$. Similarly, the function $\obsfunc(t, \state)$ is the implicit surface function representing the time-varying obstacles $\obsset(t) = \{\state \mid \obsfunc(t,\state)\le 0\}$. The BRS $\brs(t)$ and FRS $\frs(t)$ are given by

\begin{equation}
\label{eq:implicitValfuncs}
\begin{aligned}
\brs(t) &= \{\state: \valfunc(t, \state) \le 0\} \\
\frs(t) &= \{\state: \valfuncfwd(t, \state) \le 0 \}
\end{aligned}
\end{equation}

\MCnote{Not sure if this is needed, but putting it here for now} For clarity, sometimes we will denote value functions computed using the target set $\targetset(t)$, obstacle $\obsset(t)$, and Hamiltonian $\ham$ as 

\begin{equation}
\begin{aligned}
\valfunc(t, \state; \targetset, \obsset, \ham) &\quad \text{(Value function representing BRS)} \\
\valfuncfwd(t, \state; \targetset, \obsset, \ham) &\quad \text{(Value function representing FRS)}
\end{aligned}
\end{equation}

The Hamiltonian, $\ham(t, \state, \nabla \valfunc(t,\state))$ depends on the system dynamics, and the role of control and disturbance. As an example, suppose we are given a generic dynamical system $\dot \state = \fdyn(t, \state, \ctrl, \dstb)$ in which the control $\ctrl(\cdot)$ aims to reach the target $\targetset(t)$ while the disturbance aims to keep the system away from the target, then \MCnote{Should probably write down cost function to be clear}

\begin{equation}
\label{eq:ham_ex}
\begin{aligned}
\ham(t, \state, \nabla \valfunc(t,\state)) = \min_\ctrl \max_\dstb \nabla \valfunc(t,\state) \cdot \fdyn(t, \state, \ctrl, \dstb)
\end{aligned}
\end{equation}

In addition, given the Hamiltonian \eqref{eq:ham_ex}, the optimal control and disturbance can be obtained as the optimum of the optimization:

\begin{equation}
\label{eq:opt_ctrl_dstb_ex}
\begin{aligned}
\ctrl^*(t) &= \arg \max \min_\dstb \nabla \valfunc(t,\state) \cdot \fdyn(t, \state, \ctrl, \dstb)\\
\dstb^*(t) &= \arg \min \nabla \valfunc(t,\state) \cdot \fdyn(t, \state, \ctrl^*(t), \dstb)
\end{aligned}
\end{equation}

In anticipation of the many different reachable sets that will be computed, we will in general denote the optimal control and disturbance derived from a BRS $\brs(t; \targetset(\cdot), \obsset(\cdot), \ham)$ given the target set $\targetset(t)$, obstacles $\obsset(t)$, and Hamiltonian $\ham$ as 

\begin{equation}
\label{eq:opt_ctrl}
\begin{aligned}
\ctrl^*(t; \targetset(\cdot), \obsset(\cdot), \ham) &\quad \text{(optimal control)} \\
\dstb^*(t; \targetset(\cdot), \obsset(\cdot), \ham) &\quad \text{(optimal disturbance)}
\end{aligned}
\end{equation}

\MCnote{Only keep time and/or state arguments when notation gets too heavy; use subscripts to specify the different sets instead.}
