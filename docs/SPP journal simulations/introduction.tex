% !TEX root = ../SPP_IoTjournal.tex
\section{Introduction \label{sec:introduction}}
\SBnote{Reduce the number of `we's in the paper.}
Recently, there has been an immense surge of interest in the use of unmanned aerial systems (UASs) for civil applications. The applications include package delivery, aerial surveillance, disaster response, among many others \cite{Tice91, Debusk10, Amazon16, AUVSI16, BBC16}. Unlike previous uses of UASs for military purposes, civil applications will involve unmanned aerial vehicles (UAVs) flying in urban environments, potentially in close proximity of humans, other UAVs, and other important assets. As a result, government agencies such as the Federal Aviation Administration (FAA) and National Aeronautics and Space Administration (NASA) of the United States are urgently trying to develop new scalable ways to organize an airspace in which potentially thousands of UAVs can fly together \cite{FAA13, Kopardekar16}.

One essential problem that needs to be addressed for this endeavour to be successful is that of trajectory planning: how a group of vehicles in the same vicinity can reach their destinations while avoiding situations which are considered dangerous, such as collisions. Many previous studies address this problem under different assumptions. In some studies, specific control strategies for the vehicles are assumed, and approaches such as those involving induced velocity obstacles \cite{Fiorini98, Chasparis05, Vandenberg08,Wu2012} and involving virtual potential fields to maintain collision avoidance \cite{Olfati-Saber2002, Chuang07} have been used. Methods have also been proposed for real-time trajectory generation \cite{Feng-LiLian2002}, for path planning for vehicles with linear dynamics in the presence of obstacles with known motion \cite{Ahmadzadeh2009}, and for cooperative path planning via waypoints which do not account for vehicle dynamics \cite{Bellingham}. Other related work include those which consider only the collision avoidance problem without path planning. These results include those that assume the system has a linear model \cite{Beard2003, Schouwenaars2004, Stipanovic2007}, rely on a linearization of the system model \cite{Massink2001, Althoff2011}, assume a simple positional state space \cite{Lin2015}, and many others \cite{Lalish2008, Hoffmann2008, Chen2016}.

However, to make sure that a dense group of UAVs can safely fly in the close vicinity of each other, we need the capability to flexibly plan provably safe and dynamically feasible trajectories without making strong assumptions on the vehicles' dynamics and other vehicles' motion. Moreover, any trajectory planning scheme that addresses collision avoidance must also guarantee both goal satisfaction and safety of UAVs despite disturbances caused by wind and communication faults \cite{Kopardekar16}. Furthermore, unexpected scenarios such as UAV malfunctions or even UAVs with malicious intent need to be accounted for. Finally, the proposed scheme should scale well with the number of vehicles.

The problem of trajectory planning and collision avoidance under disturbances in safety-critical systems has been studied using Hamilton-Jacobi (HJ) reachability analysis, which provides guarantees on goal satisfaction and safety of optimal system trajectories \cite{Barron90, Mitchell05, Bokanowski10, Bokanowski11, Margellos11, Fisac15}. Reachability-based methods are particularly suitable in the context of UAVs because of the hard guarantees that are provided. In reachability analysis, one computes the reach-avoid set, defined as the set of states from which the system can be driven to a target set while satisfying (possibly time-varying) state constraints at all times. A major practical appeal of this approach stems from the availability of modern numerical tools, which can compute various definitions of reachable sets \cite{Sethian96, Osher02, Mitchell02, Mitchell07b}. These numerical tools, for example, have been successfully used to solve a variety of differential games, path planning problems, and optimal control problems. Concrete practical applications include aircraft auto-landing \cite{Bayen07}, automated aerial refueling \cite{Ding08}, MPC control of quadrotors \cite{Bouffard12}, and multiplayer reach-avoid games \cite{Huang11}. Despite its power, the approach becomes numerically intractable as the state space dimension increases. In particular, reachable set computations involve solving a HJ partial differential equation (PDE) or variational inequality (VI) on a grid representing a discretization of the state space, resulting in an \textit{exponential} scaling of computational complexity with respect to the dimensionality of the problem. Therefore, as such, dynamic programming-based approaches such as reachability analysis are not suitable for managing the next generation airspace, which is a large-scale system with a high-dimensional joint state space because of the high density of vehicles that needs to be accommodated \cite{Kopardekar16}.  

To overcome this problem, Sequential Path Planning (SPP) method has been proposed \cite{Chen15c}, in which vehicles are assigned a strict priority ordering. Higher-priority vehicles plan their paths without taking into account the lower-priority vehicles. Lower-priority vehicles treat higher-priority vehicles as moving obstacles. Under this assumption, time-varying formulations of reachability \cite{Bokanowski11, Fisac15} can be used to obtain the optimal and provably safe paths for each vehicle, starting from the highest-priority vehicle. Thus, the curse of dimensionality is overcome for the multi-vehicle path planning problem at the cost of a mild structural assumption, under which the computation complexity scales just \textit{linearly} with the number of vehicles. Authors in \cite{Bansal2017} and \cite{chen2016robust}, respectively, extend this method to the scenarios where disturbances and adversarial intruders are present in the system, resolving some of the practical challenges associated with the basic SPP algorithm in \cite{Chen15c}. The focus of these works, however, have mostly been on the theoretical development of SPP algorithm. Our focus in this work is instead on effectively demonstrating the linear scaling of SPP algorithm and showcasing its potential as a trajectory planning algorithm for large scale systems. In particular, our main contributions in this work are:
\begin{itemize}
\item we simulate a large-scale SPP system for two different space structures: when SPP algorithm is used for trajectory planning at a city level and at a regional level. For city level planning, we consider the San Francisco city in California, US and for regional level planning we consider the entire Bay area in California, US. The main differences emerge from the fact that the city level planning needs to take into account physcial obstacles (like buildings, etc.), and the origins and destinations are farther apart at a regional level.
\item we demonstrate the kind and the number of different vehicle trajectories that emerge out of a large scale SPP system between a given pair of origin and destination. Furthermore, we show how these trajectories change under different wind conditions and priority orderings.
\item we also show the reactivity of the control law obtained from SPP algorithm, e.g. the obtained control law is able to effectively counter the disturbances without requiring any communication with other vehicles.      
\end{itemize}

On the theoretical side, one of the limitations of the SPP algorithm proposed in \cite{chen2016robust} to account for adversarial intruders is that the algorithm might need to replan the trajectories for all SPP vehicles once the intruder disappears. Since the replanning is done in real-time, the proposed algorithm may not be scalable with the number of SPP vehicles, rendering the method unsuitable for practical implementation. In this work, we also propose a novel intruder avoidance algorithm, which will need to replan trajectories only for a fixed number of vehicles, irrespective of the total number of SPP vehicles, thus overcoming the limitations of the algorithm in \cite{chen2016robust}. Moreover, this number is a design parameter, which can be chosen based on the computational resources available during the run time. Finally, we illustrate this algorithm through a simulation at the city level.

Rest of the paper is organized as follows: in Section \ref{sec:formulation}, we formally present the SPP problem in the presence of disturbances. In Section \ref{sec:background}, we present a brief review of time-varying reachability and the Robust Trajectory Tracking (RTT) method proposed in \cite{Bansal2017} to account for disturbances. We also use this algorithm for all our simulations in this paper. In Section \ref{sec:intruder}, a novel algorithm to account for intruders has been proposed. All simulation results are in Section \ref{sec:simulations}.