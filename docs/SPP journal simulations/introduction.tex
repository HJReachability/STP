% !TEX root = STP_IoTjournal.tex
\section{Introduction \label{sec:introduction}}
Due to the recent surge of interest in the use of unmanned aerial systems (UASs) for civil applications such as package delivery, aerial surveillance, disaster response, among many others \cite{Tice91, Debusk10, Amazon16, AUVSI16, BBC16}, civilian airspace may in the near future contain up to thousands of unmanned aerial vehicles (UAVs), potentially in close proximity of humans, other UAVs, and other important assets. As a result, government agencies such as the Federal Aviation Administration (FAA) and National Aeronautics and Space Administration (NASA) of the United States are urgently trying to develop new scalable ways to organize an airspace in which potentially thousands of UAVs can fly together \cite{FAA13, Kopardekar16}.

One essential problem that needs to be addressed for this endeavor to be successful is that of trajectory planning: how a group of vehicles in the same vicinity can reach their destinations while avoiding situations which are considered dangerous, such as collisions. Many previous studies address this problem under different assumptions. In some studies, specific control strategies for the vehicles are assumed, and approaches such as those involving induced velocity obstacles \cite{Fiorini98, Chasparis05, Vandenberg08,Wu2012} and involving virtual potential fields to maintain collision avoidance \cite{Olfati-Saber2002, Chuang07} have been used. Methods have also been proposed for real-time trajectory generation \cite{Feng-LiLian2002}, for path planning for vehicles with linear dynamics in the presence of obstacles with known motion \cite{Ahmadzadeh2009}, and for cooperative path planning via waypoints which do not account for vehicle dynamics \cite{Bellingham}. Other related work include those which consider only the collision avoidance problem without path or trajectory planning. These results include those that assume the system has a linear model \cite{Beard2003, Schouwenaars2004, Stipanovic2007}, rely on a linearization of the system model \cite{Massink2001, Althoff2011}, assume a simple positional state space \cite{Lin2015}, and many others \cite{Lalish2008, Hoffmann2008, Chen2016}.

However, to make sure that a dense group of UAVs can safely fly in the close vicinity of each other, we need the capability to flexibly plan provably safe and dynamically feasible trajectories without making strong assumptions on the vehicles' dynamics and other vehicles' motion. Moreover, any trajectory planning scheme that addresses collision avoidance must also guarantee both goal satisfaction and safety of UAVs despite disturbances caused by wind and communication faults \cite{Kopardekar16}. Finally, the proposed scheme should scale well with the number of vehicles, as well as result in an intuitive airspace structure for humans to monitor and potentially adjust.

The problem of trajectory planning and collision avoidance under disturbances in safety-critical systems has been well-studied using Hamilton-Jacobi (HJ) reachability analysis, which provides guarantees on goal satisfaction and safety of optimal system trajectories \cite{Barron90, Mitchell05, Bokanowski10, Bokanowski11, Margellos11, Fisac15}. Reachability-based methods are particularly suitable in the context of UAVs because of the hard guarantees that are provided. In reachability analysis, one computes the reach-avoid set, defined as the set of states from which the system can be driven to a target set while satisfying (possibly time-varying) state constraints at all times. A major practical appeal of this approach stems from the availability of modern numerical tools, which can compute various definitions of reachable sets \cite{Sethian96, Osher02, Mitchell02, Mitchell07b}. These numerical tools, for example, have been successfully used to solve a variety of differential games, trajectory planning problems, and optimal control problems. Concrete practical applications include aircraft auto-landing \cite{Bayen07}, automated aerial refueling \cite{Ding08}, MPC control of quadrotors \cite{Bouffard12}, and multiplayer reach-avoid games \cite{Huang11}. Despite its power, the approach becomes numerically intractable as the state space dimension increases. In particular, reachable set computations involve solving a HJ partial differential equation (PDE) or variational inequality (VI) on a grid representing a discretization of the state space, resulting in an \textit{exponential} scaling of computational complexity with respect to the dimensionality of the problem. Therefore, dynamic programming-based approaches such as reachability analysis are not directly suitable for managing the next generation airspace, which is a large-scale system with a high-dimensional joint state space because of the high density of vehicles that needs to be accommodated \cite{Kopardekar16}.  

To overcome this problem, the Sequential Trajectory Planning (STP) method was proposed in \cite{Chen15c}. Here, vehicles are assigned a strict priority ordering. Higher-priority vehicles plan their trajectories without taking into account the lower-priority vehicles. Lower-priority vehicles treat higher-priority vehicles as moving obstacles. Under this assumption, time-varying formulations of reachability \cite{Bokanowski11, Fisac15} can be used to obtain the optimal and provably safe trajectories for each vehicle, starting from the highest-priority vehicle. Thus, the curse of dimensionality is overcome for the multi-vehicle trajectory planning problem at the cost of a mild structural assumption, under which the computation complexity scales just \textit{linearly} with the number of vehicles. Intuitively, STP algorithm allocates a space-time trajectory to each vehicle based on their priorities. The highest-priority vehicle gets to choose any space-time trajectory that does not collide with static obstacles, such as the optimal trajectory. The next vehicle's trajectory must not intersect with the trajectory of the highest-priority vehicle, and so on. Hence two vehicles can either follow same state tarjectory but at different times (referred to as \textit{time-separated} trajectories here on) or follow different state trajectories but at the same time (referred to as \textit{state-separated} trajectories here on), but not both. Finally, they can have different state trajectories at different times (referred to as \textit{state-time separated} trajectories here on). So by design, STP algorithm ensures that the space-time trajectories of the vehicles do not intersect, and hence a safe transition to destination is guaranteed for all vehicles.  

Authors in \cite{Bansal2017} and \cite{chen2016robust}, respectively, extend STP to the scenarios where disturbances and adversarial intruders are present in the system, resolving some of the practical challenges associated with the basic STP algorithm in \cite{Chen15c}. The focus of these works, however, have mostly been on the theoretical development of STP algorithm. Our focus in this work is instead on demonstrating the potential of STP algorithm as a provably safe trajectory planning algorithm for large-scale systems. In particular, our main contributions in this paper are as follows:
\begin{itemize}
\item We simulate large-scale multi-vehicle systems in two different urban environments under the presence of disturbances in vehicles' dynamics. First, the STP algorithm is used for trajectory planning at a city level and then at a regional level. For city level planning, we consider the city of San Francisco in California, USA, and for regional level planning we consider a part of San Francisco Bay Area in California, USA. The main differences between these two case studies are that the city level planning needs to take into account static physical obstacles such as tall buildings, whereas the origins and destinations are farther apart at the regional level. In both cases, we demonstrate that STP algorithm is able to design provably-safe trajectories despite the disturbances.
\item We demonstrate how different types of space-time trajectories emerge naturally out of STP algorithm between a given pair of origin and destination for different disturbance conditions and other problem parameters. These emerging behaviors, while being provably safe, are also intuitive and would facilitate human monitoring. 
\item We also show the reactivity of the control law obtained from STP algorithm. In other words, we show that the obtained control law is able to effectively counteract the disturbances in real-time without requiring any communication with other vehicles.
\end{itemize}

The rest of the paper is organized as follows: in Section \ref{sec:formulation}, we formally present the STP problem in the presence of disturbances. In Section \ref{sec:background}, we present a brief review of time-varying reachability, the basic STP algorithm \cite{Chen15c} in the absence of disturbances, and the Robust Trajectory Tracking (RTT) method \cite{Bansal2017} to account for disturbances. We also use this algorithm for all our simulations in this paper. Simulation results are in Sections \ref{sec:city_sim} and \ref{sec:bayArea_sim}.