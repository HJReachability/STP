% !TEX root = ../SPP_IoTjournal.tex
\section{Simulations \label{sec:simulations}}
We now illustrate the proposed algorithm using a fifty-vehicle example. 

\subsection{Setup \label{sec:simSetup}}
For this example, we will use the following dynamics for each vehicle:
\begin{equation}
\label{eq:dyn_i}
\begin{aligned}
\dot{\pos}_{x,i} &= v_i \cos \theta_i + d_{x,i} \\
\dot{\pos}_{y,i} &= v_i \sin \theta_i + d_{y,i}\\
\dot{\theta}_i &= \omega_i, \\
\underline{v} \le v_i \le \bar{v}, & |\omega_i| \le \bar{\omega}, \|(d_{x,i}, & d_{y,i}) \|_2 \le d_{r},
\end{aligned}
\end{equation}
\noindent where $\state_i = (\pos_{x,i}, \pos_{y,i}, \theta_i)$ is the state of vehicle $\veh_i$, $\pos_i = (\pos_{x,i}, \pos_{y,i})$ is the position, $\theta_i$ is the heading, and $d = (d_{x,i}, d_{y,i})$ represents $\veh_i$'s disturbances, for example wind, that affect its position evolution. The control of $\veh_i$ is $u_i = (v_i, \omega_i)$, where $v_i$ is the speed of $\veh_i$ and $\omega_i$ is the turn rate; both controls have a lower and upper bound. 

Our goal is to simulate a scenario where UAVs are flying through an urban environment. This setup can be representative of many UAV applications, such as package delivery, aerial surveillance, etc. For this purpose, we grid the San Francisco city in California, US and use it as our state space, as shown in Figure \ref{fig:sf_setup}. Each box in Figure \ref{fig:sf_setup} represents a $500m \times 500m$ area. The origin point for the vehicles is denoted by the Blue star. Four different areas in the city are chosen as the destinations for the vehicles. Mathematically, the target sets $\targetset_i$ of the vehicles are circles of radius $r$ in the position space, i.e. each vehicle is trying to reach some desired set of positions. In terms of the state space $\state_i$, the target sets are defined as

\begin{equation}
\label{eq:target_sim}
\targetset_i = \{\state_i: \|\pos_i - c_i\|_2 \le r\}
\end{equation}

\noindent where $c_i$ are centers of the target circles. In this simulation, we use $r = 0.1m$. The target centers $c_i$ are as follows:
\begin{equation} \label{eqn:NumIC}
\begin{aligned}
c_1 = (0.7, 0.2)\\
c_2 = (-0.7, 0.2)\\
c_3 = (0.7, -0.7)\\
c_4 = (-0.7, -0.7)
\end{aligned}
\end{equation}  
The four targets are represented by four circles in Figure \ref{fig:sf_setup}. The destination of each vehicle is chosen randomly from these four destinations. Finally, tall buildings in downtown San Frnacisco are used as static obstacles for SPP vehicles, denoted by black contours in Figure \ref{fig:sf_setup}.

To make our simulations as close as possible to real scenarios, we choose velocity and turn-rate bounds as $\underline{v} = 0m/s, \bar{v} = 25m/s, \bar\omega = 2 rad/s$, aligned with the modern UAV specs \cite{UAVspecs1, UAVspecs2}. The disturbance bound is chosen as $d_{r} = 11 m/s$, that corresponds to \textit{strong winds} on Beaufort wind force scale \cite{Windscale}. Note that we have used same dynamics and input bounds across all vehicles for clarity of illustration; however, our method can easily handle more general systems of the form in which the vehicles have different control bounds and dynamics.

The goal of the vehicles is to reach their destinations while avoiding a collision with the other vehicles or the static obstacles. The vehicles also need to account for the possibility of the presence of an intruder for a maximum duration of $\iat = 10s$, whose dynamics are given by \eqref{eq:dyn_i}. The joint state space of this fifty-vehicle system is 150-dimensional (150D), making the joint path planning and collision avoidance problem intractable for direct analysis using HJ reachability. Therefore, we assign a priority order to vehicles and solve the path planning problem sequentially. For this simulation, we assign a random priority order to fifty vehicles and use the algorithm proposed in Section \ref{sec:intruder}    to compute a separation between SPP vehicles so that they do not collide with each other or the intruder. We next present the results for two different scenarios: $\nva = 2$ and $\nva = 3$.

\subsection{Results \label{sec:simResults}}

\SBnote{Need to add following things:
\begin{itemize}
\item Update centers and radii of targets.
\item The technical details for the simulations, like RTT parameters, relative co-ordinate dynamics, rotation and translation of obstacles, union for obstacles, etc. 
\end{itemize}}


Focus on the following aspects:
\begin{itemize}
\item Demonstration of theory (that it avoids collision w/ other vehicles and intruders, and we reach our destinations).
\item Scaling of SPP.
\item Provide some more intuition about the solution that emerge out of theory-- Space-time separation, buffer region between vehicles, trajectories, etc.
\item Reactivity of controller to the actual disturbance (Claire: be very detailed about explaining the setup of simulation)
\item Illustrate the structure that emerge out of SPP algorithm (Almost straight line path w/ different starting times)
\item Illustrate how this structure change with change in disturbance bounds (Straight line trajectories become curvy? )
\item For method-2 results, it may be helpful to include a figure which is showing the division of space among vehicles at some time (probably right before the intruder enters).
\item The technical details for the simulations, like RTT parameters, relative co-ordinate dynamics, rotation and translation of obstacles, union for obstacles, etc. 
\end{itemize}