% !TEX root = ../../SPP_IoTjournal.tex
\subsection{Results for $6$ m/s wind and no arrival time separation\label{sec:city_simResults}}
As per Algorithm \ref{alg:rtt}, we start with $\veh_1$ and sequentially compute the optimal control policy $\tracklaw_j$ and the latest departure time $\ldt_j$ for each vehicle. To compute the error bound $\errorbound_j$ on the tracking error of vehicle $j$, we choose $R_{\text{EB}} = 5$ m and use the reduced control authority $\cset^p_j = \{(v_{r,j}, \omega_{r,j}): 11 \text{ m/s} \leq v_{r,j} \leq 13 \text{ m/s}, |\omega_{r,j}| \leq 1.2 \text{ rad/s}\}$. Given dynamics in \eqref{eq:dyn_i}, the error dynamics between $\veh_j$ and the virtual evader are given by \cite{Mitchell05}:
\begin{equation}
\label{eq:reldyn}
\begin{aligned}
\dot{e}_{x,j} &= v_{j} \cos(e_{\theta,j}) - v_{r,j} + \omega_{r,j}{e}_{y,j} + d_{x,j}\\
\dot{e}_{y,j} &= v_{r,j}\sin(e_{\theta,j}) - \omega_{r,j}{e}_{x,j} + d_{y,j}\\
\dot{e}_{\theta,j} &= \omega_{j} - \omega_{r,j},
\end{aligned}
\end{equation}    
where $e_j = ({e}_{x,j}, {e}_{y,j}, {e}_{\theta,j})$ is the tracking error in the three states of $\veh_j$. Given relative dynamics, we compute the BRS $\brs^{\text{EB}}$ using \eqref{eqn:errBound} and evaluate the infinite-horizon control invariant set $\disckernel_j$. For all the BRS computations in this simulation, we use Level Set Toolbox \cite{Mitchell07b}. In presence of moderate winds, the obtained tracking error bound is 5m. This means that given any trajectory of vehicle, winds can at most cause a deviation of 5m from this trajectory. Subsequently, the vehicle will be within a distance of 5m from the trajectory. Note that since all SPP vehicles have same dynamics, the error bound is also same for all vehicles. 

The optimal control policy for $\veh_j$ is thus given by $\tracklaw_j(e_j)$ in \eqref{eq:robust_tracking_law}. However, to compute the relative state $e_j$, the nominal trajectory $\state_{r,j}$ for $\veh_j$ is required. Using $\disckernel_j$, we compute the obstacles induced by the higher priority vehicles for $\veh_j$ and the obstacle induced by $\veh_j$ for the lower priority vehicles. These obstacles are given by \eqref{eqn:rttAugObs} and \eqref{eqn:rttObs} respectively. Note that since disturbance directly impacts the computation of tracking error bound, these obstacles also grow as disturbance magnitude increases. We will illustrate the effect of disturbance magnitude on the trajectories of vehicles in \ref{sec:city_distbEffect}.

The nominal trajectory can now be obtained by computing $\brs_j^\text{rtt}$ in \eqref{eq:rttBRS} and executing the corresponding control policy $\ctrl_j^\text{rtt}$ in \eqref{eq:RTTOptCtrl}, starting from the initial state $\state^0_j$. Finally, the latest departure time $\ldt_j$ is given by $\arg \sup_t \state^0_j \in \brs_j^\text{rtt}(t, \sta_j)$. It is important to note that since the BRS $\brs_j^\text{rtt}$ is computed backwards starting from the scheduled time of arrival $\sta_j$, (a) the latest departure time $\ldt_j$ directly depends on and varies with $\sta_j$ and (b) it directly impacts the obstacles that $\veh_j$ needs to avoid in its path towards its destination. We will illustrate the effect of the scheduled time of arrival on the trajectories of vehicles in \ref{sec:city_distbEffect}.

The resulting trajectories of vehicles for $d_{r} = 6 m/s$ and $\sta_j = 0 \forall \veh_j$ at different times are shown in Figure \ref{fig:trajectories_sf_snapshots}. As evident from the figures, the vehicles remain clear of all the static obstacles (the black contours) and make steady progress towards reachinmg their destinations. The vehicles whose destinations are relatively closer need less time to travel to their destinations and depart later. Note that the shown trajectories are simulated under uniformly random disturbance (i.e., for every vehicle, the disturbance is uniformly sampled from a circle of radius $d_{r} = 6 m/s$ at each time step), but the SPP algorithm guarantees safety and reactivity despite the worst case disturbance, as we show later in this section.
%
\begin{figure*}[!htb]
 \centering
\begin{subfigure}{0.5\textwidth}
  \includegraphics[width=\columnwidth]{figs/sf_d6sep0_s1}
  \subcaption{}
  \label{fig:sf_d6sep0_s1}
\end{subfigure}%
\begin{subfigure}{0.5\textwidth}
  \includegraphics[width=\columnwidth]{figs/sf_d6sep0_s2}
  \subcaption{}
  \label{fig:sf_d6sep0_s2}
\end{subfigure}%

\begin{subfigure}{0.5\textwidth}
  \includegraphics[width=\columnwidth]{figs/sf_d6sep0_s3}
  \subcaption{}
  \label{fig:sf_d6sep0_s3}
\end{subfigure}%
\begin{subfigure}{0.5\textwidth}
  \includegraphics[width=\columnwidth]{figs/sf_d6sep0_s4}
  \subcaption{}
  \label{fig:sf_d6sep0_s4}
\end{subfigure}%
\caption{Snapshots of vehicle trajectories at different times for uniform disturbance with $d_{r} = 6 m/s$. The vehicles remain clear of all static obstacles despite the disturbance in the dynamics.}
\label{fig:trajectories_sf_snapshots}
\end{figure*}

The full trajectories of vehicles are shown in Figure \ref{fig:sf_d6sep0}. All vehicles reach their respective destinations. A zoomed-in version of Fig. \ref{fig:sf_d6sep0} near the red-target (Figure \ref{fig:sf_d6sep0_zoomed}) illustrates that vehicles are also outside each other's danger zones (circles around the vehicles) as required. 
%
\begin{figure}[h]
  \centering
  \includegraphics[width=\columnwidth]{"figs/sf_d6sep0_zoomed"}
  \caption{Zoomed-in version of vehicle trajectories near the red target in Figure \ref{fig:sf_d6sep0}. The SPP algorithm ensures that the vehicles are outside each other's danger zones (circles around the vehicles).} 
  \label{fig:sf_d6sep0_zoomed}
\end{figure}
It is interesting to note that the vehicles going to the same destination take different trajectories. This is because all vehicles have same $\sta$ and hence the lower priority vehicles don't have the flexibility to wait for the higher priority vehicles. In order to ensure that they reach their destinations by the designated $\sta$, they take alternative tarjectories to their destinations. Thus, the vehicles' trajectories obtained in this case are predominately \textit{state-separated} trajectories, i.e., they follow different state trajectories but at the same time. 

Although the plotted trajectories are for a particular realization of the (uniformly random) disturbances, the SPP algorithm guarantees obstacle avoidance regardless of the realized disturbance. To illustrate this, we plot the trajectories of a particular vehicle for three different disturbance relaizations in Figure \ref{fig:sf_distb}. \SBnote{Explain this more once we have the figure.}

The average trajectory computation time per vehicle is \SBnote{$XYZ$s} per vehicle on a \SBnote{$Blah Blah$} computing machine. Note that all this computation is done offline and the resulting optimal policy $\tracklaw_j(e_j)$ is obtained as a lookup table. In real time, neither any computation nor any communication between vehicles is required. Only a lookup table query is required, which can be performed very efficiently in real time. This illustrates the capability of SPP as a provably safe path planning algorithm for large multi-vehicle systems.