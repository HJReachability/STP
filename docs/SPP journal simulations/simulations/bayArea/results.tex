% !TEX root = ../SPP_IoTjournal.tex
\subsection{Results \label{sec:bayArea_simResults}}
The trajectory planning for the vehicles is done using RTT algorithm, similar to that in Section \ref{sec:city_simResults}. The resulting trajectories of vehicles are shown in Figure \ref{fig:bayArea_d11sep0}. Once again, the vehicles remain clear of all other vehicles and reach their respective destinations. Given the relative separation between the scheduled times of arrival, the trajectories are predominately \textit{time-separated}, with roughly two lanes for each pair of cities (one for going from city1 to city2 and another for from city2 to city1). A high-density of vehicles is achieved in the center since the 4 paths are intersecting in the center (Richmond-Oakland, Oakland-Richmond, Berkeley-San Francisco, San Francisco-Berkeley), but the SPP algorithm ensures safety despite this high-density as shown in the zoomed-in version of center (Figure \ref{fig:bayArea_zoomed}).  

The average trajectory computation time per vehicle is $XYZ$s per vehicle on a $Blah Blah$ computing machine. Once again all the computation is done offline and only a lookup table query is required in real-time, which can be performed very efficiently. This simulation illustrates the scalability and the potential of deploying the SPP algorithm for provably safe path planning for large multi-vehicle systems.     