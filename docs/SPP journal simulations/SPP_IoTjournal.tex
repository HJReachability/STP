%%%%%%%%%%%%%%%%%%%%%%%%%%%%%%%%%%%%%%%%%%%%%%%%%%%%%%%%%%%%%%%%%%%%%%%%%%%%%%%%
%2345678901234567890123456789012345678901234567890123456789012345678901234567890
%        1         2         3         4         5         6         7         8

%\documentclass[journal]{aiaa-pretty}
\documentclass[submit]{aiaa-pretty}  

%\overrideIEEEmargins
% See the \addtolength command later in the file to balance the column lengths
% on the last page of the document

\usepackage{mathtools}    % need for sub equations
\usepackage{amsfonts}
\usepackage{graphicx}   % need for figures
\usepackage{subcaption}
\usepackage{epsfig} 
\usepackage{cancel}
\usepackage{amssymb}
\usepackage{color}
\usepackage{bm}
\usepackage[ruled,vlined,titlenotnumbered]{algorithm2e} 
\usepackage{todonotes} \setlength{\marginparwidth}{2.5cm} 
\usepackage{float}
\usepackage{cite}
\usepackage{enumitem}

\newcommand{\MCnote}{\textcolor{red}}
\newcommand{\SBnote}{\textcolor{blue}}

\newcommand{\R}{\mathbb{R}} % Real number
\newcommand{\dist}{\text{dist}} % Distance
\newcommand{\rc}{R_c} % Capture radius
\newcommand{\cradius}{\rc}
\newcommand{\N}{N} % number of agents

\newcommand{\veh}{Q} % vehicle
\newcommand{\intr}{I} % Intruder index
\newcommand{\state}{x} % state
\newcommand{\ctrl}{u} % control
\newcommand{\dstb}{d} % disturbance
\newcommand{\pos}{p} % position
\newcommand{\npos}{h} % non-position states

\newcommand{\traj}{\zeta}
\newcommand{\errstate}{e}

\newcommand{\fdyn}{f} % full dynamics
\newcommand{\cset}{\mathcal{U}} % Control set
\newcommand{\cfset}{\mathbb{U}} % control function set
\newcommand{\dset}{\mathcal{D}} % disturbance
\newcommand{\dfset}{\mathbb{D}} % disturbance function set
\newcommand{\obsset}{\mathcal{G}} % Obstacle (the one used to solve PDE)
\newcommand{\dz}{\mathcal{Z}} % danger zone
\newcommand{\sep}{\mathcal{S}} % Separation region
\newcommand{\buff}{\mathcal{B}} % Buffer region

\newcommand{\valfunc}{V} % value function
\newcommand{\valfuncfwd}{W} % value function for forwards reachable set
\newcommand{\brs}{\mathcal{V}} % backwards reachable set
\newcommand{\frs}{\mathcal{W}} % forwards reachable set
\newcommand{\pfrs}{\mathcal{P}} % projected forwards reachable set
\newcommand{\targetset}{\mathcal{L}} % target set
\newcommand{\ham}{H} % Hamiltonian
\newcommand{\fc}{l} % Final condition
\newcommand{\ic}{l} % Initial condition
\newcommand{\obsfunc}{g} % Obstacle function
\newcommand{\costate}{\lambda}

\newcommand{\disckernel}{\Omega} % Discriminating kernel

\newcommand{\edt}{t^\text{EDT}} % earliest departure time
\newcommand{\ldt}{t^\text{LDT}} % latest departure time
\newcommand{\sta}{t^\text{STA}} % scheduled time of arrival
\newcommand{\ioset}{\mathcal{O}} % Induced obstacle
\newcommand{\boset}{\mathcal{M}} % Base obstacle
\newcommand{\sosetp}{\mathcal{S}} % static obstacle in position space
\newcommand{\soset}{\ioset^\text{static}} % static obstacle in state space
\newcommand{\iat}{t^\text{IAT}} % intruder avoidance time
\newcommand{\wcttr}{t^\text{WC}} % worst case TTR

\newcommand{\basicham}{\ham^\text{basic}}

\newcommand{\tsa}{\underline{t}} % time of start of avoidance
\newcommand{\tea}{\bar{t}} % time of end of avoidance
\newcommand{\nva}{\bar{k}} % Number of Vehicles to Avoid (NVA)
\newcommand{\brd}{t^\text{BRD}} % Buffer Region Duration (BRD)
\newcommand{\trd}{t^\text{RD}} % Remaining Duration (RD)
\newcommand{\rvs}{\mathcal{N}^\text{RP}} % Re-Planning Vehicle Set
\newcommand{\dsen}{d^\text{A}} % Sensing distance
\newcommand{\avoidt}{\mathcal{A}} % Set of all avoid times

\newcommand{\errorbound}{\mathcal{E}} % Error ``bubble" between vehicle and tracking reference
\newcommand{\tracklaw}{\kappa} % Robust tracking law

\newtheorem{assumption}{Assumption}
\newtheorem{alg}{Algorithm}
\newtheorem{remark}{Remark}
\newtheorem{observation}{Observation}

\title{\LARGE \bf Provably Safe and Scalable Unmanned Aerial Vehicle Routing: A Case Study in San Francisco and the Bay Area}

\author{Mo Chen\thanks{PhD Candidate, Department of Electrical Engineering and Computer Sciences, shared first author}, Somil Bansal\thanks{PhD Student, Department of Electrical Engineering and Computer Sciences, shared first author}, Ken Tanabe\thanks{Toshiba}, and Claire J. Tomlin\thanks{Professor, Department of Electrical Engineering and Computer Sciences, Member AIAA}
}



%%%
\AIAAabstract{Provably safe and scalable multi-vehicle path planning is an important and urgent problem due to the expected increase of automation in civilian airspace in the near future. Hamilton-Jacobi (HJ) reachability is an ideal tool for analyzing such safety-critical systems and has been successfully applied to several small-scale problems. However, a direct application of HJ reachability to large scale systems is often intractable because of its exponentially-scaling computation complexity with respect to system dimension, also known as the ``curse of dimensionality''. To overcome this problem, the sequential path planning (SPP) method, which assigns strict priorities to vehicles, was previously proposed; SPP allows multi-vehicle path planning to be done with a linearly-scaling computation complexity. In this work, we demonstrate the potential of SPP algorithm for large-scale systems. In particular, we simulate large-scale multi-vehicle systems in two different urban environments, a city environment and a multi-city environment, and use the SPP algorithm for trajectory planning. SPP is able to efficiently design collision-free trajectories in both environments despite the presence of disturbances in vehicles' dynamics. To ensure a safe transition of vehicles to their destinations, our method automatically allocates space-time reservations to vehicles while accounting for the magnitude of disturbances such as wind in a provably safe way. Our simulation results show an intuitive multi-lane structure in airspace, where the number of lanes and the distance between the lanes depend on the size of disturbances and other problem parameters.}

\begin{document}
\maketitle

% Introduction
% !TEX root = SPP2.tex
\section{Introduction}
Recently, there has been an immense surge of interest in using unmanned aerial vehicles (UAVs) for civil purposes. The applications of UAVs extend well beyond package delivery, and include aerial surveillance, disaster response, and other important tasks \cite{Tice91, Debusk10, Amazon16, AUVSI16, BBC16}. Many of these applications will involve UAVs flying in an urban environment, potentially in close proximity of humans. As a result, government agencies such as the Federal Aviation Administration (FAA) and National Aeronautics and Space Administration (NASA) of the United States are urgently trying to develop new scalable ways to organize an air space in which potentially thousands of UAVs can fly \cite{FAA13, NASA16}.

One essential problem that needs to be addressed is how a group of vehicles in the same vicinity can reach their destinations while avoiding collision with each other. Several previous studies have attempted to address this problem. In some of these studies, specific control strategies for the vehicles or moving entities are assumed, and approaches such as induced velocity obstacles have been used \cite{Fiorini98, Chasparis05, Vandenberg08}. Other researchers have used ideas involving virtual potential fields to maintain collision avoidance while maintaining a specific formation \cite{Saber02, Chuang07}. Although interesting results emerge from these previous studies, simultaneous trajectory planning and collision avoidance are not considered. 

In the past, trajectory planning and collision avoidance problems in safety-critical systems have been studied using reachability analysis, which provides guarantees on the success and safety of optimal system trajectories \cite{Barron90, Mitchell05, Bokanowski10, Margellos11, Fisac15}. In reachability analysis, one computes the reachable set, defined as the set of states from which the system can be driven to a target set. Reachability analysis has been successfully used in applications involving systems with no more than two vehicles, such as pairwise collision avoidance \cite{Mitchell05}, automated in-flight refueling \cite{Ding08}, two-player reach-avoid games \cite{Huang11}, and many others \cite{Bayen07}.

%In addition to the guarantees reachability theory provides and the evident flexibility of reachability theory for analyzing vastly different systems with nonlinear dynamics, many numerical tools for solving reachability problems are also available, making the approach practically appealing \cite{Mitchell05, Sethian96, Osher02, LSToolbox}.

Despite the advantages of reachability analysis, it cannot be directly applied to scenarios involving complex high dimensional systems such as multi-vehicle systems. The computation of reachable sets involves solving a Hamilton-Jacobi (HJ) partial differential equation (PDE) on a grid representing a discretization of the state space, causing an exponential scaling of computation complexity with respect to the dimension of the system, or roughly speaking, with the number of vehicles present.

In this paper, we build on the work in \cite{Chen15}, and assume a reasonable structure in the multi-vehicle path planning problem. In the sequential path planning (SPP) scheme, vehicles are assigned some priority. Higher-priority vehicles may ignore the lower-priority vehicles, who must take into account the presence of higher-priority vehicles by treating them as induced time-varying obstacles. Unlike the work in \cite{Chen15}, we incorporate disturbances for all vehicles and consider three different assumptions on the information each of the vehicles may have access to, making the sequential path planning substantially more practical. For each of the assumed information patterns, we propose a reachability-based method to compute the induced obstacles that would guarantee collision avoidance as well as successful transit to the destination. We demonstrate and compare our proposed methods through numerical simulations.

% Problem Formulation
% !TEX root = nextUAVsched.tex
\section{Problem Formulation \label{sec:formulation}}
Consider $N$ vehicles $P_i,i=1\ldots,N$, each trying to reach one of $N$ target sets $\target_i,i=1\ldots,N$, while avoiding obstacles and collision with each other. Each vehicle $i$ has states $\x_i\in \R^{n_i}$ and travels on a domain $\amb=\obs \cup \free\in\R^p$, where $\obs$ represents the obstacles that each vehicle must avoid, and $\free$ represents all other states in the domain on which vehicles can move. Each vehicle $i = 1,2,\ldots,N$ moves with the following dynamics for $t\in[\tnow_i, \tf_i]$:

\begin{equation} \label{eq:dyn}
\dotx_i = f_i (t, \x_i, \ctrl_i), \quad\x_i(\ti_i) = \x_i^0 
\end{equation}

\noindent where $\x_i^0$ represents the initial condition of vehicle $i$, and $\ctrl_i(\cdot)$ represents the control function of vehicle $i$. In general, $f_i(\cdot,\cdot,\cdot)$ depends on the specific dynamic model of vehicle $i$, and need not be of the same form across different vehicles. Denote $\pos_i\in\R^p$ the subset of the states that represent the position of the vehicle. Given $\pos_i^0\in\free$, we define the admissible control function set for $P_i$ to be the set of all control functions such that $\pos_i(t) \in \free \forall t\ge \ti_i$. Denote the joint state space of all vehicles $\x \in \R^n$ where $n = \sum_i n_i$, and their joint control $\ctrl$.

We assume that the control functions $\ctrl_i(\cdot)$ are drawn from the set $\ctrlf_i := \{\ctrl_i: [\tnow_i, \tf_i] \rightarrow \ctrlin_i, \text{measurable}$\footnote{
A function $f:X\to Y$ between two measurable spaces $(X,\Sigma_X)$ and $(Y,\Sigma_Y)$ is said to be measurable if the preimage of a measurable set in $Y$ is a measurable set in $X$, that is: $\forall V\in\Sigma_Y, f^{-1}(V)\in\Sigma_X$, with $\Sigma_X,\Sigma_Y$ $\sigma$-algebras on $X$,$Y$.}\} where $\ctrlin_i \in \R^{n^\ctrl_i}$ is the set of allowed control inputs. Furthermore, we assume $f_i(t,\x_i, \ctrl_i)$ is bounded, Lipschitz continuous in $\x_i$ for any fixed $t,\ctrl_i$, and measurable in $t, \ctrl_i$ for each $\x_i$. Therefore given any initial state $\x_i^0$ and any control function $\ctrl_i(\cdot)$, there exists a unique, continuous trajectory $\x_i(\cdot)$ solving (\ref{eq:dyn}) \cite{coddington55}.

The goal of each vehicle $i$ is to arrive at $\target_i \subset \R^{n_i}$ at or before some scheduled time of arrival (STA) $\tf_i$ in minimum time, while avoiding obstacles and danger with all other vehicles. The target sets $\target_i$ can be used to represent desired kinematic quantities such as position and velocity and, in the case of non-holonomic systems, quantities such as heading angle.  $\tnow_i$ can be interpreted as the earliest start time (EST) of vehicle $i$, before which the vehicle may not depart from its initial state. Further, we define $\ti_i$, the latest (acceptable) start time (LST) for vehicle $i$. Our problem can now be thought of as determining the LST $\ti_i$ for each vehicle to get to $\target_i$ at or before the STA $\tf_i$, and finding a control to do this safely. If the LST is before the EST $\ti_i < \tnow_i$, then it is infeasible for vehicle $i$ to arrive at $\target_i$ at or before the STA $\tf_i$. Comparing $\ti_i$ and $\tnow_i$ is feasibility problem that may arise in practice; however, for simplicity of presentation, we will assume that $\tnow_i\le \ti_i \forall i$.

Danger is described by sets $\danger_{ij}(\x_j) \subset \amb$. In general, the definition of $\danger_{ij}$ depends on the conditions under which vehicles $i$ and $j$ are considered to be in an unsafe configuration, given the state of vehicle $j$. Here, we define danger to be the situation in which the two vehicles come within a certain radius $\Rc$ of each other: $\danger_{ij}(\x_j) = \{\x_i: \| \pos_i - \pos_j\|_2 \le \Rc \}$. Such a danger zone is also used by the FAA \cite{paglione99}. An illustration of the problem setup is shown in Figure \ref{fig:formulation}.

\begin{figure}
	\centering
	\includegraphics[width=0.35\textwidth]{"fig/formulation"}
	\caption{An illustration of the problem formulation with three vehicles. Each vehicle $P_i$ seeks to reach its target set $\target_i$ by time $t=\tf_i$, while avoiding physical obstacles $\obs$ and the danger zones of other vehicles.}
	\label{fig:formulation}
\end{figure}

In general, the above problem must be analyzed in the joint state space of all vehicles, making the solution intractable. In this paper, we will instead consider the problem of performing path planning of the vehicles in a sequential manner. Without loss of generality, we consider the problem of first fixing $i=1$ and determining the optimal control for vehicle $1$, the vehicle with the highest priority. The resulting optimal control $\ctrl_1$ sends vehicle $1$ to $\target_1$ in minimum time. 

Then, we plan the minimum time trajectory for each of the vehicles $2,\ldots,N$, in decreasing order of priority, given the previously-determined trajectories for higher-priority vehicles $1,\ldots,i-1$. We assume that all vehicles have complete information about the states and trajectories of higher-priority vehicles, and that all vehicles adhere to their planned trajectories. Thus, in planning its trajectory, vehicle $i$ treats higher-priority vehicles as known time-varying obstacles. 

With the above sequential path planning (SPP) protocol and assumptions, our problem now reduces to the following for vehicle $i$. Given $\x_j(\cdot), j=1,\ldots,i-1$, determine $\ctrl_i(\cdot)$ that maximizes $\ti_i$ and such that $x_i(\tau) \in \target_i, \tau\le \tf_i$.

% Background material
% !TEX root = ./STP_IoTjournal.tex
\section{Background \label{sec:background}}
In this section, we present the basic STP algorithm \cite{Chen15c} in which disturbances are ignored and perfect information of vehicles’ positions is assumed. This simplification allows us to clearly present the basic STP algorithm. However, in presence of disturbances, it is no longer possible to commit to exact trajectories (and hence positions), since the disturbance $\dstb_i(\cdot)$ is \textit{a priori} unknown. Thus, disturbances and incomplete information significantly complicate the STP scheme. We next present the robust trajectory tracking algorithm \cite{Bansal2017} that can be used to make basic STP approach robust to disturbances as well as to an imperfect knowledge of other vehicles' positions. All of these algorithms use time-varying reachability analysis to provide goal satisfaction and safety guarantees; therefore, we start with an overview of time-varying reachability.

% !TEX root = ../STP_IoTjournal.tex
\subsection{Time-Varying Reachability Background \label{sec:HJIVI}}
We will be using reachability analysis to compute a backward reachable set (BRS) $\brs$ given some target set $\targetset$, time-varying obstacle $\obsset(t)$, and the Hamiltonian function $\ham$ which captures the system dynamics as well as the roles of the control and disturbance. The BRS $\brs$ in a time interval $[t, t_f]$ will be denoted by

\begin{equation}
\brs(t, t_f) \quad \text{ (backward reachable set)}
\end{equation}

Several formulations of reachability are able to account for time-varying obstacles \cite{Bokanowski11, Fisac15} (or state constraints in general). For our application in STP, we utilize the time-varying formulation in \cite{Fisac15}, which accounts for the time-varying nature of systems without requiring augmentation of the state space with the time variable. In the formulation in \cite{Fisac15}, a BRS is computed by solving the following \textit{final value} double-obstacle HJ VI:

\begin{equation}
\label{eq:HJIVI_BRS}
\begin{aligned}
\max \Big\{ \min \{&D_t \valfunc(t, \state) + \ham(t, \state, \nabla \valfunc(t, \state)), \fc(\state) - \valfunc(t, \state) \}, \\
&-\obsfunc(t, \state) - \valfunc(t, \state) \Big\} = 0, \quad t \le t_f \\
&\valfunc(t_f, \state) = \max\{\fc(\state), -\obsfunc(t_f, \state)\}
\end{aligned}
\end{equation}

%In a similar fashion, the FRS is computed by solving the following \textit{initial value} HJ PDE:
%
%\begin{equation}
%\label{eq:HJIVI_FRS}
%\begin{aligned}
%D_t \valfuncfwd(t, \state) + &\ham(t, \state, \nabla \valfuncfwd(t, \state)) = 0 , \quad t \ge t_0  \\
%&\valfuncfwd(t_0, \state) = \max\{\fc(\state), -\obsfunc(t_0, \state)\}
%\end{aligned}
%\end{equation}
%
In \eqref{eq:HJIVI_BRS}, the function $\ic(\state)$ is the implicit surface function representing the target set $\targetset = \{\state: \ic(\state) \le 0\}$. Similarly, the function $\obsfunc(t, \state)$ is the implicit surface function representing the time-varying obstacles $\obsset(t) = \{\state: \obsfunc(t,\state)\le 0\}$. The BRS $\brs(t, t_f)$ is given by

%\begin{equation}
%\label{eq:implicitValfuncs}
%\begin{aligned}
%\brs(t, t_f) &= \{\state: \valfunc(t, \state) \le 0\} \\
%\frs(t_0, t) &= \{\state: \valfuncfwd(t, \state) \le 0 \}
%\end{aligned}
%\end{equation}
\begin{equation}
\label{eq:implicitValfuncs}
\brs(t, t_f) = \{\state: \valfunc(t, \state) \le 0\}
\end{equation}

Some of the reachability computations will not involve an obstacle set $\obsset(t)$, in which case we can simply set $\obsfunc(t, \state) \equiv \infty$ which effectively means that the outside maximum is ignored in \eqref{eq:HJIVI_BRS}.

The Hamiltonian, $\ham(t, \state, \nabla \valfunc(t,\state))$, depends on the system dynamics, and the role of control and disturbance. Whenever $\ham$ does not depend explicit on $t$, we will drop the argument. In addition, the Hamiltonian is an optimization that produces the optimal control $\ctrl^*(t, \state)$ and optimal disturbance $\dstb^*(t, \state)$, once $\valfunc$ is determined. For BRSs, whenever the existence of a control (``$\exists \ctrl$'') or disturbance is sought, the optimization is a minimum over the set of controls or disturbance. Whenever a BRS characterizes the behavior of the system for all controls (``$\forall \ctrl$'') or disturbances, the optimization is a maximum. We will introduce precise definitions of reachable sets, expressions for the Hamiltonian, expressions for the optimal controls as needed for the many different reachability calculations we use.
% !TEX root = STP_journal.tex
\section{STP Without Disturbances and With Perfect Information\label{sec:basic}}
In this section, we introduce the basic STP algorithm assuming that there is no disturbance affecting the vehicles, and that each vehicle knows the exact position of higher-priority vehicles. \SBnote{Although in practice, such assumptions do not hold, the description of the basic STP algorithm will introduce the notation needed for describing the subsequent, more realistic versions of STP.} We also show simulation results for the basic STP algorithm. The majority of the content in this section is taken from \cite{Chen15c}.

\subsection{Theory}
Recall that the STP vehicles $\veh_i, i=1,\ldots,N$, are each assigned a strict priority, with $\veh_j$ having a higher priority than $\veh_i$ if $j<i$. In the absence of disturbances, we can write the dynamics of the STP vehicles as

\begin{equation}
\label{eq:dyn_no_dstb}
\begin{aligned}
\dot\state_i &= \fdyn_i(\state_i, \ctrl_i), t \le \sta_i \\
\ctrl_i &\in \cset_i, \qquad i = 1 \ldots, \N
\end{aligned}
\end{equation}

%\noindent with trajectories denoted by $\traj_i(s; \state^0_i, \ldt, \ctrl_(\cdot))$.

In STP, each vehicle $\veh_i$ plans the trajectory to its target set $\targetset_i$ while avoiding static obstacles $\soset_i$ and the obstacles $\ioset_i^j(t)$ induced by higher-priority vehicles $\veh_j, j<i$. Path planning is done sequentially starting from the first vehicle and proceeding in descending priority, $\veh_1, \veh_2, \ldots, \veh_{\N}$ so that each of the trajectory planning problems can be done in the state space of only one vehicle. During its trajectory planning process, $\veh_i$ ignores the presence of lower-priority vehicles $\veh_k, k>i$, and induces the obstacles $\ioset_k^i(t)$ for $\veh_k, k>i$.

From the perspective of $\veh_i$, each of the higher-priority vehicles $\veh_j, j<i$ induces a time-varying obstacle denoted $\ioset_i^j(t)$ that $\veh_i$ needs to avoid\footnote{Note that the index $k$ in $\ioset_k^i$ denotes vehicles with lower priority than $\veh_i$, and the index $j$ in $\ioset_i^j(t)$ denotes vehicles with higher priority than $\veh_i$.}. Therefore, each vehicle $\veh_i$ must plan its trajectory to $\targetset_i$ while avoiding the union of all the induced obstacles as well as the static obstacles. Let $\obsset_i(t)$ be the union of all the obstacles that $\veh_i$ must avoid on its way to $\targetset_i$:

\begin{equation}
\label{eq:obsseti}
\obsset_i(t)  = \soset_i \cup \bigcup_{j=1}^{i-1} \ioset_i^j(t)
\end{equation}

With full position information of higher priority vehicles, the obstacle induced for $\veh_i$ by $\veh_j$ is simply

\begin{equation}
\label{eq:ioset}
\ioset_i^j(t) = \{\state_i: \|\pos_i - \pos_j(t)\|_2 \le \rc \}
\end{equation}

Each higher priority vehicle $\veh_j$ plans its trajectory while ignoring $\veh_i$. Since trajectory planning is done sequentially in descending order or priority, the vehicles $\veh_j, j<i$ would have planned their trajectories before $\veh_i$ does. Thus, in the absence of disturbances, $\pos_j(t)$ is \textit{a priori} known, and therefore $\ioset_i^j(t), j<i$ are known, deterministic moving obstacles, which means that $\obsset_i(t)$ is also known and deterministic. Therefore, the trajectory planning problem for $\veh_i$ can be solved by first computing the BRS $\brs_i^\text{basic}(t, \sta_i)$, defined as follows:
%
\begin{equation}
\label{eq:BRS_basic}
\begin{aligned}
\brs_i^\text{basic}(t, \sta_i) = & \{y: \exists \ctrl_i(\cdot) \in \cfset_i, \state_i(\cdot) \text{ satisfies \eqref{eq:dyn_no_dstb}}, \\
& \forall s \in [t, \sta_i],\state_i(s) \notin \obsset_i(s), \\
& \exists s \in [t, \sta_i], \state_i(s) \in \targetset_i, \state_i(t) = y\}
\end{aligned}
\end{equation}
%
The BRS $\brs(t, \sta_i)$ can be obtained by solving \eqref{eq:HJIVI_BRS} with $\targetset = \targetset_i$, $\obsset(t) = \obsset_i(t)$, and the Hamiltonian 
%
\begin{equation}
\label{eq:basicham}
\ham_i^\text{basic}(\state_i, \costate) = \min_{\ctrl_i\in\cset_i} \costate \cdot \fdyn_i(\state_i, \ctrl_i)
\end{equation}
%
\SBnote{Note that $\brs(t, \sta_i)$, by definition, does not contain any states from which it is inevitable to avoid the danger zone $\dz_{ij}$ (and $\obsset_i$ in general).} Given $\brs(t, \sta_i)$, the optimal control for reaching $\targetset_i$ while avoiding $\obsset_i(t)$ is then given by
%
\begin{equation}
\label{eq:basicOptCtrl}
\ctrl_i^\text{basic}(t, \state_i) = \arg \min_{\ctrl_i\in\cset_i} \costate \cdot \fdyn_i(\state_i, \ctrl_i)
\end{equation}
%
\noindent from which the trajectory $\state_i(\cdot)$ can be computed by integrating the system dynamics, which in this case are given by \eqref{eq:dyn_no_dstb}. In addition, the latest departure time $\ldt_i$ can be obtained from the BRS $\brs(t, \sta_i)$ as $\ldt_i = \arg \sup_t \{\state_i^0 \in \brs(t, \sta_i)\}$. In summary, the basic STP algorithm is given as follows:

\begin{alg}
\label{alg:basic}
\textbf{Basic STP algorithm}: Suppose we are given initial conditions $\state_i^0$, vehicle dynamics \eqref{eq:dyn_no_dstb}, target sets $\targetset_i$, and static obstacles $\soset_i, i = 1\ldots, \N$. For each $i$ in ascending order starting from $i=1$ (which corresponds to descending order of priority),
\begin{enumerate}
\item determine the total obstacle set $\obsset_i(t)$, given in \eqref{eq:obsseti}. In the case $i=1$, $\obsset_i(t) = \soset_i ~ \forall t$;
\item compute the BRS $\brs_i^\text{basic}(t, \sta_i)$ defined in \eqref{eq:BRS_basic}. The latest departure time $\ldt_i$ is then given by $\arg \sup_t \{\state^0_i \in \brs_i^\text{basic}(t, \sta_i)\}$;
\item determine the trajectory $\state_i(\cdot)$ using vehicle dynamics \eqref{eq:dyn_no_dstb}, with the optimal control  $\ctrl_i^\text{basic}(\cdot)$ given by \eqref{eq:basicOptCtrl};
\item given $\state_i(\cdot)$, compute the induced obstacles $\ioset_k^i(t)$ for each $k>i$. In the absence of disturbances, $\ioset_k^i(t)$ is given by \eqref{eq:ioset}.
\end{enumerate}
\end{alg}

\MCnote{Note that Step 1, which determines the total obstacle set, can be updated in a recursive manner by adding a new set of induced obstacles for each next vehicle: $\obsset_{i+1}(t) = \obsset_i(t) \cup \ioset_{i+1}^i(t)$. In addition, in implementation, Step 4 can be simplified by storing $\obsset_i(t)$ as a look-up table with the maximum dimensionality across all vehicle state spaces. When a vehicle plans its trajectory, irrelevant dimensions of $\obsset_i(t)$ can be ignored. This observation keeps the computational complexity of our algorithm linear with respect to the number of vehicles.}

\MCnote{As previously mentioned, the basic STP algorithm, as well as all subsequent variants of STP algorithms, will \textit{always} return a feasible trajectory that arrives at the target on time, as long as a feasible trajectory exists in the \textit{absence} of other vehicles. This is because a vehicle can simply depart early enough to avoid being blocked by higher-priority vehicles. In fact, the latest departure time $\ldt_i$ quantifies exactly when each vehicle needs to depart to arrive on time.}
% !TEX root = SPP2.tex
\subsection{Method 3: Robust Trajectory Tracking\label{sec:rtt}}
Although it is impossible to commit to and track an exact trajectory in the presence of disturbances, it may still be possible to \textit{robustly} track a \textit{nominal} trajectory with a bounded error at all times. If this can be done, then the tracking error bound can be used to determine the induced obstacles. Here, computation is done in two phases: the \textit{planning phase} and the \textit{disturbance rejection phase}. In the planning phase, we compute a nominal trajectory $\state_{r,j}(\cdot)$ that is feasible in the absence of disturbances. In the disturbance rejection phase, we compute a bound on the tracking error.%\MCnote{don't need to explain where error comes bound, imo}%, caused by a vehicle's inability to exactly track the nominal trajectory in the presence of disturbances. 

In the planning phase, planning is done for a reduced control set $\cset^p\subset\cset$, as some margin is needed to reject unexpected disturbances while tracking the nominal trajectory. In the disturbance rejection phase, we determine the error bound independently of the nominal trajectory. Let $\state_j$ and $\state_{r,j}$ denote the states of the actual vehicle $\veh_j$ and the virtual evader, respectively, and define the tracking error $e_j=\state_j-\state_{r,j}$. When the error dynamics are independent of the absolute state as in \eqref{eq:edyn} (and also (7) in \cite{Mitchell05}), we can obtain error dynamics of the form
\begin{equation}
\label{eq:edyn} % Error dynamics
\begin{aligned}
\dot{e_j} &= \fdyn_{e_j}(e_j, \ctrl_j, \ctrl_{r,j},\dstb_j), \\
\ctrl_j &\in \cset_j, \ctrl_{r,j} \in \cset^p_j, \dstb_j \in \dset_j, \quad t \leq 0
\end{aligned}
\end{equation}

To obtain bounds on the tracking error, we first conservatively estimate the error bound around any reference state $\state_{r,j}$, denoted $\errorbound_j = \{e_j: \|\pos_{e_j}\|_2 \le R_{\text{EB}}\}$,
%\begin{equation} \label{eqn:err}
%\errorbound_j = \{e_j: \|\pos_{e_j}\|_2 \le R_{\text{EB}} \}, 
%\end{equation}
\noindent where $\pos_{e_j}$ denotes the position coordinates of $e_j$ and $R_{\text{EB}}$ is a design parameter. We next solve a reachability problem with its complement $\errorbound_j^c$, the set of tracking errors violating the error bound, as the target in the space of the error dynamics. From $\errorbound_j^c$, we compute the following BRS:
\begin{equation} \label{eqn:errBound}
\begin{aligned}
&\brs^{\text{EB}}_{j}(t, 0) = \{y: \forall \ctrl_j(\cdot) \in \cfset_j, \exists \ctrl_{r, j}(\cdot) \in \cfset^\pos_j, \exists \dstb_j(\cdot) \in \dfset_i, \\
& e_j(\cdot) \text{ satisfies \eqref{eq:edyn}}, e_j(t) = y, \exists s \in [t, 0], e_j(s) \in \errorbound_j^c\}, 
\end{aligned}
\end{equation}
where the Hamiltonian to compute the BRS is given by:
\begin{equation}
\begin{aligned}
H^{\text{EB}}_{j}(e_j, \costate) &= \max_{\ctrl_j \in \cset_j} \min_{\ctrl_r \in \cset^\pos_j, \dstb_j \in \dset_j} \costate \cdot \fdyn_{e_j}(e_j, \ctrl_j, \ctrl_{r,j}, \dstb_j).
\end{aligned}
\end{equation}

Letting $t \to -\infty$, we obtain the infinite-horizon control-invariant set $\disckernel_j := \lim_{t \to -\infty} \left( \brs^{\text{EB}}_{j}(t, 0) \right)^c$. If $\disckernel_j$ is nonempty, then the tracking error $e_j$ at flight time is guaranteed to remain within $\disckernel_j \subseteq \errorbound_j$ provided that the vehicle starts inside $\disckernel_j$ and subsequently applies the feedback control law
\begin{equation}
\label{eq:robust_tracking_law}
\tracklaw_j(e_j) = \arg\max_{\ctrl_j \in \cset_j} \min_{\ctrl_r \in\cset^\pos_j, \dstb_j \in \dset_j} \costate \cdot \fdyn_{e_j}(e_j,\ctrl_j,\ctrl_{r, j},\dstb_j).
\end{equation}

The induced obstacles by each higher-priority vehicle $\veh_j$ can thus be obtained by: 
\begin{equation} 
\label{eqn:rttObs}
\begin{aligned}
\ioset_i^j(t) &=  \{\state_i: \exists y \in \pfrs_j(t), \|\pos_i - y\|_2 \le \rc \} \\
\pfrs_j(t) &= \{\pos_j: \exists \npos_j, (\pos_j, \npos_j) \in \disckernel_j  + \state_{r,j}(t)\},
\end{aligned}
\end{equation}
\noindent where the ``$+$'' in \eqref{eqn:rttObs} denotes the Minkowski sum\footnote{The Minkowski sum of sets $A$ and $B$ is the set of all points that are the sum of any point in $A$ and $B$.}. Finally, we can obtain the total obstacle set $\obsset_i(t)$ using \eqref{eq:ioset}. %Intuitively, if $\veh_j$ is tracking $\state_{r,j}(t)$, then it will remain within the error bound $\disckernel_j$ around $\state_{r,j}(t) ~\forall t$. This is precisely the set $\pfrs_j(t)$. The induced obstacles can then be obtained by augmenting a danger zone around this set. Finally, we can obtain the total obstacle set $\obsset_i(t)$ using \eqref{eq:obsseti}.

Since each vehicle $\veh_j$, $j<i$, can only be guaranteed to stay within $\disckernel_j$, we must make sure during the path planning of $\veh_i$ that at any given time, the error bounds of $\veh_i$ and $\veh_j$, $\disckernel_i$ and $\disckernel_j$, do not intersect. This can be done by augmenting the total obstacle set by $\disckernel_i$:%This can be done by choosing the induced obstacle to be the Minkowski sum\footnote{The Minkowski sum of sets $A$ and $B$ is the set of all points that are the sum of any point in $A$ and $B$.} of the error bounds. Thus,

\begin{equation} 
\label{eqn:rttAugObs}
\tilde{\obsset}_i(t) = \obsset_i(t) + \disckernel_i.
\end{equation}

Finally, given $\disckernel_i$, we can guarantee that $\veh_i$ will reach its target $\targetset_i$ if $\disckernel_i \subseteq \targetset_i$; thus, in the path planning phase, we modify $\targetset_i$ to be $\tilde{\targetset}_i := \{\state_i: \disckernel_i + \state_i \subseteq \targetset_i\}$, and compute a BRS, with the control authority $\cset^\pos_i$, that contains the initial state of the vehicle. Mathematically,

\begin{equation}
\label{eq:rttBRS}
\begin{aligned}
\brs_i^\text{rtt}(t, \sta_i) = & \{y: \exists \ctrl_i(\cdot) \in \cfset^p_i, \state_i(\cdot) \text{ satisfies \eqref{eq:dyn_no_dstb}},\\
&\forall s \in [t, \sta_i], \state_i(s) \notin \tilde{\obsset}_i(t), \\
& \exists s \in [t, \sta_i], \state_i(s) \in \tilde{\targetset}_i, \state_i(t) = y\}
\end{aligned}
\end{equation}

The Hamiltonian to compute $\brs_i^\text{rtt}(t, \sta_i)$ and the optimal control for reaching $\tilde{\targetset}_i$ are given by \eqref{eq:basicSPPHam} and \eqref{eq:optCtrl} respectively.
%$\brs_i^\text{rtt}(t, \sta_i)$ can be obtained by solving \eqref{eq:HJIVI_BRS} using the Hamiltonian: 
%\begin{equation}
%\label{eq:RTTham}
%\ham_i^\text{rtt}(\state_i, \costate) = \min_{\ctrl_i \in \cset^\pos_i } \costate \cdot \fdyn_i(\state_i, \ctrl_i)
%\end{equation}
%
%The corresponding optimal control for reaching $\tilde{\targetset}_i$ is given by:
%\begin{equation}
%\label{eq:RTTOptCtrl}
%\ctrl_i^\text{rtt}(t) = \arg \min_{\ctrl_i \in \cset^\pos_i } \costate \cdot \fdyn_i(\state_i, \ctrl_i).
%\end{equation}
The nominal trajectory $\state_{r,i}(\cdot)$ can thus be obtained by using vehicle dynamics \eqref{eq:dyn_no_dstb}, with the optimal control  $\ctrl_i^\text{rtt}(\cdot)$. From the resulting nominal trajectory $\state_{r,i}(\cdot)$, the overall control policy to reach $\targetset_i$ can be obtained via \eqref{eq:robust_tracking_law}.

% City-level Simulations
% !TEX root = ../SPP_IoTjournal.tex
\section{City Environment Simulation \label{sec:bayArea_sim}}
We next use SPP algorithm to design trajectories for a 200 vehicle UAV system where UAVs are flying through a multi-city region.
% !TEX root = ../../SPP_IoTjournal.tex
\subsection{Setup \label{sec:bayArea_simSetup}}
We grid the San Francisco Bay Area in California, US and use it as our state space, as shown in Figure \ref{fig:bayArea_setup}. We consider the UAVs flying to and from four cities: Richmond, Berkeley, Oakland, and San Francisco. The blue region in Fig. \ref{fig:bayArea_setup} represents bay. This environment is different from the city environment in Section \ref{sec:city_sim} in that now the UAVs need to fly for longer distances and through a high-density vehicle environment with strong winds, but have very few static obstacles like tall buildings.    
%
\begin{figure}
  \centering
  \includegraphics[width=\columnwidth]{figs/bayArea_setup}
  \caption{Multi-city simulation setup. A $300 km^2$ area of San Francisco Bay Area is used as the state-space for vehicles. SPP vehicles fly to and from the four cities indicated by the four circles. The simulations are performed under the strong winds condition with $d_{r} = 11 m/s$.}
  \label{fig:bayArea_setup}
\end{figure}

Each box in Figure \ref{fig:bayArea_setup} represents a $25$ km$^2$ area. The vehicles are flying to and from the four cities indicated by the four circles. The origin and the destination of each vehicle is chosen randomly from these four cities. The vehicle dynamics are given by \eqref{eq:dyn_i}. We choose velocity and turn-rate bounds as $\underline{v} = 0$ m/s, $\bar{v} = 25$ m/s, $\bar\omega = 2$ rad/s. The disturbance bound is chosen as $d_{r} = 11$ m/s, which corresponds to \textit{strong breeze} on Beaufort wind force scale \cite{Windscale}. The scheduled time of arrival $\sta$ for vehicles are chosen as $5(i-1)$ s.

The goal of the vehicles is to reach their destinations while avoiding a collision with the other vehicles. The joint state space of this 200-vehicle system is 600-dimensional, making the joint path planning and collision avoidance problem intractable for direct analysis. Therefore, we assign a priority order to vehicles and solve the path planning problem sequentially.
% !TEX root = ../SPP_IoTjournal.tex
\subsection{Results \label{sec:city_simResults}}

Focus on the following aspects:
\begin{itemize}
\item The technical details for the simulations, like RTT parameters, relative co-ordinate dynamics, rotation and translation of obstacles, union for obstacles, etc. 
\item Demonstration of theory (the vehicles avoid collision w/ other vehicles and reach their destinations).
\item Scaling of SPP.
\item Provide some more intuition about the solution that emerge out of theory-- Space-time separation, type of space-time trajectories (Almost straight line path w/ different starting times?), etc.
\item Reactivity of controller to the actual disturbance (Claire: be very detailed about explaining the setup of simulation)
\item Illustrate how the type of space-time trajectories change with change in disturbance bounds and STA
\end{itemize}
% !TEX root = ../SPP_IoTjournal.tex
\subsection{Effect of Disturbance and Scheduled Time of Arrival \label{sec:city_distbEffect}}
In this section, we illustrate how the disturbance bound $d_r$ in \eqref{eq:dyn_i} and the realtive $\sta$s of vehicles affect the vehicle trajectories. For this purpose, we simulate the SPP algorithm for four additional scenarios:
\begin{itemize}
\item Case-0: $d_r = 6m/s$, $\sta_i = 0 ~\forall i$
\item Case-1: $d_r = 11m/s$, $\sta_i = 0 ~\forall i$
\item Case-2: $d_r = 6m/s$, $\sta_i = 5(i-1) ~\forall i$
\item Case-3: $d_r = 11m/s$, $\sta_i = 5(i-1) ~\forall i$
\item Case-4: $d_r = 11m/s$, $\sta_i = 10(i-1) ~\forall i$
\end{itemize}
The interpretation $\sta_i = 5(i-1)$ is that the scheduled time of arrival of any two consecutive vehicles is separated by 5s. $d_r = 6m/s$ and $d_r = 11m/s$ correspond to the moderate winds and strong winds respectively on Beaufort wind force scale \cite{Windscale}. 

Intuitively, as $d_r$ increases, it is harder for a vehicle to closely track a particular nominal trajectory, which results in a higher tracking error bound. Thus, the vehicles need to be separated more from each other in space to ensure that they do not enter each other's danger zones. This is also evident from comparing the results corresponding to Case-0 (Fig. \ref{fig:sf_d6sep0}) and Case-1 (Fig. \ref{fig:sf_d11sep0}). As the disturbance magnitude increases from $d_r = 6m/s$ (moderate winds) to $d_r = 11m/s$ (strong winds), the vehicles' trajectories get farther apart from each other. Since $\sta$ is same for all vehicles, the vehicles’ trajectories are still predominately \textit{state-separated} trajectories.

We next compare Case-0 and Case-2. The difference between these two cases is that vehicles have different $\sta$s in Case-2. When vehicles $\veh_i$ and $\veh_{j}$ ($j>i$) have same scheduled time of arrival and are going to the same destination, they are constrained to travel at the same time to make sure they reach the destination by the designtaed $\sta$. However, since $\veh_i$ is high-priority, it gets access to the optimal trajectory (in terms of the total time of tarvel to destination) and $\veh_{j}$ has to settle for a relatively sub-optimal trajectory. Thus, all vehicles going to a particular destination take different trajectories creating a ``band" of trajectories between the origin and the destination, as shown in Figure \ref{fig:sf_d6sep0}; the high-priority vehicles take a relatively straight path between the origin and the destination whereas the low-priority vehicles take a (relatively sub-optimal) curved path. If we think of an air highway between the origin and the destination, then vehicles take different lanes of that highway to reach the destination in Case-0. Thus, the trajectories of vehicles in this case are \textit{state-separated}. However, when $\sta_j > \sta_i$, then $\veh_j$ is not bound to travel at the same time as $\veh_i$; it can wait for $\veh_i$ to depart and take a shorter path later on. Thus, vehicles travel in a single (optimal) lane in this case, as shown in Figure \ref{fig:sf_d6sep5}. In other words, they take the same trajectory to the destination, but at different times. Thus, the trajectories of vehicles in this case are \textit{time-separated}. 

Note that the exact number of lanes depends on both the disturbance and relative $\sta$s. As disturbance increases, the vehicles need to be separated more from each other to ensure safety. The relative difference in $\sta$s should be able to ensure this separation if they were to take the same lane. As shown in Figure \ref{fig:sf_d11sep5}, a difference of 5s in $\sta$s is not sufficient to achieve a single lane behavior for strong winds. However, the number of lanes are significantly less than that in Case-1 (Fig. \ref{fig:sf_d11sep0}). Finally, a difference of 10s in $\sta$s ensure that we get the single lane behavior even in the presence of strong winds, leading to \textit{time-separated} trajectories.

In summary, the relative separation in the scheduled times of arrival of vehicles determines the number of lanes between a pair of origin and destination, and more and more tarjectories become time-separated as this relative separation increases. The disturbnace magnitude, on the other hand, determines the relative separation between different lanes, and more and more tarjectories become state-separated as the disturbance increases. This behavior is also summarized in Figure \ref{fig:traj_behavior} (draw a graph with separation between TSAs on x-axis and disturbance magnitude on y-axis, and divide it into 4 regions and comment on the number and the typr of trajectories.)

\SBnote{Make sure whether the shown trajectories are nominal trajectories or if they contain disturbance. Clarify this accordingly in all the results; otherwise, it could be confusing for the readers to see the vehicles following an exact trajectory in the presence of disturbances. If these are nominal tarjectories, motivate why we are looking at nominal trajectories.}

% Bay Area level Simulations
% !TEX root = ../SPP_IoTjournal.tex
\section{City Environment Simulation \label{sec:bayArea_sim}}
We next use SPP algorithm to design trajectories for a 200 vehicle UAV system where UAVs are flying through a multi-city region.
% !TEX root = ../../SPP_IoTjournal.tex
\subsection{Setup \label{sec:bayArea_simSetup}}
We grid the San Francisco Bay Area in California, US and use it as our state space, as shown in Figure \ref{fig:bayArea_setup}. We consider the UAVs flying to and from four cities: Richmond, Berkeley, Oakland, and San Francisco. The blue region in Fig. \ref{fig:bayArea_setup} represents bay. This environment is different from the city environment in Section \ref{sec:city_sim} in that now the UAVs need to fly for longer distances and through a high-density vehicle environment with strong winds, but have very few static obstacles like tall buildings.    
%
\begin{figure}
  \centering
  \includegraphics[width=\columnwidth]{figs/bayArea_setup}
  \caption{Multi-city simulation setup. A $300 km^2$ area of San Francisco Bay Area is used as the state-space for vehicles. SPP vehicles fly to and from the four cities indicated by the four circles. The simulations are performed under the strong winds condition with $d_{r} = 11 m/s$.}
  \label{fig:bayArea_setup}
\end{figure}

Each box in Figure \ref{fig:bayArea_setup} represents a $25$ km$^2$ area. The vehicles are flying to and from the four cities indicated by the four circles. The origin and the destination of each vehicle is chosen randomly from these four cities. The vehicle dynamics are given by \eqref{eq:dyn_i}. We choose velocity and turn-rate bounds as $\underline{v} = 0$ m/s, $\bar{v} = 25$ m/s, $\bar\omega = 2$ rad/s. The disturbance bound is chosen as $d_{r} = 11$ m/s, which corresponds to \textit{strong breeze} on Beaufort wind force scale \cite{Windscale}. The scheduled time of arrival $\sta$ for vehicles are chosen as $5(i-1)$ s.

The goal of the vehicles is to reach their destinations while avoiding a collision with the other vehicles. The joint state space of this 200-vehicle system is 600-dimensional, making the joint path planning and collision avoidance problem intractable for direct analysis. Therefore, we assign a priority order to vehicles and solve the path planning problem sequentially.
% !TEX root = ../SPP_IoTjournal.tex
\subsection{Results \label{sec:city_simResults}}

Focus on the following aspects:
\begin{itemize}
\item The technical details for the simulations, like RTT parameters, relative co-ordinate dynamics, rotation and translation of obstacles, union for obstacles, etc. 
\item Demonstration of theory (the vehicles avoid collision w/ other vehicles and reach their destinations).
\item Scaling of SPP.
\item Provide some more intuition about the solution that emerge out of theory-- Space-time separation, type of space-time trajectories (Almost straight line path w/ different starting times?), etc.
\item Reactivity of controller to the actual disturbance (Claire: be very detailed about explaining the setup of simulation)
\item Illustrate how the type of space-time trajectories change with change in disturbance bounds and STA
\end{itemize}

% Conslusion
% !TEX root = ./SPP_IoTjournal.tex
\section{Conclusion}
Provably safe multi-vehicle path planning in an important problem that needs to be addressed to ensure that vehicles can fly in close proximity of each other. Recently, the SPP algorithm was proposed for multi-vehicle path planning problem that scales linearly with the number of vehicles. We illustrate the full potential of the algorithm by using it for large-scale multi-vehicle path planning problems under different flying conditions. We demonstrate how different types of space-time trajectories emerge naturally out of the algorithm for different disturbance conditions and other problem parameters. The reactivity of the obtained controller is also demonstrated under different wind conditions.

\section*{Acknowledgements}
This research is supported by ONR under the Embedded Humans MURI (N00014-16-1-2206).

%%%%%%%%%%%%%%%%%%%%%%%%%%%%%%%%%%%%%%%%%%%%%%%%%%%%%%%%%%%%%%%%%%%%%%%%%%%%%%%%
%\addtolength{\textheight}{1cm}   % This command serves to balance the column lengths
                                  % on the last page of the document manually. It shortens
                                  % the textheight of the last page by a suitable amount.
                                  % This command does not take effect until the next page
                                  % so it should come on the page before the last. Make
                                  % sure that you do not shorten the textheight too much.

\bibliographystyle{aiaa}
\bibliography{references}
\end{document}
