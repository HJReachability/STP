% !TEX root = ../STP_IoTjournal.tex
\subsection{Time-Varying Reachability Background \label{sec:HJIVI}}
We will be using reachability analysis to compute a backward reachable set (BRS) $\brs$ given some target set $\targetset$, time-varying obstacle $\obsset(t)$, and the Hamiltonian function $\ham$ which captures the system dynamics as well as the roles of the control and disturbance. The BRS $\brs$ in a time interval $[t, t_f]$ will be denoted by

\begin{equation}
\brs(t, t_f) \quad \text{ (backward reachable set)}
\end{equation}

Several formulations of reachability are able to account for time-varying obstacles \cite{Bokanowski11, Fisac15} (or state constraints in general). For our application in STP, we utilize the time-varying formulation in \cite{Fisac15}, which accounts for the time-varying nature of systems without requiring augmentation of the state space with the time variable. In the formulation in \cite{Fisac15}, a BRS is computed by solving the following \textit{final value} double-obstacle HJ VI:

\begin{equation}
\label{eq:HJIVI_BRS}
\begin{aligned}
\max \Big\{ \min \{&D_t \valfunc(t, \state) + \ham(t, \state, \nabla \valfunc(t, \state)), \fc(\state) - \valfunc(t, \state) \}, \\
&-\obsfunc(t, \state) - \valfunc(t, \state) \Big\} = 0, \quad t \le t_f \\
&\valfunc(t_f, \state) = \max\{\fc(\state), -\obsfunc(t_f, \state)\}
\end{aligned}
\end{equation}

%In a similar fashion, the FRS is computed by solving the following \textit{initial value} HJ PDE:
%
%\begin{equation}
%\label{eq:HJIVI_FRS}
%\begin{aligned}
%D_t \valfuncfwd(t, \state) + &\ham(t, \state, \nabla \valfuncfwd(t, \state)) = 0 , \quad t \ge t_0  \\
%&\valfuncfwd(t_0, \state) = \max\{\fc(\state), -\obsfunc(t_0, \state)\}
%\end{aligned}
%\end{equation}
%
In \eqref{eq:HJIVI_BRS}, the function $\ic(\state)$ is the implicit surface function representing the target set $\targetset = \{\state: \ic(\state) \le 0\}$. Similarly, the function $\obsfunc(t, \state)$ is the implicit surface function representing the time-varying obstacles $\obsset(t) = \{\state: \obsfunc(t,\state)\le 0\}$. The BRS $\brs(t, t_f)$ is given by

%\begin{equation}
%\label{eq:implicitValfuncs}
%\begin{aligned}
%\brs(t, t_f) &= \{\state: \valfunc(t, \state) \le 0\} \\
%\frs(t_0, t) &= \{\state: \valfuncfwd(t, \state) \le 0 \}
%\end{aligned}
%\end{equation}
\begin{equation}
\label{eq:implicitValfuncs}
\brs(t, t_f) = \{\state: \valfunc(t, \state) \le 0\}
\end{equation}

Some of the reachability computations will not involve an obstacle set $\obsset(t)$, in which case we can simply set $\obsfunc(t, \state) \equiv \infty$ which effectively means that the outside maximum is ignored in \eqref{eq:HJIVI_BRS}.

The Hamiltonian, $\ham(t, \state, \nabla \valfunc(t,\state))$, depends on the system dynamics, and the role of control and disturbance. Whenever $\ham$ does not depend explicit on $t$, we will drop the argument. In addition, the Hamiltonian is an optimization that produces the optimal control $\ctrl^*(t, \state)$ and optimal disturbance $\dstb^*(t, \state)$, once $\valfunc$ is determined. For BRSs, whenever the existence of a control (``$\exists \ctrl$'') or disturbance is sought, the optimization is a minimum over the set of controls or disturbance. Whenever a BRS characterizes the behavior of the system for all controls (``$\forall \ctrl$'') or disturbances, the optimization is a maximum. We will introduce precise definitions of reachable sets, expressions for the Hamiltonian, expressions for the optimal controls as needed for the many different reachability calculations we use.