% !TEX root = SPP2.tex
\section{Introduction}
Recently, there has been an immense surge of interest in using unmanned aerial vehicles (UAVs) for civil purposes. The applications of UAVs extend well beyond package delivery, and include aerial surveillance, disaster response, and other important tasks \cite{Tice91, Debusk10, Amazon16, AUVSI16, BBC16}. Many of these applications will involve UAVs flying in an urban environment, potentially in close proximity of humans. As a result, government agencies such as the Federal Aviation Administration (FAA) and National Aeronautics and Space Administration (NASA) of the United States are urgently trying to develop new scalable ways to organize an air space in which potentially thousands of UAVs could fly \cite{FAA13, NASA16}.

One of the essential problems that needs to be addressed is how a group of vehicles in the same vicinity can reach their respective destinations while avoiding collision with other vehicles. Several previous studies have attempted to address the multi-vehicle path planning and collision avoidance problem. In some of these studies, specific control strategies for the vehicles or moving entities are assumed, and approaches such as induced velocity obstacles have been used \cite{Fiorini98, Chasparis05, Vandenberg08}. \JFFnote{A bit unclear what this last sentence means in this context.} Other researchers have used ideas involving virtual potential fields to maintain collision avoidance while maintaining a specific formation \cite{Saber02, Chuang07}. Although interesting results emerge from these previous studies, simultaneous trajectory planning and collision avoidance are not considered. 

In the past, trajectory planning and collision avoidance problems in safety-critical systems have been studied using reachability analysis, which provides guarantees on the success and safety of optimal system trajectories \cite{Barron90, Mitchell05, Bokanowski10, Fisac15}\JFFnote{We should cite John Lygeros instead of Bokan-owski}. In reachability analysis, one computes the reachable set, defined as the set of states from which the system can be driven to a target set, which can model either unsafe or desirable system configurations. Reachability analysis has been successfully used in applications involving small systems and systems with no more than two vehicles, such as pairwise collision avoidance \cite{Mitchell05}, automated in-flight refueling \cite{Ding08}, two-player reach-avoid games \cite{Huang11}, and many others \cite{Bayen07, Chen15b, Chen15}.\JFFnote{SPP1 paper is not ``small systems"}

In addition to the guarantees reachability theory provides and the evident flexibility of reachability theory for analyzing vastly different systems with nonlinear dynamics, many numerical tools for solving reachability problems are also available, making the approach practically appealing \cite{Mitchell05, Sethian96, Osher02, LSToolbox}.

Despite the advantages of reachability analysis, the curse of dimensionality prevents its direct application to scenarios involving complex high dimensional systems such as multi-vehicle systems. The computation of reachable sets involves solving a Hamilton-Jacobi (HJ) partial differential equation (PDE) on a grid representing a discretization of the state space, causing an exponential scaling of computation complexity with respect to the dimension of the system, or roughly speaking, with the number of vehicles present.

In this paper, we build on the work in \cite{Chen15}, assume a slight structure in the multi-vehicle path planning problem, and perform sequential path planning. In this path planning scheme, vehicles are assigned some priority. Higher-priority vehicles may ignore the lower-priority vehicles, who must take into account the presence of higher-priority vehicles by treating them as induced time-varying obstacles. Unlike the work in \cite{Chen15}, we incorporate disturbances for all vehicles and consider three different assumptions on the information each of the vehicles may have access to, making the sequential path planning substantially more practical. For each of the assumed information patterns, we propose a reachability-based method to compute the induced obstacles that would guarantee collision avoidance as well as successful transit to the destination. We demonstrate and compare our proposed methods through numerical simulations.