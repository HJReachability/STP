% !TEX root = SPP2.tex
Fig. \ref{fig:lrc_traj} shows simulated trajectories in the situation where each vehicle assumes that higher priority vehicles use the least restrictive control, as described in \ref{sec:lrc}.

\subsection{Robust Trajectory Tracking}
Fig. \ref{fig:rtt_traj} shows the vehicle trajectories in the situation where each vehicle tracks a pre-specified trajectory and is guaranteed to stay inside a "bubble" around the trajectory. Fig. \ref{fig:rtt_rs3} shows the evolution of BRS and induced obstacles for vehicle $\veh_3$. The obstacles induced by other vehicles inhibit the evolution of the backward reachable sets, carving out thin “channels,” which can be seen at $t = -2.63$, that separate the reachable set into different “islands”. One can see how these channels and islands form by examining the time evolution of the reach-avoid set, shown in the figure.

\begin{figure}
  \centering
  \includegraphics[width=0.45\textwidth]{"fig/rtt_traj"}
  \caption{Simulated trajectories for the robust trajectory tracking method.}
  \label{fig:rtt_traj}
\end{figure}

$\ldt_i$ numbers for the four vehicles in this case are $-1.63, -3.16, -3.63$ and $-2.49$ respectively. Note that $\ldt_1$ is lower for the highest priority vehicle, $\veh_1$, in this method compared to the other two methods, as higher priority vehicles sacrifice their $\ldt$ in this method (by tracking a pre-defined trajectory and declaring a fixed obstacle size) in order to increase the $\ldt$ of lower priority vehicles, which is also evident from the $\ldt_4$ of vehicle $\veh_4$; however, since only a part of the control authority is used to compute the nominal trajectory to be tracked (and with a smaller target set, see Figure \ref{fig:rtt_rs3}), the overall $\ldt_i, i = 1, \ldots, 4$ might increase or decrease compared to the other methods depending on the specific configuration of the problem. In this example, $\ldt_i$ increases for vehicle $\veh_4$ and decreases for vehicles $\veh_2$ and $\veh_3$ compared to the other methods.     

\begin{figure}[h]
  \centering
  \includegraphics[width=0.45\textwidth]{"fig/rtt_rs3"}
  \caption{Evolution of the reachable set for the third vehicle $\veh_3$ in the robust trajectory tracking method. As the backwards reachable set grows in time, one can see that the induced obstacles carve out a channel. In particular, the obstacles at $t =-0.63$ and $t=-1.83$ are induced by vehicle $\veh_1$ and at $t =-2.63$ by vehicle $\veh_1$. Also, note that a smaller target set is used to compute the backward reachable set to ensure that the whole bubble reaches the target set by $t=0$, as the vehicle can be anywhere within the bubble.}
  \label{fig:rtt_rs3}
\end{figure}