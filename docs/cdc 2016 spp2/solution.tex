% !TEX root = SPP2.tex
\section{Background \label{sec:background}}
This section provides a brief summary of the work in \cite{Chen15}, in which SPP scheme is proposed under perfect information and absence of disturbance. Here, the dynamics of vehicle $\veh_i$ becomes

\begin{equation}
\label{eq:dyn_no_dstb}
\begin{aligned}
\dot{x}_i &= f_i(t, x_i, u_i), \quad t \in [\edt_i, \sta_i] \\
u_i &\in \cset_i \\
i &= 1,\ldots, N
\end{aligned}
\end{equation}

\noindent where the difference compared to \eqref{eq:dyn} is that the disturbance $d_i$ is no longer part of the dynamics.

In order to make the $N$-vehicle path planning problem safe and tractable, a reasonable structure is imposed to the problem: each vehicle is assigned a strict priority ordering. When planning its trajectory to its target, a higher-priority vehicle can disregard the presence of a lower priority vehicle. In contrast, a lower priority vehicle must take into account the presence of all higher priority vehicles, and plan its trajectory in a way that avoids the higher priority vehicles' danger zones. For convenience and without lost of generality, let vehicle $i$ have the $i$th highest priority and denote it as $\veh_i$. 

Under the above convention, each vehicle $\veh_i$ must take into account time-varying obstacles induced by vehicles $\veh_j, j<i$, denoted $\ioset_i^j(t)$. Optimal safe path planning of each lower-priority vehicle $\veh_i$ then consists of determining the optimal path that allows $\veh_i$ to each its target $\targetset_i$ while avoiding the moving obstacles $\obsset_j$, defined by

\begin{equation}
\obsset_i(t) = \bigcup_{j=1}^{i-1}\ioset_i^j(t)
\end{equation}

Such an optimal path planning problem can be solved by computing a backward reachable set (BRS) $\brs_i(t)$ from a target set $\targetset_i$ using formulations of HJ variational inequalities such as \cite{Bokanowski11, Fisac15}. In particular, we will utilize the formulation in \cite{Fisac15}, which does not require augmentation of the state space with the time variable.
\SBnote{...so that we do not increase the problem dimension.}
Starting from the highest-priority vehicle $\veh_1$, one computes the BRS $\brs_1(t)$, from which the optimal control and optimal trajectory $x_1(\cdot)$ to the target $\targetset_1$ can be obtained. Under the absence of disturbances and perfect information, the obstacles induced by $\veh_1$ for lower-priority vehicle $\veh_i$ is simply the danger zone centered around the position of each point $p_1(\cdot)$ on the trajectory:
\SBnote{ I will add a line on methods in [16] and [18], and state why we are using 18. Probably in the introduction.}
\begin{equation}
\ioset_i^1(t) = \{x_j: \|p_j - p_1(\cdot)\|\le\cradius\}
\end{equation}

Given $\ioset_i^j(t), j<i$, and continuing with $i = 2$, the optimal safe trajectories for each vehicle $\veh_i$ can be computed. All of the trajectories are optimal in the sense that given the requirement that $\veh_i$ must arrive at $\targetset_i$ at time $\sta_i$, the latest departure time $\ldt_i$ and the optimal control $u^*_i(\cdot)$ that guarantees arrival at $\sta_i$ can be obtained.