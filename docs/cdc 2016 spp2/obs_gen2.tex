% !TEX root = SPP2.tex
\subsection{Method 2: Least Restrictive Control \label{sec:lrc}}
If there is no centralized controller to enforce the control policy for higher priority vehicles, weaker assumptions must be made by the lower priority vehicles to ensure collision avoidance. One reasonable assumption that a lower priority vehicle can make is that all higher priority vehicles follow the least restrictive control \MCnote{LRC seems to be a common term} that would take them to their targets. This control would be given by 

\begin{equation}
\label{eq:lrctrl} % least restrictive control
u_j \in \begin{cases} \{u_j^*(t, x_j) \text{ given by } \eqref{eq:opt_ctrl_i}\} \text{ if } x_j\in \partial \brs_j(t), \\
\cset_i  \text{ otherwise}
\end{cases}
\end{equation}

Such a controller allows each higher priority vehicle to use any controller it desires, except when it is on the boundary of the BRS, $\partial \brs_j$, in which case the optimal control $u_j^*(t, x_j)$ given by \eqref{eq:opt_ctrl_i} must be used to get to the target on time. This assumption is the weakest assumption that could be made by lower priority vehicles given that the higher priority vehicles will get to their targets on time.

Suppose a lower priority vehicle $\veh_i$ assumes that higher priority vehicles $\veh_j, j < i$ use the least restrictive control strategy \eqref{eq:lrctrl}. From the perspective of the lower priority vehicle $\veh_i$, a higher priority vehicle $\veh_j$ could be in any state that is reachable from $\veh_j$'s initial state $x_j(\edt)$ and from which the target $\targetset_j$ can be reached. Mathematically, this is defined by $\veh_j$ is the intersection of the FRS from the initial state $x_j(\edt)$ and the BRS defined in \eqref{eq:brs} from the target set $\targetset_j$, $\brs_j(t) \cap \frs_j(t)$. In this situation, since $\veh_j$ cannot be assumed to be using any particular feedback control, $\frs_j(t)$ is defined in \eqref{eq:frs2} and can also be computed by solving \eqref{eq:FRS_j}.

\SBnote{traj.-based definition}
\begin{equation}
\label{eq:frs2}
\begin{aligned}
&\frs_j(t) = \{y \in \R^{n_j}: \exists u \in \cfset, \exists d \in \dfset, \\
& \quad \dot{x}_j = f_j(x_j, u_j, d_j) \Rightarrow, x_j(t) = y\}
\end{aligned}
\end{equation}

In turn, the obstacle induced by a higher priority $\veh_j$ for a lower priority vehicle $\veh_i$ is as follows:

\begin{equation}
\ioset_i^j(t) = \{x_i: \dist(\pos_i, \pfrs_j(t)) \le \cradius \}
\end{equation}

\noindent where $\pfrs_j(t)$ is given by

\begin{equation}
\pfrs_j(t) = \{p: \exists \npos_j, (p, \npos_j) \in \brs_j(t) \cap \frs_j(t)\}
\end{equation}