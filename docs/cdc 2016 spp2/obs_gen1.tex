% !TEX root = SPP2.tex
\section{Obstacle Generation \label{sec:obs_gen}}
\SBnote{Use traj.-based notation}
Obstacles can be generated in many different ways depending on the assumptions made about the information the vehicles have about each other. In each of the three obstacle generation methods that we present, the goal is to compute, for each lower priority vehicle $\veh_i$, the time-varying obstacle induced by each higher priority vehicle $\veh_j, j < i$, denoted by $\ioset_i^j(t)$. Once $\ioset_i^j(t)$ is computed, we can then solve \eqref{eq:HJIVI_i} with the union of all obstacles induced by higher priority vehicles as the total obstacle set $\obsset_i(t)$:

\begin{equation}
\obsset_i(t) = \bigcup_{j=1}^{i-1}\ioset_i^j(t)
\end{equation}

\MCnote{note first vehicle has no G_i(t)}
In general, the different methods of obstacle generation can be used in a single path planning problem, since different control strategies can be assumed for each vehicle independently. This means that $\ioset_i^j(t)$ can be computed using a different method for each $j$. For example, a more predictable vehicle may induce obstacles under the stronger assumptions in Section \ref{sec:cc}, while a vehicle that requires more control freedom may induce obstacles under the weaker assumptions in Section \ref{sec:lrc}.

\subsection{Method 1: Centralized Controller \label{sec:cc}}
In this obstacle generation method, the induced obstacle for a vehicle $\veh_i$ is computed assuming that $\veh_j$ is applying the optimal control $u_j^*(t, x_j)$ given by \eqref{eq:opt_ctrl_i}, which takes $\veh_j$ to the target in the optimal way according to the value function $V_j(t, x_j)$. If there is a centralized controller directly controlling each of the $N$ vehicles, then the control law of each vehicle can be enforced. In this case, lower priority vehicles can safely assume that higher priority vehicles are applying the enforced control law. 

From the perspective of a lower priority vehicle $\veh_i$, a higher priority vehicle $\veh_j, j < i$ induces an time-varying obstacle that represents the positions that could possibly be within the capture radius $\cradius$ of $\veh_j$, given that $\veh_j$ is executing the feedback controller $u_j^*(t, x_j)$. Determining this obstacle involves first solving a forward reachability problem. The solution gives us the forward reachable set (FRS) of $\veh_j$ starting from its initial state $x_j(\edt)$ at initial time $\edt$, denoted $\frs_j(t)$ and defined as follows:

\begin{equation}
\label{eq:frs}
\begin{aligned}
\frs_j(t) = \{y \in \R^{n_j}: \dot{x}_j = f_j(x_j, u_j^*(t, x), d_j) \\
\Rightarrow \exists d_j \in \dset_j, x_j(t) = y\}
\end{aligned}
\end{equation}

Conveniently, FRSs can be computed using a modified version of \eqref{eq:HJIVI_i}, defined in $\eqref{eq:FRS_j}$:

\begin{equation}
\label{eq:FRS_j}
\begin{aligned}
&D_t V_j(t, x_j) + H_j\left(t, x_j, D_{x_j} V\right) = 0, t \in [\edt_j, \sta_j]\\
&V_j(\sta_j, x_j) =l_j(x_j) \\
&H_j\left(t, x_j, p\right) = \min_{u_j \in \cset_j} \min_{d_j \in \dset_j} p \cdot f(t, x_j, u_j, d_j)
\end{aligned}
\end{equation}

\noindent where $l$ is chosen to be such that $\targetset = \{x_j(t_0)\}$. To impose that the optimal control is used, we substitute $u_j = u_j^*(t, x_j)$ into \eqref{eq:FRS_j} instead of minimizing over $u_j$. \SBnote{write HJVI with the optimal control plugged in}

The FRS $\frs_j(t)$ represents the set of possible states at time $t$ of a higher-priority vehicle $\veh_j$ given the worst case disturbance $d_j(\cdot)$ and given that $\veh_j$ uses the feedback controller $u_j^*(t, x)$. In order for a lower-priority vehicle $\veh_i$ to guarantee that it does not go within a distance of $\cradius$ to $\veh_j$, $\veh_i$ must stay a distance of at least $\cradius$ away from the set $\frs_j(t)$ for all possible values of the non-position states $\npos_j$. This gives the obstacle induced by a higher priority vehicle $\veh_j$ for a lower priority vehicle $\veh_i$ as follows:

\begin{equation}
\ioset_i^j(t) = \{x_i: \dist(\pos_i, \pfrs_j(t)) \le \cradius \}
\end{equation}

\noindent where the $\dist(\cdot, \cdot)$ function represents the minimum distance from a point to a set, and the set $\pfrs_j(t)$ is the set of states in the FRS $\frs_j(t)$ projected onto the states representing position $\pos_j$, and disregarding the non-position dimensions $\npos_j$:

\begin{equation}
\pfrs_j(t) = \{p: \exists \npos_j, (p, \npos_j) \in \frs_j(t)\}.
\end{equation}

