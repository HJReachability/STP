% !TEX root = SPP2.tex
\section{Numerical Simulations \label{sec:sim}}
We demonstrate our proposed methods using a four-vehicle example. Each vehicle has the following simple mode:

\begin{equation}
\label{eq:dyn_i}
\begin{aligned}
\dot{\pos}_{x,i} &= v_i \cos \theta_i + d_{x,i} \\
\dot{\pos}_{y,i} &= v_i \sin \theta_i + d_{y,i}\\
\dot{\theta}_i &= \omega_i + d_{\theta,i}, \\
\underline{v} & \le v_i \le \bar{v}, |\omega_i| \le \bar{\omega},\\
\|(d_{x,i}, & d_{y,i}) \|_2 \le d_{r}, |d_{\theta,i}| \le \bar{d_{\theta}}
\end{aligned}
\end{equation}

\noindent where $p_i = (p_{x,i}, p_{y,i})$ represent vehicle $\veh_i$'s position, $\theta_i$ represents $\veh_i$'s heading, and $d = (d_{x,i}, d_{y,i}, d_{\theta,i})$ represent the disturbance in the three states. Disturbance in each state is bounded from top and below. The control of $\veh_i$ is $u_i = (v_i, \omega_i)$, where $v_i$ is the speed of $\veh_i$ and $\omega_i$ is the turn rate; both controls have a lower and upper bound. For illustration purposes, we chose $\underline{v} = 0.5, \bar{v} = 1, \bar\omega = 1$; however, our method can easily handle the case in which these inputs differ across vehicles. The disturbance bounds are chosen as $d_{r} = 0.1$ and $\bar{d_{\theta}} = 0.2$, which correspond to a 10\% disturbance in position and a 20\% disturbance in heading. The optimal control for vehicle $i$ can be obtained by optimizing the associated Hamiltonian, $H_i(t, D_{\bm{x}_i} V_i(\bm{x}_i,t), V_i(\bm{x}_i,t))$, and is given by:

\begin{equation}
\omega_i(t) = -\bar{\omega}_i \frac{D_{\theta_i}V_i(\bm{x}_i,t)}{\left| D_{\theta_i}V_i(\bm{x}_i,t) \right|},
\end{equation}

\begin{equation}
v_i(t) =
\left \{ 
\begin{array}{ll}
\underline{v} & \mbox{ if } D_{x_i}V_i(\bm{x}_i,t) \cos \theta_i + D_{y_i}V_i(\bm{x}_i,t) \sin \theta_i \geq 0 \\
\bar{v} & \mbox{ otherwise } 
\end{array}
\right.
\end{equation}

The initial states of the vehicles are given as follows:
\begin{equation}
\begin{aligned}
\bm{x}_1^0 &= (-0.5, 0, 0), \\
\bm{x}_2^0 &= (0.5, 0, \pi), \\
\bm{x}_3^0 &= \left(-0.6, 0.6, 7\pi/4\right), \\
\bm{x}_4^0 &= \left(0.6, 0.6, 5\pi/4\right).\
\end{aligned}
\end{equation}

\noindent Each of the four vehicles have a target set $\targetset_i$ that is circular in their position $\pos_i$ centered at $c_i = (c_{x,i}, c_{y,i})$ with radius $r$:

\begin{equation}
\targetset_i = \{x_i \in \R^3: \|p_i - c_i\| \le r\}
\end{equation}

\noindent For the example shown, we chose $c_1 = (0.7, 0.2), c_2 = (-0.7, 0.2), c_3 = (0.7, -0.7), c_4 = (-0.7, -0.7)$ and $r = 0.1$. The setup of the this examples is shown in Figure (\textbf{add figure number here}).

Since the joint state space of this system is intractable for analysis using the single-obstacle HJI VI, we repeatedly solve the double-obstacle HJI VI (\ref{eq:HJIVI}) to compute the reach-avoid sets from the targets $\targetset_i$ for vehicles $1,2,3,4$, in that order, with moving obstacles induced by vehicles $j=1,\ldots,i-1$. We will also obtain $t_i^{LDT},i=1,2,3,4$, latest departure time for each vehicle in order to reach $\targetset_i$ by a scheduled time of arrival, STA, assumed to be $0$ without loss of generality (\textbf{show figures in negative time to be consistent with this notation}). Note that even though the STA is assumed to be same for all vehicles in this example for simplicity, our method can easily handle the case in which STAs are diffeenet for vehicles.

For each of the three proposed methods of computing induced obstacles, we show each vehicle's entire trajectory (in colored dotted lines). On top of the trajectory plot, we overlay each vehicle's position (colored asterisks) and heading (arrows) at a point in time in which the vehicles are in relatively dense configuration. In all cases, the vehicles are able to avoid each other's danger zones (colored dashed circles) while getting to their target sets in minimum time.

\subsection{Centralized Controller}
Fig. \ref{fig:cc_traj} shows the simulated trajectories in the situation where a centralized controller enforces each vehicle to use the optimal controller $u^*_i(t, x_i)$ according to \eqref{eq:opt_ctrl_i}, as described in Section \ref{sec:cc}.

In this case, each vehicle do not appear to deviate very much from a straight line trajectory towards its target, since the centralized controller is quite restrictive, making the possible positions of higher priority vehicles cover a small area. In the dense configuration at $t=1.5$, the vehicles are close to each other but still outside each other's danger zones.

\begin{figure}
  \centering
  \includegraphics[width=0.5\textwidth]{"fig/cc_traj"}
  \caption{Simulated trajectories in the centralized controller method.}
  \label{fig:cc_traj}
\end{figure}

Fig. \ref{fig:cc_rs3} shows the evolution of the backwards reachable set for the third vehicle (blue boundaries), as well as the obstacles induced by the higher priority vehicles $\veh_1$ and $\veh_2$ (black boundaries). The locations of the induced obstacles at different time points the actual positions of $\veh_1$ and $\veh_2$ at those times, and the size of the obstacles remain relatively small. As the backwards reachable set grows in time, one can see that the induced obstacles carve out a channel, which can be see at $t = 0000$ (\textbf{mention the $t_i^{LDT}$ numbers}).

\begin{figure}
  \centering
  \includegraphics[width=0.5\textwidth]{"fig/cc_rs3"}
  \caption{Evolution of the backwards reachable set for the third vehicle $\veh_3$ in the centralized controller method}
  \label{fig:cc_rs3}
\end{figure}

\subsection{Least Restrictive Control}
Fig. \ref{fig:lrc_traj} shows the simulated trajectories in the situation where each vehicle assumes that higher priority vehicles use the least restrictive control to reach their targets, as described in \ref{sec:lrc}. 

Here, the red vehicle has the highest priority, and takes a quick path to reach its target. 
%\begin{figure}
%  \centering
%  \includegraphics[width=0.5\textwidth]{"fig/lrc_traj"}
%  \caption{Simulated trajectories in the centralized controller method.}
%  \label{fig:lrc_traj}
%\end{figure}



%\begin{figure}
%  \centering
%  \includegraphics[width=0.5\textwidth]{"fig/lrc_rs3"}
%  \caption{Evolution of the reachable set for the third vehicle $\veh_3$ in the centralized controller method}
%  \label{fig:lrc_rs3}
%\end{figure}
(\textbf{mention the $t_i^{LDT}$ numbers})

\subsection{Robust Trajectory Tracking}
%\begin{figure}
%  \centering
%  \includegraphics[width=0.5\textwidth]{"fig/rtt_traj"}
%  \caption{Simulated trajectories in the centralized controller method.}
%  \label{fig:lrc_traj}
%\end{figure}
%
%\begin{figure}
%  \centering
%  \includegraphics[width=0.5\textwidth]{"fig/rtt_rs3"}
%  \caption{Evolution of the reachable set for the third vehicle $\veh_3$ in the centralized controller method}
%  \label{fig:rtt_rs3}
%\end{figure}
(\textbf{mention the $t_i^{LDT}$ numbers})

\section{Comparison of Proposed Methods}
This section briefly discusses the relative advantages and limitations of the proposed obstacle generation methods. Each method basically makes a trade-off between optimality (in terms of LTDs) and flexibility in control and disturbance rejection.

\subsection{Centralized Controller}
Given an order of priority, the vehicles will have the highest possible LTDs in this method for the scenario where a higher priority vehicle maximizes it's LTD as much as it can because every vehicle is taking the optimal action at all times. 
%However, it is important to note that since a higher priority vehicle only minimizes  LTDs can be much higher for lower priority vehicles in some cases, compared to the methods which jointly optimize the LTDs. 
A limitation of this method is that a centralized planner and coordinator is required to ensure that the optimal control is being applied by all vehicles at all times, and hence safety.

\subsection{Least Restrictive Control}
This method gives more control flexibility to the higher priority vehicles, as long as the control doesn't push the vehicle out of it's reach-avoid region. This flexibility, however, comes at the price of having a larger obstacle, and hence a lower LTD, for the lower priority vehicles.  

\subsection{Robust Trajectory Tracking}
Since the obstacle size is constant over time in this method, it's easier to implement from a practical standpoint. This method aims at striking a balance between LTDs across vehicles. In particular, the LTD of a higher priority vehicle can be lower in this method, compared to Method-1 for example, to have a relatively higher LTD for the lower priority vehicle, making this method particularly suitable for the scenarios where there is no strong sense of priority among vehicles. This method, however, is limited to the system which can be transformed into relative dynamics.