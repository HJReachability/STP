%%%%%%%%%%%%%%%%%%%%%%%%%%%%%%%%%%%%%%%%%%%%%%%%%%%%%%%%%%%%%%%%%%%%%%%%%%%%%%%%
%2345678901234567890123456789012345678901234567890123456789012345678901234567890
%        1         2         3         4         5         6         7         8

%\documentclass[letterpaper, 10 pt, conference]{ieeeconf}  % Comment this line out
                                                          % if you need a4paper
\documentclass[letterpaper, 10pt, conference]{ieeeconf}      % Use this line for a4
                                                          % paper

 
\IEEEoverridecommandlockouts                              % This command is only
                                                          % needed if you want to
                                                          % use the \thanks command
\overrideIEEEmargins
% See the \addtolength command later in the file to balance the column lengths
% on the last page of the document

\usepackage{amsmath}    % need for sub equations
\usepackage{amsfonts}
\usepackage{graphicx}   % need for figures
\usepackage{subcaption}
\usepackage{epsfig} 
\usepackage{cancel}
\usepackage{amssymb}
\usepackage{color}
\usepackage[ruled,vlined,titlenotnumbered]{algorithm2e} 
\usepackage{todonotes} \setlength{\marginparwidth}{1.5cm} 

\newcommand{\JFFnote}{\todo[size=\small,author=JFF,color=green]}
\newcommand{\MCnote}{\todo[size=\small,author=MC,color=cyan]}

\newcommand{\R}{\mathbb{R}}
\newcommand{\cset}{\mathcal{U}}
\newcommand{\cfset}{\mathbb{U}}
\newcommand{\dfset}{\mathbb{D}}
\newcommand{\dset}{\mathcal{D}}
\newcommand{\obsset}{\mathcal{G}} % Obstacle (the one used to solve PDE)
\newcommand{\ioset}{\mathcal{O}} % Induced obstacle
\newcommand{\brs}{\mathcal{V}} % backwards reachable set
\newcommand{\frs}{\mathcal{W}} % forwards reachable set
\newcommand{\pfrs}{\mathcal{P}} % projected forwards reachable set
\newcommand{\targetset}{\mathcal{T}}
\newcommand{\edt}{t_0}
\newcommand{\sta}{t_\text{STA}}
\newcommand{\dz}{\mathcal{A}} % Avoid set
\newcommand{\cradius}{R_c} % Capture radius
\newcommand{\pos}{p} % position
\newcommand{\npos}{h} % non-position states
\newcommand{\veh}{Q} % vehicle
\newcommand{\dist}{\text{dist}} % Distance
\newcommand{\rc}{R_c} % Capture radius
\newcommand{\errorbound}{\mathcal{E}} % Error ``bubble" between vehicle and tracking reference
\newcommand{\disckernel}{\Omega} % Discriminating kernel
\newcommand{\tracklaw}{\kappa} % Robust tracking law

\title{\LARGE \bf
Safe Sequential Path Planning of Multi-Vehicle Systems Under Presence of Disturbances and Measurement Noise}

\author{Somil Bansal*, Mo Chen*, Jaime F. Fisac, and Claire J. Tomlin
\thanks{This work has been supported in part by NSF under CPS:ActionWebs (CNS-931843), by ONR under the HUNT (N0014-08-0696) and SMARTS (N00014-09-1-1051) MURIs and by grant N00014-12-1-0609, by AFOSR under the CHASE MURI (FA9550-10-1-0567). The research of M. Chen and J. F. Fisac have received funding from the ``NSERC'' program and ``la Caixa" Foundation, respectively.}
\thanks{* Both authors contributed equally to this work. All authors are with the Department of Electrical Engineering and Computer Sciences, University of California, Berkeley. \{mochen72, somil, jfisac, tomlin\}@eecs.berkeley.edu}
}

\begin{document}
\maketitle
\thispagestyle{empty}
\pagestyle{empty}

%%%
\begin{abstract}
\end{abstract}

% !TEX root = SPP2.tex
\section{Introduction}
Recently, there has been an immense surge of interest in using unmanned aerial vehicles (UAVs) for civil purposes. The applications of UAVs extend well beyond package delivery, and include aerial surveillance, disaster response, and other important tasks \cite{Tice91, Debusk10, Amazon16, AUVSI16, BBC16}. Many of these applications will involve UAVs flying in an urban environment, potentially in close proximity of humans. As a result, government agencies such as the Federal Aviation Administration (FAA) and National Aeronautics and Space Administration (NASA) of the United States are urgently trying to develop new scalable ways to organize an air space in which potentially thousands of UAVs can fly \cite{FAA13, NASA16}.

One essential problem that needs to be addressed is how a group of vehicles in the same vicinity can reach their destinations while avoiding collision with each other. Several previous studies have attempted to address this problem. In some of these studies, specific control strategies for the vehicles or moving entities are assumed, and approaches such as induced velocity obstacles have been used \cite{Fiorini98, Chasparis05, Vandenberg08}. Other researchers have used ideas involving virtual potential fields to maintain collision avoidance while maintaining a specific formation \cite{Saber02, Chuang07}. Although interesting results emerge from these previous studies, simultaneous trajectory planning and collision avoidance are not considered. 

In the past, trajectory planning and collision avoidance problems in safety-critical systems have been studied using reachability analysis, which provides guarantees on the success and safety of optimal system trajectories \cite{Barron90, Mitchell05, Bokanowski10, Margellos11, Fisac15}. In reachability analysis, one computes the reachable set, defined as the set of states from which the system can be driven to a target set. Reachability analysis has been successfully used in applications involving systems with no more than two vehicles, such as pairwise collision avoidance \cite{Mitchell05}, automated in-flight refueling \cite{Ding08}, two-player reach-avoid games \cite{Huang11}, and many others \cite{Bayen07}.

%In addition to the guarantees reachability theory provides and the evident flexibility of reachability theory for analyzing vastly different systems with nonlinear dynamics, many numerical tools for solving reachability problems are also available, making the approach practically appealing \cite{Mitchell05, Sethian96, Osher02, LSToolbox}.

Despite the advantages of reachability analysis, it cannot be directly applied to scenarios involving complex high dimensional systems such as multi-vehicle systems. The computation of reachable sets involves solving a Hamilton-Jacobi (HJ) partial differential equation (PDE) on a grid representing a discretization of the state space, causing an exponential scaling of computation complexity with respect to the dimension of the system, or roughly speaking, with the number of vehicles present.

In this paper, we build on the work in \cite{Chen15}, and assume a reasonable structure in the multi-vehicle path planning problem. In the sequential path planning (SPP) scheme, vehicles are assigned some priority. Higher-priority vehicles may ignore the lower-priority vehicles, who must take into account the presence of higher-priority vehicles by treating them as induced time-varying obstacles. Unlike the work in \cite{Chen15}, we incorporate disturbances for all vehicles and consider three different assumptions on the information each of the vehicles may have access to, making the sequential path planning substantially more practical. For each of the assumed information patterns, we propose a reachability-based method to compute the induced obstacles that would guarantee collision avoidance as well as successful transit to the destination. We demonstrate and compare our proposed methods through numerical simulations.
% Introduction (1-1.5p)
%% Motivation
%% Related work
%% Summary of results

% !TEX root = nextUAVsched.tex
\section{Problem Formulation \label{sec:formulation}}
Consider $N$ vehicles $P_i,i=1\ldots,N$, each trying to reach one of $N$ target sets $\target_i,i=1\ldots,N$, while avoiding obstacles and collision with each other. Each vehicle $i$ has states $\x_i\in \R^{n_i}$ and travels on a domain $\amb=\obs \cup \free\in\R^p$, where $\obs$ represents the obstacles that each vehicle must avoid, and $\free$ represents all other states in the domain on which vehicles can move. Each vehicle $i = 1,2,\ldots,N$ moves with the following dynamics for $t\in[\tnow_i, \tf_i]$:

\begin{equation} \label{eq:dyn}
\dotx_i = f_i (t, \x_i, \ctrl_i), \quad\x_i(\ti_i) = \x_i^0 
\end{equation}

\noindent where $\x_i^0$ represents the initial condition of vehicle $i$, and $\ctrl_i(\cdot)$ represents the control function of vehicle $i$. In general, $f_i(\cdot,\cdot,\cdot)$ depends on the specific dynamic model of vehicle $i$, and need not be of the same form across different vehicles. Denote $\pos_i\in\R^p$ the subset of the states that represent the position of the vehicle. Given $\pos_i^0\in\free$, we define the admissible control function set for $P_i$ to be the set of all control functions such that $\pos_i(t) \in \free \forall t\ge \ti_i$. Denote the joint state space of all vehicles $\x \in \R^n$ where $n = \sum_i n_i$, and their joint control $\ctrl$.

We assume that the control functions $\ctrl_i(\cdot)$ are drawn from the set $\ctrlf_i := \{\ctrl_i: [\tnow_i, \tf_i] \rightarrow \ctrlin_i, \text{measurable}$\footnote{
A function $f:X\to Y$ between two measurable spaces $(X,\Sigma_X)$ and $(Y,\Sigma_Y)$ is said to be measurable if the preimage of a measurable set in $Y$ is a measurable set in $X$, that is: $\forall V\in\Sigma_Y, f^{-1}(V)\in\Sigma_X$, with $\Sigma_X,\Sigma_Y$ $\sigma$-algebras on $X$,$Y$.}\} where $\ctrlin_i \in \R^{n^\ctrl_i}$ is the set of allowed control inputs. Furthermore, we assume $f_i(t,\x_i, \ctrl_i)$ is bounded, Lipschitz continuous in $\x_i$ for any fixed $t,\ctrl_i$, and measurable in $t, \ctrl_i$ for each $\x_i$. Therefore given any initial state $\x_i^0$ and any control function $\ctrl_i(\cdot)$, there exists a unique, continuous trajectory $\x_i(\cdot)$ solving (\ref{eq:dyn}) \cite{coddington55}.

The goal of each vehicle $i$ is to arrive at $\target_i \subset \R^{n_i}$ at or before some scheduled time of arrival (STA) $\tf_i$ in minimum time, while avoiding obstacles and danger with all other vehicles. The target sets $\target_i$ can be used to represent desired kinematic quantities such as position and velocity and, in the case of non-holonomic systems, quantities such as heading angle.  $\tnow_i$ can be interpreted as the earliest start time (EST) of vehicle $i$, before which the vehicle may not depart from its initial state. Further, we define $\ti_i$, the latest (acceptable) start time (LST) for vehicle $i$. Our problem can now be thought of as determining the LST $\ti_i$ for each vehicle to get to $\target_i$ at or before the STA $\tf_i$, and finding a control to do this safely. If the LST is before the EST $\ti_i < \tnow_i$, then it is infeasible for vehicle $i$ to arrive at $\target_i$ at or before the STA $\tf_i$. Comparing $\ti_i$ and $\tnow_i$ is feasibility problem that may arise in practice; however, for simplicity of presentation, we will assume that $\tnow_i\le \ti_i \forall i$.

Danger is described by sets $\danger_{ij}(\x_j) \subset \amb$. In general, the definition of $\danger_{ij}$ depends on the conditions under which vehicles $i$ and $j$ are considered to be in an unsafe configuration, given the state of vehicle $j$. Here, we define danger to be the situation in which the two vehicles come within a certain radius $\Rc$ of each other: $\danger_{ij}(\x_j) = \{\x_i: \| \pos_i - \pos_j\|_2 \le \Rc \}$. Such a danger zone is also used by the FAA \cite{paglione99}. An illustration of the problem setup is shown in Figure \ref{fig:formulation}.

\begin{figure}
	\centering
	\includegraphics[width=0.35\textwidth]{"fig/formulation"}
	\caption{An illustration of the problem formulation with three vehicles. Each vehicle $P_i$ seeks to reach its target set $\target_i$ by time $t=\tf_i$, while avoiding physical obstacles $\obs$ and the danger zones of other vehicles.}
	\label{fig:formulation}
\end{figure}

In general, the above problem must be analyzed in the joint state space of all vehicles, making the solution intractable. In this paper, we will instead consider the problem of performing path planning of the vehicles in a sequential manner. Without loss of generality, we consider the problem of first fixing $i=1$ and determining the optimal control for vehicle $1$, the vehicle with the highest priority. The resulting optimal control $\ctrl_1$ sends vehicle $1$ to $\target_1$ in minimum time. 

Then, we plan the minimum time trajectory for each of the vehicles $2,\ldots,N$, in decreasing order of priority, given the previously-determined trajectories for higher-priority vehicles $1,\ldots,i-1$. We assume that all vehicles have complete information about the states and trajectories of higher-priority vehicles, and that all vehicles adhere to their planned trajectories. Thus, in planning its trajectory, vehicle $i$ treats higher-priority vehicles as known time-varying obstacles. 

With the above sequential path planning (SPP) protocol and assumptions, our problem now reduces to the following for vehicle $i$. Given $\x_j(\cdot), j=1,\ldots,i-1$, determine $\ctrl_i(\cdot)$ that maximizes $\ti_i$ and such that $x_i(\tau) \in \target_i, \tau\le \tf_i$.
% Problem formulation (1.5p)
%% A number of aircrafts aiming to reach a number of destinations respectively, on a certain schedule
%% Disturbance
%% How to guarantee they all get there and on time?
%% Feedback control (centralized server) or distributed control (naive forward reachable set)

% !TEX root = SPP2.tex
\section{Background \label{sec:background}}
This section provides a brief summary of the work in \cite{Chen15}, in which SPP scheme is proposed under perfect information and absence of disturbance. Here, the dynamics of vehicle $\veh_i$ becomes

\begin{equation}
\label{eq:dyn_no_dstb}
\begin{aligned}
\dot{x}_i &= f_i(t, x_i, u_i), \quad t \in [\edt_i, \sta_i] \\
u_i &\in \cset_i \\
i &= 1,\ldots, N
\end{aligned}
\end{equation}

\noindent where the difference compared to \eqref{eq:dyn} is that the disturbance $d_i$ is no longer part of the dynamics.

In order to make the $N$-vehicle path planning problem safe and tractable, a reasonable structure is imposed to the problem: each vehicle is assigned a strict priority ordering. When planning its trajectory to its target, a higher-priority vehicle can disregard the presence of a lower priority vehicle. In contrast, a lower priority vehicle must take into account the presence of all higher priority vehicles, and plan its trajectory in a way that avoids the higher priority vehicles' danger zones. For convenience and without lost of generality, let vehicle $i$ have the $i$th highest priority and denote it as $\veh_i$. 

Under the above convention, each vehicle $\veh_i$ must take into account time-varying obstacles induced by vehicles $\veh_j, j<i$, denoted $\ioset_i^j(t)$. Optimal safe path planning of each lower-priority vehicle $\veh_i$ then consists of determining the optimal path that allows $\veh_i$ to each its target $\targetset_i$ while avoiding the moving obstacles $\obsset_j$, defined by

\begin{equation}
\obsset_i(t) = \bigcup_{j=1}^{i-1}\ioset_i^j(t)
\end{equation}

Such an optimal path planning problem can be solved by computing a backward reachable set (BRS) $\brs_i(t)$ from a target set $\targetset_i$ using formulations of HJ variational inequalities such as \cite{Bokanowski11, Fisac15}. In particular, we will utilize the formulation in \cite{Fisac15}, which does not require augmentation of the state space with the time variable.

Starting from the highest-priority vehicle $\veh_1$, one computes the BRS $\brs_1(t)$, from which the optimal control and optimal trajectory $x_1(\cdot)$ to the target $\targetset_1$ can be obtained. Under the absence of disturbances and perfect information, the obstacles induced by $\veh_1$ for lower-priority vehicle $\veh_i$ is simply the danger zone centered around the position of each point $p_1(\cdot)$ on the trajectory:

\begin{equation}
\ioset_i^1(t) = \{x_j: \|p_j - p_1(\cdot)\|\le\cradius\}
\end{equation}

Given $\ioset_i^j(t), j<i$, and continuing with $i = 2$, the optimal safe trajectories for each vehicle $\veh_i$ can be computed. All of the trajectories are optimal in the sense that given the requirement that $\veh_i$ must arrive at $\targetset_i$ at time $\sta_i$, the latest departure time $\ldt_i$ and the optimal control $u^*_i(\cdot)$ that guarantees arrival at $\sta_i$ can be obtained.
% Solution methodology (1.5-2.5p)
%% Variational inequality to be solved
%% Backwards reachability

% Obstacle generation
% !TEX root = SPP2.tex
\section{Disturbances and Incomplete Information \label{sec:obs_gen}}
Disturbances and incomplete information significantly complicate the SPP scheme. The main difference is that the vehicle dynamics satisfy \eqref{eq:dyn} as opposed to \eqref{eq:dyn_no_dstb}. Committing to exact trajectories is therefore no longer possible, since the disturbance $d_i(\cdot)$ is a priori unknown. Thus, the induced obstacles $\ioset_i^j(t)$ are no longer just the danger zones centered around positions. We present three methods to address the above issues. The methods differ in terms of control policy information that is known to a lower priority vehicle, and have their relative advantages and disadvantages depending on the situation. The three methods are as follows:
\begin{itemize}
\item \textbf{Centralized control}: A specific control strategy is enforced upon a vehicle; this can be achieved, for example, by some central agent such as an air traffic controller.
\item \textbf{Least restrictive control}: A vehicle is required to arrive at its targets on time, but has no other restrictions on its control policy. When the control policy of a vehicle is unknown, but its timely arrive at its target can be assumed, the least restrictive control can be safely assumed by lower-priority vehicles.
\item \textbf{Robust trajectory tracking}: A vehicle declares a nominal trajectory which can be robustly tracked under disturbances.
\end{itemize}

In general, the above methods can be used in combination in a single path planning problem, with each vehicle independently having different control policies. Lower-priority vehicles would then plan their paths while taking into account the control policy of each higher-priority vehicle. For clarity, however, we will present each method as if all vehicles are using the same method of path planning.

In addition, for simplicity of explanation, we will assume that no static obstacles exist. In the situations where static obstacles do exist, the time-varying obstacles $\obsset_i(t)$ simply becomes the union of the induced obstacles $\ioset_i^j(t)$ in \eqref{eq:ioset} and the static obstacles.

\subsection{Method 1: Centralized Controller \label{sec:cc}}
The highest-priority vehicle $\veh_1$ first plans its path by computing the BRS (with $i=1$)
\vspace{-0.3em}
\begin{equation}
\label{eq:BRS}
\begin{aligned}
\brs_i(t) = \{x_i: &\exists u_i(\cdot) \in \cfset_i, \forall d_i(\cdot) \in \dfset_i, x_i(\cdot) \text{ satisfies \eqref{eq:dyn}}, \\
&\forall s \in [t, \sta_i], x_i(s) \notin \obsset_i(s), \\
&\exists s \in [t, \sta_i], x_i(s) \in \targetset_i\}
\end{aligned}
\end{equation}

Since we have assumed no static obstacles exist, we have that for $\veh_1, \obsset_1(s)=\emptyset ~ \forall s \le \sta_i$, and thus the above BRS is well-defined. This BRS can be computed by solving the HJ VI \eqref{eq:HJIVI} with the following Hamiltonian:
\vspace{-0.3em}
\begin{equation}
H_i\left(t, x_i, p\right) = \min_{u_i \in \cset_i} \max_{d_i \in \dset_i} p \cdot f_i(t, x_i, u_i, d_i)
\end{equation}

\noindent where $l_i(x_i), g_i(t,x_i),V_i(t,x_i)$ are implicit surface functions representing the target $\targetset_i, \obsset_i(t), \brs_i(t)$, respectively. From the BRS, we can obtain the optimal control
\vspace{-0.3em}
\begin{equation}
\label{eq:opt_ctrl_i}
u_i^*(t,x_i) =  \arg \min_{u_i \in \cset_i} \max_{d_i \in \dset_i} p \cdot f_i(t, x_i, u_i, d_i)
\end{equation}

The latest departure time $\ldt_i$ is then given by $\arg \inf_t x_{i0} \in \brs_i(t)$.

If there is a centralized controller directly controlling each of the $N$ vehicles, then the control law of each vehicle can be enforced. In this case, lower priority vehicles can safely assume that higher priority vehicles are applying the enforced control law. In particular, the optimal controller for getting to the target, $u^*_i(t, x_i)$ can be enforced. In this case, the dynamics of each vehicle becomes 
\vspace{-0.3em}
\begin{equation}
\label{eq:dyn_cc}
\begin{aligned}
\dot x_i &= f^*_i (t, x_i, d_i) = f_i(t, x_i, u^*_i(t,x_i), d_i) \\
d_i &\in \dset_i, \quad i = 1,\ldots, N, \quad t \in [\ldt_i, \sta_i]
\end{aligned}
\end{equation}

\noindent where $u_i$ no longer appears explicitly in the dynamics.

From the perspective of a lower-priority vehicle $\veh_i$, a higher-priority vehicle $\veh_j, j < i$ induces an time-varying obstacle that represents the positions that could possibly be within the capture radius $\cradius$ of $\veh_j$ under the dynamics $f^*_j(t, x_j, d_j)$. Determining this obstacle involves computing a forward reachable set (FRS) of $\veh_j$ starting from $x_j(\ldt) = x_{j0}$. The FRS $\frs_j(t)$ is defined as follows:
\vspace{-0.3em}
\begin{equation}
\label{eq:FRS1}
\begin{aligned}
&\frs_j(t) = \{y \in \R^{n_j}: \exists d_j(\cdot) \in \dfset_j, \\
&x_j(\cdot) \text{ satisfies \eqref{eq:dyn_cc}}, x_j(\ldt) = x_{j0}, x_j(t) = y\}
\end{aligned}
\end{equation}

Conveniently, the FRS can be computed using the following HJ VI:
\vspace{-0.4em}
\begin{equation}
\label{eq:FRS_HJIVI}
\begin{aligned}
&D_t W_j(t, x_j) + H_j\left(t, x_j, D_{x_j} W_j\right) = 0, t \in [\ldt_j, \sta_j]\\
&W_j(\ldt_j, x_j) = \bar l_j(x_j) \\
\end{aligned}
\end{equation}

\noindent with the following Hamiltonian
\begin{equation}
H_j\left(t, x_j, p\right) = \min_{d_j \in \dset_j} p \cdot f^*_j(t, x_j, d_j)
\end{equation}
\noindent where $\bar l$ is chosen to be\footnote{In practice, we define the target set to be a small region around the vehicle's initial state for computational reasons.} such that $\bar l (y) = 0 \Leftrightarrow y = x_j(\ldt)$.

The FRS $\frs_j(t)$ represents the set of possible states at time $t$ of a higher-priority vehicle $\veh_j$ given the worst case disturbance $d_j(\cdot)$ and given that $\veh_j$ uses the feedback controller $u_j^*(t, x_j)$. In order for a lower-priority vehicle $\veh_i$ to guarantee that it does not go within a distance of $\cradius$ to $\veh_j$, $\veh_i$ must stay a distance of at least $\cradius$ away from the set $\frs_j(t)$ for all possible values of the non-position states $\npos_j$. This gives the obstacle induced by a higher priority vehicle $\veh_j$ for a lower priority vehicle $\veh_i$ as follows:
\vspace{-0.4em}
\begin{equation}
\ioset_i^j(t) = \{x_i: \dist(\pos_i, \pfrs_j(t)) \le \cradius \}
\end{equation}

\noindent where the $\dist(\cdot, \cdot)$ function represents the minimum distance from a point to a set, and the set $\pfrs_j(t)$ is the set of states in the FRS $\frs_j(t)$ projected onto the states representing position $\pos_j$, and disregarding the non-position dimensions $\npos_j$:
\vspace{-0.4em}
\begin{equation}
\pfrs_j(t) = \{p: \exists \npos_j, (p, \npos_j) \in \frs_j(t)\}.
\end{equation}

Finally, taking the union of the induced obstacles $\ioset_i^j(t)$ as in \eqref{eq:ioset} gives us the time-varying obstacles $\obsset_i(t)$ needed to define and determine the BRS $\brs_i(t)$ in \eqref{eq:BRS}. Repeating this process, all vehicles will be able to plan paths that guarantee the vehicles' timely and safe arrival.
 % with feedback control
% !TEX root = SPP2.tex
\subsection{Method 2: Least Restrictive Control \label{sec:lrc}}
Here, we again begin with the highest vehicle $\veh_1$ planning its path by computing the BRS $\brs_i(t)$ in \eqref{eq:BRS}. However, if there is no centralized controller to enforce the control policy for higher-priority vehicles, weaker assumptions must be made by the lower-priority vehicles to ensure collision avoidance. One reasonable assumption that a lower-priority vehicle can make is that all higher-priority vehicles follow the least restrictive control that would take them to their targets. This control would be given by 
\vspace{-0.4em}
\begin{equation}
\label{eq:lrctrl} % least restrictive control
u_j(t, x_j)\in \begin{cases} \{u_j^*(t, x_j) \text{ given by } \eqref{eq:opt_ctrl_i}\} \text{ if } x_j(t)\in \partial \brs_j(t), \\
\cset_i  \text{ otherwise}
\end{cases}
\end{equation}

Such a controller allows each higher priority vehicle to use any controller it desires, except when it is on the boundary of the BRS, $\partial \brs_j(t)$, in which case the optimal control $u_j^*(t, x_j)$ given by \eqref{eq:opt_ctrl_i} must be used to get to the target on time. This assumption is the weakest assumption that could be made by lower priority vehicles given that the higher priority vehicles will get to their targets on time.

Suppose a lower-priority vehicle $\veh_i$ assumes that higher-priority vehicles $\veh_j, j < i$ use the least restrictive control strategy in \eqref{eq:lrctrl}. From the perspective of the lower-priority vehicle $\veh_i$, a higher-priority vehicle $\veh_j$ could be in any state that is reachable from $\veh_j$'s initial state $x_j(\ldt) = x_{j0}$ and from which the target $\targetset_j$ can be reached. Mathematically, this is defined by the intersection of a FRS from the initial state $x_j(\ldt)=x_{j0}$ and the BRS defined in \eqref{eq:BRS} from the target set $\targetset_j$, $\brs_j(t) \cap \frs_j(t)$. In this situation, since $\veh_j$ cannot be assumed to be using any particular feedback control, $\frs_j(t)$ is defined in \eqref{eq:FRS2}.
\vspace{-0.4em}
\MCnote{overloaded notation for FRS}
\begin{equation}
\label{eq:FRS2}
\begin{aligned}
\frs_j(t) &= \{y \in \R^{n_j}: \exists u_j(\cdot)\in\cfset_j, \exists d_j(\cdot) \in \dfset_j, \\
&x_j(\cdot) \text{ satisfies \eqref{eq:dyn}, }x_j(t) = y\}
\end{aligned}
\end{equation}

This FRS can be computed by solving \eqref{eq:FRS_HJIVI} with
\vspace{-0.4em}
\begin{equation}
H_j\left(t, x_j, p\right) = \min_{u_j \in \cset_j} \min_{d_j \in \dset_j} p \cdot f_j(t, x_j, u_j, d_j)
\end{equation}
\MCnote{mention without obstacles?}
In turn, the obstacle induced by a higher priority $\veh_j$ for a lower priority vehicle $\veh_i$ is as follows:
\vspace{-0.4em}
\begin{equation}
\begin{aligned}
\ioset_i^j(t) &= \{x_i: \dist(\pos_i, \pfrs_j(t)) \le \cradius \}, \text{ with} \\
\pfrs_j(t) &= \{p: \exists \npos_j, (p, \npos_j) \in \brs_j(t) \cap \frs_j(t)\}
\end{aligned}
\end{equation}

Note that the centralized controller method described in the previous section can be thought of as the ``most restrictive control'' method, in which all vehicles must use the optimal controller at all times, while the least restrictive control method allows vehicles to use any suboptimal controller that allows them to arrive at the target on time. These two methods can be considered two extremes of a spectrum in which varying degrees of optimality is assumed for higher-priority vehicles. % with intersection of FRS and BRS
% !TEX root = SPP2.tex
\subsection{Method 3: Robust Tracking of Nominal Trajectories}
A general issue with the above two methods can be the large size of the generated obstacles due to the uncertainty on other vehicles' motion. In order to reduce this uncertainty, one can have vehicles commit to approximately tracking a robustly feasible nominal trajectory obtained in the path-planning phase. If a vehicle can guarantee that it will track a time trajectory with a bounded error at all times, then this can be used to limit the size of the resulting time-varying obstacle. This approach requires a moderate amount of information sharing between vehicles: higher-priority vehicles need to declare their nominal trajectory, together with an uncertainty region, to all vehicles with lower priority. Lower-priority vehicles will then use these nominal trajectories ``expanded" by the uncertainty regions as time-varying obstacles.

To define each vehicle's uncertainty set, we need to solve a robust trajectory tracking problem.
Typically, the planning phase will not fully exploit the vehicle's full control authority, first due to energy efficiency considerations and second due to the need to leave some margin for unexpected needs and disturbance rejection. It is this margin that we will exploit in the robust tracking. Suppose that the planning in Section \ref{sec:solution} is done for a control authority $\cset^p\subset\cset$. This means that the resulting trajectory reference that the vehicle needs to track will maneuver less aggressively than the true vehicle: replicating the nominal control is therefore always possible, while the additional real-time maneuverability can be used to counteract the external disturbance and make the error dynamics asymptotically stable. In this context, robust nonlinear control techniques such as Lyapunov-based methods
\cite{Majumdar2013}\JFFnote{add citation}{ }%http://link.springer.com/10.1007/978-3-642-36279-8_33
can be used to compute robust ``funnels" around a concrete trajectory.

Instead, here we propose a reachability-based method to compute a trajectory-independent uncertainty set. In particular, we wish to find a robust controlled-invariant set in the joint state space of the vehicle and a tracking reference that may ``maneuver" arbitrarily over time, and in the presence of an unknown bounded disturbance. Taking a worst-case approach, the tracking reference can be thought of as a virtual evader vehicle that is optimally attempting to violate our desired bound on the tracking error. We therefore have a new reach-avoid game in which the controller is playing against the coordinated worst-case action of the reference and the disturbance. In general, this game will be governed by dynamics of the form:
\begin{equation}
\label{eq:jdyn} % Joint dynamics
\begin{aligned}
\dot{x} &= f(t, x, u, d) \\
\dot{x_r} &=f(t,x_r,u_r,0)\\
u &\in \cset, u_r\in\cset^p, d \in \dset \\
&t \in [0, T].
\end{aligned}
\end{equation}
Given a desired bound $\errorbound$ on the tracking error $e=x-x_r$, we can now represent the target set $\targetset$ for the adversarial reference and disturbance as the set of joint configurations where this bound is violated, that is, $\targetset = \{(x,x_r): x-x_r\not\in\errorbound\}$. In this case the backwards reachable set $\brs(t)$ is defined as
\begin{equation}
\label{eq:brs}
\begin{aligned}
&\brs(t) = \{(x,x_r):  \exists \lambda[u] \in \Lambda, \forall u \in \cfset, \eqref{eq:jdyn} \\
&\Rightarrow \exists s \in [t, T], (x(s),x_r(s)) \in \targetset(s)\},
\end{aligned}
\end{equation}
where $\Lambda$ is the set of nonanticipative strategies from $\cfset$ to $\cfset^p\times \dfset$.
With analogous definitions as those in Section \ref{sec:solution}, this set can be characterized as the negative region of the solution $V$ to a simpler case of \eqref{eq:HJIVI}:
\begin{equation}
\label{eq:HJIVI_track}
\begin{aligned}
\min\big\{&D_t V(t, z) + H\left(t, z, D_z V\right), l(t,z) - V(t, z)\big\}= 0,\\&  t\in[0,T]\\
&V(T, z) = l(T, z)\\
&H\left(t, z, p\right) = \max_{u \in \cset} \min_{u_r \in\cset^p} \min_{d \in \dset} p \cdot f_z(t,z,u,u_r,d)
\end{aligned}
\end{equation}
where for compactness of notation we denote $z=(x,x_r)$ and $f_z(t,z,u,u_r,d) = [f(t,x,u,d),f(t,x_r,u_r,0)]$. In general we will let $T\to\infty$ to obtain an all-time robust controlled invariant set.
It is worth noting that for a large class of vehicle dynamics it will be possible to exactly or approximately express the error dynamics as independent of the absolute state. In such cases, the dimensionality of the problem is reduced significantly by rewriting the problem as:
\begin{equation}
\label{eq:edyn} % Error dynamics
\begin{aligned}
\dot{e} &= f_e(t, e, u, u_r,d) \\
u &\in \cset, u_r\in\cset^p, d \in \dset \\
&t \in [0, T],
\end{aligned}
\end{equation}
now requiring the state (error) to remain in $\errorbound$, so that the target simply becomes $\targetset = \mathbb{R}^n \setminus \errorbound$.

If the complement of the reachable set $\brs$ is nonempty, then the vehicle is guaranteed to remain within $\errorbound$ of the nominal trajectory provided that it starts inside the set $\{x_r(0) + e: e\in{\brs^c}\}$. % with bubbles

% !TEX root = ../SPP_IoTjournal.tex
\subsection{Results \label{sec:city_simResults}}

Focus on the following aspects:
\begin{itemize}
\item The technical details for the simulations, like RTT parameters, relative co-ordinate dynamics, rotation and translation of obstacles, union for obstacles, etc. 
\item Demonstration of theory (the vehicles avoid collision w/ other vehicles and reach their destinations).
\item Scaling of SPP.
\item Provide some more intuition about the solution that emerge out of theory-- Space-time separation, type of space-time trajectories (Almost straight line path w/ different starting times?), etc.
\item Reactivity of controller to the actual disturbance (Claire: be very detailed about explaining the setup of simulation)
\item Illustrate how the type of space-time trajectories change with change in disturbance bounds and STA
\end{itemize}
% Numerical Simuations (1-2p)

% !TEX root = ./SPP_IoTjournal.tex
\section{Conclusion}
Provably safe multi-vehicle path planning in an important problem that needs to be addressed to ensure that vehicles can fly in close proximity of each other. Recently, the SPP algorithm was proposed for multi-vehicle path planning problem that scales linearly with the number of vehicles. We illustrate the full potential of the algorithm by using it for large-scale multi-vehicle path planning problems under different flying conditions. We demonstrate how different types of space-time trajectories emerge naturally out of the algorithm for different disturbance conditions and other problem parameters. The reactivity of the obtained controller is also demonstrated under different wind conditions.
% Conclusion (0.5p)

%%%%%%%%%%%%%%%%%%%%%%%%%%%%%%%%%%%%%%%%%%%%%%%%%%%%%%%%%%%%%%%%%%%%%%%%%%%%%%%%
%\addtolength{\textheight}{1cm}   % This command serves to balance the column lengths
                                  % on the last page of the document manually. It shortens
                                  % the textheight of the last page by a suitable amount.
                                  % This command does not take effect until the next page
                                  % so it should come on the page before the last. Make
                                  % sure that you do not shorten the textheight too much.

\bibliographystyle{IEEEtran}
\bibliography{references}
\end{document}
