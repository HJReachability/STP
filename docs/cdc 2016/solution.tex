% !TEX root = SPP2.tex
\section{Solution via double-obstacle HJI VI and SPP\label{sec:solution}}
\subsection{Double-Obstacle Hamilton-Jacobi Variational Inequality}
Our solution method takes advantage of the double-obstacle HJ approach \cite{Fisac15}, in which one computes the reachable set $\reachset(t)$ in the presence of a time-varying target set $\targetset(t)$ and time-varying obstacles $\obsset(t)$. Mathematically, we are given a system with state $z$ evolving according to the following ODE:

\begin{equation}
\label{eq:fdyn} % Full dynamics
\begin{aligned}
\dot{z} &= f(t, z, u, d) \\
z(0) &= z_0 \\
u &\in \cset, d \in \dset \\
&t \in [0, T]
\end{aligned}
\end{equation}

After defining some target set $\targetset(t)$, we compute the reachable set $\reachset(t)$, defined by

\begin{equation}
\begin{aligned}
&\targetset(t) = \{z: \exists u \in \cfset, \forall \gamma[u] \in \Gamma, \eqref{eq:fdyn} \\
&\Rightarrow \exists s \in [t, T], z(s) \in \targetset(s) \wedge z(\tau) \notin \obsset(\tau) \forall \tau \in [t, s]\}
\end{aligned}
\end{equation}

\noindent where $\cfset$ is the set of measurable functions satisfying control constraints at every $t$, and $\Gamma$ is the set of non-anticipative strategies \cite{Mitchell05}. Intuitively, the reachable set is the set of states from which exists a control such that for all non-anticipative disturbances, the system is driven into the target set $\targetset(t)$ in the time horizon $[t, T]$ without first entering the obstacle set $\obsset(t)$.

Given the target set $\targetset(t)$ specified as an implicit surface function such that $\targetset(t) = \{z: l(t, z) \le 0\}$, the reachable set can be obtained as the implicit surface function such that $\reachset(t) = \{z: V(t, z) \le 0\}$, where $V(t, z)$ is the viscosity solution \cite{Crandall83} to the following HJ variational inequality:

\begin{equation}
\label{eq:HJIVI}
\begin{aligned}
\max\Big\{&\min\big\{D_t V(t, z) + H\left(t, z, D_z V\right), l(t,z) - V(t, z)\big\}\\
& g(t, z) - V(t, z)\Big\} = 0, \quad t\in[0,T]\\
&V(T, x) = \max\big\{l(T, x), g(T, x)\big\}
\end{aligned}
\end{equation}

\begin{equation*}
H\left(z, p\right) = \min_{u \in \cset} p \cdot f(z, u, d).
\end{equation*}

\noindent where $g(t, x)$ is the implicit surface function representing $\obsset(t)$: $\obsset(t) = \{z: g(t, x) \le 0\}$. After the reachable set is computed, the optimal control can be obtained as follows:

\begin{equation}
\label{eq:opt_ctrl}
u^* = \arg \min_{u \in \cset} H\left(t, z, D_z V\right)
\end{equation}

In theory, one could define the state to be the joint states of all vehicles, $z = (x_1, x_2, \ldots, x_N)$, define the dynamics \eqref{eq:fdyn} to follow $\eqref{eq:dyn}$, the target set $\targetset$ to correspond to the situation in which all vehicles have arrived at their targets $\targetset_i, i = 1, \ldots, N$, and the obstacle set $\obsset$ to correspond to the union of all the danger zones $\dz_{ij}$. Then, \eqref{eq:HJIVI} could be solved to obtain $\reachset(t)$, and then the joint optimal control would be given by \eqref{eq:opt_ctrl}.

However, practically, the dimensionality of the joint state $z$ would be extremely high. In fact, for even the simplest vehicle models, solving \eqref{eq:HJIVI} would be intractable for more than two vehicles. Therefore, we propose the sequential path planning method, which allows \eqref{eq:HJIVI} to be solved in the state space of each vehicle, making the computation complexity scale linearly, as opposed to exponentially, with the number of vehicles.

\subsection{Sequential Path Planning}
- Priorities
- Treat higher priority vehicles as obstacles
\subsection{Obstacle Generation}
- Forward reachable set
\subsubsection{Centralized Planning}
\subsubsection{Distributed Planning}

\subsection{State Measurement Updates}