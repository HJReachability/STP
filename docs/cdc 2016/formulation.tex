% !TEX root = SPP2.tex
\section{Problem Formulation \label{sec:formulation}}
Consider $N$ vehicles whose joint dynamics described by the time-varying ordinary differential equation

\begin{equation}
\label{eq:dyn}
\begin{aligned}
\dot{x}_i = f_i(t, x_i, u_i, d_i) \\
u_i \in \cset_i \\
d_i \in \dset_i \\
i = 1,\ldots, N
\end{aligned}
\end{equation}

where $x_i \in \R^{n_i}$ is the state of the $i$th vehicle, $u_i$ is the control and of the $i$th vehicle, and $d_i$ is the disturbance experienced by the $i$th vehicle. In general, the physical meaning of $x_i$ and the dynamics $f_i$ depend on the specific dynamic model of vehicle $i$, and need not be the same across the different vehicles. 

We assume that the control functions $u_i(\cdot), d_i(\cdot)$ are drawn from the set of measurable functions\footnote{
A function $f:X\to Y$ between two measurable spaces $(X,\Sigma_X)$ and $(Y,\Sigma_Y)$ is said to be measurable if the preimage of a measurable set in $Y$ is a measurable set in $X$, that is: $\forall V\in\Sigma_Y, f^{-1}(V)\in\Sigma_X$, with $\Sigma_X,\Sigma_Y$ $\sigma$-algebras on $X$,$Y$.}. Furthermore, we assume $f_i(t,x_i, u_i, d_i)$ is bounded, Lipschitz continuous in $x_i$ for any fixed $t, u_i, d_i$, and measurable in $t, u_i, d_i$ for each $x_i$. Therefore given any initial state $x_i^0$ and any control function $u_i(\cdot)$, there exists a unique, continuous trajectory $x_i(\cdot)$ solving \eqref{eq:dyn} \cite{Coddington55}.

For convenience, let $\pos_i \in \R^\pos$ denote the position of vehicle $i$; note that $\pos_i$ in most practical cases would be a subset of the state $x_i$. Denote the rest of the states $\npos_i$, so that $x_i = (\pos_i, \npos_i)$. Under the worst case disturbance, each vehicle aims to get to some set of target states, denoted $\targetset_i \subset \R^{n_i}$ at some scheduled time of arrival $\sta$. On its way to the target set $\targetset_i$, each vehicle must avoid the danger zones $\dz_{ij}(t)$ of all other vehicles $j\neq i$ for all time. In general, the danger zone can be defined to capture any undesirable configuration between vehicle $i$ and vehicle $j$. For simplicity, in this paper we define $\dz_{ij}(t)$ as

\begin{equation}
\dz_{ij}(t) = \{x_i \in \R^{n_i}: \|\pos_i - \pos_j(t)\| \le \cradius \},
\end{equation}

\noindent the interpretation of which is that a vehicle is another vehicle's danger zone if the two vehicles are within a distance of $\cradius$ apart.

The problem of driving each of the vehicles in \eqref{eq:dyn} into their respective target sets $\targetset_i$ would be in general a differential game of dimension $\sum_i n_i$. Due to the exponential scaling of the complexity of the state space with the problem dimension, an optimal solution is computationally intractable.

In this paper, we impose a mild structure to the general problem in order to trade complexity for optimality: we assign a priority to each vehicle. While traveling to its target set, a vehicle may ignore the presence of lower priority vehicles, but must take full responsibility for avoiding higher priority vehicles. Such a joint path planning scheme makes intuitive and practical sense, and the priorities can be assigned, for example, on a first-come first-serve basis.

Recently, \cite{Chen15} described how such a sequential path planning algorithm can be implemented using a HJ reachability approach without taking into account the presence of the disturbances $d_i$ and limited information available to each vehicle. In this paper, we extend the work in \cite{Chen15} to consider these practically important aspects of the problem. In particular, we answer the following inter-dependent questions that were not previously addressed:

\begin{enumerate}
\item How can each vehicle guarantee that it will reach its target set without getting into any danger zones, despite the disturbances it experiences?
\item How can each vehicle take into account the disturbances that other vehicles experience?
\item How should each vehicle robustly handle situations with limited information about the state and intention of other vehicles?
\end{enumerate}