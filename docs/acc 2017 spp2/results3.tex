% !TEX root = SPP2.tex
\subsection{Robust Trajectory Tracking}
Fig. \ref{fig:rtt_traj} shows the vehicle trajectories in the situation where each vehicle tracks a pre-specified trajectory and is guaranteed to stay inside a ``bubble" around the trajectory. Fig. \ref{fig:rtt_rs3} shows the evolution of BRS and induced obstacles for vehicle $\veh_3$. The obstacles induced by other vehicles inhibit the evolution of the BRS, carving out thin “channels,” which can be seen at $t = -2.59$, that separate the BRS into different “islands”. One can see how these channels and islands form by examining the time evolution of the BRS set.

\begin{figure}
  \centering
  \includegraphics[width=0.40\textwidth]{"fig/rtt_traj"}
  \caption{Simulated trajectories for the robust trajectory tracking method.}
  \label{fig:rtt_traj}
  \vspace{-2em}
\end{figure}

$\ldt_i$ numbers for the four vehicles in this case are $-1.61, -3.16, -3.57$ and $-2.47$ respectively. In this method, vehicles use reduced control authority for path planning towards a reduced-size effective target set. As a result, higher-priority vehicles tend to have higher $\ldt$ compared to the other two methods, as evident from $\ldt_1$. Because of this ``sacrifice" by the higher-priority vehicles during the path planning phase, the $\ldt$ of lower-priority vehicles may increase compared to that in the other methods, as evident from $\ldt_4$. Overall, it is unclear whether $\ldt_i$ for a vehicle would increase or decrease compared to the other methods, as $\ldt_i$ is increased by a conservative path planning by higher-priority vehicles, and decreased by a conservative path planning of $\veh_i$. 

\begin{figure}[h]
  \centering
  \includegraphics[width=0.40\textwidth]{"fig/rtt_rs3"}
  \caption{Evolution of the BRS for $\veh_3$ in the robust trajectory tracking method. As the BRS grows in time, the induced obstacles carve out a channel. Note that a smaller target set is used to compute the BRS to ensure that the vehicle reaches the target set by $t=0$ for any allowed tracking error.}
  \label{fig:rtt_rs3}
  \vspace{-1em}
\end{figure}
\vspace{-0.2em}