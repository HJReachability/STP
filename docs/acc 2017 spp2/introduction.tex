% !TEX root = SPP2.tex
\section{Introduction}
Recently, there has been an immense surge of interest in using unmanned aerial vehicles (UAVs) for civil purposes. The applications of UAVs extend well beyond package delivery, and include aerial surveillance, disaster response, and other important tasks \cite{Tice91, Debusk10, Amazon16, AUVSI16, BBC16}. Many of these applications will involve UAVs flying in an urban environment, potentially in close proximity of humans. As a result, government agencies such as the Federal Aviation Administration (FAA) and National Aeronautics and Space Administration (NASA) of the United States are urgently trying to develop new scalable ways to organize an air space in which potentially thousands of UAVs can fly \cite{FAA13, NASA16,Kopardekar16}.

One essential problem that needs to be addressed is how a group of vehicles in the same vicinity can reach their destinations while avoiding collision with each other. Several previous studies have attempted to address this problem. In some of these studies, specific control strategies for the vehicles or moving entities are assumed, and approaches such as induced velocity obstacles have been used \cite{Fiorini98, Chasparis05, Vandenberg08}. Other researchers have used ideas involving virtual potential fields to maintain collision avoidance while maintaining a specific formation \cite{Saber02, Chuang07}. Although interesting results emerge from these previous studies, simultaneous trajectory planning and collision avoidance are not considered. 

In the past, trajectory planning and collision avoidance problems in safety-critical systems have been studied using reachability analysis, which provides guarantees on the success and safety of optimal system trajectories \cite{Barron90, Mitchell05, Bokanowski10, Bokanowski11, Margellos11, Fisac15}. In reachability analysis, one computes the reachable set, defined as the set of states from which the system can be driven to a target set. Reachability analysis has been successfully used in applications involving systems with no more than two vehicles, such as pairwise collision avoidance \cite{Mitchell05}, automated in-flight refueling \cite{Ding08}, two-player reach-avoid games \cite{Huang11}, and many others \cite{Bayen07}.

Despite the advantages of reachability analysis, it cannot be directly applied to scenarios involving complex high dimensional systems such as multi-vehicle systems. The computation of reachable sets involves solving a Hamilton-Jacobi (HJ) partial differential equation (PDE) on a grid representing a discretization of the state space, causing an exponential scaling of computation complexity with respect to the dimension of the system, or roughly speaking, with the number of vehicles present. In \cite{Chen15}, the authors presented the sequential path planning (SPP), in which vehicles are assigned a strict priority ordering. Higher-priority vehicles ignore the lower-priority vehicles, who must take into account the presence of higher-priority vehicles by treating them as induced time-varying obstacles. Under this structure, computation complexity scales just \textit{linearly} with the number of vehicles. In addition, such a structure has the potential to flexibly divide up the airspace for the use of many UAVs, which is an important task in NASA's concept of operations for unmanned aerial systems traffic management \cite{Kopardekar16}. However, the previous formulation assumes an absense of disturbances and perfect information about all vehicles. Unfortunately, perfect information about other vehicles' states and control strategies cannot be realistically assumed, and disturbances would make it impossible to commit to exact trajectories as required in \cite{Chen15}.

In order for any path planning scheme to be viable, perfect information cannot be assumed, and disturbances must be accounted for. To take advantage of the computation benefits of the SPP scheme while resolving some of its practical challenge, in this paper we accomplish the following:

\begin{itemize}
\item Incorporate disturbances into the vehicle models,
\item analyze three different assumptions on the information to which each vehicle may have access.
\end{itemize}

For each assumed information pattern, we propose a reachability-based method to compute the induced obstacles that would guarantee collision avoidance as well as successful transit to the destination. We demonstrate and compare our proposed methods through numerical simulations.