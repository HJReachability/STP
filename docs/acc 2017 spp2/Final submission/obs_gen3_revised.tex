% !TEX root = ../SPP_journal.tex
\subsection{Method 3: Robust Trajectory Tracking\label{sec:rtt}}
Even though it is impossible to commit to and track an exact trajectory in the presence of disturbances, it may still be possible to instead \textit{robustly} track a feasible \textit{nominal} trajectory with a bounded error at all times. If this can be done, then the tracking error bound can be used to determine the induced obstacles. Here, computation is done in two phases: the \textit{planning phase} and the \textit{disturbance rejection phase}. In the planning phase, we compute a nominal trajectory $\state_{r,j}(\cdot)$ that is feasible in the absence of disturbances. In the disturbance rejection phase, we compute a bound on the tracking error.%\MCnote{don't need to explain where error comes bound, imo}%, caused by a vehicle's inability to exactly track the nominal trajectory in the presence of disturbances. 

In the planning phase, planning is done for a reduced control set $\cset^p\subset\cset$, as some margin is needed to reject unexpected disturbances while tracking the nominal trajectory. In the disturbance rejection phase, we determine the error bound independently of the nominal trajectory. Let $\state_j$ and $\state_{r,j}$ denote the states of the actual vehicle $\veh_j$ and the virtual evader, respectively, and define the tracking error $e_j=\state_j-\state_{r,j}$. When the error dynamics are independent of the absolute state as in \eqref{eq:edyn} (and also (7) in \cite{Mitchell05}), we can obtain error dynamics of the form
\begin{equation}
\label{eq:edyn} % Error dynamics
\begin{aligned}
\dot{e_j} &= \fdyn_{e_j}(e_j, \ctrl_j, \ctrl_{r,j},\dstb_j), \\
\ctrl_j &\in \cset_j, \ctrl_{r,j} \in \cset^p_j, \dstb_j \in \dset_j, \quad t \leq 0
\end{aligned}
\end{equation}

To obtain bounds on the tracking error, we first conservatively estimate the error bound around any reference state $\state_{r,j}$, denoted $\errorbound_j = \{e_j: \|\pos_{e_j}\|_2 \le R_{\text{EB}}\}$,
%\begin{equation} \label{eqn:err}
%\errorbound_j = \{e_j: \|\pos_{e_j}\|_2 \le R_{\text{EB}} \}, 
%\end{equation}
\noindent where $\pos_{e_j}$ denotes the position coordinates of $e_j$ and $R_{\text{EB}}$ is a design parameter. We next solve a reachability problem with its complement $\errorbound_j^c$, the set of tracking errors violating the error bound, as the target in the space of the error dynamics. From $\errorbound_j^c$, we compute the following BRS:
\begin{equation} \label{eqn:errBound}
\begin{aligned}
\brs^{\text{EB}}_{j}(t, 0) = & \{y: \forall \ctrl_j(\cdot) \in \cfset_j, \exists \ctrl_{r, j}(\cdot) \in \cfset^\pos_j, \exists \dstb_j(\cdot) \in \dfset_i, \\
& e_j(\cdot) \text{ satisfies \eqref{eq:edyn}}, e_j(t) = y, \\
& \exists s \in [t, 0], e_j(s) \in \errorbound_j^c\}, 
\end{aligned}
\end{equation}
where the Hamiltonian to compute the BRS is given by:
\begin{equation}
\begin{aligned}
H^{\text{EB}}_{j}(e_j, \costate) &= \max_{\ctrl_j \in \cset_j} \min_{\ctrl_r \in \cset^\pos_j, \dstb_j \in \dset_j} \costate \cdot \fdyn_{e_j}(e_j, \ctrl_j, \ctrl_{r,j}, \dstb_j).
\end{aligned}
\end{equation}

Letting $t \to -\infty$, we obtain the infinite-horizon control-invariant set $\disckernel_j := \lim_{t \to -\infty} \left( \brs^{\text{EB}}_{j}(t, 0) \right)^c$. If $\disckernel_j$ is nonempty, then the tracking error $e_j$ at flight time is guaranteed to remain within $\disckernel_j \subseteq \errorbound_j$ provided that the vehicle starts inside $\disckernel_j$ and subsequently applies the feedback control law
\begin{equation}
\label{eq:robust_tracking_law}
\tracklaw_j(e_j) = \arg\max_{\ctrl_j \in \cset_j} \min_{\ctrl_r \in\cset^\pos_j, \dstb_j \in \dset_j} \costate \cdot \fdyn_{e_j}(e_j,\ctrl_j,\ctrl_{r, j},\dstb_j).
\end{equation}

The induced obstacles by each higher-priority vehicle $\veh_j$ can thus be obtained by: 
\begin{equation} 
\label{eqn:rttObs}
\begin{aligned}
\ioset_i^j(t) &=  \{\state_i: \exists y \in \pfrs_j(t), \|\pos_i - y\|_2 \le \rc \} \\
\pfrs_j(t) &= \{\pos_j: \exists \npos_j, (\pos_j, \npos_j) \in \disckernel_j  + \state_{r,j}(t)\},
\end{aligned}
\end{equation}
\noindent where the ``$+$'' in \eqref{eqn:rttObs} denotes the Minkowski sum\footnote{The Minkowski sum of sets $A$ and $B$ is the set of all points that are the sum of any point in $A$ and $B$.}. Finally, we can obtain the total obstacle set $\obsset_i(t)$ using \eqref{eq:ioset}. %Intuitively, if $\veh_j$ is tracking $\state_{r,j}(t)$, then it will remain within the error bound $\disckernel_j$ around $\state_{r,j}(t) ~\forall t$. This is precisely the set $\pfrs_j(t)$. The induced obstacles can then be obtained by augmenting a danger zone around this set. Finally, we can obtain the total obstacle set $\obsset_i(t)$ using \eqref{eq:obsseti}.

Since each vehicle $\veh_j$, $j<i$, can only be guaranteed to stay within $\disckernel_j$, we must make sure during the path planning of $\veh_i$ that at any given time, the error bounds of $\veh_i$ and $\veh_j$, $\disckernel_i$ and $\disckernel_j$, do not intersect. This can be done by augmenting the total obstacle set by $\disckernel_i$:%This can be done by choosing the induced obstacle to be the Minkowski sum\footnote{The Minkowski sum of sets $A$ and $B$ is the set of all points that are the sum of any point in $A$ and $B$.} of the error bounds. Thus,

\begin{equation} 
\label{eqn:rttAugObs}
\tilde{\obsset}_i(t) = \obsset_i(t) + \disckernel_i.
\end{equation}

Finally, given $\disckernel_i$, we can guarantee that $\veh_i$ will reach its target $\targetset_i$ if $\disckernel_i \subseteq \targetset_i$; thus, in the path planning phase, we modify $\targetset_i$ to be $\tilde{\targetset}_i := \{\state_i: \disckernel_i + \state_i \subseteq \targetset_i\}$, and compute a BRS, with the control authority $\cset^\pos_i$, that contains the initial state of the vehicle. Mathematically,

\begin{equation}
\label{eq:rttBRS}
\begin{aligned}
\brs_i^\text{rtt}(t, \sta_i) = & \{y: \exists \ctrl_i(\cdot) \in \cfset^p_i, \state_i(\cdot) \text{ satisfies \eqref{eq:dyn_no_dstb}},\\
&\forall s \in [t, \sta_i], \state_i(s) \notin \tilde{\obsset}_i(t), \\
& \exists s \in [t, \sta_i], \state_i(s) \in \tilde{\targetset}_i, \state_i(t) = y\}
\end{aligned}
\end{equation}

The Hamiltonian to compute $\brs_i^\text{rtt}(t, \sta_i)$ and the corresponding optimal control for reaching $\tilde{\targetset}_i$ are given by \eqref{eq:basicSPPHam} and \eqref{eq:optCtrl} respectively.
%$\brs_i^\text{rtt}(t, \sta_i)$ can be obtained by solving \eqref{eq:HJIVI_BRS} using the Hamiltonian: 
%\begin{equation}
%\label{eq:RTTham}
%\ham_i^\text{rtt}(\state_i, \costate) = \min_{\ctrl_i \in \cset^\pos_i } \costate \cdot \fdyn_i(\state_i, \ctrl_i)
%\end{equation}
%
%The corresponding optimal control for reaching $\tilde{\targetset}_i$ is given by:
%\begin{equation}
%\label{eq:RTTOptCtrl}
%\ctrl_i^\text{rtt}(t) = \arg \min_{\ctrl_i \in \cset^\pos_i } \costate \cdot \fdyn_i(\state_i, \ctrl_i).
%\end{equation}
The nominal trajectory $\state_{r,i}(\cdot)$ can thus be obtained by using vehicle dynamics \eqref{eq:dyn_no_dstb}, with the optimal control  $\ctrl_i^\text{rtt}(\cdot)$. From the resulting nominal trajectory $\state_{r,i}(\cdot)$, the overall control policy to reach $\targetset_i$ can be obtained via \eqref{eq:robust_tracking_law}.