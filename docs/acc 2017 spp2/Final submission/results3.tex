% !TEX root = SPP2.tex
\subsection{Robust Trajectory Tracking}
In the planning phase, we reduced the maximum turn rate of the vehicles from $1$ to $0.6$, and the speed range from $[0.5, 1]$ to exactly $0.75$ (constant speed). With these reduced control authorities, we determined from the disturbance rejection phase that a nominal trajectory from the planning phase can be robustly tracked within a distance of $0.075$.

Fig. \ref{fig:allTrajs} shows vehicle trajectories in the situation where each vehicle robustly tracks a nominal trajectory. Fig. \ref{fig:rtt_rs3} shows the BRS evolution and induced obstacles for $\veh_3$. %The obstacles induced by other vehicles inhibit the evolution of the BRS, carving out thin “channels,” which can be seen at $t = -2.59$, that separate the BRS into different “islands”. %One can see how these channels and islands form by examining the time evolution of the BRS set.
%
%\begin{figure}[H]
%  \centering
%  \includegraphics[width=0.40\textwidth]{"fig/rtt_traj"}
%  \caption{Simulated trajectories for the robust trajectory tracking method.}
%  \label{fig:rtt_traj}
%  \vspace{-1em}
%\end{figure}
%
\begin{figure}[H]
  \centering
  \includegraphics[width=0.45\textwidth]{"fig/rtt_rs3"}
  \caption{Evolution of the BRS for $\veh_3$ in the robust trajectory tracking method. Note that a smaller target set is used to ensure target reaching for any allowed tracking error.}
  \label{fig:rtt_rs3}
\end{figure}

In this case, the $\ldt_i$ values for the four vehicles are $-1.61, -3.16, -3.57$ and $-2.47$ respectively. In this method, vehicles use reduced control authority for path planning towards a reduced-size effective target set. As a result, higher-priority vehicles tend to have lower $\ldt$ compared to the other two methods, as evident from $\ldt_1$. Because of this ``sacrifice" made by the higher-priority vehicles during the path planning phase, the $\ldt$'s of lower-priority vehicles may increase compared to those in the other methods, as evident from $\ldt_4$. Overall, it is unclear how $\ldt_i$ will change for a vehicle compared to the other methods, as the conservative path planning increases $\ldt_i$ for higher-priority vehicles and decreases $\ldt_i$ for lower-priority vehicles.