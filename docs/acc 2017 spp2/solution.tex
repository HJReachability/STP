% !TEX root = SPP2.tex
\section{Background \label{sec:background}}
This section provides a brief summary of \cite{Chen15}, in which the SPP scheme is proposed under perfect information and absence of disturbance. Here, the dynamics of $\veh_i$ becomes

\begin{equation}
\label{eq:dyn_no_dstb}
\begin{aligned}
\dot{x}_i &= f_i(t, x_i, u_i), \quad t \le \sta_i \\
u_i &\in \cset_i, \quad i = 1,\ldots, N
\end{aligned}
\end{equation}

\noindent where the difference compared to \eqref{eq:dyn} is that the disturbance $d_i$ is no longer part of the dynamics.

In order to make the $N$-vehicle path planning problem safe and tractable, a reasonable structure is imposed to the problem: each vehicle is assigned a strict priority ordering. When planning its trajectory to its target, a higher-priority vehicle can disregard the presence of a lower priority vehicle. In contrast, a lower-priority vehicle must take into account the presence of all higher-priority vehicles, and plan its trajectory in a way that avoids the higher-priority vehicles' danger zones. For convenience and without lost of generality, let $\veh_i$ be the vehicle with the $i$th highest priority. 

Under the above convention, each vehicle $\veh_i$ must take into account time-varying obstacles induced by vehicles $\veh_j, j<i$, denoted $\ioset_i^j(t)$, and represents the set of states that could possibly be in the danger zone of $\veh_j$. Optimal safe path planning of each lower-priority vehicle $\veh_i$ then consists of determining the optimal path that allows $\veh_i$ to reach its target $\targetset_i$ while avoiding the time-varying obstacles $\obsset_i$, defined by

\begin{equation}
\label{eq:ioset}
\obsset_i(t) = \bigcup_{j=1}^{i-1}\ioset_i^j(t)
\end{equation}

Such an optimal path planning problem can be solved by computing a backward reachable set (BRS) $\brs_i(t)$ from a target set $\targetset_i$ using formulations of HJ variational inequalities (VI) such as \cite{Barron90, Bokanowski10, Bokanowski11, Fisac15}. For example, to compute BRSs under the presence of time-varying obstacles, the authors in \cite{Bokanowski11} augmented system with the time variable, and then applied reachability theory for time-invariant systems. To avoid increasing the problem dimension and save computation time, for the simulations of this paper we utilized the formulation in \cite{Fisac15}, which does not require augmentation of the state space with the time variable.

Starting from the highest-priority vehicle $\veh_1$, one computes the BRS $\brs_1(t)$, from which the optimal control and optimal trajectory $x_1(\cdot)$ to the target $\targetset_1$ can be obtained. Under the absence of disturbances and perfect information, the obstacles induced by $\veh_1$ for lower-priority vehicle $\veh_i$ is simply the danger zone centered around the position of each point $p_1(\cdot)$ on the trajectory:
\vspace{-0.5em} 
\begin{equation}
\ioset_i^1(t) = \{x_j: \|p_j - p_1(\cdot)\|\le\cradius\}
\end{equation}

Given $\ioset_i^j(t), j<i$, and continuing with $i = 2$, the optimal safe trajectories for each vehicle $\veh_i$ can be computed. All of the trajectories are optimal in the sense that given the requirement that $\veh_i$ must arrive at $\targetset_i$ at time $\sta_i$, the latest departure time $\ldt_i$ and the optimal control $u^*_i(\cdot)$ that guarantees arrival at $\sta_i$ can be obtained.

To compute $\brs_i(t)$ using the method in \cite{Fisac15}, we solve the following HJ VI:
\vspace{-0.5em} 
\begin{equation}
\label{eq:HJIVI}
\begin{aligned}
\max\Big\{&\min\big\{D_t V_i(t, x_i) + H_i\left(t, x_i, D_{x_i} V_i\right), \\
&l_i(x_i) - V_i(t, x_i)\big\}, -g_i(t, x_i) - V_i(t, x_i)\Big\} = 0 \\
&\quad t\le\sta\\
&V_i(\sta, x_i) = \max\big\{l_i(x_i), -g_i(0, x_i)\big\}\\ 
\end{aligned}
\end{equation}

\begin{equation}
H_i\left(t, x_i, p\right) = \min_{u_i \in \cset} p \cdot f_i(t, x_i, u_i)
\end{equation}

\noindent where $p$ is the gradient of the value function, $D_{x_i} V_i$, and $l_i(x_i), g_i(t,x_i),V_i(t,x_i)$ are implicit surface functions representing the target $\targetset_i$, the time-varying obstacles $\obsset_i(t)$, and the backward reachable set $\brs_i(t)$, respectively: 

\begin{equation}
\label{eq:impl_surf}
\begin{aligned}
x_i \in \targetset_i &\Leftrightarrow l_i(x_i) \le 0 \\
x_i(t) \in \obsset_i(t) &\Leftrightarrow g_i(t,x_i) \le 0 \\
x_i(t) \in \brs_i(t) &\Leftrightarrow V_i(t, x_i) \le 0
\end{aligned}
\end{equation}

The optimal control is given by
 \vspace{-0.4em} 
\begin{equation}
\label{eq:optCtrl}
u^*_i\left(t, x_i\right) = \arg \min_{u_i \in \cset} p \cdot f_i(t, x_i, u_i)
\end{equation}